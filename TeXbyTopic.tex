%%%%%%%%%%%%%%%%%%%%%%%%%%%%%%%%%%%%%%%%%%%%%%%%%%%%%%%%%%%%%%%%
%%%%
%%%% This file belongs with the repository of "TeX by Topic"
%%%% copyright Victor Eijkhout 1991-2023,
%%%% released under the GNU Free Documentation License
%%%%
%%%%%%%%%%%%%%%%%%%%%%%%%%%%%%%%%%%%%%%%%%%%%%%%%%%%%%%%%%%%%%%%

\documentclass[twoside,letterpaper,openright]{rapport3}

%\nofiles

\usepackage{comment,makeidx,multicol}
\usepackage{svn-multi}
\svnid{$Id: TeXbyTopic.tex 3 2008-02-02 11:19:13Z victor $}
\usepackage{newcent}
\renewcommand{\ttdefault}{cmtt}
\addtolength{\oddsidemargin}{-.5in}
\addtolength{\evensidemargin}{-.5in}
\addtolength{\textwidth}{1.2in}
\addtolength{\textheight}{15pt}

\usepackage[pdftex,colorlinks,plainpages=true,pagebackref=true]{hyperref}

\begin{comment}
\usepackage{german}
% german
\righthyphenmin=3
\mdqoff
\captionsenglish
\end{comment}
\usepackage[english]{babel} 
{\catcode`"=13 \gdef"#1{\ifx#1"\discretionary{}{}{}\fi\relax}}
\def\mdqon{\catcode`"=13\relax}
\def\mdqoff{\catcode`"=12\relax}
\makeindex
\hyphenation{ex-em-pli-fies}

\usepackage{fancyhdr}
% headers & footers
\pagestyle{fancy}
% foot
\lfoot[\thepage]{\protect\small\protect\it Victor Eijkhout -- \protect\TeX\ by Topic}
\rfoot[{\protect\small\protect\it Victor Eijkhout -- \protect\TeX\ by Topic}]{\thepage}
\cfoot{}
% head is actually defined later.
\chead{}
\def\normalheads{
\lhead[\let\\\relax \let\uppercase\relax \leftmark]{\relax}
\rhead[\relax]{\let\\\relax \let\uppercase\relax \rightmark}
}
\lhead{}\rhead{}

\newdimen\tempdima \newdimen\tempdimb

% these are fine
\def\handbreak{\\ \message{^^JManual break!!!!^^J}}
\def\nl{\protect\\}\def\n#1{{\tt #1}}
\def\cs#1{\texorpdfstring{{\tt\char`\\#1}}{\textbackslash#1}} %\def\cs#1{{\tt\char`\\#1}}
\let\csc\cs
\def\lb{{\tt\char`\{}}\def\rb{{\tt\char`\}}}
\def\gr#1{\texorpdfstring{$\langle$#1$\rangle$}{<#1>}} %\def\gr#1{$\langle$#1$\rangle$}
\def\key#1{{\tt#1}}
\def\alt{}\def\altt{}%this way in manstijl
\def\ldash{\unskip\ --\nobreak\ \ignorespaces}
\def\rdash{\unskip\nobreak\ --\ \ignorespaces}
% check these
\def\hex{{\tt"}}
\def\ascii{{\sc ascii}}
\def\ebcdic{{\sc ebcdic}}
\def\IniTeX{Ini\TeX}\def\LamsTeX{LAMS\TeX}\def\VirTeX{Vir\TeX}
\def\AmsTeX{Ams\TeX}
\def\TeXbook{the \TeX\ book}\def\web{{\sc web}}
\def\eTeX{(e\TeX\ extension)\ \ignorespaces}

% needs major thinking
\newenvironment{disp}{\begin{quotation}}{\end{quotation}}
\newenvironment{Disp}{\begin{quotation}}{\end{quotation}}
\newenvironment{tdisp}{\begin{quotation}}{\end{quotation}}
\newenvironment{example}{\begin{quotation}}{\end{quotation}}
\newenvironment{inventory}{\begin{description}\raggedright}{\end{description}}
\newenvironment{glossinventory}{\begin{description}}{\end{description}}
\def\gram#1{\gr{#1}}%???
%
% index
%
\def\indexterm#1{\emph{#1}\index{#1}}
\def\indextermsub#1#2{\emph{#1 #2}\index{#1!#2}}
\def\indextermbus#1#2{\emph{#1 #2}\index{#2!#1}}
\def\term#1\par{\index{#1}}
\def\howto#1\par{}
%\def\cstoidx#1\par{\index{#1@\cs{#1}}}
\def\thecstoidx#1\par{\index{#1@\protect\csname #1\endcsname}}
\def\thecstoidxsub#1#2{\index{#1, #2@\protect\csname #1\endcsname, #2}\ignorespaces}
\def\csterm#1{\csidx{#1}\cs{#1}}
\def\csidx#1{\index{#1@\cs{#1}}}%{\cstoidx #1\par\cs{#1}}

\begin{document}

\def\tmc{\tracingmacros=2 \tracingcommands\tracingmacros}

%%%%%%%%%%%%%%%%%%%
\makeatletter
\def\snugbox{\hbox\bgroup\setbox\z@\vbox\bgroup
    \leftskip\z@
    \bgroup\aftergroup\make@snug
    \let\next=}
\def\make@snug{\par\sn@gify\egroup \box\z@\egroup}
\def\sn@gify
   {\skip\z@=\lastskip \unskip
    \advance\skip\z@\lastskip \unskip
    \unpenalty
    \setbox\z@\lastbox
    \ifvoid\z@ \nointerlineskip \else {\sn@gify} \fi
    \hbox{\unhbox\z@}\nointerlineskip
    \vskip\skip\z@
    }

\newdimen\fbh \fbh=60pt % dimension for easy scaling:
\newdimen\fbw \fbw=60pt % height and width of character box

\newdimen\dh \newdimen\dw % height and width of current character box
\newdimen\lh % height of previous character box
\newdimen\lw \lw=.4pt % line weight, instead of default .4pt

\def\hdotfill{\noindent
    \leaders\hbox{\vrule width 1pt height\lw 
                  \kern4pt 
                  \vrule width.5pt height\lw}\hfill\hbox{}
    \par}
\def\hlinefill{\noindent
    \leaders\hbox{\vrule width 5.5pt height\lw         }\hfill\hbox{}
    \par}
\def\stippel{$\qquad\qquad\qquad\qquad$}
\makeatother
%%%%%%%%%%%%%%%%%%%

\begin{comment}
\def\SansSerif{\Typeface:macHelvetica }
\def\SerifFont{\Typeface:macTimes }
\def\SansSerif{\Typeface:bsGillSans }
\def\SerifFont{\Typeface:bsBaskerville }
\end{comment}
\let\SansSerif\relax \def\italic{\it}
\let\SerifFont\relax \def\MainFont{\rm}
\let\SansSerif\relax
\let\SerifFont\relax
\let\PopIndentLevel\relax \let\PushIndentLevel\relax
\let\ToVerso\relax \let\ToRecto\relax

\begin{comment}
\def\stop@command@suffix{stop}
\let\PopListLevel\PopIndentLevel
\let\FlushRight\relax
\let\flushright\FlushRight
\let\SetListIndent\LevelIndent
\def\awp{\ifhmode\vadjust{\penalty-10000 }\else
    \penalty-10000 \fi}
\end{comment}
\let\awp\relax
\let\PopIndentLevel\relax \let\PopListLevel\relax

\showboxdepth=-1

%\input figs
\def\endofchapter{\vfill\noindent}

\title{\TeX\ by Topic, A \TeX nician's Reference}
\date{document revision 1.5, 2019}
\author{Victor Eijkhout}
\maketitle

\thispagestyle{empty}
\hbox{}\vfill\hbox{}\pagebreak
\setcounter{page}{1}

  \begin{minipage}[h]{1.0\linewidth}\parskip 15pt
  Copyright \copyright\  1991-2013 Victor Eijkhout.\\
  Permission is granted to copy, distribute and/or modify this document
  under the terms of the GNU Free Documentation License, Version 1.2
  or any later version published by the Free Software Foundation;
  with no Invariant Sections, no Front-Cover Texts, and no Back-Cover
  Texts.  A copy of the license is included in the section entitled "GNU
  Free Documentation License".

This document is based on the book \TeX\ by Topic,
copyright 1991-2019 Victor Eijkhout. This book was
printed in~1991 by Addison-Wesley UK, ISBN 0-201-56882-9, reprinted
in~1993, pdf version first made freely available in~2001.

Cover design (lulu.com version): Joanna K. Wozniak (jokwoz@gmail.com)

  \end{minipage}
\pagebreak

\tableofcontents

\pagebreak
\addcontentsline{toc}{section}{License}
\paragraph*{\bf GNU Free Documentation License}

Version 1.2, November 2002

  Copyright \copyright\ 2000,2001,2002  Free Software Foundation, Inc.
  51 Franklin St, Fifth Floor, Boston, MA  02110-1301  USA
  Everyone is permitted to copy and distribute verbatim copies
  of this license document, but changing it is not allowed.

0. PREAMBLE

The purpose of this License is to make a manual, textbook, or other
functional and useful document "free" in the sense of freedom: to
assure everyone the effective freedom to copy and redistribute it,
with or without modifying it, either commercially or noncommercially.
Secondarily, this License preserves for the author and publisher a way
to get credit for their work, while not being considered responsible
for modifications made by others.

This License is a kind of "copyleft", which means that derivative
works of the document must themselves be free in the same sense. It
complements the GNU General Public License, which is a copyleft
license designed for free software.

We have designed this License in order to use it for manuals for free
software, because free software needs free documentation: a free
program should come with manuals providing the same freedoms that the
software does. But this License is not limited to software manuals; it
can be used for any textual work, regardless of subject matter or
whether it is published as a printed book. We recommend this License
principally for works whose purpose is instruction or reference.

1. APPLICABILITY AND DEFINITIONS

This License applies to any manual or other work, in any medium, that
contains a notice placed by the copyright holder saying it can be
distributed under the terms of this License. Such a notice grants a
world-wide, royalty-free license, unlimited in duration, to use that
work under the conditions stated herein. The "Document", below, refers
to any such manual or work. Any member of the public is a licensee,
and is addressed as "you". You accept the license if you copy, modify
or distribute the work in a way requiring permission under copyright
law.

A "Modified Version" of the Document means any work containing the
Document or a portion of it, either copied verbatim, or with
modifications and/or translated into another language.

A "Secondary Section" is a named appendix or a front-matter section of
the Document that deals exclusively with the relationship of the
publishers or authors of the Document to the Document's overall
subject (or to related matters) and contains nothing that could fall
directly within that overall subject. (Thus, if the Document is in
part a textbook of mathematics, a Secondary Section may not explain
any mathematics.) The relationship could be a matter of historical
connection with the subject or with related matters, or of legal,
commercial, philosophical, ethical or political position regarding
them.

The "Invariant Sections" are certain Secondary Sections whose titles
are designated, as being those of Invariant Sections, in the notice
that says that the Document is released under this License. If a
section does not fit the above definition of Secondary then it is not
allowed to be designated as Invariant. The Document may contain zero
Invariant Sections. If the Document does not identify any Invariant
Sections then there are none.

The "Cover Texts" are certain short passages of text that are listed,
as Front-Cover Texts or Back-Cover Texts, in the notice that says that
the Document is released under this License. A Front-Cover Text may be
at most 5 words, and a Back-Cover Text may be at most 25 words.

A "Transparent" copy of the Document means a machine-readable copy,
represented in a format whose specification is available to the
general public, that is suitable for revising the document
straightforwardly with generic text editors or (for images composed of
pixels) generic paint programs or (for drawings) some widely available
drawing editor, and that is suitable for input to text formatters or
for automatic translation to a variety of formats suitable for input
to text formatters. A copy made in an otherwise Transparent file
format whose markup, or absence of markup, has been arranged to thwart
or discourage subsequent modification by readers is not Transparent.
An image format is not Transparent if used for any substantial amount
of text. A copy that is not "Transparent" is called "Opaque".

Examples of suitable formats for Transparent copies include plain
ASCII without markup, Texinfo input format, LaTeX input format, SGML
or XML using a publicly available DTD, and standard-conforming simple
HTML, PostScript or PDF designed for human modification. Examples of
transparent image formats include PNG, XCF and JPG. Opaque formats
include proprietary formats that can be read and edited only by
proprietary word processors, SGML or XML for which the DTD and/or
processing tools are not generally available, and the
machine-generated HTML, PostScript or PDF produced by some word
processors for output purposes only.

The "Title Page" means, for a printed book, the title page itself,
plus such following pages as are needed to hold, legibly, the material
this License requires to appear in the title page. For works in
formats which do not have any title page as such, "Title Page" means
the text near the most prominent appearance of the work's title,
preceding the beginning of the body of the text.

A section "Entitled XYZ" means a named subunit of the Document whose
title either is precisely XYZ or contains XYZ in parentheses following
text that translates XYZ in another language. (Here XYZ stands for a
specific section name mentioned below, such as "Acknowledgements",
"Dedications", "Endorsements", or "History".) To "Preserve the Title"
of such a section when you modify the Document means that it remains a
section "Entitled XYZ" according to this definition.

The Document may include Warranty Disclaimers next to the notice which
states that this License applies to the Document. These Warranty
Disclaimers are considered to be included by reference in this
License, but only as regards disclaiming warranties: any other
implication that these Warranty Disclaimers may have is void and has
no effect on the meaning of this License.

2. VERBATIM COPYING

You may copy and distribute the Document in any medium, either
commercially or noncommercially, provided that this License, the
copyright notices, and the license notice saying this License applies
to the Document are reproduced in all copies, and that you add no
other conditions whatsoever to those of this License. You may not use
technical measures to obstruct or control the reading or further
copying of the copies you make or distribute. However, you may accept
compensation in exchange for copies. If you distribute a large enough
number of copies you must also follow the conditions in section 3.

You may also lend copies, under the same conditions stated above, and
you may publicly display copies.

3. COPYING IN QUANTITY

If you publish printed copies (or copies in media that commonly have
printed covers) of the Document, numbering more than 100, and the
Document's license notice requires Cover Texts, you must enclose the
copies in covers that carry, clearly and legibly, all these Cover
Texts: Front-Cover Texts on the front cover, and Back-Cover Texts on
the back cover. Both covers must also clearly and legibly identify you
as the publisher of these copies. The front cover must present the
full title with all words of the title equally prominent and visible.
You may add other material on the covers in addition. Copying with
changes limited to the covers, as long as they preserve the title of
the Document and satisfy these conditions, can be treated as verbatim
copying in other respects.

If the required texts for either cover are too voluminous to fit
legibly, you should put the first ones listed (as many as fit
reasonably) on the actual cover, and continue the rest onto adjacent
pages.

If you publish or distribute Opaque copies of the Document numbering
more than 100, you must either include a machine-readable Transparent
copy along with each Opaque copy, or state in or with each Opaque copy
a computer-network location from which the general network-using
public has access to download using public-standard network protocols
a complete Transparent copy of the Document, free of added material.
If you use the latter option, you must take reasonably prudent steps,
when you begin distribution of Opaque copies in quantity, to ensure
that this Transparent copy will remain thus accessible at the stated
location until at least one year after the last time you distribute an
Opaque copy (directly or through your agents or retailers) of that
edition to the public.

It is requested, but not required, that you contact the authors of the
Document well before redistributing any large number of copies, to
give them a chance to provide you with an updated version of the
Document.

4. MODIFICATIONS

You may copy and distribute a Modified Version of the Document under
the conditions of sections 2 and 3 above, provided that you release
the Modified Version under precisely this License, with the Modified
Version filling the role of the Document, thus licensing distribution
and modification of the Modified Version to whoever possesses a copy
of it. In addition, you must do these things in the Modified Version:

A. Use in the Title Page (and on the covers, if any) a title distinct
from that of the Document, and from those of previous versions (which
should, if there were any, be listed in the History section of the
Document). You may use the same title as a previous version if the
original publisher of that version gives permission.  B. List on the
Title Page, as authors, one or more persons or entities responsible
for authorship of the modifications in the Modified Version, together
with at least five of the principal authors of the Document (all of
its principal authors, if it has fewer than five), unless they release
you from this requirement.  C. State on the Title page the name of the
publisher of the Modified Version, as the publisher.  D. Preserve all
the copyright notices of the Document.  E. Add an appropriate
copyright notice for your modifications adjacent to the other
copyright notices.  F. Include, immediately after the copyright
notices, a license notice giving the public permission to use the
Modified Version under the terms of this License, in the form shown in
the Addendum below.  G. Preserve in that license notice the full lists
of Invariant Sections and required Cover Texts given in the Document's
license notice.  H. Include an unaltered copy of this License.  I.
Preserve the section Entitled "History", Preserve its Title, and add
to it an item stating at least the title, year, new authors, and
publisher of the Modified Version as given on the Title Page. If there
is no section Entitled "History" in the Document, create one stating
the title, year, authors, and publisher of the Document as given on
its Title Page, then add an item describing the Modified Version as
stated in the previous sentence.  J. Preserve the network location, if
any, given in the Document for public access to a Transparent copy of
the Document, and likewise the network locations given in the Document
for previous versions it was based on. These may be placed in the
"History" section. You may omit a network location for a work that was
published at least four years before the Document itself, or if the
original publisher of the version it refers to gives permission.  K.
For any section Entitled "Acknowledgements" or "Dedications", Preserve
the Title of the section, and preserve in the section all the
substance and tone of each of the contributor acknowledgements and/or
dedications given therein.  L. Preserve all the Invariant Sections of
the Document, unaltered in their text and in their titles. Section
numbers or the equivalent are not considered part of the section
titles.  M. Delete any section Entitled "Endorsements". Such a section
may not be included in the Modified Version.  N. Do not retitle any
existing section to be Entitled "Endorsements" or to conflict in title
with any Invariant Section.  O. Preserve any Warranty Disclaimers.  If
the Modified Version includes new front-matter sections or appendices
that qualify as Secondary Sections and contain no material copied from
the Document, you may at your option designate some or all of these
sections as invariant. To do this, add their titles to the list of
Invariant Sections in the Modified Version's license notice. These
titles must be distinct from any other section titles.

You may add a section Entitled "Endorsements", provided it contains
nothing but endorsements of your Modified Version by various
parties--for example, statements of peer review or that the text has
been approved by an organization as the authoritative definition of a
standard.

You may add a passage of up to five words as a Front-Cover Text, and a
passage of up to 25 words as a Back-Cover Text, to the end of the list
of Cover Texts in the Modified Version. Only one passage of
Front-Cover Text and one of Back-Cover Text may be added by (or
through arrangements made by) any one entity. If the Document already
includes a cover text for the same cover, previously added by you or
by arrangement made by the same entity you are acting on behalf of,
you may not add another; but you may replace the old one, on explicit
permission from the previous publisher that added the old one.

The author(s) and publisher(s) of the Document do not by this License
give permission to use their names for publicity for or to assert or
imply endorsement of any Modified Version.

5. COMBINING DOCUMENTS

You may combine the Document with other documents released under this
License, under the terms defined in section 4 above for modified
versions, provided that you include in the combination all of the
Invariant Sections of all of the original documents, unmodified, and
list them all as Invariant Sections of your combined work in its
license notice, and that you preserve all their Warranty Disclaimers.

The combined work need only contain one copy of this License, and
multiple identical Invariant Sections may be replaced with a single
copy. If there are multiple Invariant Sections with the same name but
different contents, make the title of each such section unique by
adding at the end of it, in parentheses, the name of the original
author or publisher of that section if known, or else a unique number.
Make the same adjustment to the section titles in the list of
Invariant Sections in the license notice of the combined work.

In the combination, you must combine any sections Entitled "History"
in the various original documents, forming one section Entitled
"History"; likewise combine any sections Entitled "Acknowledgements",
and any sections Entitled "Dedications". You must delete all sections
Entitled "Endorsements."

6. COLLECTIONS OF DOCUMENTS

You may make a collection consisting of the Document and other
documents released under this License, and replace the individual
copies of this License in the various documents with a single copy
that is included in the collection, provided that you follow the rules
of this License for verbatim copying of each of the documents in all
other respects.

You may extract a single document from such a collection, and
distribute it individually under this License, provided you insert a
copy of this License into the extracted document, and follow this
License in all other respects regarding verbatim copying of that
document.

7. AGGREGATION WITH INDEPENDENT WORKS

A compilation of the Document or its derivatives with other separate
and independent documents or works, in or on a volume of a storage or
distribution medium, is called an "aggregate" if the copyright
resulting from the compilation is not used to limit the legal rights
of the compilation's users beyond what the individual works permit.
When the Document is included in an aggregate, this License does not
apply to the other works in the aggregate which are not themselves
derivative works of the Document.

If the Cover Text requirement of section 3 is applicable to these
copies of the Document, then if the Document is less than one half of
the entire aggregate, the Document's Cover Texts may be placed on
covers that bracket the Document within the aggregate, or the
electronic equivalent of covers if the Document is in electronic form.
Otherwise they must appear on printed covers that bracket the whole
aggregate.

8. TRANSLATION

Translation is considered a kind of modification, so you may
distribute translations of the Document under the terms of section 4.
Replacing Invariant Sections with translations requires special
permission from their copyright holders, but you may include
translations of some or all Invariant Sections in addition to the
original versions of these Invariant Sections. You may include a
translation of this License, and all the license notices in the
Document, and any Warranty Disclaimers, provided that you also include
the original English version of this License and the original versions
of those notices and disclaimers. In case of a disagreement between
the translation and the original version of this License or a notice
or disclaimer, the original version will prevail.

If a section in the Document is Entitled "Acknowledgements",
"Dedications", or "History", the requirement (section 4) to Preserve
its Title (section 1) will typically require changing the actual
title.

9. TERMINATION

You may not copy, modify, sublicense, or distribute the Document
except as expressly provided for under this License. Any other attempt
to copy, modify, sublicense or distribute the Document is void, and
will automatically terminate your rights under this License. However,
parties who have received copies, or rights, from you under this
License will not have their licenses terminated so long as such
parties remain in full compliance.

10. FUTURE REVISIONS OF THIS LICENSE

The Free Software Foundation may publish new, revised versions of the
GNU Free Documentation License from time to time. Such new versions
will be similar in spirit to the present version, but may differ in
detail to address new problems or concerns. See
http://www.gnu.org/copyleft/.

Each version of the License is given a distinguishing version
number. If the Document specifies that a particular numbered version
of this License "or any later version" applies to it, you have the
option of following the terms and conditions either of that specified
version or of any later version that has been published (not as a
draft) by the Free Software Foundation. If the Document does not
specify a version number of this License, you may choose any version
ever published (not as a draft) by the Free Software Foundation.

\pagebreak
\mark{}
\addcontentsline{toc}{section}{Preface}

\paragraph*{\bf Preface}
To the casual observer, \TeX\
is not a state-of-the-art typesetting system.
No flashy multilevel menus and interactive manipulation
of text and graphics dazzle the onlooker.
On a less superficial level, however, \TeX\ is a very sophisticated
program, first of all because of the ingeniousness of its
built-in algorithms for such things as paragraph breaking
and make-up of mathematical formulas, and
second because of its almost complete programmability.
The combination of these factors makes it possible for \TeX\
to realize almost every imaginable layout in a highly automated
fashion.

Unfortunately, it also means that \TeX\ has an
unusually large number of commands and parameters,
and that programming \TeX\ can be far from easy.
Anyone wanting to program in \TeX, and maybe
even the ordinary user, would seem to need two books:
a~tutorial that gives a first glimpse of the many
nuts and bolts of \TeX, and after that
a~systematic, complete reference manual.
This book tries to fulfil the latter function.
A~\TeX er who has already made a start
(using any of a number of introductory books
on the market)
should be able to use this book indefinitely thereafter.

In this volume the universe of \TeX\ is presented as
about forty different subjects, each in a separate
chapter.
Each chapter starts out with a list of control sequences
relevant to the topic of that chapter
and proceeds to treat the 
theory of the topic. 
Most chapters conclude with remarks and examples.

Globally, the chapters are ordered as follows. 
The chapters on basic mechanisms are first,
the chapters on text treatment and mathematics are next,
and finally there are some
chapters on output and aspects of \TeX's connections to
the outside world.
%
The book also contains a glossary of \TeX\
commands, tables,
and indexes by example, by control sequence, and by subject.
The subject index refers for most concepts to
only one page, where most of the information
on that topic can be found, as well as references
to the locations of related information.

This book does not treat any specific \TeX\ macro package.
Any parts of the plain format that are treated are those
parts that belong to the `core' of plain \TeX: they
are also present in, for instance, \LaTeX.
Therefore, most remarks about the plain format
are true for \LaTeX, as well as most other formats.
Putting it differently,
if the text refers to the plain format, this should be taken
as a contrast to pure \IniTeX, not to \LaTeX.
By way of illustration, occasionally macros from plain \TeX\
are explained that do not belong to the core.

\medskip\noindent
{\bf Acknowledgment}\nl
I am indebted to Barbara Beeton, Karl Berry, and Nico Poppelier,
who read previous versions of this book. Their comments
helped to improve the presentation.
Also I~would like to thank the participants of
the discussion lists \TeX hax, \TeX-nl, and {\tt comp.text.tex}.
Their questions and answers gave me much food for thought.
Finally, any acknowledgement in a book about \TeX\ ought to
include Donald Knuth for inventing \TeX\ in the
first place. This book is no exception.

\begin{flushright}
 Victor Eijkhout\\
 Urbana, Illinois, August 1991\\
 Knoxville, Tennessee, May 2001\\
 Austin, Texas, December 2013
\end{flushright}
\pagebreak
\normalheads

\chapter{The Structure of the \TeX\ Processor}

This book treats the various aspects of \TeX\ in chapters
that are concerned with relatively small, well-delineated,
topics. In this chapter, therefore, 
a global picture of the way \TeX\ operates will be given.
Of necessity, many details will be omitted here, but all of
these are treated in later chapters. On the other hand,
the few examples given in this chapter will be repeated
in the appropriate places later on; they are included here
to make this chapter self-contained.

\section{Four \TeX\protect\ processors}

The way \TeX\ processes its input can be viewed as
happening on four levels. One might  say that
the \TeX\ processor is split into four separate units,
each one accepting the output of the previous stage, and
delivering the input for the next stage. The input of
the first stage is then the \n{.tex} input file; the output
of the last stage is a \n{.dvi} file.

For many purposes it is most convenient, and most insightful,
to consider these four levels of processing as happening
after one another, each one accepting the {\em completed\/}
output of the previous level. In reality this is not true:
all levels are simultaneously
active, and there is interaction between them.

The four levels are (corresponding roughly
to the `eyes', `mouth', `stomach', and `bowels' respectively
in Knuth's original terminology) as follows.
\begin{enumerate}\item
The input processor. This is the piece of \TeX\ that
accepts input lines from the file system of whatever computer
\TeX\ runs on, and turns them into tokens.
Tokens are the internal objects of \TeX:
there are character tokens that constitute the typeset
text, and control sequence tokens that are commands 
to be processed by the next two levels.
\item The expansion processor. 
Some but not all of the tokens generated in the first level
\ldash macros, conditionals, and a number
of primitive \TeX\ commands \rdash  are subject to expansion.
Expansion is the process that replaces some (sequences of)
tokens by other (or no) tokens.
\item The execution processor. 
Control sequences that are not expandable are executable,
and this execution takes place on the third level of the
\TeX\ processor.

One part of the activity here concerns changes to
\TeX's internal state: assignments (including
macro definitions) are typical activities in this
category. The other major thing happening on this level
is the construction of horizontal, vertical, and
mathematical lists.
\item The visual processor. 
In the final level of processing
the visual part of \TeX\ processing is performed. Here
horizontal lists are broken into paragraphs, 
vertical lists are broken into pages,
and  formulas are built out of math lists. 
Also the output to the \n{dvi} file takes place on this level.
The algorithms working here are not accessible to the user,
but they can be influenced by a number of parameters.
\end{enumerate}

\section{The input processor}

The input processor of \TeX\ is that part of \TeX\ that
translates whatever characters it gets from the input file
into tokens. The output of this processor is a stream
of tokens: a token list. Most tokens fall into one of two categories:
character tokens and control sequence tokens. 
The remaining category is that of the parameter tokens;
these will not be treated in this chapter.

\subsection{Character input}

For simple input text, characters are made into
character tokens. However, \TeX\ can ignore input characters:
a row of spaces in the input is usually equivalent to just one
space. Also, \TeX\ itself can insert tokens that do not correspond
to any character in the input, for instance the space token
at the end of the line, or the \cs{par} token after an empty line.

Not all character tokens signify characters to be typeset.
\altt
Characters fall into sixteen categories \ldash each one
specifying a certain function that a character can have \rdash 
of which only two contain the characters that will be
typeset. The other categories contain such characters 
as~\n{\char`\{}, \n{\char`\}}, 
\n\&, and~\n\#. A~character token can be considered
as a pair of numbers: the character code \ldash typically the \ascii\
code \rdash  and the category code.
It is possible to change
the category code that is associated with a particular
character code.

When the escape character (by default~\cs{}$\,$) appears in the input,
\TeX's behaviour in forming tokens is more complicated. 
Basically,
\TeX\ builds a control sequence by taking a number of characters
from the input and lumping them together into a single token.

The behaviour with which \TeX's input processor 
reacts to category codes can be described
as a machine that switches between three internal states:
$N$,~new line; $M$,~middle of line; $S$,~skipping spaces.
These states and the transitions between them are treated
in Chapter~\ref{mouth}.

\subsection{Two-level input processing}

\TeX's input processor is in fact itself a two-level processor.
Because of limitations of the terminal, the editor, or the operating
system, the user may not be able to input certain desired characters.
Therefore, \TeX\ provides a mechanism to access
with two superscript characters all of the available character
positions. This may be considered
a separate stage of \TeX\ processing, taking place
prior to the three-state machine mentioned above.

For instance, the sequence \verb>^^+> is replaced by~\n{k} because
the \ascii{} codes of \n k and \n + differ by~64. 
Since this replacement takes place before tokens are formed,
writing \verb>\vs^^+ip 5cm> has the same effect as
\verb>\vskip 5cm>. Examples more useful than this exist.

Note that this first stage is a transformation from
characters to characters, without considering category
codes. These come into play only in the second phase
of input processing where characters are converted
to character tokens by coupling the category code
to the character code.

\section{The expansion processor}

\TeX's  expansion processor accepts a stream of tokens
and, if possible, 
expands the tokens in this stream one by one
until only unexpandable tokens remain.
Macro expansion is the clearest example of this:
if a control sequence is a macro name, it is replaced
(together possibly with parameter tokens) by 
the definition text of the macro.

Input for the expansion processor is provided mainly
by the input processor. The stream of tokens coming
from the first stage of \TeX\ processing is subject
to the expansion process, and the result is a stream
of unexpandable tokens which is fed to the execution processor.

However, the expansion processor comes into play 
also when (among others) an \cs{edef} or \cs{write} is processed.
The parameter token list of these commands is
expanded very much as if the lists had been
on the top level, instead of the argument to a command.

\subsection{The process of expansion}

Expanding a token consists of the following steps:
\begin{enumerate}
\item See whether the token is expandable. 
\item If the token is unexpandable, pass it to the token
      list currently being built, and take on the next token. 

\item If the token is expandable, replace it by its expansion.
      For macros without parameters, and a few primitive commands
      such as \cs{jobname}, this is indeed a simple replacement.
      Usually, however, \TeX\ needs to absorb some argument tokens from
      the stream in order to be able to form the replacement
      of the current token.
      For instance, if the token was a macro with parameters,
      sufficiently many tokens need to be absorbed to form
      the arguments corresponding to  these parameters.

\item Go on expanding, starting with the first token of the
      expansion. 
\end{enumerate}
%
Deciding whether a token is expandable is
a simple decision. Macros and active characters, 
conditionals, and a number of primitive \TeX\ commands
(see the list on page~\pageref{expand:lijst})
are expandable, other tokens are not.
Thus the expansion processor replaces macros by their expansion,
it evaluates conditionals and eliminates any irrelevant parts of 
these, but tokens such as \cs{vskip} and character tokens,
including characters such as dollars and braces, are passed untouched.

\subsection{Special cases: \cs{expandafter}, \cs{noexpand}, and \cs{the}}

As stated above,
after a token has been expanded, \TeX\ will start expanding
the resulting tokens. At first sight the \cs{expandafter}
command would seem to be an exception to this rule, because
it expands only one step. What actually happens is that
the sequence \begin{disp}\cs{expandafter}\gr{token$_1$}\gr{token$_2$}\end{disp}
is replaced by 
\begin{disp}\gr{token$_1$}\gr{\italic expansion of token$_2$}\end{disp}
and this replacement is in fact reexamined by the expansion
processor.

Real exceptions do exist, however. If the 
current token is the \cs{noexpand} command, the next
token is considered for the moment to be unexpandable:
it is handled as if it were \cs{relax}, and it is
passed to the token list being built.

For example,
in the macro definition
\begin{verbatim}
\edef\a{\noexpand\b}
\end{verbatim}
the replacement text \verb>\noexpand\b> is expanded at definition 
time. The expansion of \cs{noexpand} is the next token, with
a temporary meaning of \cs{relax}. Thus, when the expansion
processor tackles the next token, the~\cs{b}, it will consider
that to be unexpandable, and just pass it to the token list
being built, which is the replacement text of the macro.

Another exception is that the tokens
resulting from \cs{the}\gr{token variable}
are not expanded further if this statement occurs
inside an \cs{edef} macro definition.

\subsection{Braces in the expansion processor}

Above, it was said that braces are passed as unexpandable
character tokens. In general this is true. For instance,
the \cs{romannumeral} command is handled by the expansion
processor; when confronted with 
\begin{verbatim}
\romannumeral1\number\count2 3{4 ...
\end{verbatim} 
\TeX\ will expand until the brace is encountered:
if \cs{count2} has the value of zero, the result will be
the roman numeral representation of~\n{103}.

As another example, \begin{verbatim}
\iftrue {\else }\fi
\end{verbatim}
is handled by the expansion processor 
completely analogous to
\begin{disp}\cs{iftrue} {\italic a}\cs{else} {\italic b}\cs{fi}\end{disp}
The result is a character token, independent of its category.

However, in the context of macro expansion 
the expansion  processor will 
recognize braces. 
First of all, a balanced pair of braces marks off a group of tokens
to be passed as one argument.
If a macro has an argument \begin{verbatim}
\def\macro#1{ ... }
\end{verbatim}
one can call it with a single token, as in
\begin{verbatim}
\macro 1 \macro \$
\end{verbatim}
or with a group of tokens, surrounded by braces
\begin{verbatim}
\macro {abc} \macro {d{ef}g}
\end{verbatim}


Secondly, when the arguments for a macro with
parameters are read, no expressions with unbalanced braces
are accepted. In 
\begin{verbatim}
\def\a#1\stop{ ... }
\end{verbatim}
the argument consists of all
tokens up to the first occurrence of \cs{stop}
that is not in braces: in
\begin{verbatim}
\a bc{d\stop}e\stop
\end{verbatim}
the argument of~\cs{a} is \verb>bc{d\stop}e>.
Only balanced expressions
are accepted here.

\section{The execution processor}

The execution processor builds lists: horizontal, vertical,
and math lists. Corresponding to these lists, it works
in horizontal, vertical, or math mode. Of these three modes
`internal' and `external' variants exist.
In addition to building lists, this part of the \TeX\ processor
also performs mode-independent processing, such as
assignments.

Coming out of the expansion processor is a stream of
unexpandable tokens to be processed by
the execution processor. 
\relax From the point of view of the execution processor, this
stream contains two types of tokens:
\begin{itemize}
\item Tokens signalling an assignment (this includes
      macro definitions), and
      other tokens signalling actions
      that are independent of the mode, such
      as \cs{show} and \cs{aftergroup}.
\item Tokens that build lists:
      characters, boxes, and glue. The way they are handled
      depends on the current mode.
\end{itemize}

Some objects can be used in any mode; for instance boxes
can appear in horizontal, vertical, and math lists.
The effect of such an object will of course still depend on the mode.
Other objects are  specific for one mode.
For instance, characters (to be more precise:
character tokens of categories 11 and~12), 
are intimately connected to horizontal mode:
if the execution processor 
is in vertical mode when it encounters a character, it will
switch to horizontal mode.

Not all character tokens signal characters to be typeset:
the execution processor can also encounter math shift
characters (by default~\n{\char`\$}) and beginning/end of group
characters (by default \n{\char`\{} and~\n{\char`\}}).
Math shift characters let \TeX\ enter or exit
math mode, and braces let it enter or exit a~new level of
grouping.

One control sequence handled by the execution processor 
deserves special mention: \cs{relax}.
This control sequence is not expandable, but the execution
is to do nothing. Compare the effect of \cs{relax} in
\begin{verbatim}
\count0=1\relax 2
\end{verbatim}
with that of \cs{empty}
defined by
\begin{verbatim}
\def\empty{}
\end{verbatim}
in 
\begin{verbatim}
\count0=1\empty 2
\end{verbatim}
In the first case the expansion
process that is forming the number stops at \cs{relax} and
the number {\tt 1} is assigned; in the second case 
\cs{empty} expands to nothing, so {\tt 12} is assigned.

\section{The visual processor}

\TeX's output processor encompasses those algorithms that
are outside direct user control: paragraph breaking,
alignment, page breaking, math typesetting, and \n{dvi} file
generation. Various parameters control the operation
of these parts of \TeX.

Some of these algorithms return their results in a form that
can be handled by the execution processor. For instance,
a paragraph that has been broken into lines is added to
the main vertical list as a sequence of horizontal boxes
with intermediate glue and penalties. Also, the page breaking
algorithm stores its result in \cs{box255}, so output
routines can dissect it. On the other hand, a math formula
can not be broken into pieces, and, naturally, 
shipping a box to the \n{dvi} file is irreversible.

\section{Examples}

\subsection{Skipped spaces}

Skipped spaces provide an illustration of the view that
\TeX's levels of processing accept the completed input
of the previous level. Consider the commands
\begin{verbatim}
\def\a{\penalty200}
\a 0
\end{verbatim} 
This is {\italic not\/} equivalent to
\begin{verbatim}
\penalty200 0
\end{verbatim} 
which would place a penalty of \n{200}, and
typeset the digit~\n0. Instead it expands to
\begin{verbatim}
\penalty2000
\end{verbatim}
because the space after \cs{a} is skipped in the
input processor. Later stages of processing then receive
the sequence \begin{verbatim}
\a0
\end{verbatim}

\subsection{Internal quantities and their representations}

\TeX\ uses various sorts of internal quantities,
such as integers and dimensions. These internal
quantities have an external representation,
which is a string of characters, such as 
\n{4711} or~\n{91.44cm}.

Conversions between the internal value and the external
representation take place on two different levels,
depending on what direction the conversion goes.
A~string of characters is converted to an internal
value in assignments such as
\begin{verbatim}
\pageno=12 \baselineskip=13pt
\end{verbatim}
or statements such as
\begin{verbatim}
\vskip 5.71pt
\end{verbatim}
and all of these statements are handled by the execution
processor.

On the other hand, the conversion of the internal
values into a representation as a string of
characters is handled by the expansion processor.
For instance, \begin{verbatim}
\number\pageno \romannumeral\year
\the\baselineskip
\end{verbatim}
are all processed by expansion.

As a final example, suppose \verb>\count2=45>, and
consider the statement
\begin{verbatim}
\count0=1\number\count2 3
\end{verbatim}
The expansion processor tackles \verb>\number\count2>
to give the characters \n{45}, and the space after
the \n 2 does not end the number being assigned:
it only serves as a delimiter
of the number of the \cs{count} register.
In the next stage of processing, the execution processor
will then see the statement
\begin{verbatim}
\count0=1453
\end{verbatim}
and execute this.

%\endinput

%%%% end of input file [bigpic]

\chapter{Category Codes and Internal States}\label{mouth}

When characters are read, 
\TeX\ assigns them
category codes. The reading mechanism has three internal
states, and transitions between these states are affected
by category codes of characters in the input.
This chapter describes how \TeX\ reads its input and
how the category codes of characters influence the
reading behaviour. Spaces and line ends are discussed.

\label{cschap:endlinechar}\label{cschap:ignorespaces}\label{cschap:catcode}\label{cschap:char32}\label{cschap:obeylines}\label{cschap:obeyspaces}
\begin{inventory}
\item [\cs{endlinechar}]  
      The character code of the end-of-line character 
      appended to input lines.
      \IniTeX\ default:~13.
\item [\cs{par}]  
      Command to close off a paragraph and go into vertical mode.
      Is generated by empty lines.

\item [\cs{ignorespaces}]   
      Command that reads and expands until something is
      encountered that is not a \gr{space token}.

\item [\cs{catcode}] 
      Query or set category codes.

\item [\cs{ifcat}]  
      Test whether two characters have the same category code.

\item [\cs{\char32}]
      Control space.
      Insert the same amount of space that a space token would
      when \cs{spacefactor}${}=1000$.

\item [\cs{obeylines}]
      Macro in plain \TeX\ to make line ends significant.

\item [\cs{obeyspaces}]
      Macro in plain \TeX\ to make (most) spaces significant.
\end{inventory}

\section{Introduction}

\TeX's input processor scans input lines from a file or terminal, and
makes tokens out of the characters.
The input processor can be viewed as
a simple finite state automaton with three internal states; 
depending on the state its scanning behaviour may differ.
This automaton will be treated here both from the point of view of the
internal states and of the category codes governing the
transitions.

\section{Initial processing}

Input from a file (or from the user terminal, but this
will not be mentioned specifically
most of the time) is handled one line at a time.
Here follows a discussion of what exactly is an input line
for \TeX.

Computer systems differ with respect to 
\index{line! input}\index{line! end}\index{machine independence}
the exact definition of an input
\mdqon
line. The carriage return/""line feed
\mdqoff
\message{slash-dash}%
sequence terminating a line is most common,
but some systems use just a line feed, and
some systems with fixed record length (block) storage do not have
a line terminator at all. Therefore \TeX\ has its
own way of terminating an input line.

\begin{enumerate}
\item An input line is read from an input file  (minus the
line terminator, if any).
\item Trailing spaces are removed (this is for the systems
with block storage, and it prevents confusion because these
spaces are hard to see in an editor).
\item The \csterm{endlinechar} is appended. By default this is character code~13,
  which is the \ascii{} code of the \gram{return} character.
  However, note that \TeX\ is independent of \ascii.
If the value of \cs{endlinechar} is negative
\label{append:elc}%
or more than~255 (this was 127 in versions of \TeX\ older
than version~3; see page~\pageref{2vs3} for more differences),
no character is appended. 
The effect then is the same as
if the line were to end with a comment character.
\end{enumerate}


Computers may also differ in the character encoding
(the most common schemes are \ascii{} and \ebcdic{}), so \TeX\
converts the characters that are read from the file to its
own character codes. These codes are then used exclusively,
so that \TeX\ will perform the same on any system.
For more on this, see Chapter~\ref{char}.

\section{Category codes}

Each of the 256 character codes (0--255) has an
associated \indexterm{category code}, though not necessarily always the same one.
There are 16 categories, numbered 0--15. 
When scanning the input, \TeX\
thus forms character-code--category-code pairs.
The input processor sees only these pairs; from them are formed
character tokens, control sequence tokens, and parameter tokens.
These tokens are then passed to \TeX's expansion and execution
processes.

A~character token is a character-code--category-code
pair that is passed unchanged.
A~control sequence token consists of one or more characters
preceded by an escape character; see below.
Parameter tokens are also explained below.

This is the list of the categories, together with a brief
description. More elaborate explanations follow in this and
later chapters.
\begin{enumerate} \message{set counter}%\SetCounter:item=-1
\setcounter{enumi}{-1}
\item\label{ini:esc}\index{category!0} Escape character; this signals
  the start of a control sequence. \IniTeX\ makes the backslash
  \verb-\- (code~92) an escape character.
\item\index{category!1} Beginning of group; such a character causes
  \TeX\ to enter a new level of grouping. The plain format makes the
  open brace \verb-{- \mdqon a beginning"-of-group character.  \mdqoff
\item\index{category!2} End of group; \TeX\ closes the current level
  of grouping.  Plain \TeX\ has the closing brace \verb-}- as
  end-of-group character.
\item\index{category!3} Math shift; this is the opening and closing
  delimiter for math formulas. Plain \TeX\ uses the dollar
  sign~\verb-$- for this.
\item\index{category!4} Alignment tab; the column (row) separator in
  tables made with \cs{halign} (\cs{valign}). In plain \TeX\ this is
  the ampersand~\verb-&-.
\item\index{category!5}\label{ini:eol} End of line; a character that
  \TeX\ considers to signal the end of an input line.
  \IniTeX\ assigns this code to character code~13,
  which is the \ascii{} value of the \gram{return} character.
  Not coincidentally, 13~is also the value that \IniTeX\ assigns to
  the \cs{endlinechar} parameter; see above.
\item\index{category!6} Parameter character; this indicates parameters
  for macros.  In plain \TeX\ this is the hash sign~\verb-#-.
\item\index{category!7} Superscript; this precedes superscript
  expressions in math mode. It is also used to denote character codes
  that cannot be entered in an input file; see below.  In plain
  \TeX\ this is the circumflex~\verb-^-.
\item\index{category!8} Subscript; this precedes subscript expressions
  in math mode.  In plain \TeX\ the underscore~\verb-_- is used for
  this.
\item\index{category!9} Ignored; characters of this category are
  removed from the input, and have therefore no influence on further
  \TeX\ processing. In plain \TeX\ this is the \gr{null} character,
  that is, code~0.
\item\index{category!10}\label{ini:sp} Space; space characters receive
  special treatment.  \IniTeX\ assigns this category to the \ascii{}
  \gr{space} character, code~32.
\item\index{category!11}\label{ini:let} Letter; in \IniTeX\ only the
  characters \n{a..z}, \n{A..Z} are in this category. Often, macro
  packages make some `secret' character (for instance~\n@) into a
  letter.
\item\index{category!12}\label{ini:other} Other; \IniTeX\ puts
  everything that is not in the other categories into this
  category. Thus it includes, for instance, digits and punctuation.
\item\index{category!13} Active; active characters function as a
  \TeX\ command, without being preceded by an escape character.  In
  plain \TeX\ this is only the tie character~\verb-~-, which is
  defined to produce an unbreakable space; see page~\pageref{tie}.
\item\index{category!14}\label{ini:comm} Comment character; from a
  comment character onwards, \TeX\ considers the rest of an input line
  to be comment and ignores it. In \IniTeX\ the per cent sign \verb-%-
  is made a comment character.
\item\index{category!15}\label{ini:invalid} Invalid character; this
  category is for characters that should not appear in the
  input. \IniTeX\ assigns the \ascii\ \gr{delete} character, code~127,
  to this category.
\end{enumerate}

The user can change the mapping 
of character codes to category codes
with the \csterm{catcode}\ command (see Chapter~\ref{gramm}
for the explanation of concepts such as~\gr{equals}):
\begin{disp}\cs{catcode}\gram{number}\gr{equals}\gram{number}.\end{disp}
In such a statement, the first number is often given in the form
\begin{disp}\verb>`>\gr{character}\quad or\quad \verb>`\>\gr{character}\end{disp}
both of which denote the character code of the character
(see pages \pageref{char:code} and~\pageref{int:denotation}).

The plain format defines
\csterm{active}
\begin{verbatim}
\chardef\active=13
\end{verbatim} 
so that one can write statements such as
\begin{verbatim}
\catcode`\{=\active
\end{verbatim}
The \cs{chardef} command is  treated
on pages \pageref{chardef} and~\pageref{num:chardef}.

The \LaTeX\ format has the control sequences
\begin{verbatim}
\def\makeatletter{\catcode`@=11 }
\def\makeatother{\catcode`@=12 }
\end{verbatim}
in order to switch on and off the `secret' character~\n@
(see below).

The \cs{catcode} command can also be used to query category
codes: in 
\begin{verbatim}
\count255=\catcode`\{
\end{verbatim}
it yields a number, which can be assigned.

Category codes can be tested by
\begin{disp}\cs{ifcat}\gr{token$_1$}\gr{token$_2$}\end{disp}
\TeX\ expands whatever is after \cs{ifcat} until two 
unexpandable tokens are found; these are then compared
with respect to their category codes. Control sequence
tokens are considered to have category code~16\index{category!16},
which makes them all equal to each other, and unequal to
all character tokens.
Conditionals are treated further in Chapter~\ref{if}.

\section{From characters to tokens}

The input processor
of \TeX\ scans input lines from a file or from the
user terminal, and converts the characters in the input
to tokens. There are three types of tokens.
\begin{itemize}\item Character tokens: any character that is
	passed on its own to \TeX's
further levels of processing with an appropriate
category code attached.
\item Control sequence tokens, of which there are two kinds:
	an escape character 
\ldash that is,\message{ldash nobreak?}
a character of category~0\index{category!0} \rdash  followed
by a string of `letters' is
lumped together into a {\em control word}, which is a single token.
An escape character followed by a single character that is not of
category~11\index{category!11}, letter, is made into a 
\indextermsub{control}{symbol}.
If the distinction between control word and control symbol is
irrelevant, both are called 
\indextermsub{control}{sequence}.

The control symbol that results from an escape character followed
\csidx{\char32}%
by a space character is called 
\indextermbus{control}{space}.

\item Parameter tokens: a parameter character \ldash that is, a
  character of category~6\index{category!6}, by default~\verb=#=
  \rdash followed by a digit \n{1..9} is replaced by a parameter
  token.  Parameter tokens are allowed only in the context of macros
  (see Chapter~\ref{macro}).

A macro parameter character followed by another macro parameter
character (not necessarily with the same character code)
is replaced by a single character token.
This token has category~6 (macro parameter), and the character
code of the second parameter character.
The most common instance is of this is
replacing \n{\#\#} by~\n{\#$_6$}, where the subscript
denotes the category code.

\end{itemize}

\section{The input processor as a finite state automaton}
\label{input:states}

\TeX's input processor can be considered to be a finite state 
automaton with three \indextermbus{internal}{states},
that is, at any moment in time it is in one of three states,
and after transition to another state there is no memory of the
previous states. 

\subsection{State {\italic N}: new line}

State {\italic N} is entered at the beginning of each new input line,
and that is the only time \TeX\ is in this state.
In state~{\italic N} all space tokens
(that is, characters of category~10\index{category!10})
are ignored; an end-of-line character
is converted into a \cs{par} token.  All other tokens bring \TeX\ into
state~{\italic M}.

\subsection{State {\italic S}: skipping spaces}

State {\italic S} is entered in any mode after a control word or
control space (but after no other control symbol),
or, when in state~{\italic M}, after a space.
In this state all subsequent spaces or end-of-line characters
in this input line are discarded.

%\spoint State {\italic M}: middle of line
\subsection{State {\italic M}: middle of line}

By far the most common state is~{\italic M}, `middle of line'.
It is entered after characters of categories
1--4, 6--8, and 11--13, and after control symbols
other than control space.
An end-of-line character encountered in this state
results in a space token.

%% \input figflow \message{left align flow diagram}
%% \vskip12pt plus 1pt minus 4pt\relax %before spoint skip
%% \begin{tdisp}%\PopIndentLevel
%% \leavevmode\relax
%% %\figmouth
%% \message{fig mouth missing}
%% \end{tdisp}

\input figs1
\begin{quotation}
  \figmouth
\end{quotation}

%\point[hathat] Accessing the full character set
\section{Accessing the full character set}
\label{hathat}

Strictly speaking, \TeX's input processor
is not a finite state automaton.
This is because during the scanning of the input line
all trios consisting of two {\sl equal\/} superscript characters 
\index{\char94\char94\ replacement}
(category code~7\index{category!7}) and a subsequent character
(with character code~$<128$)
are replaced by a single character with a character
code in the range 0--127,
differing by 64 from that of the original character.

This mechanism can be used, for instance, to access positions in a font
corresponding to character codes that cannot
be input, for instance because they are \ascii{} control characters.
The most obvious examples are the \ascii{} \gr{return}
and \gr{delete} characters; the corresponding 
positions 13 and 127 in a font are
accessible as \verb>^^M> and~\verb>^^?>.
However, since the category of \verb>^^?> is 15\index{category!15}, invalid,
that has to be changed before character 127 can be accessed.

In \TeX3 this mechanism has been 
modified and extended to access 256 characters:
any quadruplet \verb-^^xy- where both \n x and \n y are lowercase
hexadecimal digits \n0--\n9, \n a--\n f, 
is replaced by a character in the
range 0--255, namely the character the number of which is
represented hexadecimally as~\n{xy}.
This imposes a slight restriction on the applicability
of the earlier mechanism: if, for instance, \verb>^^a>
is typed to produce character~33, then a following
\n0--\n9, \n{a}--\n{f} will be misunderstood.

While this process makes \TeX's input processor
somewhat more powerful
than a true finite state automaton,
it does not interfere with the rest of
the scanning. Therefore it is conceptually simpler to pretend that
such a replacement of triplets or quadruplets
of characters, starting with~\verb>^^>, is performed in advance. 
In actual practice this is not possible,
because an
input line may assign category code~7\index{category!7} to some 
character other than the circumflex, thereby 
influencing its further processing.


%\point Transitions between internal states
\section{Transitions between internal states}

Let us now discuss the effects on the internal state
of \TeX's input processor when
certain category codes are encountered in the input. 

%\spoint 0: escape character
\subsection{0: escape character}
\index{escape!character|see{character, escape}}

When an \indextermbus{escape}{character} is encountered,
\TeX\ starts forming a control sequence token.
Three different types of control sequence can result,
depending on the category code of the character that
follows the escape character.

\begin{itemize}\item
If the character following the escape is of category~11\index{category!11},
letter, then \TeX\ combines the escape,
that character and all following
characters of category~11, into a control word.
After that \TeX\
goes into state~{\italic S}, skipping spaces.
\item
With a character of category~10\index{category!10}, space, a control
symbol called control space results, and \TeX\ goes into
state~{\italic S}.
\item
With a character of any other category code 
a control symbol results, and \TeX\ goes into state~{\italic M},
middle of line.
\end{itemize}

The letters of a control sequence name have to be all on one line;
a control sequence name is not continued on the next line
if the current line ends with a comment sign, or if (by letting
\cs{endlinechar} be outside the range~0--255) 
there is no terminating character.

%\spoint 1--4, 7--8, 11--13: non-blank characters
\subsection{1--4, 7--8, 11--13: non-blank characters}

Characters of category codes 1--4, 7--8, and 11--13 are made
into tokens, and \TeX\ goes into state~{\italic M}.

%\spoint 5: end of line
\subsection{5: end of line}

Upon encountering an end-of-line character, 
\TeX\ discards the rest of the
line, and starts processing the next line,
in state~{\italic N}. If the current state was~{\italic N},
that is, if the
line so far contained at most spaces, a~\cs{par} token
is inserted; if the state was~{\italic M}, a~space token is inserted,
and in state~{\italic S} nothing is inserted.

Note that by `end-of-line character' a character with category
code~5 is meant. This is not necessarily the \cs{endlinechar},
nor need it appear at the end of the line.
See below for further remarks on line ends.

%\spoint 6: parameter
\subsection{6: parameter}

A \indextermbus{parameter}{character} \ldash usually~\verb=#= \rdash  can be
followed by either a digit \n{1..9} 
in the context of macro definitions
\altt
or by another parameter character. 
In the first case a `parameter token' results,
in the second case only a single parameter character
is passed on as a character token for further processing.
In either case \TeX\ goes into state~{\italic M}.

A parameter character can also appear on its own in an
alignment preamble (see Chapter~\ref{align}).

%\spoint 7: superscript
\subsection{7: superscript}

A superscript character is handled like most non-blank
characters, except in the case where it is followed
by a  superscript character of the same character code.
The process
that replaces these two characters plus the following character
(possibly two characters in \TeX3) by another character
was described above.

%\spoint 9: ignored character
\subsection{9: ignored character}

Characters of category 9 are ignored; \TeX\ remains in the same state.

%\spoint 10: space
\subsection{10: space}

A token with category code 10 \ldash this is called a \gr{space token},
irrespective of the character code \rdash 
is ignored in states {\italic N} and~{\italic S} 
(and the state does not change); 
in state~{\italic M} \TeX\ goes into state~{\italic S}, inserting
a token that has category~10 and character code~32 
(\ascii{} space).
This implies that the character code of the space token may change
from the character that was actually input.

%\spoint 14: comment
\subsection{14: comment}

A comment character causes \TeX\ to discard 
the rest of the line, including the comment character.
In particular, the end-of-line character is not seen,
so even if the comment was encountered in state~{\italic M}, no space
token is inserted.

%\spoint 15: invalid
\subsection{15: invalid}

Invalid characters cause an error message. \TeX\ remains in
the state it was in.
However, in the context of a control symbol an invalid character
is acceptable. Thus \verb>\^^?> does not cause any error messages.

%\point[cat12] Letters and other characters
\section{Letters and other characters}
\label{cat12}

In most programming languages identifiers can consist
of both letters and digits (and possibly some other
character such as the underscore), but control sequences in \TeX\
are only allowed to be formed out of characters of category~11,
letter. Ordinarily, the digits and punctuation symbols have
category~12, other character.
However, there are contexts where \TeX\ itself
generates a string of characters, all of which have
category code~12, even if that is not their usual
category code.

This happens when the operations 
\cs{string},
\cs{number},
\cs{romannumeral},
\cs{jobname},
\cs{fontname},
\cs{meaning},
and \cs{the}
are used to generate a stream of character tokens.
If any of the characters delivered by such a command
is a space character (that is, character code~32), 
it receives category code~10, space.

For the extremely rare case where a hexadecimal digit has been
hidden in a control sequence, \TeX\ allows \n A$_{12}$--\n F$_{12}$
to be hexadecimal digits, in addition to the ordinary
\n A$_{11}$--\n F$_{11}$ (here
the subscripts denote the category codes).

For example,
\begin{disp}\verb>\string\end>\quad gives four character tokens\quad
\n{\char92$_{12}$e$_{12}$n$_{12}$d$_{12}$} \end{disp}
Note that the \indextermbus{escape}{character}~\texttt{\char`\\}$_{12}$\label{use:escape}
is used in the output only because the
value of \cs{escapechar} is the character code for the
backslash. Another value of \cs{escapechar} leads to another
character in the output of \cs{string}. 
The \cs{string} command is treated further in Chapter~\ref{char}.

Spaces can wind up in control sequences:
\begin{disp}\verb>\csname a b\endcsname>\end{disp} gives a control sequence
token in which one of the three characters is a space.
Turning this control sequence token into a string of characters
\begin{disp}\verb>\expandafter\string\csname a b\endcsname>\end{disp}
gives \n{\char92$_{12}$a$_{12}$\char32$_{10}$b$_{12}$}.


As a more practical example, suppose there exists a sequence
of input files \n{file1.tex}, \n{file2.tex}\label{ex:jobnumber},
and we want to
write a macro that finds the number of the input file
that is being processed. One approach would be to write
\begin{verbatim}
\newcount\filenumber  \def\getfilenumber file#1.{\filenumber=#1 }
\expandafter\getfilenumber\jobname.
\end{verbatim}
where the letters \n{file} in the parameter text of the
macro (see Section~\ref{param:text}) absorb that part of the
jobname, leaving the number as the sole parameter.

However, this is slightly incorrect: the letters \n{file} resulting
from the \cs{jobname} command have category code~12, instead of
11 for the ones in the definition of \cs{getfilenumber}.
This can be repaired as follows:
\begin{verbatim}
{\escapechar=-1
 \expandafter\gdef\expandafter\getfilenumber
       \string\file#1.{\filenumber=#1 }
}
\end{verbatim}
Now the sequence \verb>\string\file> gives the four
letters \n{f$_{12}$i$_{12}$l$_{12}$e$_{12}$}; 
the \cs{expandafter} commands let this be executed prior to
the macro definition;
the backslash is omitted because we put\handbreak \verb>\escapechar=-1>.
Confining this value to a group makes it necessary to use~\cs{gdef}.


\section{The \lowercase{\n{\char92par}} token}

\TeX\ inserts a \csterm{par}\ token into the input after
an \indextermbus{empty}{line}, that is, when 
encountering a character with category code~5,
end of line, in state~{\italic N}.
It is good to realize when exactly this happens:
since \TeX\ leaves state~{\italic N}
when it encounters any token but a space,
a~line giving a \cs{par} can only contain characters
of category~10. In particular, it cannot end with a comment
character. Quite often this fact is used the other way around:
if an empty line is wanted for the layout of the input
one can put a comment sign on that line.


Two consecutive empty lines generate two \cs{par} tokens.
For all practical purposes this is equivalent to one \cs{par},
because after the first one \TeX\ enters vertical mode, and
in vertical mode a \cs{par} only
exercises the page builder,
and clears the paragraph shape parameters.

A \cs{par} is also inserted into the input when \TeX\ sees a
\gram{vertical command} in unrestricted horizontal mode.
After the \cs{par} has been read and expanded, the
vertical command is examined anew (see Chapters~\ref{hvmode}
and~\ref{par:end}).

The \cs{par} token may also be inserted by the \cs{end}
command that finishes off the run of \TeX; see Chapter~\ref{output}.

It is important to realize that \TeX\ does what it normally does
when encountering an empty line
(which is ending a paragraph)
only because of the default definition of the \cs{par} token.
By redefining \cs{par} the behaviour
caused by empty lines and vertical commands can be changed completely,
and  interesting special effects can be achieved.
In order to continue to be able  to cause the actions normally
associated with \cs{par}, the synonym \cs{endgraf} is
available in the plain format. See further Chapter~\ref{par:end}.

The \cs{par} token is not allowed to be part of a macro
argument, unless the macro has been declared to be \cs{long}.
A \cs{par} in the argument of a non-\cs{long} macro
prompts \TeX\ to give a `runaway argument' message.
Control sequences that have been \cs{let} to \cs{par}
(such as \cs{endgraf}) are allowed, however.

\section{Spaces}

This section treats some of the aspects of the
\indextermbus{space}{character} and \indextermbus{space}{token} in the
initial processing stages of \TeX. The topic of spacing in text
typesetting is treated in Chapter~\ref{space}.


\subsection{Skipped spaces}

From the discussion of the internal states of \TeX's 
input processor
it is clear that some spaces in the input never reach the
output; in fact they never get past the input processor.
These are for instance the spaces at the beginning
of an input line, and the spaces following the one
that lets \TeX\ switch to state~{\italic S}.


On the other hand, line ends can generate spaces (which are not
in the input) that may wind up in the output.
There is a third kind of space: the spaces that get past the
input processor,
or are even generated there, but still do not wind up in the
output. These are the \gram{optional spaces} that the 
syntax of \TeX\ allows in various places.

%\spoint Optional spaces
\subsection{Optional spaces}

The syntax of \TeX\ has the concepts of \indextermbus{optional}{spaces}
and `one optional space':
\begin{disp}\gr{one optional space} $\longrightarrow$
\gr{space token} $|$ \gr{empty}\nl
\gr{optional spaces} $\longrightarrow$
\gr{empty} $|$ \gr{space token}\gr{optional spaces}\end{disp}
In general, \gr{one optional space} is allowed after
numbers and glue specifications, while \gr{optional spaces} are
allowed whenever a space can occur inside a number
(for example, between a minus sign and the digits of the number)
or glue specification (for example, between \n{plus} and \n{1fil}).
Also, the definition of \gr{equals} allows \gr{optional spaces}
before the \n= sign.

Here are some examples of optional spaces.

\begin{itemize} 
\item A number can be delimited by \gr{one optional space}. 
This prevents accidents (see Chapter~\ref{number}), 
and it speeds up processing, as \TeX\ can 
detect more easily where the \gram{number} being read ends.
Note, however, that not every `number' is a \gram{number}:
for instance the {\tt 2} in \cs{magstep2} is not a number,
but the  single token that is the parameter of the
\cs{magstep} macro. Thus a space or line end after this
is significant. Another example is a parameter number,
for example~\n{\#1}: since at most nine parameters are allowed, scanning
one digit after the parameter character suffices.

\item From the grammar of \TeX\ 
it follows that the
keywords \n{fill} and \n{filll}
consist of \n{fil} and
separate {\tt l}$\,$s, each of which is a keyword
(see page~\pageref{keywords} for a more elaborate discussion),
and hence can be followed by optional spaces. 
Therefore forms such as \hbox{\n{fil L l}} are also valid.
This is a potential source of strange accidents.
In most cases, appending a \cs{relax} token prevents
such mishaps.

\item The primitive command \csterm{ignorespaces}\ 
may come in handy as the final command in a macro definition.
As it gobbles up
optional spaces, it can be used to prevent spaces following the
closing brace of an argument from winding up in the output
inadvertently. For example, in
\begin{verbatim}
\def\item#1{\par\leavevmode
    \llap{#1\enspace}\ignorespaces}
\item{a/}one line \item{b/} another line \item{c/}
yet another
\end{verbatim} 
the \cs{ignorespaces} prevents spurious
spaces in the second and third item.
An empty line
after \cs{ignorespaces} will still insert a \cs{par}, however.
\end{itemize}

%\spoint Ignored and obeyed spaces
\subsection{Ignored and obeyed spaces}

After control words spaces are ignored. This is not an
instance of optional spaces, but it is due to the fact that
\TeX\ goes into state~{\italic S}, skipping spaces, after control
words. Similarly an end-of-line character is skipped
after a control word.

Numbers are delimited by only \gr{one optional space},
but still
\begin{disp}\n{a\char92 count0=3\char32\char32b}\quad gives\quad `ab',\end{disp}
because \TeX\ goes into state~{\italic S} after the first
space token. The second space is therefore skipped 
in the input processor of \TeX; it never becomes a space token.

Spaces are skipped furthermore when \TeX\ is in state~{\italic N},
newline. When \TeX\ is processing in vertical mode
space tokens (that is, spaces that were not skipped)
are ignored. For example, the space inserted (because of the line end)
after the first box in
\begin{verbatim}
\par
\hbox{a}
\hbox{b}
\end{verbatim}
has no effect.

Both plain \TeX\ and \LaTeX\ define a command \cs{obeyspaces}
\altt
that makes spaces significant: after one space other spaces are no
longer ignored. In both cases the basis is
\altt
\begin{verbatim}
\catcode`\ =13 \def {\space}
\end{verbatim}
However, there is a difference between the two cases:
in plain \TeX\ \begin{verbatim}
\def\space{ }
\end{verbatim}
while in \LaTeX\ \begin{verbatim}
\def\space{\leavevmode{} }
\end{verbatim}
although the macros bear other names there.

The difference between the two macros becomes
apparent in the context of \cs{obeylines}:
each line end is then a \cs{par} command, implying that
each next line is started in vertical mode.
An active space is expanded by the plain macro to a space token, 
which is ignored in vertical mode.
The active spaces in \LaTeX\ will immediately switch to horizontal
mode, so that each space is significant.

\subsection{More ignored spaces}

There are three further places where \TeX\ will ignore space tokens.
\alt
\begin{enumerate}
\item When \TeX\ is looking for
an undelimited macro argument it will accept the
first token (or group) that is not a space. This is treated
in Chapter~\ref{macro}.

\item In math mode space tokens are ignored (see Chapter~\ref{math}).

\item After an alignment tab character spaces are ignored
(see Chapter~\ref{align}).
\end{enumerate}

\subsection{\gr{space token}}

Spaces are anomalous in \TeX.
For instance, the \cs{string} operation 
assigns category code~12\index{category!12} to all
characters except spaces; they receive category~10\index{category!10}.
Also, as was said above, \TeX's input processor converts (when in
state~{\italic M}) all tokens with category code~10 into real spaces:
they get character code~32.
Any character token with category~10 is called
\gram{space token}\indexterm{space! token}.
Space tokens with character
code not equal to 32 are called \indextermbus{funny}{spaces}.

\begin{example} After giving the character \n Q 
the category code of a space character, 
and using it in a definition
\begin{verbatim}
\catcode`Q=10 \def\q{aQb}
\end{verbatim}
we get
\begin{verbatim}
\show\q
macro:-> a b
\end{verbatim}
because the input processor
changes the character code of the funny space
in the definition.
\end{example}

Space tokens with character codes other than 32 can be
created using, for instance, \cs{uppercase}.
However, `since the various forms of
space tokens are almost identical in behaviour, there's no
point dwelling on the details'; see~\cite{Knuth:TeXbook}~p.~377.


%\spoint Control space
\subsection{Control space}

The `control space' command \verb-\-\n{\char32}
\csidx{\char32}
contributes the amount of space that a \gr{space token} would
when the \verb=\spacefactor= is~1000.
A~control space
is not treated like a space token, or like a macro
expanding to one (which is how \cs{space} is defined in plain \TeX).
For instance, \TeX\ ignores spaces
at the beginning of an input line, but
control space is a \gr{horizontal command}, so it 
makes \TeX\ switch from vertical to horizontal mode
(and insert an indentation box).
See  Chapter~\ref{space} for the space factor, and
chapter~\ref{hvmode} for horizontal and vertical modes.

%\spoint `\n{\char32}'
\subsection{`\n{\char32}'}

The explicit symbol `\n{\char32}' for a space
is character~32 in the Computer Modern typewriter typeface.
However, switching to \cs{tt} is not sufficient to get
spaces denoted this way, because spaces will still
receive special treatment in the input processor.

One way to
let spaces be typeset by \n{\char32}
is to set
\begin{verbatim}
\catcode`\ =12
\end{verbatim}
\TeX\ will then take a space as the instruction to
typeset character number~32. Moreover, subsequent spaces
are not skipped, but also typeset this way: state~{\italic S}
is only entered after a character with category code~10.
Similarly, spaces after a control sequence are made
visible by changing the category code of the space character.

\section{More about line ends}

\TeX\ accepts lines from an input file, excluding any line
terminator that may be used.
Because of this, \TeX's behaviour here is not dependent
on the operating system and the \indextermsub{line}{terminator}
it uses (\key{CR}-\key{LF},
\key{LF}, or none at all for block storage).
From the input line any trailing spaces are removed.
The reason for this is historic; it has to do with 
the block storage mode on \key{IBM} mainframe computers.
For some computer-specific problems with end-of-line
characters, see~\cite{B:ctrl-M}.

A~terminator character is then appended
with a character code of \cs{endlinechar}, 
unless this parameter has a value that
is negative or more than~255. 
Note that this terminator character
need not have category code~5\index{category!5}, end of line.

\subsection{Obeylines}

Every once in a while it is desirable that the output uses the same
line breaking as the input.
\message{Check spurious space obeylines+1}%
\csidx{obeylines}\howto Change the meaning of the line end\par
The following piece of code does the trick:
\begin{verbatim}
\catcode`\^^M=13 %
\def^^M{\par}% 
\end{verbatim}
The \cs{endlinechar} character is here made active,
and its meaning becomes \cs{par}.
The comment signs prevent \TeX\ from seeing the terminator of the
\alt
lines of this definition, and expanding it since it is active.

However, it takes some care to embed this code in a macro.
The definition
\begin{verbatim}
\def\obeylines{\catcode`\^^M=13 \def^^M{\par}}
\end{verbatim}
will be misunderstood:
\TeX\ will discard everything
after the second \verb>^^M>, because this has category code~5.
Effectively, this line is then
\begin{verbatim}
\def\obeylines{\catcode`\^^M=13 \def
\end{verbatim}
To remedy this,
the definition itself has to be
performed in a context where \verb>^^M> is an active
character:
\begin{verbatim}
{\catcode`\^^M=13 %
 \gdef\obeylines{\catcode`\^^M=13 \def^^M{\par}}%
}
\end{verbatim}
Empty lines in the  input are not taken into account
in this definition: these disappear, because two consecutive \cs{par}
tokens are (in this case) equivalent to one. 
A slightly modified definition for the line end as
\begin{verbatim}
\def^^M{\par\leavevmode}
\end{verbatim}
remedies this:
now every line end forces \TeX\ to start a paragraph. For empty
lines this will then be an empty paragraph.

%\spoint Changing the \cs{\endlinechar}
\subsection{Changing the \cs{endlinechar}}

Occasionally you may want to change the \cs{endlinechar}, or
the \cs{catcode} of the ordinary line terminator \verb.^^M.,
for instance to obtain special effects such as macros where 
the argument is terminated by the line end.
See page~\pageref{pick:eol} for a worked-out example.

There are  a couple of traps. Consider the following:
\begin{verbatim}
{\catcode`\^^M=12 \endlinechar=`\^^J \catcode`\^^J=5
...
... }
\end{verbatim}
This causes unintended output of both character~13 (\verb-^^M-)
and~10 (\verb-^^J-), caused by the line terminators of the
first and last line.

Terminating the first and  last line with a comment works,
but replacing the first line by the two lines
\begin{verbatim}
{\endlinechar=`\^^J \catcode`\^^J=5
\catcode`\^^M=12
\end{verbatim}
is also a solution.

Of course, in many cases it is not necessary to substitute
another end-of-line character; a~much simpler solution 
is then to put
\begin{verbatim}
\endlinechar=-1 
\end{verbatim}
which treats all lines as if they end with a comment.

%\spoint More remarks about the end-of-line character
\subsection{More remarks about the end-of-line character}

The character that \TeX\ appends at the end of an input line
is treated like any other character. Usually one is not aware
of this, as its category code is special, but there are a few
ways to let it be processed in an unusual way.

\begin{example} Terminating an input line with \verb>^^> will
(ordinarily, when \cs{endlinechar} is~13) give `M' in the output, 
which is the 
\ascii{} character with code~13+64.
\end{example}

\begin{example} If \verb>\^^M> has been defined,
terminating an input line with a backslash will execute this command.
The plain format defines
\begin{verbatim}
\def\^^M{\ }
\end{verbatim}
which makes a `control return' equivalent to a control space.
\end{example}

%\point More about the input processor
\section{More about the input processor}

%\spoint The input processor as a separate process
\subsection{The input processor as a separate process}

\TeX's levels of processing are all working at the
same time and incrementally, but conceptually they can often be
considered to be separate processes that each accept the
completed output of the previous stage. The juggling with
spaces provides a nice illustration for this.

Consider the definition
\begin{verbatim}
\def\DoAssign{\count42=800}
\end{verbatim}
and the call
\begin{verbatim}
\DoAssign 0
\end{verbatim}
The input processor, the part
of \TeX\ that builds tokens, in scanning this call
skips the space before the zero, so the expansion of this
call is
\begin{verbatim}
\count42=8000
\end{verbatim}
It would be incorrect to reason
`\cs{DoAssign} is read, then expanded, the space delimits the
number 800, so 800 is assigned and the zero is printed'.
Note that the same would happen if the zero appeared on the next line.

Another illustration shows that optional spaces appear in a different
stage of processing from that for skipped spaces:
\begin{disp}\verb>\def\c.{\relax}>\nl
     \verb>a\c.>{\tt\char32 b}\end{disp}
expands to
\begin{disp}\n{a\cs{relax}\char32 b}\end{disp}
which gives as output\begin{disp} `a b'\end{disp}
because spaces after the \cs{relax} control sequence are only
skipped when the line is first read, not when it is expanded.
The fragment
\begin{disp} \verb-\def\c.{\ignorespaces}-\nl \verb-a\c. b-\end{disp}
on the other hand, expands to
\begin{disp}\n{a\cs{ignorespaces}\char32 b}\end{disp}
Executing the \cs{ignorespaces} command removes the subsequent
space token, so the output is \begin{disp} `ab'.\end{disp}
In both definitions
the period after \cs{c} is a delimiting token; it is used here
to prevent spaces from being skipped.

%\spoint The input processor not as a separate process
\subsection{The input processor not as a separate process}

Considering the tokenizing of \TeX\ to be a separate process
is a convenient view, but sometimes it leads to confusion.
The line
\begin{verbatim}
\catcode`\^^M=13{}
\end{verbatim}
makes the line end active,
and subsequently gives an `undefined control sequence' error
for the line end of this line itself. Execution of the commands
on the line thus influences the scanning process of that
same line.

By contrast,
\begin{verbatim}
\catcode`\^^M=13
\end{verbatim}
does not give an error.
The reason for this is that \TeX\ reads the line end while it is still
scanning the number~13; that is, at a time when the assignment
has not been performed yet.
The line end is then converted to the optional space character
delimiting the number to be assigned.

%\spoint Recursive invocation of the input processor
\subsection{Recursive invocation of the input processor}

Above, the activity of replacing a parameter
character plus a digit by a parameter token was described
as something similar to the lumping together of letters
into  a control sequence token. Reality is somewhat more
complicated than this. \TeX's token scanning mechanism
is invoked both for input from file and for input from
lists of tokens such as the macro definition. Only in the
first case is the terminology of internal states applicable.

Macro parameter characters are treated the same in both
cases, however. If this were not the case it would
not be possible to write things such as
\begin{verbatim}
\def\a{\def\b{\def\c####1{####1}}}
\end{verbatim}
See page \pageref{nest:def} for an explanation of such
nested definitions.

%\point The \verb@- convention
\section{The \n{@} convention}

Anyone who has ever browsed through either the plain format or
the \LaTeX\ format will have noticed that a lot of control sequences
contain an `at' sign:~\verb-@-. These are control sequences that
are meant to be inaccessible to the ordinary user.

Near the beginning of the format files the instruction
\begin{verbatim}
\catcode`@=11
\end{verbatim}
occurs, making the at sign into a letter,
meaning that it can be used in control sequences. Somewhere near the
end of the format definition the at sign is made `other' again:
\begin{verbatim}
\catcode`@=12
\end{verbatim}

Now why is it that users cannot
call a control sequence with an at sign
directly, although they can call macros that contain lots of those
`at-definitions'? The reason is that the control sequences
containing an \n@ are internalized by \TeX\ at definition time,
after which they are a token, not a string of characters. 
Macro expansion then
just inserts such tokens, and at that time the category codes
of the constituent characters do not matter any more.

%%%% end of input file [mouth]

%\InputFile:char
%%%% this is input file [char]
%\subject[char] Characters
\endofchapter
\chapter{Characters}\label{char}

Internally, \TeX\ represents characters by their (integer) 
character code. This chapter treats those codes, and the
commands that have access to them.

\label{cschap:char}\label{cschap:chardef}\label{cschap:accent}\label{cschap:uccode}\label{cschap:lccode}\label{cschap:uppercase}\label{cschap:lowercase}\label{cschap:string}\label{cschap:escapechar}
\begin{inventory}
\item [\cs{char}]
      Explicit denotation of a character to be typeset. 

\item [\cs{chardef}] 
      Define a control sequence to be a synonym for
      a~character code.

\item [\cs{accent}] 
      Command to place accent characters.

\item [\cs{if}]
      Test equality of character codes. 

\item [\cs{ifx}]
      Test equality of both character and category codes.

\item [\cs{let}]
      Define a control sequence to be a synonym of a token.

\item [\cs{uccode}] 
      Query or set
      the character code that is the uppercase variant of a given code.

\item [\cs{lccode}]
      Query or set
      the character code that is the lowercase variant of a given code.

\item [\cs{uppercase}]
      Convert the \gr{general text} argument to its uppercase form.

\item [\cs{lowercase}] 
      Convert the \gr{general text} argument to its lowercase form.

\item [\cs{string}]
      Convert a token to a string of one or more characters.
\item [\cs{escapechar}]
      Number of the character that is to be used 
      for the escape character
      when control sequences are being converted
      into character tokens. \IniTeX\ default:~92~(\cs{}).
\end{inventory}

%\point[char:code] Character codes
\section{Character codes}
\label{char:code}

Conceptually it is easiest to think that \TeX\ works with
characters internally, but in fact
\TeX\ works with integers: the \indextermsub{character}{codes}. 

The way characters are encoded in a computer may differ
from system to system.
Therefore \TeX\ uses its own scheme of character codes.
Any character that is read from a file (or from the user terminal)
is converted to a character code according to the
character code table.
A~category code is then assigned based on this (see Chapter~\ref{mouth}).
The character code table is based on the 7-bit \ascii{} table
for numbers under~128 (see Section~\ref{sec:asciitable}).

There is an explicit conversion between characters
(better:  character tokens)
and  character codes  using the left quote (grave, back quote)
character~\n{`{}}:
at all places where \TeX\ expects a \gram{number} you
can use the left quote followed by a character
token or
a single-character control sequence.
Thus both \verb.\count`a. and \verb.\count`\a. are synonyms
for \verb.\count97.. See also Chapter~\ref{number}.

The possibility of a single-character control
sequence is necessary in certain cases such as
\begin{disp}\verb>\catcode`\%=11>\quad or\quad \verb>\def\CommentSign{\char`\%}>\end{disp}
which would be misunderstood if the backslash were left out.
For instance
\begin{verbatim}
\catcode`%=11
\end{verbatim}
would consider
the \n{=11} to be a comment.
Single-character
control sequences can be formed from characters with any
category code.

After the conversion to character codes any connection
with external representations has disappeared. Of course,
for most characters  the visible output will `equal' the input
(that is, an `\n{a}' causes an~`a').
There are exceptions, however, even among the common symbols.
In the Computer Modern
roman fonts there are no `less than' and `greater than'
\message{Check <>! Dammit!}%
signs, so the input `\verb.<>.' will give `<>' in the output.
%{\MathRMx<>}

In order to make \TeX\ machine independent at the output
side, the character codes are also used in the \n{dvi} file:
opcodes $n=0\ldots127$ denote simply the instruction `take
character $n$ from the current font'. The complete definition
of the opcodes in a \n{dvi} file can be found in~\cite{Knuth:TeXprogram}.


%\point Control sequences for characters
\section{Control sequences for characters}

There are a number of ways in which a control sequence can denote
a character. The \cs{char} command specifies a character to be
typeset; the \cs{let} command introduces
a synonym for a character token, that is,
the combination of character code and category code.

%\point Denoting characters to be typeset: \cs\char
\subsection{Denoting characters to be typeset: \protect\cs{char}}

The \csterm{char} command can be used to generate characters in the
output that are difficult to have in the input. This can be because
they are \TeX\ special characters, or because their font position
corresponds to an unprintable ascii character.

Characters can be denoted numerically by, for example,
\verb.\char98..
This command tells \TeX\ to add character number~98 of the
current font to the horizontal list currently under construction.

Instead of decimal notation, it is often more convenient to
use octal or hexadecimal notation. For octal the single quote is used:
\verb.\char'142.; hexadecimal uses the double quote: \verb.\char"62..
%
Note that using two single quotes, as \verb.\char''62., is incorrect;
the process that replaces two quotes by a double quote works at a
later stage of processing (the visual processor) than number scanning
(the execution processor).

Because of the explicit conversion to character codes by the
back quote character it is also possible to get a~`b' \ldash provided
that you are using a font organized a bit like the \ascii{} table \rdash
with \verb.\char`b.  or \verb.\char`\b..

The \cs{char} command looks superficially a bit like
the \verb-^^- substitution mechanism (Chapter~\ref{mouth}).
Both mechanisms access characters without directly denoting them.
However, the \verb-^^- mechanism operates in a very early stage of
processing (in the input processor of \TeX,
but before category code
assignment); the \cs{char} command, on the other hand,
comes in the final stages of processing. 
In effect it says `typeset character number
so-and-so'.

There is a construction to let a control sequence stand
for some character code: the \csterm{chardef}\ command.
The syntax of this is \label{chardef}
\begin{disp}\cs{chardef}\gram{control sequence}\gr{equals}\gram{number}, 
\end{disp}
where the number can be an explicit
representation or a counter value, but it can also be
a character code
obtained using the left quote command (see above; 
the full definition of \gr{number} is given in Chapter~\ref{number}). 
In the plain format 
the latter possibility is used in
definitions such as
\begin{verbatim}
\chardef\%=`\%
\end{verbatim}
which could have been given equivalently as
\begin{verbatim}
\chardef\%=37
\end{verbatim}
After this command, the control symbol \verb>\%>
used on its own is a synonym for \verb>\char37>,
that is, the command to typeset character~37
(usually the per cent character).

A control sequence that has been defined with a \cs{chardef}
command can also be used as a \gr{number}.
This fact is used in  allocation commands such as 
\cs{newbox} (see Chapters~\ref{number} and~\ref{alloc}).
Tokens defined with \cs{mathchardef} can also be used this
way.

\subsection{Implicit character tokens: \protect\cs{let}}

Another construction defining a control sequence
to stand for (among other things)
a character is~\cs{let}\csidx{let}:
\begin{disp}\cs{let}\gr{control sequence}\gr{equals}\gr{token}\end{disp}
with a character token on the right hand side of the (optional)
equals sign. The result is called an \indextermbus{implicit}{character} token.
(See page~\pageref{let} for a further discussion of~\cs{let}.)

In the
plain format there are for instance synonyms for
the open and close brace:
\begin{verbatim}
\let\bgroup={ \let\egroup=}
\end{verbatim}
The resulting control sequences are called `implicit braces'
(see Chapter~\ref{group}).

Assigning characters by \cs{let}
is different from defining control sequences by \cs{chardef}, 
in the sense that \cs{let}
makes the control sequence stand for the combination
of a character code and category code. 

As an example
\begin{verbatim}
\catcode`|=2 % make the bar an end of group
\let\b=|  % make \b a bar character
{\def\m{...}\b \m
\end{verbatim}
gives an `undefined control sequence \cs{m}'
because the \cs{b} closed the group inside which \cs{m}
was defined. On the other hand,
\begin{verbatim}
\let\b=| % make \b a bar character
\catcode`|=2  % make the bar character end of group
{\def\m{...}\b \m
\end{verbatim}
leaves one group open, and it prints a vertical bar
(or whatever is in position 124 of the current font).
The first of these examples
implies that even when the braces have been redefined
(for instance into active characters for macros that
format C code) the beginning-of-group and end-of-group
functionality is available through the control sequences
\cs{bgroup} and~\cs{egroup}.

Here is
another example to show
that implicit character tokens are hard to distinguish
from real character tokens. After the above sequence
\begin{verbatim}
\catcode`|=2 \let\b=|
\end{verbatim}
the tests
\begin{verbatim}
\if\b|
\end{verbatim}
and
\begin{verbatim}
\ifcat\b}
\end{verbatim}
are both true.

Yet another example can be found in the plain format:
the commands
\begin{verbatim}
\let\sp=^ \let\sb=_ 
\end{verbatim}
allow people without an
underscore or circumflex on their keyboard to 
make sub- and superscripts in mathematics.
For instance:
\begin{disp}\verb>x\sp2\sb{ij}>\quad gives\quad $x\sp2\sb{ij}$\end{disp}
If a person typing in the format itself does not have
these keys, some further tricks are needed:\label{spsb:truc}
\begin{verbatim}
{\lccode`,=94 \lccode`.=95 \catcode`,=7 \catcode`.=8
\lowercase{\global\let\sp=, \global\let\sb=.}}
\end{verbatim}
will do the job; see below for an explanation of lowercase codes.
The \verb>^^> method as it was in \TeX\ version~2
(see page~\pageref{hathat}) cannot be used here,
as it would require typing two characters that can ordinarily
not be input.
With the extension in \TeX\ version~3 it would also be possible
to write
\begin{verbatim}
{\catcode`\,=7
\global\let\sp=,,5e \global\let\sb=,,5f}
\end{verbatim}
denoting the codes 94 and 95 hexadecimally.

Finding out just what a control sequence has been defined to be with
\cs{let} can be done using \cs{meaning}:
the sequence
\begin{verbatim}
\let\x=3 \meaning\x
\end{verbatim}
gives
`\n{the character 3}'.

%\point Accents
\section{Accents}

\emph{Accents}\index{accents} can be placed by the
\gr{horizontal command}~\csterm{accent}
\label{character}:
\begin{disp}\cs{accent}\gr{8-bit number}\gr{optional assignments}%
     \gr{character}\end{disp} 
where \gr{character} is a character of
category 11\index{category!11} or~12\index{category!12},
a~\cs{char}\gr{8-bit number} command, or a~\cs{chardef} token. If none
of these four types of \gr{character} follows, the accent is taken to
be a \cs{char} command itself; this gives an accent `suspended in
mid-air'. Otherwise the accent is placed on top of the following
character.  Font changes between the accent and the character can be
effected by the \gr{optional assignments}.

An unpleasant implication of the fact that an \cs{accent} command
has to be followed by a \gr{character} is that it is not
possible to place an accent on a ligature, or
two accents on top of each other.
In some languages, such as Hindi or Vietnamese,
such double accents do occur.
Positioning accents on top of each other is possible,
however, in math mode.

The width of a character with an accent is the same as that of
the unaccented character. \TeX\ assumes that the 
accent as it appears in the font file
is properly positioned for a character that is as high
as the x-height of the font; for characters with other heights
it correspondingly lowers or raises the accent.

No genuine under-accents exist in \TeX. They are
implemented as low placed over-accents. A~way of handling
them more correctly would be to write a macro that
measures the following character, and raises or drops
the accent accordingly.
The cedilla macro~\csterm{c}
in plain \TeX\ does something along these lines. However,
it does not drop the accent for characters with descenders.

The horizontal positioning of an accent is controlled by
\cs{fontdimen1}, \indextermsub{slant}{per point}. Kerns are used
for the horizontal movement. Note that, although they
are inserted automatically, these kerns are classified
as {\italic explicit\/} kerns. Therefore they inhibit hyphenation
in the parts of the word before and after the kern.

As an example of kerning for accents, 
here follows the dump of a horizontal list.
\message{maybe italic correction for extra line}
\begin{verbatim}
\setbox0=\hbox{\it \`l}
\showbox0
\end{verbatim}
gives
\begin{verbatim}
\hbox(9.58334+0.0)x2.55554
.\kern -0.61803 (for accent)
.\hbox(6.94444+0.0)x5.11108, shifted -2.6389
..\tenit ^^R
.\kern -4.49306 (for accent)
.\tenit l
\end{verbatim}
Note that the accent is placed first, so afterwards the italic
correction of the last character is still available.

\section{Testing characters}

Equality of character codes is tested by \cs{if}:
\begin{disp}\cs{if}\gr{token$_1$}\gr{token$_2$}\end{disp}
Tokens following this conditional are expanded until two
unexpandable tokens are left. The condition is then true
if those tokens are character tokens with the same character
code, regardless of category code. 

An unexpandable control
sequence is considered to have character code 256 and
category code~16\index{category!16}
(so that it is unequal to anything except
another control sequence), except in the case
where it had been \cs{let} to a non-active character token.
In that case it is considered to have the character code
and category code of that character. This was mentioned above.

The test \cs{ifcat} for category codes was mentioned
in Chapter~\ref{mouth}; the test
\begin{disp}\cs{ifx}\gr{token$_1$}\gr{token$_2$}\end{disp}
can be used to test for category code and character code
simultaneously.
The tokens following this test are not expanded.
However, if they are macros, \TeX\
tests their expansions for equality.

Quantities defined by \cs{chardef} can be tested with
\cs{ifnum}:
\begin{verbatim}
\chardef\a=`x \chardef\b=`y \ifnum\a=\b % is false 
\end{verbatim}
based on the fact (see Chapter~\ref{number}) that
\gr{chardef token}s can be used as numbers.

See also section~\ref{sec:charactertests}


\section{Uppercase and lowercase}

%\spoint[uc/lc] Uppercase and lowercase codes
\subsection{Uppercase and lowercase codes}
\label{uc/lc}

To each of the character codes correspond\csidx{lccode}\csidx{uccode}
an \indextermsub{uppercase}{code}\index{code!uppercase|see{uppercase, code}}
and a \indextermsub{lowercase}{code}\index{code!lowercase|see{lowercase, code}}
(for still more codes see below).
These can be assigned
by 
\begin{Disp}\cs{uccode}\gram{number}\gr{equals}\gram{number}\end{Disp}
and 
\begin{Disp}\cs{lccode}\gram{number}\gr{equals}\gram{number}.\end{Disp}
In \IniTeX\ codes \verb-`a..`z-, \verb-`A..`Z- have uppercase code
\label{ini:uclc}
\verb-`A..`Z- and lowercase code \verb-`a..`z-.
All other character codes have both uppercase and lowercase
code zero.

%\spoint[upcase] Uppercase and lowercase commands
\subsection{Uppercase and lowercase commands}
\label{upcase}

The commands \verb-\uppercase{...}- and \verb-\lowercase{...}-
\csidx{uppercase}\csidx{lowercase}%
go through their argument lists, replacing all character 
codes of explicit character tokens
by their uppercase and lowercase code respectively
if these are non-zero,
without changing the category codes. 

The argument of \cs{uppercase} and \cs{lowercase}
is a \gr{general text}, which is defined as
\begin{Disp} \gr{general text} $\longrightarrow$ \gr{filler}\lb
      \gr{balanced text}\gr{right brace}\end{Disp}
(for the definition of \gr{filler} see Chapter~\ref{gramm})
meaning that the left brace can be implicit, but the closing
right brace must be an explicit character token with category
code~2. \TeX\ performs expansion to find the opening
brace.

Uppercasing and lowercasing are executed in the execution processor;
they are not `macro expansion' activities
like \cs{number} or \cs{string}.
The sequence (attempting to produce~\cs{A})
\begin{verbatim}
\expandafter\csname\uppercase{a}\endcsname
\end{verbatim}
gives an error (\TeX\ inserts an \cs{endcsname} before   the
\cs{uppercase} because \cs{uppercase} is unexpandable), but
\begin{verbatim}
\uppercase{\csname a\endcsname}
\end{verbatim}
works.

As an example of the correct use of \cs{uppercase}, here
is a macro that tests if a character is uppercase:
\begin{verbatim}
\def\ifIsUppercase#1{\uppercase{\if#1}#1}
\end{verbatim}
The same test can be
performed by \verb>\ifnum`#1=\uccode`#1>.

Hyphenation of words starting with an uppercase character,
that is, a character not equal to its own \cs{lccode},
is subject to the \cs{uchyph} parameter: if this
is positive, hyphenation of capitalized words is allowed.
See also Chapter~\ref{line:break}.

%\spoint Uppercase and lowercase forms of keywords
\subsection{Uppercase and lowercase forms of keywords}

Each character in \TeX\ keywords, such as \n{pt}, can be
given in uppercase or lowercase form. 
For instance, \n{pT}, \n{Pt}, \n{pt}, and~\n{PT} all have
the same meaning. \TeX\ does not use
the \cs{uccode} and \cs{lccode} tables here to
determine the lowercase form. Instead it
converts uppercase characters to lowercase by adding~32
\ldash the \ascii{} difference between uppercase and lowercase
characters \rdash to their character code. This has some implications
for implementations of \TeX\ for non-roman alphabets;
see page 370 of \TeXbook, \cite{Knuth:TeXbook}.

\subsection{Creative use of \cs{uppercase} and \cs{lowercase}}

The fact that \cs{uppercase} and \cs{lowercase} do not change
category codes can sometimes be used to create certain
character-code--category-code combinations that would
otherwise be difficult to produce. See for instance the
explanation of the \cs{newif} macro in Chapter~\ref{if},
and another example on page~\pageref{spsb:truc}.

For a slightly different application, consider the
problem (solved by Rainer Sch\"opf) of,
given a counter \verb-\newcount\mycount-, writing character
number \verb-\mycount- to the terminal.
Here is a solution:
%\begin{verbatim}
%\lccode`a=\mycount \chardef\terminal=16
%\lowercase{\write\terminal{a}}
%\end{verbatim}
\begin{verbatim}
\lccode`a=\mycount \chardef\terminal=16
\end{verbatim}
\begin{verbatim}
\lowercase{\write\terminal{a}}
\end{verbatim}
The \cs{lowercase} command effectively changes the 
argument of the \cs{write} command from~`\n a'
into whatever it should be.

%\point[codename] Codes of a character
\section{Codes of a character}
\label{codename}

Each character code has a number of \gr{codename}s 
associated\indexterm{codenames}
with it. These are integers in various ranges that determine
how the character is treated in various contexts, or
how the occurrence of that character changes the workings
of \TeX\ in certain contexts.

The code names are as follows:
\begin{description}\item [\cs{catcode}]
\gr{4-bit number} (0--15); the category to which a character belongs.
This is treated in Chapter~\ref{mouth}.
\item [\cs{mathcode}]
\gr{15-bit number} (0--\verb-"7FFF-) or \verb-"8000-;
determines how a character is treated
in math mode. See Chapter~\ref{mathchar}.
\item [\cs{delcode}]
\gr{27-bit number} (0--\n{\hex7$\,$FFF$\,$FFF});
determines how a character is treated after
\cs{left} or \cs{right} in math mode.
See page~\pageref{delcodes}.
\item [\cs{sfcode}]
integer; determines how spacing is affected after this character.
See Chapter~\ref{space}.
\item [\cs{lccode}, \cs{uccode}]
\gr{8-bit number} (0-255); lowercase and
uppercase codes \rdash these were treated above.
\end{description}

%\point Converting tokens into character strings
\section{Converting tokens into character strings}

The command \csterm{string} takes the next token and expands it
into a string of separate characters. Thus
\begin{verbatim}
\tt\string\control
\end{verbatim}
will give \cs{control} in the
output, and
\begin{verbatim}
\tt\string$
\end{verbatim}
will give~\verb-$-, but, noting that the string 
operation comes after the tokenizing,
\begin{verbatim}
\tt\string%
\end{verbatim}
will {\em not\/} give~\verb$%$,
because the comment
sign is removed by \TeX's input processor.
Therefore, this command will `string' the first token on the next line.

The \cs{string} command is executed by the expansion processor, thus
it is expanded unless explicitly inhibited (see Chapter~\ref{expand}).

%\spoint Output of control sequences
\subsection{Output of control sequences}

In the above examples the typewriter font was selected, because
the Computer Modern roman font does not have a backslash character.
However,
\TeX\ need not have used the backslash character to display
a control sequence: it uses character number \csterm{escapechar}.
This same value is also used when a control sequence is
output with \cs{write}, \cs{message}, or \cs{errmessage},
and it is used in the output of \cs{show}, \cs{showthe} and \cs{meaning}.
If \cs{escapechar} is negative or more than~255,
the escape character is not
output; the default value (set in \IniTeX) is~92, the number
of the backslash character.

For use in a  \cs{write} statement the \cs{string} can 
in some circumstances be
replaced  by \cs{noexpand} (see page~\pageref{expand:write}).

%\spoint Category codes of a \cs{string}
\subsection{Category codes of a \cs{string}}

The characters that are the result of a \cs{string} command have 
category code~12\index{category!12}, except for any spaces in 
a stringed control sequence;
they have category code~10\index{category!10}. Since inside a control
sequence there are no category codes, 
any spaces resulting from \cs{string} are
of necessity only space {\em characters}, that is,
characters with code~32.
However, \TeX's input processor converts
all space tokens that have a character code other than~32
into character tokens with character code~32, 
so the chances are pretty slim that
`funny spaces' wind up in control sequences.

Other commands with the same behaviour with respect to 
category codes as \cs{string}, are
\cs{number},
\cs{romannumeral}, \cs{jobname}, \cs{fontname}, \cs{meaning},
and \cs{the}.




%%%% end of input file [char]

%\InputFile:fontfam
%%%% this is input file [fontfam]
%\subject[font] Fonts
\endofchapter
\chapter{Fonts}\label{font}

In text mode \TeX\ takes characters from a `current font'.
This chapter describes how \indexterm{fonts} are identified to \TeX,
and what attributes a font can have.

\label{cschap:font}\label{cschap:fontname}\label{cschap:nullfont}\label{cschap:hyphenchar}\label{cschap:defaulthyphenchar}\label{cschap:fontdimen}\label{cschap:char47}\label{cschap:noboundary}
\begin{inventory}
\item [\cs{font}] 
      Declare the identifying control sequence of a font.

\item [\cs{fontname}] 
      The external name of a font.

\item [\cs{nullfont}] 
      Name of an empty font that \TeX\ uses in emergencies.


\item [\cs{hyphenchar}] 
      Number of the hyphen character of a font.

\item [\cs{defaulthyphenchar}] 
      Value of \cs{hyphenchar} when a font is loaded.
      Plain \TeX\ default:~\verb>`\->.

\item [\cs{fontdimen}] 
      Access various parameters of fonts.

\item [\cs{char47}]
      Italic correction.

\item [\cs{noboundary}] 
      Omit implicit boundary character.
\end{inventory}

%\point Fonts
\section{Fonts}

In  \TeX\ terminology a font is the set of characters that
is contained in one external font file. 
During processing, \TeX\ decides from
what font a character should be taken. This decision is
taken separately for text mode and math mode.

When \TeX\ is processing ordinary text, characters are taken
from the `current font'. 
External font file names are coupled to  control sequences
by   statements such as
\begin{verbatim}
\font\MyFont=myfont10
\end{verbatim}
which makes \TeX\ load the file \n{myfont10.tfm}.
Switching the current font to the font described in that file
is then done by
\begin{verbatim}
\MyFont
\end{verbatim}
The status of the current font
can be queried: the sequence
\begin{verbatim}
\the\font
\end{verbatim}
produces the control sequence for the current font.

Math mode completely ignores the current font. Instead
it looks  at the `current family', which can contain
three fonts: one for text style, one for script style,
and one for scriptscript style. This is treated
in Chapter~\ref{mathchar}.

See \cite{S} for a consistent terminology of fonts and typefaces.

With `virtual fonts' (see~\cite{K:virt}) it is possible that
what looks like one font to \TeX\ resides in more than
one physical font file.
\alt
See further page~\pageref{virtual:fonts}.

%\point Font declaration
\section{Font declaration}

Somewhere during a run of \TeX\ or \IniTeX\ 
\csidx{font}
the coupling between an internal identifying control sequence
and the external file name of a font has to be made.
The syntax of the command for this is
\begin{disp}\cs{font}\gr{control sequence}\gr{equals}%
\gr{file name}\gr{at clause}\end{disp} 
where
\begin{disp}\gr{at clause} $\longrightarrow$ \n{at} \gr{dimen}
$|$ \n{scaled} \gr{number} $|$ \gr{optional spaces}\end{disp}
Font declarations are local to a group.

By the \gr{at clause} the user specifies that some
magnified version of the font is wanted. The \gr{at clause} comes
in two forms: if the font is given \n{scaled}~{\italic f\/}, \TeX\
multiplies all its font dimensions for that font by~$f/1000$; 
if the font
has a design size~{\italic d\/}\n{pt} and 
the \gr{at clause} is \n{at}~{\italic p\/}\n{pt},
\TeX\ multiplies all font data by~$p/d$.
The presence of an \gr{at clause} makes no difference for
the external font file (the \n{.tfm} file)
that \TeX\ reads for the font; it just multiplies
the font dimensions by a constant.


After such a font declaration, using the defined control sequence
will set the current font to the font of the 
control sequence.

%\spoint Fonts and \n{tfm} files
\subsection{Fonts and \n{tfm} files}

The external file needed for the font is a \n{tfm} 
(\TeX\ font metrics) file,
which is taken independent of any  \gr{at clause}
in the \cs{font} declaration. If the \n{tfm}
file has been loaded already (for instance by \IniTeX\
when it constructed the format),
an assignment of that font file can be reexecuted
without needing recourse to the \n{tfm} file.

Font design sizes are given in the font metrics files.
The \n{cmr10} font, for instance, has a design size
of 10~point. However, there is not much in the font
that actually has a size of 10~points: the opening and closing
parentheses are two examples, but capital
letters are considerably smaller.

%\spoint Querying the current font and font names
\subsection{Querying the current font and font names}

It was already mentioned above that the control sequence
which set the current font can be retrieved by the
command \verb>\the\font>. This is a special case of
\begin{Disp}\cs{the}\gr{font}\end{Disp} where 
\begin{disp}\gr{font} $\longrightarrow$
\cs{font} $|$ \gr{fontdef token} $|$ \gr{family member}\nl
\gr{family member} $\longrightarrow$ 
\gr{font range}\gr{4-bit number}\nl
\gr{font range} $\longrightarrow$ 
\cs{textfont} $|$ \cs{scriptfont} $|$ \cs{scriptscriptfont}\end{disp}
A \gr{fontdef token} is a control sequence defined by \cs{font},
or the predefined control sequence \cs{nullfont}.
The concept of \gr{family member} is only 
relevant in math mode.

Also, the
\csidx{fontname}%
external name of fonts can be retrieved:
\begin{Disp}\cs{fontname}\gr{font}\end{Disp}
gives a sequence of character tokens of category~12\index{category!12}
(but space characters get category~10\index{category!10}) that spells the font file
name, plus an \gr{at clause} if applicable.

\begin{example} After
\begin{verbatim}
\font\tenroman=cmr10 \tenroman
\end{verbatim}
the calls
\verb>\the\font> and \verb>\the\tenroman> both give \cs{tenroman}.
The call \verb>\fontname\tenroman> gives \n{cmr10}.
\end{example}

%\spoint \cs{nullfont}
\subsection{\cs{nullfont}}

\TeX\ always knows a font that has no characters: the \csterm{nullfont}.
If no font has been specified, or if in math mode a family member
is needed that has not been specified, 
\TeX\ will take its characters from the nullfont.
This control sequence qualifies as a \gr{fontdef token}:
it acts like any other control sequence that stands for a font;
it just does not have an associated \n{tfm} file.

%\point Font information
\section{Font information}

During a run of \TeX\ the main information needed about the
\index{\n{tfm} files}%
font consists of the dimensions of the characters.
\TeX\ finds these in the font metrics files, which usually have
extension \n{.tfm}. Such files
contain \begin{itemize} \item global information: the \cs{fontdimen}
parameters, and some other information,
\item dimensions and the italic corrections of characters, and
\altt 
\item ligature and kerning programs for characters.
	\end{itemize}
Also, the design size of a font is specified in the \n{tfm} file;
see above. The definition of the \n{tfm} format can be found
in~\cite{Knuth:TeXprogram}.

%\spoint[font:dims] Font dimensions
\subsection{Font dimensions}
\label{font:dims}

Text fonts need to have at least seven \csterm{fontdimen} parameters
to describe \indextermsub{font}{dimensions}
(but \TeX\ will take zero for unspecified parameters);
math symbol and math extension fonts have more
(see page~\pageref{fam23:fontdims}).
For text fonts the minimal set of seven comprises the following:
\begin{enumerate} \item the slant per point; this dimension is used
    for the proper horizontal positioning of accents;
\item the interword space: this is used unless the user
    specifies an explicit \cs{spaceskip};
    see Chapter~\ref{space};
\item interword stretch: the stretch component of the interword
    space;
\item interword shrink: the shrink component of
    the interword space;
\item the x-height: the value of
    the \gr{internal unit} \n{ex}, which is usually about the
    height of the lowercase letter~`x'; 
\item the quad width:
    the value of the \gr{internal unit} \n{em}, which is
    approximately the width of the capital letter~`M'; and
\item the extra space: the space added to the interword space
at the end of sentences (that is, when \cs{spacefactor}${}\geq2000$)
unless the user specifies an explicit \cs{x\-space\-skip}.
\end{enumerate}

Parameters 1 and~5 are purely information about the font
and there is no point in varying them.
The values of other parameters can be changed in order to
adjust spacing; see Chapter~\ref{space} for examples
of changing parameters 2, 3, 4, and~7.

Font dimensions can be altered in a \gr{font
  assignment}\index{assignment!font}, which is a \gr{global
  assignment}\index{assignment!global} (see
page~\pageref{global:assign}):
\begin{Disp}\cs{fontdimen}\gr{number}\gr{font}\gr{equals}\gr{dimen}
\end{Disp} See above for the definition of \gr{font}.

%\spoint Kerning
\subsection{Kerning}

Some combinations of characters should be moved closer
together than would be the case if their bounding boxes
were to be just abutted. This fine spacing is called \indexterm{kerning},
and a proper kerning is as essential to a font as the
design of the letter shapes.

Consider as an example\message{Kerning!}
\begin{Disp} `Vo' versus the unkerned variant `V\hbox{}o'\end{Disp}

Kerning in \TeX\ is controlled by information in the
\n{tfm} file, and is therefore outside the influence of the
user. The \n{tfm} file can be edited, however (see Chapter~\ref{TeXcomm}).

The \cs{kern} command has (almost) nothing to do with the
phenomenon of kerning; it is explained in Chapter~\ref{glue}.

%\spoint Italic correction
\subsection{Italic correction}

The primitive control symbol \csterm{/} inserts the 
\indexterm{italic correction}
of the previous character or ligature.
Such a correction may be necessary owing to the definition
of the \indextermbus{bounding}{box} of a character. This box always
has vertical sides, and the width of the character as \TeX\
perceives it is the distance between these sides.
However, in order to achieve proper spacing  for slanted or
italic typefaces, characters may very well project outside their
bounding boxes. The italic correction is then needed if
such an overhanging character is followed by a 
character from a non-slanting typeface.

Compare for instance\message{Visible italic correction!}
\begin{Disp} `{\italic\TeX} has'
to `{\italic\TeX\/} has',
\end{Disp} where the second version was typed as
\begin{verbatim}
{\italic\TeX\/} has
\end{verbatim}

The size of the italic correction of each character
is determined by font information
in the font metrics file; for the Computer Modern fonts it is
approximately half the `overhang' of the characters;
see~\cite{K:partE}.
Italic correction is not the same as \cs{fontdimen1}, slant
per point. That font dimension is used only for positioning
accents on top of characters.

An italic correction can only be inserted if the previous item
processed 
by \TeX\ was a character or ligature. Thus the
following solution for roman text inside an italic passage
does not work:
\begin{verbatim}
{\italic Some text {\/\roman not} emphasized}
\end{verbatim}
The italic correction has no effect here,
because the previous item is glue.

%\spoint Ligatures
\subsection{Ligatures}

Replacement of character sequences by \indexterm{ligatures} is controlled
by information in the \n{tfm} file of a font.
Ligatures are formed from \gr{character} commands:
sequences such as \n{fi} are replaced by `fi' in some fonts.

Other ligatures traditionally in use are
between \n{ff}, \n{ffi}, \n{fl}, and \n{ffl};
in some older works \n{ft} and \n{st} can be found,
and similarly to the \n{fl} ligature \n{fk} and \n{fb}
can also occur.

Ligatures in \TeX\ can be formed between explicit character
tokens, \cs{char} commands, and \gr{chardef token}s.
For example,
the sequence \verb-\char`f\char`i- is replaced by the
`fi' ligature, if such a ligature is part of the font.

Unwanted ligatures can be suppressed in a number of ways:
the unwanted ligature `\hbox{halflife}' can 
for instance be prevented by
\begin{disp} \verb>half{}life>, \verb>half{l}ife>, \verb>half\/life>,
      or \verb>half\hbox{}life>\end{disp}
but the solution using italic correction is not equivalent
to the others.

%\spoint Boundary ligatures
\subsection{Boundary ligatures}

Each word is surrounded by a left and a right
boundary character (\TeX3 only).
This makes phenomena possible
such as the two different sigmas in Greek:
one at the end of a word, and one for every other position.
This can be realized through a ligature with the
boundary character. A~\csterm{noboundary} command immediately
before or after a word suppresses the boundary character
at that place.

In general, the ligature mechanism has become more complicated
with the transition to \TeX\ version~3; see~\cite{K:TeX23}.

%%%% end of input file [fontfam]

\endofchapter

\chapter{Boxes}\label{boxes}\index{boxes|(}

The horizontal and vertical boxes of \TeX\ are containers for
pieces of horizontal and vertical lists.
Boxes can be stored in box registers. 
This chapter treats box registers and such
aspects of boxes as their dimensions, and the way their components
are placed relative to each other.

\label{cschap:hbox}\label{cschap:vbox}\label{cschap:vtop}\label{cschap:vsplit}\label{cschap:box}\label{cschap:setbox}\label{cschap:copy}\label{cschap:ifhbox}\label{cschap:ifvbox}\label{cschap:ifvoid2}\label{cschap:newbox}\label{cschap:unhbox}\label{cschap:unvbox}\label{cschap:unhcopy}\label{cschap:unvcopy}\label{cschap:ht}\label{cschap:dp}\label{cschap:wd}\label{cschap:boxmaxdepth}\label{cschap:splitmaxdepth}\label{cschap:badness}\label{cschap:hfuzz}\label{cschap:vfuzz}\label{cschap:hbadness}\label{cschap:vbadness}\label{cschap:overfullrule}\label{cschap:hsize}\label{cschap:vsize}\label{cschap:lastbox}\label{cschap:raise}\label{cschap:lower}\label{cschap:moveleft}\label{cschap:moveright}\label{cschap:everyhbox}\label{cschap:everyvbox}
\begin{inventory}
\item [\cs{hbox}] 
      Construct a horizontal box.
\item [\cs{vbox}] 
      Construct a vertical box with reference point of the last item.
\item [\cs{vtop}] 
      Construct a vertical box with reference point of the first item.
\item [\cs{vcenter}] 
      Construct a vertical box vertically centred
      on the math axis; this command can only be used in math mode.

\item [\cs{vsplit}] 
      Split off the top part of a vertical box. 

\item [\cs{box}] 
      Use a box register, emptying it. 

\item [\cs{setbox}] 
      Assign a box to a box register.

\item [\cs{copy}] 
      Use a box register, but retain the contents. 

\item [\cs{ifhbox \cs{ifvbox}}]
\mdqon
      Test whether a box register contains a horizontal/""vertical box.
\mdqoff

\item [\cs{ifvoid}] 
      Test whether a box register is empty.


\item [\cs{newbox}] 
      Allocate a new box register. 

\item [\cs{unhbox \cs{unvbox}}]
      Unpack a box register containing a horizontal/vertical box,
      adding the contents to the current horizontal/vertical list,
      and emptying the register. 

\item [\cs{unhcopy \cs{unvcopy}}]
      The same as \cs{unhbox}$\,$/$\,$\cs{unvbox},
      but do not empty the register. 

\item [\cs{ht \cs{dp} \cs{wd}}]
      Height/depth/width of the box in a box register. 

\item [\cs{boxmaxdepth}] 
      Maximum allowed depth of boxes.
      Plain \TeX\ default:~\cs{maxdimen}.

\item [\cs{splitmaxdepth}]
      Maximum allowed depth of boxes generated by \cs{vsplit}.

\item [\cs{badness}] 
      Badness of the most recently constructed box.

\item [\cs{hfuzz \cs{vfuzz}}]
      Excess size that \TeX\ tolerates before it considers  
\mdqon
      a horizontal/""vertical box overfull.
\mdqoff

\item [\cs{hbadness \cs{vbadness}}]
      Amount of tolerance before \TeX\ reports an underfull 
\mdqon
      or overfull  horizontal/""vertical box.
\mdqoff

\item [\cs{overfullrule}] 
      Width of the rule that is printed to indicate 
      overfull horizontal boxes.

 
\item [\cs{hsize}] 
      Line width used for text typesetting inside a vertical box.

\item [\cs{vsize}] 
      Height of the page box.


\item [\cs{lastbox}] 
      Register containing the last item added to the current list, 
      if this was a box.

\item [\cs{raise \cs{lower}}]
      Adjust vertical positioning of a box in horizontal mode. 

\item [\cs{moveleft \cs{moveright}}]
      Adjust horizontal positioning of a box in vertical mode. 

\item [\cs{everyhbox \cs{everyvbox}}]
\mdqon
      Token list inserted at the start of a horizontal/""vertical box.
\mdqoff

\end{inventory}

\section{Boxes}

In this chapter we shall look at boxes. Boxes are containers
for pieces of horizontal or vertical lists.
Boxes that are needed more than once can be stored in box registers.

When \TeX\ expects a \gr{box}, any of the following forms
is admissible:
\begin{itemize}
\item \cs{hbox}\gr{box specification}\lb\gr{horizontal material}\rb
\item \cs{vbox}\gr{box specification}\lb\gr{vertical material}\rb
\item \cs{vtop}\gr{box specification}\lb\gr{vertical material}\rb
\item \cs{box}\gr{8-bit number}
\item \cs{copy}\gr{8-bit number}
\item \cs{vsplit}\gr{8-bit number}\n{to}\gr{dimen}
\item \cs{lastbox}
\end{itemize}
A \gr{box specification} is defined as\label{box:spec}
\begin{disp}\gr{box specification} $\longrightarrow$ \gr{filler}
\nl\indent$|$ \n{to} \gr{dimen}\gr{filler} 
          $|$ \n{spread} \gr{dimen}\gr{filler}
\end{disp}
An \gr{8-bit number} is a number in the range~0--255.

The braces surrounding box material define a group;
they can be explicit characters
of categories 1 and~2 respectively,
or control sequences \cs{let} to such characters;
see also below.


A \gr{box} can in general be used in horizontal, vertical,
and math mode, but see below for the \cs{lastbox}.
The connection between
boxes and modes is explored further in Chapter~\ref{hvmode}.

The box produced by \cs{vcenter} \ldash a command that is allowed only in
math mode \rdash  is not a \gr{box}. For instance,
it can not be assigned with \verb=\setbox=; see further
Chapter~\ref{math}.

The \cs{vsplit} operation is treated in Chapter~\ref{page:break}.

\section{Box registers}

There are 256 box registers, numbered 0--255. 
Either a box register is  empty (`void'), or it contains a horizontal
or vertical box.
This section discusses specifically box {\em registers};
the sizes of boxes, and the way material is arranged inside them,
is treated below.

%\spoint Allocation: \cs{newbox}
\subsection{Allocation: \cs{newbox}}

The plain \TeX\ \csterm{newbox} macro allocates an unused
box register:
\begin{verbatim}
\newbox\MyBox 
\end{verbatim}
after which one can say
\begin{verbatim}
\setbox\MyBox=...
\end{verbatim}
or
\begin{verbatim}
\box\MyBox
\end{verbatim}
and so on.
Subsequent calls to this macro give subsequent box numbers;
this way macro collections can allocate their own boxes
without fear of collision with other macros.

The number of the box is assigned by \cs{chardef}
(see Chapter~\ref{alloc}). 
This implies that \cs{MyBox} is equivalent to,
and can be used as, a~\gr{number}.
The control sequence
\altt
\cs{newbox} is an \cs{outer} macro.
Newly allocated box registers are initially empty.


\subsection{Usage: \cs{setbox}, \cs{box}, \cs{copy}}

A~register is filled by assigning a \gr{box}
\csidx{setbox}%
to it:
\begin{Disp}\verb>\setbox>\gr{number}\gr{equals}\gr{box}\end{Disp}
For example, the \gr{box} can be explicit
\begin{Disp}\verb>\setbox37=\hbox{...}>\quad or\quad \verb>\setbox37=\vbox{...}>
\end{Disp}
or it can be a box register:
\begin{verbatim}
\setbox37=\box38
\end{verbatim}
Usually, box numbers will have been assigned by a \cs{newbox}
command.

The box in a box register is appended
by the commands \cs{box} and~\cs{copy}
to whatever list \TeX\ is building: the call
\begin{verbatim}
\box38
\end{verbatim}
appends box~38.
To save memory space, box registers become empty by using them:
\TeX\ assumes that after you have inserted a box by
calling \csterm{box}$nn$ in some mode, you do not need the
contents of that register any more and empties it.
In case you {\em do\/} need the contents of
a box register more than once, 
you can \csterm{copy} it. Calling \cs{copy}$nn$ is
equivalent to \cs{box}$nn$ in all respects except that
the register is not cleared.

It is possible to unwrap the contents of a box register
by `unboxing' it using the commands \cs{unhbox} and \cs{unvbox},
and their copying versions \cs{unhcopy} and \cs{unvcopy}.
Whereas a box can be used in any mode, the
unboxing operations can only be used in the appropriate mode,
since in effect they contribute a partial
horizontal or vertical list (see also Chapter~\ref{hvmode}).
See below for more information on unboxing registers.

%\spoint Testing: \cs{ifvoid}, \cs{ifhbox}, \cs{ifvbox}
\subsection{Testing: \cs{ifvoid}, \cs{ifhbox}, \cs{ifvbox}}

Box
registers can be tested for their contents:
\begin{disp}\cs{ifvoid}\gr{number}\end{disp}
is true if the box register is empty.
Note that an empty, or `void',
box register is not the same as a register containing an empty box.
An empty box is still either a horizontal or a vertical box;
a~void register can be used as both.

The test
\begin{disp}\cs{ifhbox}\gr{number}\end{disp}
is true if the box register contains a horizontal box;
\begin{disp}\cs{ifvbox}\gr{number}\end{disp}
is true if the box register contains a vertical box.
Both tests are false for void registers.

%\spoint[lastbox] The \cs{lastbox}
\subsection{The \cs{lastbox}}
\label{lastbox}

When \TeX\ has built a partial list, the last box in this
list is accessible as the \csterm{lastbox}. This behaves
like a box register, so you can remove the last box from  the
list by assigning the \cs{lastbox} to some  box register. 
If the last item on the current list is not a box,
the \cs{lastbox} acts like a void box register.
It is not possible to get hold of the last box
in the case of the main vertical list.
The \cs{lastbox} is then always void.

As an example, the statement
\begin{verbatim}
{\setbox0=\lastbox}
\end{verbatim}
removes
the last box from the current list, assigning it to box
register~0. Since this assignment occurs inside a group,
the register is cleared at the end of the group.
At the start of a paragraph this can be used to remove the
indentation box (see Chapter~\ref{par:start}).
Another example of \cs{lastbox} can be found on page~\pageref{varioset}.

Because the \verb-\lastbox- is always empty in external vertical mode,
it is not possible to get hold of boxes that have been 
added to the page. However, it is possible to dissect
the page once it is in \cs{box255}, for instance doing
\begin{verbatim}
\vbox{\unvbox255{\setbox0=\lastbox}}
\end{verbatim}
inside the output routine.

If boxes in vertical mode have been shifted by \cs{moveright}
or \cs{moveleft}, or if boxes in horizontal mode  have
been raised by \cs{raise} or lowered by \cs{lower}, 
any information about this
displacement due to such a command is lost when
the \cs{lastbox} is taken from the list.

\section{Natural dimensions of boxes}

\subsection{Dimensions of created horizontal boxes}

Inside an \csterm{hbox} all constituents are lined up next to each other,
with their reference points on the baseline of the box,
unless they are moved explicitly in the vertical direction
by \cs{lower} or~\cs{raise}.

The resulting width of the box is the sum of the widths
of the components. Thus the width of
\begin{verbatim}
\hbox{\hskip1cm}
\end{verbatim}
is positive, and the width of
\begin{verbatim}
\hbox{\hskip-1cm}
\end{verbatim}
is negative. By way of example,
\begin{disp}\verb>a\hbox{\kern-1em b}-->\end{disp}
gives as output
\begin{disp}\leavevmode\hphantom{b}a\hbox{\kern-1em b}--\end{disp}
\message{check align input/output}
which shows that a horizontal box can have negative
width.

The height and depth of an \cs{hbox} are the
maximum amount that constituent boxes project above and
below the baseline of the box. They are non-negative when the
box is created.

The commands \cs{lower} and \cs{raise} are the only possibilities
for vertical movement inside an \cs{hbox} (other than
including a \cs{vbox} inside the \cs{hbox}, of course);
a~\gr{vertical command} \ldash such as \cs{vskip} \rdash 
is not allowed in a horizontal box, and
\cs{par}, although allowed,
does not do anything inside a horizontal box.

%\spoint Dimensions of created vertical boxes
\subsection{Dimensions of created vertical boxes}

Inside a \csterm{vbox} vertical material is lined up with the
\csidx{vtop}%
reference points on the vertical line through the reference
point of the box,
unless components are moved explicitly in the horizontal direction
by \csterm{moveleft} or~\csterm{moveright}.

The  reference point of a vertical box
is always located at the left boundary of the box.
The width of a vertical box
is then the maximal amount that any material in the
box sticks to the right of the reference point.
Material to the left of the reference point is
not taken into account in the width.
Thus the result of
\begin{disp}\verb>a\vbox{\hbox{\kern-1em b}}-->\end{disp}
is
\begin{disp}\leavevmode\hphantom{b}a\vbox{\hbox{\kern-1em b}}--\end{disp}
This should be contrasted with the above example.


The calculation of height and depth is different
for vertical boxes constructed by \cs{vbox} and \cs{vtop}.
The ground rule is that
a \cs{vbox} has a reference point that lies on
the baseline of its last component,
and a \cs{vtop} has its reference point on the baseline of the
first component.
In general, the depth (height) of a \cs{vbox} (\cs{vtop})
\alt
can be non-zero if the last (first) item is a box or rule.

The height of a \cs{vbox} is then the sum of the heights and
depths of all components except the last, plus the height
of that last component; the depth of the \cs{vbox} is the
depth of its last component.
The depth of a \cs{vtop}
is the sum of the depth of the first component and the heights
and depths of all subsequent material; its height is the
height of the first component.

However, the actual rules are a bit
more complicated when the first component of a \cs{vtop}
or the last component of a \cs{vbox} is not a box or rule.
If the last component of a \cs{vbox} is a kern or a glue,
the depth of that box is zero; a \cs{vtop}'s 
height is zero
unless its first component is a box or rule.
\altt
(Note the asymmetry in these definitions; see below for
an example illustrating this.)
The depth of a \cs{vtop}, then, is equal to the total
height plus depth of all enclosed material minus
the height of the \cs{vtop}.

There is a limit on the depth of vertical boxes:
if the depth of a \cs{vbox} or \cs{vtop}
calculated by the above rules would exceed
\csterm{boxdepth},
the reference point of the box
is moved down by the excess amount. 
More precisely, the excess depth is added to the 
natural height of the box. If the box had a \n{to} or
\n{spread} specification, any glue is set anew to take
the new height into account.

Ordinarily,
\cs{boxmaxdepth} is set to the maximum dimension
possible in \TeX. It is for instance reduced during some of
the calculations  in the plain \TeX\ output routine;
see Chapter~\ref{output}.

\subsection{Examples}

Horizontal boxes are relatively straightforward. Their width is the
distance between the `beginning' and the `end' of the
box, 
and consequently the width is not necessarily positive.
With
\begin{verbatim}
\setbox0=\hbox{aa} \setbox1=\hbox{\copy0 \hskip-\wd0}
\end{verbatim}
the \cs{box1} has width zero;
\begin{Disp} \verb-/\box1/-\quad gives\quad
`{\setbox0=\hbox{aa}\setbox1=\hbox{\copy0 \hskip-\wd0}/\box1/}\kern.75em'
\end{Disp}
The height and depth of a horizontal box cannot be negative: in
\begin{verbatim}
\setbox0=\hbox{\vrule height 5pt depth 5pt}
\setbox1=\hbox{\raise 10pt \box0}
\end{verbatim}
the \cs{box1} has depth \n{0pt} and height~\n{15pt}

Vertical boxes are more troublesome than horizontal boxes.
Let us first treat their width.
After
\begin{verbatim}
\setbox0=\hbox{\hskip 10pt}
\end{verbatim}
the box in the
\cs{box0} register has a width of \n{10pt}. Defining
\begin{verbatim}
\setbox1=\vbox{\moveleft 5pt \copy0}
\end{verbatim}
the \cs{box1} will have width \n{5pt}; material to the
left of the reference point is not accounted for in the
width of a vertical box. With
\begin{verbatim}
\setbox2=\vbox{\moveright 5pt \copy0}
\end{verbatim}
the \cs{box2} will have width \n{15pt}.

The depth of a \cs{vbox} is the depth of the last item if
that is a box, so
\begin{verbatim}
\vbox{\vskip 5pt \hbox{\vrule height 5pt depth 5pt}}
\end{verbatim}
has height \n{10pt} and depth \n{5pt}, 
and
\begin{verbatim}
\vbox{\vskip -5pt \hbox{\vrule height 5pt depth 5pt}}
\end{verbatim}
has height \n{0pt} and depth~\n{5pt}.
With a glue or kern as the last item in the box, the resulting depth
is zero, so 
\begin{verbatim}
\vbox{\hbox{\vrule height 5pt depth 5pt}\vskip 5pt}
\end{verbatim}
has height \n{15pt} and depth~\n{0pt};
\begin{verbatim}
\vbox{\hbox{\vrule height 5pt depth 5pt}\vskip -5pt}
\end{verbatim}
has height \n{5pt} and depth~\n{0pt}.

The height of a \cs{vtop} behaves (almost) the same with respect to
the first item of the box, as the depth of a \cs{vbox} does
with respect to the last item. Repeating the above examples with
a \cs{vtop} gives the following:
\begin{verbatim}
\vtop{\vskip 5pt \hbox{\vrule height 5pt depth 5pt}}
\end{verbatim}
has height \n{0pt} and depth \n{15pt}, 
and
\begin{verbatim}
\vtop{\vskip -5pt \hbox{\vrule height 5pt depth 5pt}}
\end{verbatim}
has height \n{0pt} and depth~\n{5pt};
\begin{verbatim}
\vtop{\hbox{\vrule height 5pt depth 5pt} \vskip 5pt}
\end{verbatim}
has height \n{5pt} and depth~\n{10pt}, and
\begin{verbatim}
\vtop{\hbox{\vrule height 5pt depth 5pt} \vskip -5pt}
\end{verbatim}
has height \n{5pt} and depth~\n{0pt}.

%\point More about box dimensions
\section{More about box dimensions}

%\spoint Predetermined dimensions
\subsection{Predetermined dimensions}

The size of a box can be specified in advance
with a \gr{box specification}; see above for the syntax.
Any glue
in the box is then set in order to reach the required size.
Prescribing the size of the box is done by
\begin{disp}\cs{hbox} \n{to} \gr{dimen} \n{\lb...\rb},
     \cs{vbox} \n{to} \gr{dimen} \n{\lb...\rb}\end{disp}
If stretchable or shrinkable glue is present in the box,
it is stretched or shrunk in order to give the box the
specified size. Associated with this glue setting is a badness value
(see Chapter~\ref{glue}). If no stretch or shrink \ldash whichever
is necessary \rdash  is present, the resulting box will be underfull
or overfull respectively. Error reporting for over/underfull
boxes is treated below.

Another command to let a box have a size other than
the natural size is
\begin{disp}\cs{hbox} \n{spread} \gr{dimen} \n{\lb...\rb},
     \cs{vbox} \n{spread} \gr{dimen} \n{\lb...\rb}\end{disp}
which tells \TeX\ to set the glue in such a way that
the size of the box is a  specified amount more than the 
natural size.

Box specifications for \cs{vtop} vertical boxes are
somewhat difficult to interpret. \TeX\ constructs a \cs{vtop}
by first making a \cs{vbox}, including 
glue settings induced by a \gr{box specification};
then it computes the height and depth by the above rules.
Glue setting is described in Chapter~\ref{glue}.

%\spoint Changes to box dimensions
\subsection{Changes to box dimensions}

The dimensions of a box register are accessible by the
commands \csterm{ht}, \csterm{dp}, and~\csterm{wd};
for instance \cs{dp13} gives the depth of box~13.
However, not only can boxes be measured this way;
by assigning values to these
dimensions \TeX\ can even be fooled into thinking that
a box has a  size different from its actual.
However, changing the dimensions of a box does not change
anything about the contents; in particular it does not
change the way the glue is set.


Various formats use this in `smash' macros: the macro defined by
\csidx{smash}%
\begin{verbatim}
\def\smash#1{{\setbox0=\hbox{#1}\dp0=0pt \ht0=0pt \box0\relax}}
\end{verbatim}
places its argument but annihilates its height and depth;
\altt
that is, the output does show the whole box, but further calculations
by \TeX\ act as if the height and depth were zero.

Box dimensions can be changed only by setting them.
They are \gr{box dimen}s, which can only be set
in a \gr{box size assignment}, and not, for instance
changed with \cs{advance}.

Note that a \gr{box size assignment}\index{assignment!box size} is a
\gr{global assignment}\index{assignment!global} its effect transcends
  any groups in which it occurs (see Chapter~\ref{group}).  Thus the
  output of
\begin{verbatim}
\setbox0=\hbox{---} {\wd0=0pt} a\box0b
\end{verbatim}
is `{\setbox0=\hbox{---}{\wd0=0pt}a\box0b}\kern.5em'.

The limits that hold on the dimensions with which a 
box can be created (see above) do not hold for explicit changes to the
\mdqon
size of a box: the assignment \cs{dp0=}""\n{-2pt} for a 
\mdqoff
horizontal box is perfectly admissible.

\subsection{Moving boxes around}

In a horizontal box all constituent elements are lined up
with their reference points at the same height as the 
reference point of the box. Any box inside a horizontal
box can be lifted or dropped using the macros
\csterm{raise} and~\csterm{lower}.

Similarly, in a vertical box all constituent elements
are lined up with their reference points underneath one another,
in line with the reference point of the box.
Boxes can now be moved sideways by the macros 
\csterm{moveleft} and~\csterm{moveright}.

Only boxes can be shifted thus; these operations cannot 
be applied to, for instance, characters or rules.

\subsection{Box dimensions and box placement}

\TeX\ places the components of horizontal and
vertical lists by maintaining a reference line and a
current position on that line. For horizontal lists
the reference line is the baseline of the surrounding
\cs{hbox}; for vertical lists it is the vertical line
through the reference point of the surrounding \cs{vbox}.

In horizontal mode a component is placed as follows.
The current position coincides initially
with the reference point of the surrounding box. After that,
the following actions are carried out.
\begin{enumerate} \item If the component has been shifted by
\cs{raise} or \cs{lower}, shift the current
position correspondingly.
\item If the component is a horizontal box, use
this algorithm recursively for its contents; 
if it is a vertical box, go up  by the height of this box,
putting  a new current position for the enclosed vertical list there,
and place its components using the algorithm for vertical
lists below.
\item Move the current position (on the reference line)
to the right by the width of the component.
\end{enumerate}

For the list in a vertical box \TeX's  current position is
initially at the upper left corner of that box, as explained above,
and the reference line is the vertical line through that point;
it also runs through the reference point of the box.
Enclosed components are then placed as follows.
\begin{enumerate} \item If a component has been shifted using
\cs{moveleft} or \cs{moveright}, shift the current position
accordingly.
\item Put the component with its upper left corner at the
current position.
\item If the component is a vertical box, use this algorithm
recursively for its contents; if it is a horizontal box,
its reference point can be found below  the current position
by the height of the box. Put the current position for that
box there, and use the above algorithm for horizontal lists.
\item Go down by the height plus depth of the box
(that is, starting at the upper left corner of the box)
on the  reference line,
and continue processing vertically.
\end{enumerate}
Note that the above processes do not describe the construction
of boxes. That would (for instance)
involve for vertical boxes the insertion
of baselineskip glue. Rather, it describes the way the components
of a finished box are arranged in the output.

\subsection{Boxes and negative glue}

Sometimes it is useful to have boxes overlapping instead of
line up. An easy way to do this is to use negative glue.
In horizontal mode
\begin{verbatim}
{\dimen0=\wd8 \box8 \kern-\dimen0}
\end{verbatim}
places box 8 without moving the current location.

More versatile are the macros \csterm{llap} and \csterm{rlap}\label{rlap},
defined as
\begin{verbatim}
\def\llap#1{\hbox to 0pt{\hss #1}}
\end{verbatim}
and
\begin{verbatim}
\def\rlap#1{\hbox to 0pt{#1\hss}}
\end{verbatim}
that allow material to protrude left or right from the
current location.
The \cs{hss} glue is equivalent to \verb>\hskip 0pt plus 1fil minus 1fil>, 
which absorbs any positive or negative width
of the argument of \cs{llap} or \cs{rlap}.

\begin{example} The sequence
\begin{verbatim}
\llap{\hbox to 10pt{a\hfil}}
\end{verbatim}
is effectively the same as
\begin{verbatim}
\hbox{\hskip-10pt \hbox to 10pt{a\hfil}}
\end{verbatim}
which has a total width of~\n{0pt}.
\end{example}

\section{Overfull and underfull boxes}
\label{over/underfull}

If a box has a  size specification \TeX\ will
stretch or shrink glue in the box. For glue with
only finite stretch or shrink components the {\em badness\/}
(see Chapter~\ref{line:break}) of stretching or shrinking
is computed.
In \TeX\ version~3 the badness
\csidx{badness}%
\thecstoidxsub{TeX}{version 3}
of the box most recently
constructed is available for inspection
by the user through the \cs{badness} parameter. Values for
badness range 0--$10\,000$, but if the box is overfull
it is~$1\,000\,000$.

When \TeX\ considers the badness too large,
it gives a diagnostic message. Let us first consider error reporting
for horizontal boxes.

Horizontal boxes of which the glue has to stretch are never reported if
\csidx{vbadness}%
\csterm{hbadness}${}\geq10\,000$; otherwise \TeX\ reports them
as `underfull' if their badness is more than \cs{hbadness}.

Glue shrinking can lead to `overfull' boxes: a box is called
\csidx{vfuzz}%
overfull if the available shrink is less than the shrink
necessary to meet the box specification. An overfull box
is only reported if the difference in shrink is more than
\csterm{hfuzz}, or if \cs{hbadness}${}<100$ (and it turns out that
using all available shrinkability has badness~$100$).

\begin{example} Setting \verb>\hfuzz=1pt> will let \TeX\ ignore
boxes that can not shrink enough if they lack less than~\n{1pt}.
In
\begin{verbatim}
\hbox to 1pt{\hskip3pt minus .5pt}
\end{verbatim}
\begin{verbatim}
\hbox to 1pt{\hskip3pt minus 1.5pt}
\end{verbatim}
only the first box will give an error message:
it is \n{1.5pt} too big, whereas the second lacks
\n{.5pt} which is less than \cs{hfuzz}.
\end{example}

Also, boxes that shrink but that are not overfull can be reported:
if a box is `tight', that is, if it uses at least half its
shrinkability, \TeX\ reports this fact if the
computed badness (which is between 13 and~100) is more than
\cs{hbadness}.

For horizontal and vertical boxes this error reporting is almost
the same, with parameters \cs{vbadness} and \cs{vfuzz}.
The difference is that for horizontal overfull boxes
\TeX\ will draw a rule to the right of the box that has the
same height as the box, and width \csterm{overfullrule}.
No overfull rule ensues if 
the \cs{tabskip} glue in an \cs{halign} cannot be
shrunk enough.

\section{Opening and closing boxes}

The opening and closing braces of a box can be either explicit, that
is, character tokens of category 1\index{category!1}
and~2\index{category!2}, or implicit, a control sequence \verb=\let=
to such a character.  After the opening brace the \csterm{everyhbox} or
\csterm{everyvbox} tokens are inserted.  If this box appeared in a
\csterm{setbox} assignment any \csterm{afterassignment} token is
inserted even before the `everybox' tokens.

\begin{example} \label{every:box:assign}
\begin{verbatim}
\everyhbox{b}
\afterassignment a
\setbox0=\hbox{c}
\showbox0
\end{verbatim}
gives
\begin{verbatim}
> \box0=
\hbox(6.94444+0.0)x15.27782
.\tenrm a
.\tenrm b
.\kern0.27779
.\tenrm c
\end{verbatim}
\end{example}

Implicit braces can be used to let a box be opened or closed
by a macro, for example:
\begin{verbatim}
\def\openbox#1{\setbox#1=\hbox\bgroup}
\def\closebox#1{\egroup\DoSomethingWithBox#1}
\openbox0 ... \closebox0
\end{verbatim}
This mechanism can be used to scoop up paragraphs:
\begin{verbatim}
\everypar{\setbox\parbox=
    \vbox\bgroup
         \everypar{}
         \def\par{\egroup\UseBox\parbox}}
\end{verbatim}
Here the \cs{everypar} opens the box and lets the text be
set in the box: starting for instance
\begin{verbatim}
Begin a text ...
\end{verbatim}
gives the equivalent of
\begin{verbatim}
\setbox\parbox=\vbox{Begin a text ...
\end{verbatim}
Inside the box \cs{par} has been redefined, so
\begin{verbatim}
... a text ends.\par
\end{verbatim}
is equivalent to
\begin{verbatim}
... a text ends.}\Usebox\parbox
\end{verbatim}

In this example, the \cs{UseBox} command can only treat the
box as a whole; if the elements of the box should somehow
be treated separately another approach is necessary.
In 
\begin{verbatim}
\everypar{\setbox\parbox=
  \vbox\bgroup\everypar{}%
       \def\par{\endgraf\HandleLines
                \egroup\box\parbox}}
\def\HandleLines{ ... \lastbox ... }
\end{verbatim}
the macro \cs{HandleLines} can have access to successive
elements from the vertical list of the paragraph.
See also the example on page~\pageref{varioset}.

%\point Unboxing
\section{Unboxing}

Boxes can be unwrapped by the commands \csterm{unhbox} and
\csterm{unvbox}, and by their copying versions 
\csterm{unhcopy} and \csterm{unvcopy}. 
These are horizontal and vertical commands
(see Chapter~\ref{hvmode}), considering that in effect
they contribute a partial horizontal or vertical list.
It is not possible to \cs{unhbox} a register
containing a \cs{vbox} or vice versa,
but a void box register can both be \cs{unhbox}ed and
\cs{unvbox}ed.

Unboxing takes the contents of a box in a box register and appends
them to the surrounding list; any glue can then 
be set anew. Thus
\begin{verbatim}
\setbox0=\hbox to 1cm{\hfil} \hbox to 2cm{\unhbox0}
\end{verbatim}
is completely equivalent to 
\begin{verbatim}
\hbox to 2cm{\hfil}
\end{verbatim}
and not to
\begin{verbatim}
\hbox to 2cm{\kern1cm}
\end{verbatim}

The intrinsically horizontal nature of \cs{unhbox} is
\csidx{leavevmode}%
used to define
\begin{verbatim}
\def\leavevmode{\unhbox\voidb@x}
\end{verbatim}
This command switches from vertical mode to horizontal without
adding anything to the horizontal list. 
However, the subsequent \cs{indent} caused by this transition
adds an indentation box.
In horizontal mode the \cs{leavevmode} command has no effect.
Note that here it is not necessary to use \cs{unhcopy},
because the register is empty anyhow. 

Beware of the following subtlety: unboxing in vertical
mode does not add interline glue between the box contents and
any preceding item. 
Also, the value of \cs{prevdepth} is not
changed, so glue between the box contents and any following
item will  occur only if there was something preceding the box;
interline glue will be based on the depth of that preceding item.
Similarly, unboxing in horizontal mode does not influence
the \cs{spacefactor}.

%\point Text in boxes
\section{Text in boxes}

Both horizontal and vertical boxes can contain text. However,
\index{boxes !text in}
the way text is treated differs. 
In horizontal boxes
the text is placed in one straight line, and the width of
the box is in principle the natural width of the text 
(and other items) contained in it. No \gram{vertical command}s
are allowed inside a horizontal box, and \cs{par} does
nothing in this case.

For vertical boxes the situation is radically different.
As soon as a character, or any other \gram{horizontal command}
(see page~\pageref{h:com:list}),
is encountered in a vertical box, \TeX\ starts building a paragraph
in unrestricted horizontal mode, that is, just as if the paragraph
were directly part of the page.
At the occurrence of a \gram{vertical command}
(see page~\pageref{v:com:list}), or at the end
of the box, the paragraph is broken into lines using the
current values of parameters such as~\cs{hsize}.

Thus
\begin{verbatim}
\hbox to 3cm{\vbox{some reasonably long text}}
\end{verbatim}
will {\sl not\/} give a paragraph of width 3 centimetres 
(it gives an overfull horizontal box if \cs{hsize}${}>{}$\n{3cm}).
However,
\begin{verbatim}
\vbox{\hsize=3cm some reasonably long text}
\end{verbatim}
will be 3 centimetres wide.

A paragraph of text inside a vertical box is broken into
lines, which are packed in horizontal boxes.
These boxes  are then stacked
in internal vertical mode, possibly with
\cs{baselineskip} and \cs{lineskip} separating them
(this is treated in Chapter~\ref{baseline}).
This process is also used for text on the page; the boxes
are then stacked in outer vertical mode.

If the internal vertical list is empty, no \cs{parskip}
glue is added at the start of a paragraph.

Because text in a horizontal box is not
\label{wide:vbox}%
broken into lines, there is a further
difference between text in restricted and unrestricted
horizontal mode. In restricted horizontal mode no
discretionary nodes and whatsit items changing the
value of the current language are inserted.
This may give problems if the text is subsequently
unboxed to form part of a paragraph.

See Chapter~\ref{line:break} for an explanation of these
items, and \cite{Downs} for a way around this problem.

%\point Assorted remarks
\section{Assorted remarks}

%\spoint Forgetting the \cs{box}
\subsection{Forgetting the \cs{box}}

After \verb.\newcount\foo., one can use \cs{foo} on its own
to get the \cs{foo} counter.
For  boxes, however, one has to use \verb.\box\foo. to get
the \cs{foo} box.
The reason for this is that there exists
no separate \cs{boxdef} command, so \cs{chardef} is
used (see Chapter~\ref{alloc}). 

\begin{example}
Suppose \verb.\newbox\foo. allocates box register~25; then
typing \cs{foo} is equivalent  to typing
\verb.\char25..
\end{example}

%\spoint Special-purpose boxes
\subsection{Special-purpose boxes}

Some   box registers 
have a special
purpose:
\begin{itemize}
\item \cs{box255} is by used \TeX\ internally 
 to give the page to the output routine.
\item \cs{voidb@x} is the number of 
 a box register allocated in 
 \n{plain.tex}; it is supposed to be empty always.
 It is used in the macro \cs{leavevmode} and others.
\item when a new \cs{insert} is created with the plain \TeX\
 \cs{newinsert} macro, a \cs{count},
 \cs{dimen}, \cs{skip}, and \cs{box} all with the same number
 are reserved for that insert.
 The numbers for these registers count down from~254.
\end{itemize}


%\spoint The height of a vertical box in horizontal mode
\subsection{The height of a vertical box in horizontal mode}

In horizontal mode a vertical box is placed with its
reference point aligned vertically with the reference
point of the surrounding box. 
\TeX\ then traverses its contents starting at the left
upper corner; that is, the point that lies above the reference
point by a distance of the height of the box.
Changing the height of the box  implies then that the
contents of the box are placed at a different height.

Consider as an example
\begin{verbatim}
\hbox{a\setbox0=\vbox{\hbox{b}}\box0 c}
\end{verbatim}
which gives
\begin{disp}\leavevmode\hbox{a\setbox0=\vbox{\hbox{b}}\box0 c}\end{disp}
and
\begin{verbatim}
\hbox{a\setbox0=\vbox{\hbox{b}}\ht0=0cm \box0 c}
\end{verbatim}
which gives
\begin{disp}\leavevmode\hbox{a\setbox0=\vbox{\hbox{b}}\ht0=0cm \box0 c}\end{disp}

By contrast, changing the width of a box placed in vertical
mode has no effect on its placement.

%\spoint More subtleties with vertical boxes
\subsection{More subtleties with vertical boxes}

Since there are two kinds of vertical boxes, the \cs{vbox} and
the \cs{vtop}, using these two kinds nested may lead to
confusing results. For instance,
\begin{verbatim}
\vtop{\vbox{...}}
\end{verbatim}
is completely equivalent to just
\begin{verbatim}
\vbox{...}
\end{verbatim}

It was stated above that
the depth of a \cs{vbox} is zero if the last item
is a kern or  glue, and the height of a \cs{vtop} is
zero unless the first item in it is a box.
The above examples used a kern for that first or last item, 
but if, in the case of a \cs{vtop}, 
this item is not a glue or kern, one is apt to
overlook the effect that it has on the surrounding box.
For instance,
\begin{verbatim}
\vtop{\write16{...}...}
\end{verbatim}
has zero height,
because the write instruction
is packed into a `whatsit' item that is placed on the current,
that is, the vertical, list. 
The remedy here is
\begin{verbatim}
\vtop{\leavevmode\write16{...}...}
\end{verbatim}
which puts the whatsit in the beginning of the paragraph,
instead of above it.

Placement of items in a vertical list is sometimes
a bit tricky. There is for instance a difference between
how vertical and horizontal boxes are treated in a
vertical list. Consider the following examples.
After \cs{offinterlineskip} the first example\begin{verbatim}
\vbox{\hbox{a}
      \setbox0=\vbox{\hbox{(}}
      \ht0=0pt \dp0=0pt \box0
      \hbox{ b}}
\end{verbatim}
gives \begin{disp}\offinterlineskip\leavevmode\vbox{\hbox{a}
      \setbox0=\vbox{\hbox{(}}
      \ht0=0pt \dp0=0pt \box0
      \hbox{ b}}
\end{disp}
while a slight variant\begin{verbatim}
\vbox{\hbox{a}
      \setbox0=\hbox{(}
      \ht0=0pt \dp0=0pt \box0
      \hbox{ b}}
\end{verbatim}
gives
\begin{disp}\offinterlineskip\leavevmode\vbox{\hbox{a}
      \setbox0=\hbox{(}
      \ht0=0pt \dp0=0pt
      \box0
      \hbox{ b}}
\end{disp}
The difference is caused by the fact that horizontal boxes
are placed with respect to their reference point, but vertical
boxes with respect to their upper left corner.

%\spoint Hanging the \cs{lastbox} back in the list
\subsection{Hanging the \cs{lastbox} back in the list}

You can pick the last box off a vertical list that has been
compiled in (internal) vertical mode.
However, if you try to hang it back in the list the vertical
spacing may go haywire. If you just hang it back, 
\begin{verbatim}
\setbox\tmpbox=\lastbox
\usethetmpbox \box\tmpbox
\end{verbatim}
baselineskip glue is added a second time. If you `unskip' prior
to hanging the box back, 
\begin{verbatim}
\setbox\tmpbox=\lastbox \unskip
\usethetmpbox \box\tmpbox
\end{verbatim}
things go wrong in a more subtle way.
The \gram{internal dimen} \cs{prevdepth} 
(which controls interline glue; see Chapter~\ref{baseline})
will have a
value based on the last box, but what you need for the proper
interline glue is a depth based on one box earlier.
The solution is not to unskip, 
but to specify \cs{nointerlineskip}:
\begin{verbatim}
\setbox\tmpbox=\lastbox
\usethetmpbox \nointerlineskip \box\tmpbox
\end{verbatim}


%\spoint[varioset] Dissecting paragraphs with \cs{lastbox}
\subsection{Dissecting paragraphs with \cs{lastbox}}
\label{varioset}

Repeatedly applying \cs{last...} and \cs{un...} macros
\howto Take a paragraph apart\par
can be used to take a paragraph apart.
Here is an example of that.

\indent\vbox{\message{Check vario look!}
\hyphenpenalty10000 \exhyphenpenalty10000 \parindent 10pt\relax
\advance\hsize by -2\parindent \rightskip 0pt\relax
\newif\ifsnap \spaceskip=\fontdimen2\font plus 2\fontdimen2\font
\def\eatlines{
    \setbox2\lastbox    % check the last line
    \ifvoid2\global\snaptrue
    \else                      % if it's not empty
    \unskip\unpenalty          % take whatever is
    {\eatlines}                % above it;
    \setbox4\hbox{\unhcopy2}   % collapse this line
\message{^^Jsnap \the\wd4 vs \the\wd2^^J}
%\hbox{\copy4 .4} \hbox{\copy 2 .2}
    \ifdim\wd4<.97\wd2 % if the difference is too large, maybe snap
        \ifsnap \message{^^JNot snapping just now ^^J}
          \hbox to \hsize{\unhbox2} \global\snapfalse % we just snapped: not again
        \else   \message{^^JSnapping!^^J}
          \box4 \global\snaptrue    % safe to snap; remember snap
        \fi
    \else \message{^^Jfull length, no snap^^J}
      \box2 \global\snapfalse
    \fi
    \fi}
\advance\hsize by -.8in \advance \leftskip by .5 in\relax
\noindent In typesetting advertisement copy, a way of justifying
paragraphs has become popular in recent years
that is somewhere between flushright and raggedright
setting.
Lines that would stretch beyond certain limits
are set with their glue at natural width. This single paragraph
is but an example of this procedure; the macros are given
 next.\par\eatlines}\par

\begin{verbatim}
\newbox\linebox \newbox\snapbox
\def\eatlines{
    \setbox\linebox\lastbox    % check the last line
    \ifvoid\linebox
    \else                      % if it's not empty
    \unskip\unpenalty          % take whatever is
    {\eatlines}                % above it;
                               % collapse the line
    \setbox\snapbox\hbox{\unhcopy\linebox} 
                       % depending on the difference
    \ifdim\wd\snapbox<.98\wd\linebox
          \box\snapbox % take the one or the other,
    \else \box\linebox \fi
    \fi}
\end{verbatim}
This macro can be called as
\begin{verbatim}
\vbox{ ... some text ... \par\eatlines}
\end{verbatim}
or it can be inserted automatically
with \cs{everypar}; see~\cite{E1}.

In the macro \cs{eatlines}, the \cs{lastbox} is taken
from a vertical list. If the list is empty
the last box will test true on \cs{ifvoid}.
These boxes containing lines from a paragraph
are actually horizontal boxes: the test
\cs{ifhbox} applied to them would give a true
result.

%%%% end of input file [boxes]

%\InputFile:modes
%%%% this is input file [modes]
%\subject[hvmode]  Horizontal and \nl Vertical Mode
\endofchapter
\index{boxes|)}

\chapter{Horizontal and Vertical Mode}\label{hvmode}
\label{chap:hvmode}

At any point in its processing \TeX\ is in some \indexterm{mode}.
There are six modes, divided in three categories:
\begin{enumerate}
\item horizontal mode and restricted horizontal mode,
\item vertical mode and internal vertical mode, and
\item math mode and display math mode.
\end{enumerate}
The math modes will be treated elsewhere (see page~\pageref{math:modes}).
Here we shall look
at the horizontal and vertical modes, the kinds of objects
that can occur in the corresponding lists, and the
commands that are exclusive for one mode or the other. 

\label{cschap:vadjust}\label{cschap:showlists}
\begin{inventory}
\item [\cs{ifhmode}] 
      Test whether the current mode is (possibly restricted) horizontal mode.

\item [\cs{ifvmode}] 
      Test whether the current mode is (possibly internal) vertical mode.

\item [\cs{ifinner}] 
      Test whether the current mode is an internal mode.

\item [\cs{vadjust}] 
      Specify vertical material for the enclosing vertical list
      while in horizontal mode.

\item [\cs{showlists}] 
      Write to the log file the contents of the partial lists 
      currently being built in all modes.
\end{inventory}

%\point Horizontal and vertical mode
\section{Horizontal and vertical mode}

When not typesetting mathematics, \TeX\ is in horizontal
or vertical mode, building horizontal or vertical lists
respectively. Horizontal mode is typically used to
make lines of text; vertical mode is typically used
to stack the lines of a paragraph on top of each other.
Note that
these modes
are different from the internal states of \TeX's input processor
(see page~\pageref{input:states}).

%\spoint Horizontal mode
\subsection{Horizontal mode}

The main activity in \indextermbus{horizontal}{mode} is building lines of text.
Text on the page and text in a \cs{vbox} or \cs{vtop} is built in
horizontal mode (this might be called `paragraph mode');
if the text is in an \cs{hbox} there is only one line
of text, and the corresponding mode is the restricted
horizontal mode.

In horizontal mode all material is added to a horizontal list.
If this list is built in unrestricted horizontal mode, it
will later be broken into lines and added to the surrounding vertical list.

Each element of a \indextermbus{horizontal}{list} is one of the following:
\begin{itemize} \item a box (a character, ligature, \cs{vrule},
or a \gr{box}),
\item a discretionary break,
\item a whatsit (see Chapter~\ref{io}),
\item vertical material enclosed in \cs{mark},
\cs{vadjust}, or \cs{insert},
\item 
\mdqon
glue or leaders, a kern, a penalty, or a math-on/""off item.
\mdqoff
\end{itemize}
The items in the last point are all discardable.
\emph{Discardable items}\index{discardable items}
are called that, because they disappear in
a break. Breaking of horizontal
lists is discussed in Chapter~\ref{line:break}.

\subsection{Vertical mode}

\emph{Vertical mode}\index{mode!vertical}
can be used to stack items on top of one another.
Most of the time, these items are boxes 
containing the lines of paragraphs.

Stacking material can take place inside a 
vertical box, but the
items that are stacked can also 
appear by themselves on the page. In the latter case
\TeX\ is in vertical mode; in the former case, inside a
vertical box, \TeX\ operates in internal vertical mode.

In vertical mode all material is added to a vertical list.
If this list is built in external vertical mode, it
will later be broken when pages are formed.

Each element of a \indextermbus{vertical}{list} is one of the following:
\begin{itemize}
\item a box (a horizontal or vertical box or an \cs{hrule}),
\item a whatsit,
\item a mark,
\item glue or leaders, a kern, or a penalty.
\end{itemize}
The items in the last point are all discardable.
Breaking of vertical lists
is discussed in Chapter~\ref{page:break}.

There are a few exceptional conditions at the beginning
of  a vertical list: the value of \cs{prevdepth} is set
to \n{-1000pt}. Furthermore, no \cs{parskip} glue is added
at the top of an internal vertical list; 
at the top of the main vertical list (the top of the
`current page') no glue or other discardable items
are added, and \cs{topskip} glue is added when the
first box is placed on this list
(see Chapters \ref{page:shape} and~\ref{page:break}).

%\point Horizontal and vertical commands
\section{Horizontal and vertical commands}

Some commands are so intrinsically horizontal or vertical
in nature that they force \TeX\ to go into that mode, if
possible. A~command that forces \TeX\ into horizontal mode
is called a \gr{horizontal command}; similarly a command that
forces \TeX\ into vertical mode is called a
\gr{vertical command}.

However, not all transitions are possible:
\TeX\ can switch from both vertical modes to 
(unrestricted) horizontal mode and back
through horizontal and vertical commands, but no transitions
to or from restricted horizontal mode are possible
(other than by enclosing horizontal boxes in vertical boxes or
the other way around).
A~vertical command in restricted horizontal mode thus gives
an error; the \cs{par} command in restricted horizontal mode
has no effect.

The \indextermbus{horizontal}{commands} are the following:
\label{h:com:list}
\begin{itemize}
\item any \gr{letter}, \gr{otherchar}, \cs{char}, 
a control sequence defined by \cs{chardef}, or \cs{noboundary};
\item \cs{accent}, \cs{discretionary}, the discretionary
hyphen~\verb|\-| and control space~\verb|\|\n{\char32};
\item \cs{unhbox} and \cs{unhcopy};
\item \cs{vrule} and the
\gr{horizontal skip} commands
\cs{hskip}, \cs{hfil}, \cs{hfill}, \cs{hss}, and \cs{hfilneg};
\item \cs{valign};
\item math shift (\n\$).
\end{itemize}

The \indextermbus{vertical}{commands} are the following:
\label{v:com:list}
\begin{itemize}
\item \cs{unvbox} and \cs{unvcopy};
\item \cs{hrule} and the \gr{vertical skip} commands
 \cs{vskip}, \cs{vfil}, \cs{vfill}, \cs{vss}, and \cs{vfilneg};
\item \cs{halign};
\item \cs{end} and \cs{dump}.
\end{itemize}
Note that the vertical commands do not include \cs{par};
nor are \cs{indent} and \cs{noindent} horizontal commands.

The connection between boxes and modes is explored below;
see Chapter~\ref{rules} for more on the connection between
rules and modes.

\section{The internal modes}

The \indextermbus{restricted horizontal}{mode} and 
\indextermbus{internal vertical}{mode}
are those variants of horizontal mode and vertical mode
that hold inside an \cs{hbox} and \cs{vbox} (or \cs{vtop}
or \cs{vcenter}) respectively.
However, restricted horizontal mode is rather more
restricted in nature than internal vertical mode.
The third internal mode is non-display math mode
(see Chapter~\ref{math}).

%\spoint Restricted horizontal mode
\subsection{Restricted horizontal mode}

The main difference between restricted horizontal mode,
the mode in an \cs{hbox}, and unrestricted horizontal mode,
the mode in which paragraphs in vertical boxes
and on the page are built,
is that you cannot break out of restricted horizontal mode: 
\cs{par}~does nothing in this mode. 
Furthermore, a~\gram{vertical command} in restricted horizontal
mode gives an error. 
In unrestricted horizontal mode it would cause a
\cs{par} token to be inserted and vertical mode to be entered
(see also Chapter~\ref{par:end}).

%\spoint Internal vertical mode
\subsection{Internal vertical mode}

Internal vertical mode, the vertical mode inside
a~\cs{vbox}, is a lot like external vertical
mode, the mode in which pages are built.
A~\gram{horizontal command} in internal vertical mode,
for instance, is perfectly valid:
\TeX\ then starts building a paragraph in
unrestricted horizontal mode.

One difference is that the commands
\cs{unskip} and \cs{unkern} have no effect
in external vertical mode, and
\cs{lastbox} is always empty in external vertical mode.
See further pages \pageref{lastbox} and~\pageref{unskip}.

The entries of alignments (see Chapter~\ref{align}) are 
processed in internal modes: restricted horizontal mode
for the entries of an \cs{halign}, and internal vertical
mode for the entries of a~\cs{valign}.
The material in \cs{vadjust}   and \cs{insert} items
is also processed in internal vertical mode; furthermore,
\TeX\ enters this mode when processing the \cs{output} token list.

The commands \cs{end} and \cs{dump} (the latter exists only in \IniTeX)
are not allowed in
internal vertical mode; furthermore, \cs{dump} is not allowed
inside a group (see Chapter~\ref{TeXcomm}).


%\point[hvbox]  Boxes and modes
\section{Boxes and modes}
\label{hvbox}

There are horizontal and vertical boxes, and there is
horizontal and vertical mode. Not surprisingly, there is
a connection between the boxes and the modes.
One can ask about this connection in two ways.

%\spoint What box do you use in what mode?
\subsection{What box do you use in what mode?}

This is the wrong question. Both horizontal  and vertical boxes
can be used in both horizontal and vertical mode. 
Their placement is determined by the prevailing mode at that moment.

%\spoint What mode holds in what box?
\subsection{What mode holds in what box?}

This is the right question.
When an \cs{hbox} starts, \TeX\ is in restricted horizontal
mode. Thus everything in a horizontal box is lined up horizontally.

When a \cs{vbox} is started, \TeX\ is in internal vertical mode.
Boxes of both kinds and other items are then stacked
on top of each other.


%\spoint Mode-dependent behaviour of boxes
\subsection{Mode-dependent behaviour of boxes}

Any \gr{box} (see Chapter \ref{boxes} for the full definition)
can be used in horizontal, vertical, and math mode.
Unboxing commands, however, are specific for horizontal or vertical mode.
Both \cs{unhbox} and \cs{unhcopy} are \gr{horizontal command}s,
so they can make \TeX\ switch from vertical to horizontal
mode; 
both \cs{unvbox} and \cs{unvcopy} are \gr{vertical command}s,
so they can make \TeX\ switch from horizontal to vertical
mode.

In horizontal mode the \cs{spacefactor} is set to 1000
after a box has been placed. In vertical mode the
\cs{prevdepth} is set to the depth of the box placed.
Neither statement holds for
unboxing commands: after an \cs{unhbox} or \cs{unhcopy} the 
spacefactor is not altered, and after \cs{unvbox} or \cs{unvcopy}
the \cs{prevdepth} remains unchanged.
After all, these commands do not add a box,
but a piece of a~(horizontal or vertical) list.

The operations \cs{raise} and \cs{lower} can only be
applied to a box in horizontal mode; similarly, \cs{moveleft} and
\cs{moveright} can only be applied in vertical mode.


%\point Modes and glue
\section{Modes and glue}

Both in horizontal and vertical mode
\TeX\ can insert glue items the size of which is
determined by the preceding object in the list.

For horizontal mode the amount of glue that is inserted
for a space token depends on the \cs{spacefactor} of
the previous object in the list. This is treated
in Chapter~\ref{space}.

In vertical mode \TeX\ inserts glue to keep boxes at a certain
distance from each other. This glue is influenced by the
height of the current item and the depth of the previous one.
The depth of items is recorded in the \cs{prevdepth} parameter
(see Chapter~\ref{baseline}).

The two quantities \cs{prevdepth} 
and \cs{spacefactor} 
use the same internal register of \TeX. Thus the \cs{prevdepth}
can be used or asked only in vertical mode, and the \cs{spacefactor}
only in horizontal mode.

%\point[migrate] Migrating material
\section{Migrating material}
\label{migrate}

The three control sequences \cs{insert}, \cs{mark}, and \cs{vadjust}
can be given in a paragraph 
(the first two can also occur
in vertical mode) to specify \indexterm{migrating material}:
material that will wind up on the
surrounding vertical list rather than on the current list.
Note that this need not be 
the main vertical list: it can be a vertical box
containing a paragraph of text. In this case a \cs{mark}
or \cs{insert} command will not reach the page breaking algorithm.

When several migrating items are specified in a certain line
of text, their left-to-right order is preserved when they are
placed on the surrounding vertical list. These items are placed
directly after the horizontal box containing the line of text
in which they were specified: they come before any
penalty or glue items that are automatically inserted
(see page~\pageref{between:lines}).

%\spoint \cs{vadjust}
\subsection{\cs{vadjust}}

The command
\csidx{vadjust}%
\begin{disp}\cs{vadjust}\gr{filler}\lb\gr{vertical mode material}\rb\end{disp}
is only allowed in horizontal and math modes (but it is
not a \gr{horizontal command}).
Vertical mode material specified by \cs{vadjust} is moved from
the horizontal list in which the command is given
to the surrounding vertical list, directly after the box
in which it occurred.

In the current line
\vadjust{\setbox0=\hbox{$\bullet$\hskip1em}\ht0=0pt \dp0=0pt \llap{\box0}}
a \cs{vadjust} item was placed to put the bullet in the margin.


Any vertical material in a \cs{vadjust} item is processed
in internal vertical mode, even though it will wind up
on the main vertical list. For instance, the \cs{ifinner}
test is true in a \cs{vadjust}, and at the start
\mdqon
of the vertical material \cs{prevdepth}$=$""\n{-1000pt}.
\mdqoff

%\point Testing modes
\section{Testing modes}

The three conditionals \cs{ifhmode}, \cs{ifvmode}, and
\cs{ifinner} can distinguish between the four modes of
\TeX\ that are not math modes.
The \cs{ifinner} test is true if \TeX\ is in 
restricted horizontal mode or internal vertical mode
(or in non-display math mode).
Exceptional condition: during a \cs{write} \TeX\
is in a `no mode' state. The tests \cs{ifhmode},
\cs{ifvmode}, and \cs{ifmmode} are then all false.

Inspection of all current lists, including the `recent
contributions' (see Chapter~\ref{page:break}),
is possible through the command \csterm{showlists}\label{showlists}.
This command writes to the log file the contents of all
lists that are being built at the moment the command is given.

Consider the example
\begin{verbatim}
a\hfil\break b\par 
c\hfill\break d
\hbox{e\vbox{f\showlists
\end{verbatim}
Here the first paragraph has been broken into two lines, and
these have been added to the current page. The second paragraph
has not been concluded or broken into lines.

The log file shows the following. \TeX\ was busy
building a paragraph (starting with an indentation box
\n{20pt} wide):\begin{verbatim}
### horizontal mode entered at line 3
\hbox(0.0+0.0)x20.0
\tenrm f
spacefactor 1000
\end{verbatim}
This paragraph was inside a vertical box:\begin{verbatim}
### internal vertical mode entered at line 3
prevdepth ignored
\end{verbatim}
The vertical box was in  a horizontal box, 
\begin{verbatim}
### restricted horizontal mode entered at line 3
\tenrm e
spacefactor 1000
\end{verbatim}
which was part of
an as-yet unfinished paragraph:\begin{verbatim}
### horizontal mode entered at line 2
\hbox(0.0+0.0)x20.0
\tenrm c
\glue 0.0 plus 1.0fill
\penalty -10000
\tenrm d
etc.
spacefactor 1000
\end{verbatim}
Note how the infinite glue and the \cs{break} penalty
are still part of the horizontal list.

Finally, the first paragraph has been broken into lines and 
added to the current page:\begin{verbatim}
### vertical mode entered at line 0
### current page:
\glue(\topskip) 5.69446
\hbox(4.30554+0.0)x469.75499, glue set 444.75497fil
.\hbox(0.0+0.0)x20.0
.\tenrm a
.\glue 0.0 plus 1.0fil
.\penalty -10000
.\glue(\rightskip) 0.0
\penalty 300
\glue(\baselineskip) 5.05556
\hbox(6.94444+0.0)x469.75499, glue set 464.19943fil
.\tenrm b
.\penalty 10000
.\glue(\parfillskip) 0.0 plus 1.0fil
.\glue(\rightskip) 0.0
etc.
total height 22.0 plus 1.0
 goal height 643.20255
prevdepth 0.0
\end{verbatim}



%%%% end of input file [modes]

%\InputFile:number
%%%% this is input file [number]
%\subject[number]  Numbers
\endofchapter
\chapter{Numbers}\label{number}

In this chapter integers and their
denotations will be treated,
the conversions that are possible either way, 
allocation and use of \cs{count} registers, and
arithmetic with integers.

\label{cschap:number}\label{cschap:romannumeral}\label{cschap:count}\label{cschap:countdef}\label{cschap:newcount}\label{cschap:advance}\label{cschap:multiply}\label{cschap:divide}
\begin{inventory} 
\item [\cs{number}] 
      Convert a \gr{number} to decimal representation. 

\item [\cs{romannumeral}] 
      Convert a positive \gr{number} to lowercase roman representation.

\item [\cs{ifnum}] 
      Test relations between numbers.

\item [\cs{ifodd}] 
      Test whether a number is odd.

\item [\cs{ifcase}] 
      Enumerated case statement.


\item [\cs{count}] 
      Prefix for count registers. 

\item [\cs{countdef}] 
      Define a control sequence to be a synonym for
      a~\cs{count} register.

\item [\cs{newcount}] 
      Allocate an unused \cs{count} register. 

\item [\cs{advance}] 
      Arithmetic command to add to or subtract from 
      a~\gr{numeric variable}.

\item [\cs{multiply}] 
      Arithmetic command to multiply a \gr{numeric variable}.

\item [\cs{divide}] 
      Arithmetic command to divide a \gr{numeric variable}.

\end{inventory}


\section{Numbers and \gr{number}s}

An important part of the grammar of \TeX\ is the rigorous definition
of a \gr{number}\index{numbers}, the syntactic entity that
\TeX\ expects when semantically an \indexterm{integer} is
expected. This definition will take the largest part of this
chapter. Towards the end, \cs{count} registers, arithmetic, and tests
for numbers are discussed.

For clarity of discussion a distinction will be made
here between integers and numbers, 
but note that a \gr{number} can be both
an `integer' and a `number'.
`Integer'  will be taken to denote a mathematical number:
a~quantity that can be added or multiplied.
`Number' will be taken to refer to the printed representation
of an integer: a string of digits, in other words.

\section{Integers}

Quite a few different sorts of objects can function
as integers in \TeX. In this section they will all
be treated, accompanied by the relevant lines from
the grammar of \TeX.

First of all, an integer can be positive or negative:
\begin{disp}\gr{number} $\longrightarrow$ 
\gr{optional signs}\gr{unsigned number}\nl
\gr{optional signs} $\longrightarrow$ \gr{optional spaces}\nl
\indent $|$ \gr{optional signs}\gr{plus or minus}\gr{optional spaces}
\end{disp}

A first possibility for an unsigned integer is a string of digits
in decimal, octal, or hexadecimal notation.
Together with the alphabetic constants these will be named
here \gr{integer denotation}.
Another possibility for an integer is an
internal integer quantity, an \gr{internal integer};
together with the denotations these form the 
\gr{normal integer}s.
Lastly an integer can be a \gr{coerced integer}: 
an internal \gr{dimen} or \gr{glue}
quantity that is converted to an integer value.
\begin{disp}\gr{unsigned number} $\longrightarrow$ \gr{normal integer}
$|$ \gr{coerced integer}\nl
\gr{normal integer} $\longrightarrow$ \gr{integer denotation}
$|$ \gr{internal integer}\nl
\gr{coerced integer} $\longrightarrow$ \gr{internal dimen}
$|$ \gr{internal glue}\end{disp}
All of these possibilities will be treated in sequence.


%\spoint[int:denotation] Denotations: integers
\subsection{Denotations: integers}
\label{int:denotation}

Anything that looks like a number
can be used as a \gr{number}: thus \verb-42- is a number.
However, bases other than decimal can also be used:
\begin{verbatim}
'123
\end{verbatim}
is the octal notation for $1\times8^2+2\times8^1+3\times8^0=83$,
and
\begin{verbatim}
"123
\end{verbatim}
is the hexadecimal notation
for $1\times16^2+2\times16^1+3\times16^0=291$.
\begin{disp}\gr{integer denotation} $\longrightarrow$
\gr{integer constant}\gr{one optional space} \nl
\indent $|$ \n{\char`\'}\gr{octal constant}\gr{one optional space}\nl
\indent $|$ \n{\char`\"}\gr{hexadecimal constant}\gr{one optional space}
\end{disp}
The octal digits are \n0--\n7; a~digit \n8 or~\n9 following an
octal denotation is not part of the number: 
after
\begin{verbatim}
\count0='078
\end{verbatim}
the \cs{count0} will have the value~7, and the 
digit~\n8 is typeset.

The hexadecimal digits are \n0--\n9, \n A--\n F, where the \n A--\n F
can have category code 11\index{category!11}
or~12\index{category!12}. The latter has a somewhat far-fetched
justification: the characters resulting from a \cs{string} operation
have category code~12.  Lowercase \n a--\n f are not hexadecimal
digits, although (in \TeX3) they are used for hexadecimal notation in
the `circumflex method' for accessing all character codes (see
Chapter~\ref{char}).

%\spoint Denotations: characters
\subsection{Denotations: characters}

A character token is a pair consisting of a character code,
which is a~number in the range 0--255, 
and a category code. Both of these codes are accessible,
and can be used as a \gr{number}. 

The character code of a character token, or of a single letter
control sequence, is accessible through the left quote command:
both \verb-`a- and~\verb-`\a- denote the character code of~{\tt a},
which can be used as an integer. 
\begin{disp}\gr{integer denotation} $\longrightarrow$ 
\n{\char`\`}\gr{character token}\gr{one optional space}\end{disp}

In order to emphasize that accessing the character code is
in a sense using a denotation, the syntax of \TeX\ allows
an optional space after such a `character constant'.
The left quote must have category~12\index{category!12}.

%\spoint Internal integers
\subsection{Internal integers}

The class of \gr{internal integers} can
be split into five parts. 
The \gr{codename}s and \gr{special integer}s
will be treated separately below; furthermore, there are the following.

\begin{itemize} \item The contents of \cs{count} registers;
either explicitly used by writing for instance \cs{count23},
or by referring to such a register by means of a
control sequence 
that was defined by \cs{countdef}:
after
\begin{verbatim}
\countdef\MyCount=23
\end{verbatim}
\cs{MyCount} is called a
\gr{countdef token}, and it is fully equivalent to \cs{count23}.

\item All parameters of \TeX\ that hold integer values;
this includes obvious ones such as \cs{linepenalty}, but
also parameters such as
\cs{hyphenchar}\gr{font} and \cs{parshape}
(if a paragraph shape has been defined for $n$ lines,
using \cs{parshape} in the context of a \gr{number}
will yield this value of~$n$). 

\item\label{num:chardef} Tokens defined by \cs{chardef} or \cs{mathchardef}.
After
\begin{verbatim}
\chardef\foo=74
\end{verbatim}
the control sequence \cs{foo}
can be used on its own to mean \cs{char74}, but in a context
where a \gr{number} is wanted it can be used to denote~74:
\begin{verbatim}
\count\foo
\end{verbatim}
is equivalent to \verb=\count74=.
This fact is
exploited in the allocation routines for registers (see
Chapter~\ref{alloc}).

A control sequence thus defined by \cs{chardef} is called a 
\gr{chardef token}; if it is defined by \cs{mathchardef} it
is called a \gr{mathchardef token}.

\end{itemize}

Here is the full list:
\begin{disp}\gr{internal integer} $\longrightarrow$
\gr{integer parameter} \nl
\indent $|$ \gr{special integer} $|$ \cs{lastpenalty}\nl
\indent $|$ \gr{countdef token} $|$ \cs{count}\gr{8-bit number}\nl
\indent $|$ \gr{chardef token} $|$ \gr{mathchardef token}\nl
\indent $|$ \gr{codename}\gr{8-bit number}\nl
\indent $|$ \cs{hyphenchar}\gr{font} $|$ \cs{skewchar}\gr{font}
$|$ \cs{parshape}\nl
\indent $|$ \cs{inputlineno} $|$ \cs{badness}\nl
\gr{integer parameter} $\longrightarrow$\vadjust{\nobreak}
$|$ \cs{adjdemerits} $|$ \cs{binoppenalty}\nl
\indent $|$ \cs{brokenpenalty} $|$ \cs{clubpenalty} $|$ \cs{day}%
\nl
\indent $|$ \cs{defaulthyphenchar} $|$ \cs{defaultskewchar} \nl
\indent $|$ \cs{delimiterfactor} $|$ \cs{displaywidowpenalty} \nl
\indent $|$ \cs{doublehyphendemerits} $|$ \cs{endlinechar} 
        $|$ \cs{escapechar}\nl
\indent $|$ \cs{exhypenpenalty} $|$ \cs{fam} $|$ \cs{finalhyphendemerits}\nl
\indent $|$ \cs{floatingpenalty} $|$ \cs{globaldefs} $|$ \cs{hangafter}\nl
\indent $|$ \cs{hbadness} $|$ \cs{hyphenpenalty}
        $|$ \cs{interlinepenalty}\nl
\indent $|$ \cs{linepenalty} $|$ \cs{looseness} $|$ \cs{mag}\nl
\indent $|$ \cs{maxdeadcycles} $|$ \cs{month} \nl
\indent $|$ \cs{newlinechar} $|$ \cs{outputpenalty} $|$ \cs{pausing}\nl
\indent $|$ \cs{postdisplaypenalty} $|$ \cs{predisplaypenalty}\nl
\indent $|$ \cs{pretolerance} $|$ \cs{relpenalty} $|$ \cs{showboxbreadth}\nl
\indent $|$ \cs{showboxdepth} $|$ \cs{time} $|$ \cs{tolerance}\nl
\indent $|$ \cs{tracingcommands} $|$ \cs{tracinglostchars}
        $|$ \cs{tracingmacros}\nl
\indent $|$ \cs{tracingonline} $|$ \cs{tracingoutput}
        $|$ \cs{tracingpages}\nl
\indent $|$ \cs{tracingparagraphs} $|$ \cs{tracingrestores}
        $|$ \cs{tracingstats}\nl
\indent $|$ \cs{uchyph} $|$ \cs{vbadness} $|$ \cs{widowpenalty}
        $|$ \cs{year}
\end{disp}

Any internal integer can function as an \gr{internal unit},
which  \ldash preceded by \gr{optional spaces} \rdash  
can serve as a \gr{unit of measure}.
Examples of this are given in Chapter~\ref{glue}.

%\spoint Internal integers: other codes of a character
\subsection{Internal integers: other codes of a character}

The \cs{catcode} command
(which was  described in Chapter~\ref{mouth}) 
is a \gr{codename}, and like the other code names
it can be used as an integer.
\begin{disp}\gr{codename} $\longrightarrow$ \cs{catcode} $|$ \cs{mathcode}
$|$ \cs{uccode} $|$ \cs{lccode}\nl \indent $|$ \cs{sfcode} $|$ \cs{delcode}
\end{disp}
A~\gr{codename} has to be followed by an \gr{8-bit number}.

Uppercase and lowercase codes were treated in Chapter~\ref{char};
the \cs{sfcode} is treated
in Chapter~\ref{space};
the \cs{mathcode} and~\cs{delcode} are treated in
Chapter~\ref{mathchar}.

%\spoint[special:int:list] \gr{special integer}
\subsection{\gr{special integer}}
\label{special:int:list}

One of the subclasses of the internal integers is
that of the special integers.
\begin{disp}\gr{special integer} $\longrightarrow$
\cs{spacefactor} $|$ \cs{prevgraf}\nl
\indent $|$ \cs{deadcycles} $|$ \cs{insertpenalties}
\end{disp}
An assignment to any of these is called an \gr{intimate
assignment}, and is automatically global
(see Chapter~\ref{group}).

%\spoint Other internal quantities: coersion to integer
\subsection{Other internal quantities: coersion to integer}

\TeX\ provides a conversion between dimensions and integers:
if an integer is expected, a \gr{dimen} or \gr{glue} used
in that context is converted by taking its 
(natural) size
in scaled points\index{point!scaled}.
However, only \gr{internal dimen}s and \gr{internal glue}
can be used this way: no dimension or glue denotations
can be coerced to integers.

%\spoint Trailing spaces
\subsection{Trailing spaces}

The syntax of \TeX\ defines integer denotations (decimal,
octal, and hexadecimal) and `back-quoted' character tokens
to be followed by \gr{one optional space}. This means that
\TeX\ reads the token after the number, absorbing it
if it was a space token, and backing up if it was not.

Because \TeX's input processor goes into the state `skipping spaces'
after it has seen one space token, this
scanning behaviour implies that
integer denotations can be followed by
arbitrarily many space characters in the input. 
Also, a line end is admissible.
However, only one space token is allowed. 

%\point Numbers
\section{Numbers}

\TeX\ can perform an implicit \indextermsub{number}{conversion} from a string
of digits to an integer. Conversion from a representation
in decimal, octal, or hexadecimal notation was
treated above. The conversion the other way,
from an \gr{internal integer} to a printed representation,
has to be performed explicitly.
\TeX\ provides two conversion routines,
\csterm{number}, to decimal, and \csterm{romannumeral} to
\indexterm{roman numerals}.
The command \cs{number} is equivalent to \cs{the}
when followed by an internal integer.
These commands are performed in the expansion processor of \TeX, that is,
they are expanded whenever expansion has not been inhibited.

Both commands
yield a string of tokens with category code~12\index{category!12};
their argument is a~\gr{number}.
Thus \verb-\romannumeral51-, \verb-\romannumeral\year-,
and~\verb-\number\linepenalty- are valid, and so is \verb-\number13-.
Applying \cs{number} to a denotation has some uses:
it removes leading zeros and superfluous plus and minus signs.

A roman numeral is a string of lowercase `roman digits',
which are characters of category code~12.
The sequence\howto Uppercase roman numberals\par
\begin{verbatim}
\uppercase\expandafter{\romannumeral ...}
\end{verbatim}
gives uppercase roman numerals.
This works because \TeX\  expands
tokens in order to find the opening brace of the argument
of \verb=\uppercase=. If \cs{romannumeral} is applied to
a negative number, the result is simply empty.

%\point Integer registers
\section{Integer registers}

Integers can be stored in \csterm{count} registers:
\begin{Disp}\cs{count}\gr{8-bit number}\end{Disp}
is an \gr{integer variable} and an \gr{internal integer}.
As an integer variable it can be used in a 
\gr{variable assignment}:
\begin{Disp}\gr{variable assignment} $\longrightarrow$
     \gr{integer variable}\gr{equals}\gr{number} $|$ \dots\end{Disp}
As an internal integer it can be used as a \gr{number}:
\begin{Disp}\gr{number} $\rightarrow$ \gr{optional signs}\gr{internal integer}
     $|$ \dots
\end{Disp} 

Synonyms for \cs{count} registers can be introduced by the
\csterm{countdef} command in a \gr{shorthand definition}:
\begin{Disp}\cs{countdef}\gr{control sequence}\gr{equals}\gr{8-bit number}
\end{Disp} A control sequence defined this way
is called a \gr{countdef token}, and it serves as an
\gr{internal integer}.

The plain \TeX\ macro \csterm{newcount}
(which is declared \cs{outer}) uses the \cs{countdef} command
to allocate an unused \cs{count} register.
Counters 0--9 are scratch registers, like all
registers with numbers~0--9. 
However, counters 0--9 are used for page identification
in the \n{dvi} file (see Chapter~\ref{TeXcomm}), 
so they should be used as scratch
registers only inside a group.
Counters 10--22 are
used for plain \TeX's bookkeeping of allocation of registers.
Counter 255 is also scratch.

%\point Arithmetic
\section{Arithmetic}

The user can perform some \indexterm{arithmetic}
in \TeX, and
\TeX\ also performs arithmetic internally. User arithmetic
is concerned only with integers; the internal arithmetic
is mostly on fixed-point quantities, and only in the
case of glue setting on floating-point numbers.

%\spoint Arithmetic statements
\subsection{Arithmetic statements}

\TeX\ allows the user to
\csidx{advance}%
\csidx{multiply}%
\csidx{divide}%
perform some arithmetic on integers. The statement
\begin{Disp}\cs{advance}\gr{integer variable}\gr{optional \n{by}}%
     \gr{number}\end{Disp} 
adds the value of the \gr{number}
 \ldash which may be negative \rdash  to the \gr{integer variable}.
Similarly,
\begin{Disp}\cs{multiply}\gr{integer variable}\gr{optional \n{by}}%
     \gr{number}\end{Disp} 
multiplies the value of the \gr{integer variable}, and
\begin{Disp}\cs{divide}\gr{integer variable}\gr{optional \n{by}}%
     \gr{number}\end{Disp}
divides an \gr{integer variable}.

Multiplication and division are also available for any so-called
\gr{numeric variable}: their most general form is
\begin{disp}\cs{multiply}\gr{numeric variable}\gr{optional \n{by}}\gr{number}
\end{disp} where
\begin{disp}\gr{numeric variable} $\longrightarrow$
\gr{integer variable} $|$ \gr{dimen variable}\nl
\indent $|$ \gr{glue variable} $|$ \gr{muglue variable}\end{disp}

The result of an arithmetic operation should not exceed
$2^{30}$ in absolute value.

Division of integers yields an integer; that is, the remainder
is discarded. This raises the question of how rounding is performed
when either operand is negative. In such cases \TeX\ performs
the division with the absolute values of the operands, and
takes the negative of the result if exactly one operand was negative.

%\spoint Floating-point arithmetic
\subsection{Floating-point arithmetic}

Internally some \indextermsub{floating-point}{arithmetic}
is performed, namely
in the calculation of glue set ratios.
%and slant for accents!!
However, machine-dependent aspects of rounding cannot
influence the decision process of \TeX, so machine independence
of \TeX\ is guaranteed in this respect (sufficient
accuracy of rounding is enforced by the \n{Trip} test of~\cite{K:trip}).

%\spoint Fixed-point arithmetic
\subsection{Fixed-point arithmetic}

All fractional arithmetic in \TeX\ is performed in
\indextermsub{fixed-point}{arithmetic}
of `scaled integers': multiples of~$2^{-16}$.
This ensures the machine independence of \TeX.
Printed representations of scaled integers are rounded
to 5 decimal digits.

In ordinary 32-bit implementations of \TeX\ the largest
integers are $2^{31}-1$ in absolute size.
The user is not allowed to specify
dimensions larger in absolute size than~$2^{30}-1$: two
such dimensions can be added or subtracted without
overflow on a 32-bit system. This makes \n{maxdimen} equal to $2^{14}$
point; see section~\ref{sec:maxdimen}.

%\point Number testing
\section{Number testing}

The most general test for integers in \TeX\ is
\begin{disp}\cs{ifnum}\gr{number$_1$}\gr{relation}\gr{number$_2$}\end{disp}
where \gr{relation} is a~\n<, \n>, or~\n= character,
all of category~12\index{category!12}.

Distinguishing between odd and even numbers is done
by \begin{disp}\cs{ifodd}\gr{number}\end{disp}

A numeric case statement is provided by
\begin{disp}\cs{ifcase}\gr{number}\gr{case$_0$}\cs{or}\n{...}\cs{or}%
     \gr{case$_n$}\cs{else}\gr{other cases}\cs{fi}\end{disp}
where the \cs{else}-part is optional. The tokens for \gr{case$_i$}
are processed if the number turns out to be~$i$; other cases are
skipped, similarly to what ordinarily happens in conditionals
(see Chapter~\ref{if}).

%\point Remarks
\section{Remarks}

%\spoint Character constants
\subsection{Character constants}

In formats and macro collections numeric constants
are often needed. There are several ways to implement these
in \TeX. 

Firstly,
\begin{verbatim}
\newcount\SomeConstant \SomeConstant=42
\end{verbatim}
This is wasteful, as it uses up a \cs{count} register.

Secondly,
\begin{verbatim}
\def\SomeConstant{42}
\end{verbatim}
Better but accident prone: \TeX\ has to expand to find the number
 \ldash which in itself is a slight overhead \rdash  and may inadvertently
expand some tokens that should have been left alone.

Thirdly,
\begin{verbatim}
\chardef\SomeConstant=42
\end{verbatim}
This one is fine.
A \gr{chardef token} has the same status as a \cs{count}
register: both are \gr{internal integer}s.
Therefore a number defined this way can be used everywhere that
a \cs{count} register is feasible.
For large numbers the \cs{chardef} can be replaced by \cs{mathchardef},
which runs to \verb>"7FFF>${}=32\,767$.
Note that a \gr{mathchardef token} can usually only appear
in math mode, but in the context of a number it can appear anywhere.

%\spoint Expanding too far / how far
\subsection{Expanding too far / how far}

It is a common mistake to write pieces of \TeX\ code
where \TeX\ will inadvertently expand something because it
is trying to compose a number. For example:
\begin{verbatim}
\def\par{\endgraf\penalty200}
...\par \number\pageno
\end{verbatim}
Here the page number will be absorbed into the value of the penalty.

Now consider 
\begin{verbatim}
\newcount\midpenalty \midpenalty=200
\def\par{\endgraf\penalty\midpenalty}
...\par \number\pageno
\end{verbatim}
Here the page number is not scooped up by mistake:
\TeX\ is trying to locate a \gr{number} after the \cs{penalty},
and it finds a \gr{countdef token}. This is {\em not\/}
converted to a representation in digits, so there is never any
danger of the page number being touched.

It is possible to convert a \gr{countdef token} first to
a representation in digits before assigning it:
\begin{verbatim}
\penalty\number\midpenalty
\end{verbatim}
and this brings back again all previous problems of expansion.


%%%% end of input file [number]

%\InputFile:glue
%%%% this is input file [glue]
%\subject[glue]  Dimensions and Glue
\endofchapter
\chapter{Dimensions and Glue}\label{glue}

In \TeX\ vertical and horizontal white space
can have a possibility to adjust itself through `stretching' or
`shrinking'. An~adjustable white space is called \indexterm{glue}.
This chapter treats all technical concepts related to
dimensions and glue, and it explains how the badness of stretching or shrinking
a  certain amount is calculated.

\label{cschap:dimen}\label{cschap:dimendef}\label{cschap:newdimen}\label{cschap:skip}\label{cschap:skipdef}\label{cschap:newskip}\label{cschap:hskip}\label{cschap:hfil}\label{cschap:hfilneg}\label{cschap:hfill}\label{cschap:hss}\label{cschap:vskip}\label{cschap:vfil}\label{cschap:vfill}\label{cschap:vfilneg}\label{cschap:vss}\label{cschap:kern}\label{cschap:lastkern}\label{cschap:lastskip}\label{cschap:unkern}\label{cschap:unskip}\label{cschap:removelastskip}\label{cschap:advance2}\label{cschap:multiply2}\label{cschap:divide2}
\begin{inventory}
\item [\cs{dimen}] 
      Dimension register prefix.

\item [\cs{dimendef}] 
      Define a control sequence to be a synonym for
      a~\cs{dimen} register.

\item [\cs{newdimen}] 
      Allocate an unused dimen register. 

\item [\cs{skip}] 
      Skip register prefix.

\item [\cs{skipdef}] 
      Define a control sequence to be a synonym for
      a~\cs{skip} register.

\item [\cs{newskip}]
      Allocate an unused skip register.

\item [\cs{ifdim}] 
      Compare two dimensions. 

\item [\cs{hskip}]  
      Insert in horizontal mode a glue item.

\item [\cs{hfil}]
      Equivalent to
      \verb-\hskip 0cm plus 1fil-.

\item [\cs{hfilneg}]
      Equivalent to
      \verb/\hskip 0cm plus -1fil/.

\item [\cs{hfill}]
      Equivalent to
      \verb-\hskip 0cm plus 1fill-.

\item [\cs{hss}]
      Equivalent to 
      \verb-\hskip 0cm plus 1fil minus 1fil-.

\item [\cs{vskip}]  
      Insert in vertical mode a glue item.

\item [\cs{vfil}]
      Equivalent to 
      \verb-\vskip 0cm plus 1fil-.

\item [\cs{vfill}]
      Equivalent to 
      \verb-\vskip 0cm plus 1fill-.

\item [\cs{vfilneg}]
      Equivalent to 
      \verb-\vskip 0cm minus 1fil-.

\item [\cs{vss}]
      Equivalent to 
      \verb-\vskip 0cm plus 1fil minus 1fil-.

\item [\cs{kern}]  
      Add a kern item to the current horizontal or vertical list.

\item [\cs{lastkern}] 
      If the last item on the current list was a kern, the size of it.

\item [\cs{lastskip}] 
      If the last item on the current list was a~glue, the size of it.

\item [\cs{unkern}] 
      If the last item of the current list was a~kern, remove it.

\item [\cs{unskip}] 
      If the last item of the current list was a~glue, remove it.

\item [\cs{removelastskip}]
      Macro to append the negative of the \cs{lastskip}.

\item [\cs{advance}] 
      Arithmetic command to add to or subtract from
      a~\gr{numeric variable}.

\item [\cs{multiply}] 
      Arithmetic command to multiply a~\gr{numeric variable}.

\item [\cs{divide}] 
      Arithmetic command to divide a~\gr{numeric variable}.

\end{inventory}



%\point Definition of \gr{glue} and \gr{dimen}
\section{Definition of \gr{glue} and \gr{dimen}}

This section gives
the syntax of the quantities
\gr{dimen} and \gr{glue}. 
In the next section the practical aspects of glue are treated.

Unfortunately the terminology for glue is slightly confusing.
The syntactical quantity~\gr{glue} is a dimension (a distance) with
\mdqon
possibly a stretch and/""or shrink component.
\mdqoff
In order to add a glob of `glue' (a white space) to a list one has to
let a \gr{glue} be preceded by commands such as \cs{vskip}.


%\spoint Definition of dimensions
\subsection{Definition of dimensions}

A~\gr{dimen} is what \TeX\ expects to see when
it needs to indicate a dimension; it can be positive or negative.
\begin{disp}\gr{dimen} $\longrightarrow$ \gr{optional signs}%
     \gr{unsigned dimen}\end{disp}
The unsigned part of a \gr{dimen} can be
\begin{disp}\gr{unsigned dimen} $\longrightarrow$ \gr{normal dimen}
     $|$ \gr{coerced dimen}\nl
     \gr{normal dimen} $\longrightarrow$ \gr{internal dimen}
     $|$ \gr{factor}\gr{unit of measure}\nl
     \gr{coerced dimen} $\longrightarrow$ \gr{internal glue}
     \end{disp}
That is, we have the following three cases:
\begin{itemize} \item an \gr{internal dimen}; this is
 any register or parameter of \TeX\ that has a \gr{dimen} value:
 \begin{disp}\PopIndentLevel\gr{internal dimen} $\longrightarrow$
      \gr{dimen parameter}\nl
      \indent $|$ \gr{special dimen} $|$ \cs{lastkern}\nl
      \indent $|$ \gr{dimendef token} $|$ \cs{dimen}\gr{8-bit number}\nl
      \indent $|$ \cs{fontdimen}\gr{number}\gr{font}\nl
      \indent $|$ \gr{box dimension}\gr{8-bit number}\nl
      \gr{dimen parameter} $\longrightarrow$ \cs{boxmaxdepth}\nl
      \indent $|$ \cs{delimitershortfall} $|$ \cs{displayindent}\nl
      \indent $|$ \cs{displaywidth} $|$ \cs{hangindent}\nl
      \indent $|$ \cs{hfuzz} $|$ \cs{hoffset} $|$ \cs{hsize}\nl
      \indent $|$ \cs{lineskiplimit} $|$ \cs{mathsurround}\nl
      \indent $|$ \cs{maxdepth} $|$ \cs{nulldelimiterspace}\nl
      \indent $|$ \cs{overfullrule} $|$ \cs{parindent}\nl
      \indent $|$ \cs{predisplaysize} $|$ \cs{scriptspace}\nl
      \indent $|$ \cs{splitmaxdepth} $|$ \cs{vfuzz}\nl
      \indent $|$ \cs{voffset} $|$ \cs{vsize}
 \end{disp}
\item  a dimension denotation, 
 consisting of \gr{factor}\gr{unit of measure},
 for example \verb>0.7\vsize>; or
\item an \gr{internal glue} (see below) 
 coerced to a dimension by omitting
 the stretch and shrink components, for example \cs{parfillskip}.
\end{itemize}

A dimension denotation is a somewhat complicated entity:
\begin{itemize} \item a \gr{factor} is an integer denotation,
 a decimal constant denotation (a number with an integral and
 a fractional part),
 or an \gr{internal integer}
 \begin{disp}\PopIndentLevel
      \gr{factor} $\longrightarrow$ \gr{normal integer} 
      $|$ \gr{decimal constant}\nl
      \gr{normal integer} $\longrightarrow$ \gr{integer denotation}\nl
      \indent $|$ \gr{internal integer}\nl
      \gr{decimal constant} $\longrightarrow$ \n{.$_{12}$}
      $|$ \n{,$_{12}$}\nl
      \indent $|$ \gr{digit}\gr{decimal constant}\nl
      \indent $|$ \gr{decimal constant}\gr{digit}
 \end{disp}
 An internal integer is a parameter that is `really' an
\alt
 integer (for instance, \cs{count0}), and not coerced from a dimension or glue.
 See Chapter~\ref{number}
 for the definition of various kinds of integers.
\item a \gr{unit of measure} can be 
 a \gr{physical unit}, that is, an ordinary unit such as~\n{cm} 
 (possibly preceded by \n{true}),
 an internal unit such as~\n{em}, but also an \gr{internal integer}
 (by conversion to scaled points\index{point!scaled}),
 an \gr{internal dimen}, or an \gr{internal glue}.
 \begin{disp}\PopIndentLevel
      \gr{unit of measure} $\longrightarrow$
      \gr{optional spaces}\gr{internal unit}\nl
      \indent $|$ 
      \gr{optional \n{true}}\gr{physical unit}\gr{one optional space}\nl 
      \gr{internal unit} $\longrightarrow$ 
      \n{em}\gr{one optional space}\nl
      \indent $|$ \n{ex}\gr{one optional space}
              $|$ \gr{internal integer}\nl
      \indent $|$ \gr{internal dimen} $|$ \gr{internal glue}
      \end{disp}
\end{itemize}

Some \gr{dimen}s are called \gr{special dimen}s:\label{special:dimen:list}
\begin{disp}\gr{special dimen} $\longrightarrow$ \cs{prevdepth}\nl
     \indent $|$ \cs{pagegoal} $|$ \cs{pagetotal} $|$ \cs{pagestretch}\nl
     \indent $|$ \cs{pagefilstretch} $|$ \cs{pagefillstretch}\nl
     \indent $|$ \cs{pagefilllstretch} $|$ \cs{pageshrink} $|$ \cs{pagedepth}
     \end{disp}
An assignment to any of these is
called an \gr{intimate assignment}, and it is automatically
global (see Chapter~\ref{group}). The meaning of these 
dimensions is explained in Chapter \ref{page:break}, with the
exception of \cs{prevdepth} which is treated in
Chapter~\ref{baseline}.

%\spoint Definition of glue
\subsection{Definition of glue}

A \gr{glue} is either some form of glue variable, or
a glue denotation with explicitly indicated stretch and
shrink. Specifically,
\begin{disp}\gr{glue} $\longrightarrow$ \gr{optional signs}\gr{internal glue}
     $|$ \gr{dimen}\gr{stretch}\gr{shrink}\nl
     \gr{internal glue} $\longrightarrow$ \gr{glue parameter}
     $|$ \cs{lastskip}\nl 
     \indent $|$ \gr{skipdef token} $|$ \cs{skip}\gr{8-bit number}\nl
     \gr{glue parameter} $\longrightarrow$ \cs{abovedisplayshortskip}\nl
     \indent $|$ \cs{abovedisplayskip} $|$ \cs{baselineskip}\nl
     \indent $|$ \cs{belowdisplayshortskip} $|$ \cs{belowdisplayskip}\nl
     \indent $|$ \cs{leftskip} $|$ \cs{lineskip} $|$ \cs{parfillskip}
             $|$ \cs{parskip}\nl
     \indent $|$ \cs{rightskip} $|$ \cs{spaceskip}
             $|$ \cs{splittopskip} $|$ \cs{tabskip}\nl
     \indent $|$ \cs{topskip} $|$ \cs{xspaceskip}
\end{disp}
The stretch and shrink components in a glue denotation
are optional, but when both are specified they have to
be given in sequence; they are defined as
\begin{disp}
\gr{stretch} $\longrightarrow$ \n{plus} \gr{dimen}
      $|$ \n{plus}\gr{fil dimen} $|$ \gr{optional spaces}\nl
\gr{shrink} $\longrightarrow$ \n{minus} \gr{dimen}
      $|$ \n{minus}\gr{fil dimen} $|$ \gr{optional spaces}\nl
\gr{fil dimen} $\longrightarrow$ \gr{optional signs}\gr{factor}%
     \gr{fil unit}\gr{optional spaces}\nl
\gr{fil unit} $\longrightarrow$ \n{ $|$ fil $|$ fill $|$ filll}
\end{disp}

The actual definition of \gr{fil unit} is recursive
(see Chapter~\ref{gramm}), but these are the only valid
possibilities.

%\spoint Conversion of \gr{glue} to \gr{dimen}
\subsection{Conversion of \gr{glue} to \gr{dimen}}

The grammar rule
\begin{disp}\gr{dimen} $\longrightarrow$
     \gr{factor}\gr{unit of measure}
\end{disp}
has some noteworthy consequences, caused by the fact
that a \gr{unit of measure} need not look like a `unit of measure'
at all (see the list above).

For instance, from this definition we conclude that the statement
\begin{verbatim}
\dimen0=\lastpenalty\lastpenalty
\end{verbatim}
is
syntactically correct because \cs{lastpenalty} can function
both as an integer and as \gr{unit of measure} by taking
its value in scaled points.
After \verb>\penalty8> the \cs{dimen0} thus defined will
have a size of~\n{64sp}.

More importantly, consider the case where the \gr{unit of measure} is
an \gr{internal glue}, that is, any sort of glue parameter.
Prefixing such a glue with a number (the \gr{factor})
makes it a valid \gr{dimen} specification.
Thus
\begin{verbatim}
\skip0=\skip1
\end{verbatim}
is very different
from
\begin{verbatim}
\skip0=1\skip1
\end{verbatim}
The first statement makes
\cs{skip0} equal to \cs{skip1}, the second converts
the \cs{skip1} to a \gr{dimen} before assigning it.
In other words, the \cs{skip0} defined by the second statement
has no stretch or shrink.


%\spoint Registers for \cs{dimen} and \cs{skip}
\subsection{Registers for \cs{dimen} and \cs{skip}}

\TeX\ has registers for storing \gr{dimen} and \gr{glue}
values: the \csterm{dimen} and \csterm{skip} registers
respectively. These are accessible by the expressions
\begin{disp}\cs{dimen}\gr{number}\end{disp} and
\begin{disp}\cs{skip}\gr{number}\end{disp}
As with all registers of \TeX, these registers are
numbered~0--255.

Synonyms for registers can be made with the \csterm{dimendef} and
\csterm{skipdef} commands. Their syntax is
\begin{Disp}\cs{dimendef}\gr{control sequence}\gr{equals}\gr{8-bit number}
\end{Disp}
and 
\begin{Disp}\cs{skipdef}\gr{control sequence}\gr{equals}\gr{8-bit number}\end{Disp}
For example, after \verb-\skipdef\foo=13- using \cs{foo}
is equivalent to using \cs{skip13}.

Macros \csterm{newdimen} and \csterm{newskip} exist in plain \TeX
for allocating an unused dimen or skip register.
These macros are defined to be \cs{outer} in the plain format.

\subsection{Arithmetic: addition}

As for integer variables, arithmetic operations exist for
\csidx{advance}%
\indextermbus{arithmetic on}{glue}\index{arithmetic! on glue|see{glue, arithmetic on}}:
dimen, glue, and muglue (mathematical glue; see page~\pageref{muglue})
variables.

The expressions
\begin{Disp}\cs{advance}\gr{dimen variable}\gr{optional \n{by}}%
     \gr{dimen}\nl
     \cs{advance}\gr{glue variable}\gr{optional \n{by}}%
     \gr{glue}\nl
     \cs{advance}\gr{muglue variable}\gr{optional \n{by}}%
     \gr{muglue}\end{Disp}
add to the size of a dimen, glue, or muglue.

Advancing a \gr{glue variable} by \gr{glue} is done by
adding the natural sizes, and the stretch and shrink components.
Because \TeX\ converts between \gr{glue} and \gr{dimen},
it is possible to write for instance
\begin{verbatim}
\advance\skip1 by \dimen1
\end{verbatim}
or
\begin{verbatim}
\advance\dimen1 by \skip1
\end{verbatim}
In the first case  \cs{dimen1} is coerced to \gr{glue} without
stretch or shrink; in the second case the \cs{skip1} is coerced
to a \gr{dimen} by taking its natural size.

%\spoint Arithmetic: multiplication and division
\subsection{Arithmetic: multiplication and division}

Multiplication and division operations exist for glue
\csidx{multiply}%
\csidx{divide}%
and dimensions. One may for instance write
\begin{verbatim}
\multiply\skip1 by 2
\end{verbatim}
which multiplies the natural size, and the stretch and shrink
components of \cs{skip1} by~2.

The second operand of a \cs{multiply} or \cs{divide}
operation can only be a \gr{number}, that is, an integer.
Introducing the notion of \gr{numeric variable}:
\begin{disp}\gr{numeric variable} $\longrightarrow$ \gr{integer variable}
     $|$ \gr{dimen variable} \nl
     \indent $|$ \gr{glue variable} $|$ \gr{muglue variable}\end{disp}
these operations take the form
\begin{Disp}\cs{multiply}\gr{numeric variable}\gr{optional \n{by}}%
\gr{number}\end{Disp} 
and
\begin{Disp}\cs{divide}\gr{numeric variable}\gr{optional \n{by}}%
\gr{number}\end{Disp}

Glue and dimen can be multiplied by 
non-integer quantities:
\begin{verbatim}
\skip1=2.5\skip2
\dimen1=.78\dimen2
\end{verbatim}
However, in the first line the \cs{skip2} is first coerced
to a \gr{dimen} value by omitting its stretch and shrink.

%\point More about dimensions
\section{More about dimensions}

%\spoint Units of measurement
\subsection{Units of measurement}

In \TeX\ dimensions can be indicated in the following
\indexterm{units of measurement}:
\begin{description} 
\item [centimetre]
    denoted \n{cm} or 
\item [millimetre]
	denoted \n{mm}; these are SI~units ({\italic Syst\`eme International
	d'Unit\'es}, the
	international system of standard units of measurements).
\item [inch]
\n{in}; more common in the Anglo-American world.
One inch is 2.54~centimetres.
\item [pica]
    denoted \n{pc}; one pica is 12~points.
\item [point]
    denoted \n{pt}\index{point}; the common system
for Anglo-American printers. One inch is 72.27 points.
\item [didot point]
    denoted \n{dd}. This is the common system for continental European printers;
    1157~didot points are 1238~points.
\item [cicero]
    denoted \n{cc}; one cicero is 12~didot points.
\item [big point]
    denoted \n{bp}; one inch is 72 big points.
\item [scaled point]
    denoted \n{sp}; this is the smallest unit in \TeX, and all measurements
    are integral multiples of one scaled point\index{point!scaled}.
    There are $2^{16}=65\,536$ scaled points in a~point.
\end{description}

Decimal fractions can be written using both the
Anglo-American system with the decimal point
(for example, \n{1in}=\n{72.27pt})
and the continental European system with a decimal
comma; \n{1in}=\n{72,27pt}.

Internally \TeX\ works with multiples of a smallest 
dimension: the  scaled point\index{point!scaled|textbf}.
Dimensions larger (in absolute value) than $2^{30}-1$\n{sp},
which is about 5.75~metres or 18.9~feet, are illegal.

Both the pica system and the didot system are of French
origin: in 1737 the type founder Pierre Simon Fournier
introduced typographical points based on the French foot.
Although at first he introduced a system based on lines and
points, he later took the point as unit:
there are 72 points in an inch,
which is one-twelfth of a foot. 
About 1770 another founder, Fran\c{c}ois Ambroise Didot, introduced
points based on the more common, and slightly longer,
`pied du roi'.

%\spoint Dimension testing
\subsection{Dimension testing}

Dimensions and natural sizes of glue can be compared with
the \cs{ifdim} test. This takes the form
\begin{disp}\cs{ifdim}\gr{dimen$_1$}\gr{relation}\gr{dimen$_2$}\end{disp}
where the relation can be an \n>, \n<, or~\n= token, 
all of category~12\index{category!12}.

%\spoint Defined dimensions
\subsection{Defined dimensions}
\label{sec:maxdimen}

\begin{inventory}
\item [\cs{z@}]
 \n{0pt}

\item [\cs{maxdimen}] 
      \n{16383.99999pt}; the largest legal dimension, about $5.75$ meters.
\end{inventory}

These \gr{dimen}s are predefined in the plain format;
for instance
\begin{verbatim}
\newdimen\z@ \z@=0pt
\end{verbatim}
Using such abbreviations for commonly used dimensions
has at least two advantages. First of all it saves main memory
if such a dimension occurs in a macro: a control sequence
is one token, whereas a string such as \n{0pt} takes three.
Secondly, it saves time in processing, as \TeX\ does not need
to perform conversions to arrive at the correct type of
object.

Control sequences such as \cs{z@}
are only available to a user who changes the
category code of the `at' sign. Ordinarily, these control sequences
appear only in the macros defined in packages such as the
plain format.

%\point More about glue
\section{More about glue}

Glue items can be added to a vertical list with one of the
\alt
commands \csterm{vskip}\gr{glue}, \cs{vfil}, \cs{vfill}, \cs{vss} or
\cs{vfilneg}; 
glue items can be added to a horizontal list with one of the
commands \csterm{hskip}\gr{glue}, \cs{hfil}, \cs{hfill}, \cs{hss} or
\cs{hfilneg}. We will now treat the properties of glue.

%\spoint Stretch and shrink
\subsection{Stretch and shrink}

In the syntax given above, \gr{glue} was defined as having
\begin{itemize}
\item a `natural size', which is a \gr{dimen}, and optionally
\item a \indexterm{stretch} and \indexterm{shrink}
  component\indexterm{stretch}\indexterm{shrink}
  \index{glue!stretch component of}\index{glue!shrink
    component of} built out of a \gr{fil dimen}.
\end{itemize}

Each list that \TeX\ builds has amounts of stretch and shrink
(possibly zero),
which are the sum of the
stretch and shrink components of individual pieces of glue in the list. 
Stretch and shrink are used if the context in which the list
appears requires it to assume a size that is different from
its natural size.

There is an important difference in behaviour between stretch
and shrink components when they are finite \ldash that is,
when the \gr{fildimen} is not \n{fil}(\n{l}(\n{l})). 
A~finite amount of shrink is indeed the maximum shrink
that \TeX\ will take: the amount of glue specified
as
\begin{verbatim}
5pt minus 3pt
\end{verbatim}
can shrink to \n{2pt}, but not further.
In contrast to this, a finite amount of stretch 
can be stretched arbitrarily far. 
Such arbitrary stretching
has a large `badness', however.
Badness calculation is treated below.

\begin{itemize}
\item The sequence with natural size \n{20pt}
\begin{verbatim}
\hskip 10pt plus 2pt \hskip 10pt plus 3pt
\end{verbatim}
has \n{5pt} of stretch, but it has no shrink.
\item In
\begin{verbatim}
\hskip 10pt minus 2pt \hskip 10pt plus 3pt
\end{verbatim}
there is \n{3pt} of stretch, and \n{2pt} of shrink,
so its minimal size is~\n{18pt}. 
\item 
Positive shrink is not the same as negative stretch:
\begin{verbatim}
\hskip 10pt plus -2pt \hskip 10pt plus 3pt
\end{verbatim}
looks a lot like the previous example, but it cannot
be shrunk as there are no \hbox{\n{minus}\gr{dimen}}
specifications. It does have \n{1pt} of stretch, however.
\item 
The command \csterm{hfilneg} has negative stretch \n{plus -1fil},
so it can be used to cancel positive fill stretch that was introduced
by another glue item.
\item %This is another example of negative amounts of shrink and stretch.
It is not possible to stretch
glue (in the informal sense) by shrinking it (in the technical
sense):
\begin{verbatim}
\hbox to 5cm{a\hskip 0cm minus -1fil}
\end{verbatim}
is an underfull box, because \TeX\ looks for a \n{plus}~\gr{dimen}
specification when it needs to stretch the contents.
\item
Finally,
\begin{verbatim}
\hskip 10pt plus -3pt \hskip 10pt plus 3pt
\end{verbatim}
can neither stretch nor shrink.
The fact that there is only stretch
available means that the sequence cannot
shrink. However, the stretch components cancel out: the 
total stretch is zero. Another way of looking at this
is to consider that for each point that the second glue item would
stretch, the first one would `stretch back' one point.
\end{itemize}

Any amount of infinite stretch or shrink overpowers all
finite stretch or shrink available:
\begin{verbatim}
\hbox to 5cm{\hskip 0cm plus 16384pt 
              text\hskip 0cm plus 0.0001fil}
\end{verbatim}
has the \n{text} at the extreme left of the box.
There are three orders of `infinity', each  one infinitely
stronger than the previous one:
\begin{verbatim}
\hbox to 5cm{\hskip 0cm plus 16384fil
              text\hskip 0cm plus 0.0001fill}
\end{verbatim}
and
\begin{verbatim}
\hbox to 5cm{\hskip 0cm plus 16384fill
              text\hskip 0cm plus 0.0001filll}
\end{verbatim}
both have the \n{text} at the left end of the box.

The available infinities are:
\begin{itemize}
\item \csterm{hfil}:
      Equivalent to\csidx{hfil}
      \verb-\hskip 0cm plus 1fil-.
\item \csterm{hfilneg}:
      Equivalent to\csidx{hfilneg} 
      \verb/\hskip 0cm plus -1fil/.
\item \csterm{hfill}:
      Equivalent to
      \verb-\hskip 0cm plus 1fill-.\csidx{hfill} 
\item \csterm{hss}:
      Equivalent to 
      \verb-\hskip 0cm plus 1fil minus 1fil-.\csidx{hss} 
\item \csterm{vfil}:
      Equivalent to 
      \verb-\vskip 0cm plus 1fil-.\csidx{vfil}
\item \csterm{vfill}:
      Equivalent to 
      \verb-\vskip 0cm plus 1fill-.\csidx{vfill} 
\item \csterm{vfilneg}:
      Equivalent to 
      \verb-\vskip 0cm minus 1fil-.\csidx{vfilneg} 
\item \csterm{vss}:
      Equivalent to 
      \verb-\vskip 0cm plus 1fil minus 1fil-.\csidx{vss} 
\end{itemize}

%\spoint Glue setting
\subsection{Glue setting}

In the process of \indextermsub{glue}{setting}, the desired width (or height)
of a box is compared with the natural dimension of its contents,
which is the sum of all natural dimensions of boxes and globs of glue.
If the two differ, any available stretchability or shrinkability is used
to bridge the gap.
To attain the desired dimension of the box
only the glue of the highest available order is set:
each piece of glue of that order is stretched or shrunk by the
same ratio.

For example, in
\begin{verbatim}
\hbox to 6pt{\hskip 0pt plus 3pt \hskip 0pt plus 9pt}
\end{verbatim}
the natural size of the box is~\n{0pt}, and
the total stretch is~\n{12pt}. In order to obtain a box
of~\n{6pt} each glue item is set with a stretch ratio
of~$1/2$. Thus the result is equivalent to
\begin{verbatim}
\hbox {\hskip 1.5pt \hskip 4.5pt}
\end{verbatim}
Only the highest order of stretch or shrink is used:
in
\begin{verbatim}
\hbox to 6pt{\hskip 0pt plus 1fil \hskip 0pt plus 9pt}
\end{verbatim}
the second glue  will assume its natural size of~\n{0pt},
and only the first   glue will be stretched.

\TeX\ will never exceed the maximum value of a finite
amount of shrink.
A~box that cannot be shrunk enough is called `overfull'.
Finite stretchability can be exceeded to provide an
escape in difficult situations; however, \TeX\ is likely 
to give an \verb-Underfull \hbox- message about this
(see page~\pageref{over/underfull}).
For an example of infinite shrink see page~\pageref{rlap}.

%\spoint Badness
\subsection{Badness}

When stretching or shrinking a list \TeX\ calculates 
\indexterm{badness}\index{badness! calculation}
badness based on the
ratio between actual stretch and the amount of stretch
present in the line. See Chapter~\ref{line:break}
for the application  of badness to the paragraph algorithm.

The formula for badness of a list that is stretched (shrunk) is
\label{bad:form}\message{Check roman min}
\begin{disp} $\displaystyle b=\hbox{min}\left(10\,000,
100\times \left({\hbox{actual amount stretched (shrunk)}
\over\hbox{possible amount of stretch (shrink)}}\right)^3\right)$\end{disp}
In reality \TeX\ uses a slightly different formula that is
easier to calculate, but behaves the same. Since glue setting is
one of the main activities of \TeX, this must be performed
as efficiently as possible.

This formula lets the badness be a reasonably small number
if the glue set ratio (the fraction in the above expression)
is reasonably small, but will let it grow rapidly once
the ratio is more than~1. Badness is infinite if the
glue would have to shrink more than the allotted amount;
stretching glue beyond its maximum is possible, so this
provides an  escape for very difficult lines of text or pages.

In \TeX3, the \cs{badness} parameter records the badness
of the most recently formed box.

%\spoint Glue and breaking
\subsection{Glue and breaking}

\TeX\ can break lines and pages in several kinds of places.
One of these places is before a glue item. 
The glue is then discarded. For line breaks this is treated
in Chapter~\ref{line:break}, 
for page breaks see Chapter~\ref{page:break}.

There are two macros in plain \TeX, \csterm{hglue} and \csterm{vglue},
that give non-disappearing glue in horizontal and
vertical mode respectively. For the horizontal case this is
accomplished by
placing:
\begin{verbatim}
\vrule width 0pt \nobreak \hskip ...
\end{verbatim}
Because \TeX\ breaks at the front end of glue,
this glue will always stay attached to the rule,
and will therefore never disappear.
The actual macro definitions are somewhat more complicated,
because they take care to preserve the \cs{spacefactor} and the
\cs{prevdepth}.

%\spoint \cs{kern}
\subsection{\cs{kern}}

The \csterm{kern} command specifies
a~kern item in whatever mode \TeX\ is currently
in. A~kern item is much like a glue item without
stretch or shrink.
It differs from glue in that it is
in general not a legal breakpoint. Thus in
\begin{verbatim}
.. text .. \hbox{a}\kern0pt\hbox{b}
\end{verbatim}
\TeX\ will not break lines in between the boxes; in
\begin{verbatim}
.. text .. \hbox{a}\hskip0pt\hbox{b}
\end{verbatim}
a line can be broken in between the boxes.

However, if a kern is followed by glue, \TeX\ can break at the
kern (provided that it is not in math mode). 
In horizontal mode
both the kern and the glue then disappear in the break.
In vertical mode they are discarded when they are moved to
the (empty) current page after the material before
the break has been disposed of by the output routine 
(see Chapter~\ref{page:break}).

%\spoint Glue and modes
\subsection{Glue and modes}

All horizontal skip commands are \gr{horizontal command}s and
all vertical skip commands are \gr{vertical commands}s.
This means that, for instance, an \cs{hskip} command
makes \TeX\ start a paragraph if it is given in vertical mode.
The \cs{kern} command can be given in both modes.

%\spoint The last  glue item in a list: backspacing
\subsection{The last  glue item in a list: backspacing}

The last glue item in a list can be measured, and
it can be removed in all modes but external vertical mode.
The internal variables
\csterm{lastskip} and  \csterm{lastkern} can be used
to measure the last glob of glue in all modes;
if the last glue was not a skip or kern respectively
they give~\n{0pt}.
In math mode the \cs{lastskip}
functions as \gr{internal muglue}, but in general
it classifies as \gr{internal glue}.
The \cs{lastskip} and \cs{lastkern}
are also \n{0pt} if that was the size of the last glue or
kern item on the list.

The operations\label{unskip}
\csterm{unskip} and \csterm{unkern} remove the last item of a list,
if this is a glue or kern respectively. They have no effect
in external vertical mode; in that case the
best substitute is 
\verb=\vskip-\lastskip= 
and~\verb=\kern-\lastkern=.

In the process of paragraph building \TeX\ itself performs
an important \cs{unskip}: a~paragraph ending with a
white line will have a space token inserted by \TeX's input processor.
This is removed by an \cs{unskip} before the \cs{parfillskip} glue
(see Chapter~\ref{par:end}) is inserted.

Glue is treated by \TeX\ as a special case of leaders,
which becomes apparent when \cs{unskip} is applied to
leaders: they are removed.

%\spoint Examples of backspacing
\subsection{Examples of backspacing}

The plain \TeX\ macro \csterm{removelastskip} is defined
as
\begin{verbatim}
\ifdim\lastskip=0pt \else \vskip-\lastskip \fi
\end{verbatim}
If the last item on the list was a glue, this macro will
backspace by its value, provided its natural size was not zero.
In all other cases, nothing is added to the list.

Sometimes an intelligent version of commands such as \cs{vskip}
is necessary, in the sense that two subsequent skip commands
should result only in the larger of the two glue amounts.
On page~\pageref{skip:scheme} such a macro is used:
\begin{verbatim}
\newskip\tempskipa
\def\vspace#1{\tempskipa=#1\relax
    \ifvmode \ifdim\tempskipa<\lastskip 
             \else \vskip-\lastskip \vskip\tempskipa
             \fi
    \else \vskip\tempskipa \fi}
\end{verbatim}
First of all, this tests whether the mode is vertical; 
if not, the argument can safely be placed.
Copying the argument into a skip register is necessary
because \cs{v\-space}\verb>{2pt plus 3pt}> would lead to 
problems in an \verb>\ifdim#1<\lastskip> test.

If the surrounding mode was vertical, the argument
should only be placed if it is not less than what is
already there. The macro would be incorrect
if the test  read
\begin{verbatim}
        \ifdim\tempskipa>\lastskip 
            \vskip-\lastskip \vskip\tempskipa
        \fi
\end{verbatim}
In this case the sequence
\begin{verbatim}
... last word.\par \vspace{0pt plus 1fil}
\end{verbatim}
would not place any glue, because after
the \cs{par} we are in vertical mode and
\cs{lastskip} has a value of \n{0pt}.

%\spoint Glue in trace output
\subsection{Glue in trace output}

If the workings of \TeX\ are traced by setting
\cs{tracingoutput} positive, or if \TeX\ 
writes a box to the log file 
(because of a \cs{showbox} command, or because it
is overfull or underfull),
glue  is denoted by the control sequence \cs{glue}.
This is not a \TeX\ command; it merely indicates the presence
of glue in the current list.

The box representation that \TeX\ generated from,
\alt
for instance, \cs{showbox}
inserts a space after every explicit \cs{kern},
but no space is inserted after an implicit
kern that was inserted by the kerning information in the font
\n{tfm} file. Thus \hbox{\verb-\kern 2.0pt-} denotes a kern
that was inserted by the user or by a macro, and
\verb-\kern2.0pt- denotes an implicit kern.

Glue that is inserted automatically (\cs{topskip}, \cs{baselineskip},
et cetera) is denoted by name in \TeX's trace output.
For example, the box
\begin{verbatim}
\vbox{\hbox{Vo}\hbox{b}}
\end{verbatim}
looks like
\begin{verbatim}
\vbox(18.83331+0.0)x11.66669
.\hbox(6.83331+0.0)x11.66669
..\tenrm V
..\kern-0.83334
..\tenrm o
.\glue(\baselineskip) 5.05556
.\hbox(6.94444+0.0)x5.55557
..\tenrm b
\end{verbatim}
Note the implicit kern inserted between `V' and~`o'.

%%%% end of input file [glue]

%\InputFile:rules
%%%% this is input file [rules]
%\subject[rules]  Rules and Leaders
\endofchapter
\chapter{Rules and Leaders}\label{rules}

Rules and leaders are two ways of getting \TeX\ to draw a line.
Leaders are more general than rules: they can also fill
available space with copies of a certain box. This chapter
explain how rules and leaders work, and how they interact with modes.

\label{cschap:hrule}\label{cschap:vrule}\label{cschap:leaders}\label{cschap:cleaders}\label{cschap:xleaders}
\begin{inventory}
\item [\cs{hrule}] 
      Rule that spreads in horizontal direction.

\item [\cs{vrule}] 
      Rule that spreads in vertical direction.

\item [\cs{leaders}] 
      Fill a specified amount of space with a rule or copies of box.

\item [\cs{cleaders}] 
      Like \verb=\leaders=, but with box leaders 
      any excess space is split equally before and after the leaders.

\item [\cs{xleaders}] 
      Like \verb=\leaders=, but with box leaders any excess space is 
      spread equally before, after, and between the boxes.

\end{inventory}

%\point Rules
\section{Rules}

\TeX's rule commands give
\indexterm{rules}:
rectangular black patches with horizontal and vertical sides.
Most of the times, a rule command will give output that
looks like a rule, but~\hbox{\vrule height 1.5ex width 1.5ex}
can also be produced by a rule.

\TeX\ has both horizontal and vertical rules, 
but the names do not necessarily imply anything about the shape.
They do, however, imply something about modes:
an \csterm{hrule} command can only be used in vertical mode,
and a \csterm{vrule} only in horizontal mode.
In fact, an \cs{hrule} is a \gr{vertical command}, and a \cs{vrule}
is a \gr{horizontal command}, so \TeX\ may change
modes when encountering these commands.

Why then is a \cs{vrule} called a {\em vertical\/} rule?
The reason is that a \cs{vrule} can expand arbitrarily
far in the vertical direction: if its height and depth are not
specified explicitly it will take as much room as its
surroundings allow\altt.

\begin{example}
\begin{verbatim}
\hbox{\vrule\ text \vrule}
\end{verbatim}
looks like \begin{disp}\leavevmode\hbox{\vrule\ text \vrule}\end{disp}
and
\begin{verbatim}
\hbox{\vrule\ A gogo! \vrule}
\end{verbatim}
looks like
\begin{disp}\leavevmode\hbox{\vrule\ A gogo! \vrule}\end{disp}
\end{example}

For the \cs{hrule} command a similar statement is true:
a horizontal rule can spread to assume the width of
its surroundings. Thus 
\begin{verbatim}
\vbox{\hbox{One line of text}\hrule}
\end{verbatim}
looks like
\begin{disp}\leavevmode\vtop{\hbox{One line of text}\hrule}\end{disp}


%\spoint Rule dimensions
\subsection{Rule dimensions}

Horizontal and vertical rules have a default thickness:
\begin{Disp} \cs{hrule}\quad is the same as\quad \verb-\hrule height.4pt depth0pt-
\end{Disp}
and 
\begin{Disp} \cs{vrule}\quad is the same as\quad \verb-\vrule width.4pt- \end{Disp}
and if the remaining dimension remains unspecified, the rule
extends in that direction to fill the enclosing box.

Here is the formal specification of how to indicate rule sizes:
\begin{disp}\gr{vertical rule} $\longrightarrow$ 
                        \cs{vrule}\gr{rule specification}\nl
     \gr{horizontal rule} $\longrightarrow$
                        \cs{hrule}\gr{rule specification}\nl
     \gr{rule specification} $\longrightarrow$
                        \gr{optional spaces} \nl \indent$|$
                        \gr{rule dimensions}\gr{rule specification}\nl
     \gr{rule dimension} $\longrightarrow$
                        \n{width}\gr{dimen} $|$ \n{height}\gr{dimen} $|$
                        \n{depth}\gr{dimen}
     \end{disp}
If a rule dimension is specified twice, the second instance
takes precedence over the first. This makes it possible
to override the default dimensions. For instance,
after
\alt
\howto Change the default dimensions of rules\par
\begin{verbatim}
\let\xhrule\hrule  \def\hrule{\xhrule height .8pt}
\end{verbatim}
the macro \cs{hrule} gives a horizontal rule
of double the original height, and it is still possible
with
\begin{verbatim}
\hrule height 2pt
\end{verbatim}
to specify other heights.

It is possible to specify all three dimensions; then
\begin{verbatim}
\vrule height1ex depth0pt width1ex
\end{verbatim}
and
\begin{verbatim}
\hrule height1ex depth0pt width1ex
\end{verbatim}
look the same.
Still, each of them can be used only in the appropriate mode.

%\point Leaders
\section{Leaders}

Rules are intimately connected to modes, which makes it easy
to obtain some effects. For instance, a typical application
of a vertical rule looks like
\begin{verbatim}
\hbox{\vrule width1pt\ Important text! \vrule width 1pt}
\end{verbatim}
which gives
\begin{disp}\leavevmode\hbox{\vrule width 1pt\ Important text! 
                      \vrule width 1pt}\end{disp}
However, one might want to have a horizontal rule
in horizontal mode for effects such as
\begin{disp}\leavevmode
\vbox{\hbox to 5cm{$\longleftarrow$\hfil 5cm\hfil$\longrightarrow$}
    \hbox to 5cm{from here\leaders\hrule\hfil to there}}\end{disp}
An \cs{hrule} can not be used in horizontal mode, and
a vertical rule will not spread automatically.

However, there is a way to use an \cs{hrule} command in
horizontal mode and a \cs{vrule} in vertical mode,
and that is with \indexterm{leaders}, so called because
they lead your eye across the page. 
A~leader command tells \TeX\
to fill a~specified space, in whatever mode it is in,
with as many copies of some box or rule specification
as are needed. For instance, the above example
was given as
\begin{disp}\verb>\hbox to 5cm{from here\leaders\hrule\hfil to there}>\end{disp}
that is, with an \cs{hrule} that was allowed to stretch along
an \cs{hfil}.
Note that the leader was given a horizontal skip,
corresponding to the horizontal mode in which it appeared.

A general leader command looks like
\begin{Disp} \gr{leaders}\gr{box or rule}%
      \gr{vertical/horizontal skip}\end{Disp}
where \gr{leaders} is \cs{leaders}, \cs{cleaders}, 
or~\cs{xleaders}, a \gr{box~or~rule}
is a~\gr{box}, \cs{vrule}, or~\cs{hrule}, and the
lists of horizontal and vertical skips appear in Chapter~\ref{hvmode}.
Leaders can be used in all three modes. However, the
appropriate kind of skip must be specified. 

A horizontal (vertical) box containing leaders has at least
the height and depth (width) of the \gr{box~or~rule} used
in the leaders, even if, as can happen in the case of box leaders,
no actual leaders are placed.

%\spoint Rule leaders
\subsection{Rule leaders}

\emph{Rule leaders}\index{leaders!rule}
fill the specified amount of space with a rule
\csidx{leaders}%
extending in the direction of the skip specified.
The other dimensions of the resulting rule leader
are determined by the sort of rule that is used:
either dimensions can be specified explicitly, or
the default values can be used.

For instance, 
\begin{verbatim}
\hbox{g\leaders\hrule\hskip20pt f}
\end{verbatim}
gives \begin{disp}\leavevmode\hbox{g\leaders\hrule\hskip20pt f}\end{disp}
because a horizontal rule has a default height of~\n{.4pt}.
On the other hand,
\begin{verbatim}
\hbox{g\leaders\vrule\hskip20pt f}
\end{verbatim}
gives \begin{disp}\leavevmode\hbox{g\leaders\vrule\hskip20pt f}\end{disp}
because the height and depth of a vertical rule
by default fill the surrounding box.

Spurious rule dimensions are ignored: in horizontal mode
\begin{verbatim}
\leaders\hrule width 10pt \hskip 20pt
\end{verbatim}
is equivalent to
\begin{verbatim}
\leaders\hrule \hskip 20pt
\end{verbatim}

If the width or height-plus-depth
of either the skip or the box is negative, 
\TeX\ uses ordinary glue instead of leaders.

%\spoint Box leaders
\subsection{Box leaders}

Box leaders fill the available spaces with copies of
a given box, instead of with a rule. 

\newbox\centerdot  \setbox\centerdot=\hbox{\hskip.7em.\hskip.7em}

For all of the following examples, assume that a box register
has been allocated:
\begin{verbatim}
\newbox\centerdot  \setbox\centerdot=\hbox{\hskip.7em.\hskip.7em}
\end{verbatim}
Now the output of
\begin{verbatim}
\hbox to 8cm {here\leaders\copy\centerdot\hfil there}
\end{verbatim}
is 
\begin{disp}\leavevmode\hbox to 8cm {here\leaders\copy\centerdot\hfil there}
\end{disp} That is, copies of the box register fill up the
available space.

Dot leaders, as in the above example, are often used for
tables of contents. In such applications it is desirable that
dots on subsequent lines are vertically aligned.
The \cs{leaders} command does this automatically:
\begin{verbatim}
\hbox to 8cm {here\leaders\copy\centerdot\hfil there}
\hbox to 8cm {over here\leaders\copy\centerdot\hfil over there}
\end{verbatim}
gives \begin{disp}\leavevmode
\vtop{\hbox to 8cm {here\leaders\copy\centerdot\hfil there}
\hbox to 8cm {over here\leaders\copy\centerdot\hfil over there\strut}}
\end{disp}
The mechanism behind this is the following:
\TeX\ acts as if an infinite row of boxes starts (invisibly) at 
the left edge of the surrounding box, 
and the row of copies actually placed is 
merely the part of this row that is not obscured by
the other contents of the box.

Stated differently, box leaders are a window on an infinite
row of boxes, and the row starts at the left edge of the
surrounding box. Consider the following example:
\begin{verbatim}
\hbox to 8cm {\leaders\copy\centerdot\hfil}
\hbox to 8cm {word\leaders\copy\centerdot\hfil}
\end{verbatim}
which gives
\begin{disp}\leavevmode\vtop{\hbox to 8cm {\leaders\copy\centerdot\hfil}
\hbox to 8cm {word\leaders\copy\centerdot\hfil\strut}}\end{disp}
The row of leaders boxes becomes visible as soon as it
does not coincide with other material.

The above discussion only talked about leaders in horizontal
mode. Leaders can equally well be placed in vertical mode;
for box leaders the `infinite row' then starts at the top
of the surrounding box.

%\spoint Evenly spaced leaders
\subsection{Evenly spaced leaders}

Aligning subsequent box leaders in the way described above
means that the white space before and after the
leaders will in general be different.
If vertical alignment is not
an issue it may be aesthetically more pleasing to have
the leaders evenly spaced.
The \csterm{cleaders} command is like \cs{leaders},
except that it splits excess space before and after the leaders
into two equal parts, centring the row of boxes in the
available space.

\begin{example}\message{check verbatim indentation}
\begin{verbatim}
\hbox to 7.8cm {here\cleaders\copy\centerdot\hfil there}
\hbox to 7.8cm {here is\cleaders\copy\centerdot\hfil there}
\end{verbatim}
gives \begin{disp}\leavevmode\vbox{
\hbox to 7.8cm {here\cleaders\copy\centerdot\hfil there}
\hbox to 7.8cm {here is\cleaders\copy\centerdot\hfil there\strut}
}\end{disp}
The `expanding leaders' \csterm{xleaders} spread excess space evenly
between the boxes, with equal globs of glue before, after,
and in between leader boxes.
\end{example}

\begin{example}
\begin{verbatim}
\hbox to 7.8cm{here\hskip.7em
      \xleaders\copy\centerdot\hfil  \hskip.7em there}
\end{verbatim}
gives \begin{disp}\leavevmode
\hbox to 7.8cm {here\hskip.7em\xleaders\copy\centerdot\hfil\hskip.7em there}
\end{disp} Note that the glue in the leader box is balanced here
with explicit glue before and after the leaders;
leaving out these glue items, as in
\begin{verbatim}
\hbox to 7.8cm {here\xleaders\copy\centerdot\hfil there}
\end{verbatim}
gives \begin{disp}\leavevmode
\hbox to 7.8cm {here\xleaders\copy\centerdot\hfil there}
\end{disp}
which is clearly not what was intended.
\end{example}

%\point Assorted remarks
\section{Assorted remarks}

%\spoint Rules and modes
\subsection{Rules and modes}

Above it was explained how rules can only occur in the 
appropriate modes. Rules also influence mode-specific
quantities:
no baselineskip is added before rules in 
vertical mode. In order to prevent glue after rules,
\TeX\ sets \cs{prevdepth} to
\n{\hbox{-}1000pt}
(see Chapter~\ref{baseline}).
Similarly the \cs{spacefactor} is set to 1000 after a \cs{vrule}
in horizontal mode (see Chapter~\ref{line:break}).


%\spoint[par:leaders:end] Ending a paragraph with leaders
\subsection{Ending a paragraph with leaders}
\label{par:leaders:end}

An attempt to simulate an \cs{hrule} at the end of a paragraph by
\howto End a paragraph with leaders\par
\begin{verbatim}
\nobreak\leaders\hrule\hfill\par
\end{verbatim}
does not work. The reason for this is that \TeX\
performs an \cs{unskip} at the end of a paragraph,
which removes the leaders. Normally this \cs{unskip} removes
any space token inserted by the input processor after the
last line. Remedy: stick an \verb.\hbox{}. at the end of
the leaders.

%\spoint Leaders and box registers
\subsection{Leaders and box registers}

In the above examples the leader box was inserted with
\cs{copy}. The output of
\begin{verbatim}
\hbox to 8cm {here\leaders\box\centerdot\hfil there}
\hbox to 8cm {over here\leaders\box\centerdot\hfil 
                   over there}
\end{verbatim}
is
\begin{disp}\leavevmode
     \vtop{\hbox to 8cm {here\leaders\box\centerdot\hfil there}
           \hbox to 8cm {over here\leaders\box\centerdot\hfil over there}
           }\end{disp}
The box register is emptied after the first leader command,
but more than one copy is placed in that first command.

%\spoint Output in leader boxes
\subsection{Output in leader boxes}

Any \cs{write}, \cs{openout}, or \cs{closeout} operation
appearing in leader boxes is ignored. 
Otherwise such an operation would be executed once for every
copy of the box that would be shipped out.

%\spoint Box leaders in trace output
\subsection{Box leaders in trace output}

The dumped box representation obtained from,
for instance, \cs{tracingoutput}
does not write out box leaders in full: only the total size and
one copy of the box used are dumped. In particular,
the surrounding white space before and after the leaders
is not indicated.

%\spoint Leaders and shifted margins
\subsection{Leaders and shifted margins}

If margins have been shifted,
leaders may look different
depending on how the shift has been realized.
For an illustration of how \cs{hangindent} and \cs{leftskip}
influence the look of leaders, consider the following
examples, where
\begin{verbatim}
\setbox0=\hbox{K o }
\end{verbatim}
The horizontal boxes above  the leaders
\altt
serve to indicate the starting point of the row of leaders.

First
\begin{verbatim}
\hbox{\leaders\copy0\hskip5cm}
\noindent\advance\leftskip 1em
      \leaders\copy0\hskip5cm\hbox{}\par
\end{verbatim}
gives\message{examples on}
\begin{disp}\leavevmode\vbox{\leftskip=0pt \hsize=7cm
\setbox0=\hbox{K o }
\hbox{\leaders\copy0\hskip5cm}
\noindent\advance\leftskip 1em
      \leaders\copy0\hskip5cm\hbox{}\par
    }\end{disp}
Then
\begin{verbatim}
\hbox{\kern1em\hbox{\leaders\copy0\hskip5cm}}
\hangindent=1em \hangafter=-1 \noindent
      \leaders\copy0\hskip5cm\hbox{}\par
\end{verbatim}
gives (note the shift with respect to the previous example)
\begin{disp}\leavevmode\vbox{\leftskip=0pt \hsize=7cm
\setbox0=\hbox{K o }
\hbox{\kern1em\hbox{\leaders\copy0\hskip5cm}}
\hangindent=1em \hangafter=-1 \noindent
      \leaders\copy0\hskip5cm\hbox{}\par}\end{disp}
\message{one page}
In the first paragraph the \cs{leftskip} glue only obscures
the first leader box; in the second paragraph the hanging
indentation actually shifts the orientation point for the 
row of leaders. Hanging indentation is performed in \TeX\
by a \cs{moveright} of the boxes containing the lines
of the paragraph.

%%%% end of input file [rules]

%\InputFile:group
%%%% this is input file [group]
%\subject[group] Grouping
\endofchapter
\chapter{Grouping}\label{group}

\TeX\ has a grouping mechanism that is able to confine most
changes to a~particular locality. This chapter explains
what sort of actions can be local, and how groups are formed.

\label{cschap:bgroup}\label{cschap:egroup}\label{cschap:begingroup}\label{cschap:endgroup}\label{cschap:aftergroup}\label{cschap:global}\label{cschap:globaldefs}
\begin{inventory}
\item [\cs{bgroup}] 
Implicit beginning of group character.
\item [\cs{egroup}] 
Implicit end of group character.
\item [\cs{begingroup}] 
Open a group that must be closed with \cs{endgroup}.
\item [\cs{endgroup}] 
Close a group that was opened with \cs{begingroup}.
\item [\cs{aftergroup}] 
Save the next token for insertion after the current group ends.
\item [\cs{global}] 
Make assignments, macro definitions, and arithmetic global.
\item [\cs{globaldefs}] 
Parameter for overriding \cs{global} prefixes.
\IniTeX\ default:~0.
\end{inventory}

%\point The grouping mechanism
\section{The grouping mechanism}

A group is a sequence of tokens starting with a
`beginning of group' token,
and ending with an `end of group'
token, and in which all such tokens are properly balanced. 

The \indexterm{grouping} mechanism of \TeX\ is not the same as
the block structured `scoping'
of ordinary programming languages.
Most languages with block structure are only able  to have
local definitions. \TeX's grouping mechanism is stronger: 
most assignments made inside a group
are local to that group unless explicitly indicated otherwise,
and outside the group old values are restored.

An example of local definitions
\begin{verbatim}
{\def\a{b}}\a
\end{verbatim}
gives an `undefined control sequence'
message because \cs{a} is only defined inside the group.
Similarly, the code
\begin{verbatim}
\count0=1 {\count0=2 } \showthe\count0
\end{verbatim}
will display the value~1; the assignment made inside the group
is undone at the end of the group.


Bookkeeping of values that are to be restored outside the group
is done through the mechanism
of the \indexterm{save stack}. Overflow of the save stack is treated
in Chapter~\ref{error}. The save stack is also used for
a few other purposes: in calls such as \hbox{\verb>\hbox to 100pt{...}>}
the specification \hbox{\n{to 100pt}} is put on the save
stack before a new level of grouping is opened.

In order to prevent a lot of trouble with the save stack,
\IniTeX\ does not allow dumping a format inside a group.
The \cs{end} command is allowed to occur inside a group,
but \TeX\ will give a diagnostic message about this.

The \cs{aftergroup} control sequence saves a token for
insertion after the current group. Several tokens can be
set aside by this command, and they are inserted in the left-to-right
order in which they were stated.
This is treated in Chapter~\ref{expand}.


%\point[global:assign] Local and global assignments
\section{Local and global assignments}
\index{assignment!global|(}
\index{assignment!local|(}
\label{global:assign}

An assignment or macro definition
is usually made global by prefixing it with \csterm{global},
but non-zero values of the \gr{integer parameter}
\csterm{globaldefs} override \cs{global}
specifications: if \cs{globaldefs} is positive every assignment
is implicitly prefixed with \cs{global}, and if
\cs{globaldefs} is negative, \cs{global} is
ignored. Ordinarily this parameter is zero.

Some assignment are always global: the \gr{global assignment}s are
\begin{description}%\FlushRight:no
\item [\gr{font assignment}]
assignments involving \cs{fontdimen}, \cs{hyphenchar}, 
and \cs{skew\-char}.
\item [\gr{hyphenation assignment}]
\cs{hyphenation} and \cs{patterns} commands
(see Chapter~\ref{line:break}).
\item [\gr{box size assignment}]
altering box dimensions with \cs{ht}, \cs{dp}, and~\cs{wd}
(see Chapter~\ref{boxes}).
\item [\gr{interaction mode assignment}]
run modes for a \TeX\ job (see Chapter~\ref{run}).
\item [\gr{intimate assignment}]
assignments to a \gr{special integer} or \gr{special dimen};
see %Chapters \ref{number} and~\ref{glue}.
pages \pageref{special:int:list} and~\pageref{special:dimen:list}.
\end{description}

\index{assignment!global|)}
\index{assignment!local|)}

\section{Group delimiters}

The open and closing \indextermsub{group}{delimiters} can be character
tokens of category code~1\index{category!1} for `beginning of group'
and code~2\index{category!2} for `end of group'
(\indextermbus{explicit}{braces}), or control sequence tokens that are
\cs{let} to such characters (\indextermbus{implicit}{braces}), such as
the \cs{bgroup} and \cs{egroup} in plain \TeX.  Implicit and explicit
braces can match to delimit a group; see for instance the example in
section~\ref{sec:aftergroup}.

Groups can also be delimited by \csterm{begingroup} and
\csterm{endgroup}. These two control sequences must
be used together: they cannot be matched with implicit
or explicit braces, nor can they function as the braces
surrounding, for instance, boxed material.

Delimiting with \cs{begingroup} and \cs{endgroup} can
\label{begin:end:macros}%
provide a limited form of run-time error checking. 
In between these two group delimiters an excess
open or close brace would result in
\begin{verbatim}
\begingroup ... } ... \endgroup
\end{verbatim}
or
\begin{verbatim}
\begingroup ... { ... \endgroup
\end{verbatim}
In both cases \TeX\ gives an error message about improper
balancing. Using \cs{bgroup} and \cs{egroup} here would
make an error much harder to find, because of the incorrect
matching that would occur. This idea is used in the environment
macros of several formats.

The choice of the brace characters for the beginning and end of group
characters is not hard-wired in \TeX. It is arranged
\csidx{bgroup}%
\csidx{egroup}%
like this in the plain format:
\begin{verbatim}
\catcode`\{=1 % left brace is begin-group character
\catcode`\}=2 % right brace is end-group character
\end{verbatim}
Implicit braces have also been defined in the plain format:
\begin{verbatim}
\let\bgroup={ \let\egroup=}
\end{verbatim}

Special cases are the following:
\begin{itemize} \item The replacement text of a macro must be enclosed
in  explicit beginning and end of group character tokens.
\item  The open and close braces for boxes, \cs{vadjust},
and \cs{insert} can be implicit. This makes it possible
to define, for instance
\begin{verbatim}
\def\openbox#1{\setbox#1=\hbox\bgroup}
\def\closebox#1{\egroup\box#1}
\openbox{15}Foo bar\closebox{15}
\end{verbatim}
\item The right-hand side of a token list assignment and the
argument of the commands \cs{write}, \cs{message}, \cs{errmessage}, 
\cs{uppercase}, \cs{lowercase}, 
\cs{special}, and \cs{mark} is a \gr{general text}, defined
as
\begin{Disp} \gr{general text} $\longrightarrow$ \gr{filler}\lb
      \gr{balanced text}\gr{right brace}\end{Disp}
meaning that the left brace can be implicit, but the closing
right brace must be an explicit character token with category
code~2\index{category!2}.
\end{itemize}

In cases where an implicit left brace suffices, and where
expansion is not explicitly inhibited, \TeX\ will
expand tokens until a left brace is encountered. This
is the basis for such constructs as
\verb=\uppercase\expandafter{\romannumeral80}=,
which in this unexpanded form do not adhere to the
syntax. If the first unexpandable token is not a left
brace \TeX\ gives an error message.

The grammar of \TeX\ (see Chapter~\ref{gramm})  uses
\gr{left brace} and \gr{right brace} for explicit
characters, that is, character tokens,
and \n{\lb} and~\n{\rb} 
for possibly implicit characters,
\altt
that is, control sequences that have been \cs{let} to such
explicit characters.

\section{More about braces}
\index{braces|(}

\subsection{Brace counters}

\TeX\ has two counters for keeping  track of grouping levels:
the {\it master counter} and the {\it balance counter}.
Both of these counters are syntactic counters: they count the
explicit brace character tokens, but are not affected by implicit
braces (such as \cs{bgroup}) that are semantically equivalent
to an explicit brace.

The balance counter handles braces in all cases except in
alignment. Its workings are intuitively clear: it goes up
by one for every opening and down for every closing
brace that is not being skipped. Thus
\begin{verbatim}
\iffalse{\fi
\end{verbatim}
increases the balance counter if
this statement is merely scanned (for instance if it
appears in a macro definition text); if this statement
is executed the brace is skipped, so there is no effect on
the balance counter.

The master counter is more tricky;
it is used in alignments instead of the balance counter.
This counter records all braces, even when they are skipped
such as in \verb>\iffalse{\fi>.
For this counter uncounted skipped braces are still possible:
the alphabetic constants \n{`\lb} and \n{`\rb} have
no effect on this counter when they are
use by the execution processor as a~\gr{number};
they do affect this counter when they are seen by the 
input processor (which merely sees characters, and not
the context).

%\spoint The brace as a token
\subsection{The brace as a token}

Explicit braces are character tokens, and as such they are
unexpandable. This implies that they survive until the
last stages of \TeX\ processing. For example,
\begin{verbatim}
\count255=1{2}
\end{verbatim}
will assign~1 to \cs{count255},
and print~`2', because the
opening brace functions as a delimiter for the number~1.
Similarly
\begin{verbatim}
f{f}
\end{verbatim}
will prevent \TeX\ from forming
an `\hbox{ff}' ligature.

From the fact that braces are unexpandable,
it follows that their nesting is independent
of the nesting of conditionals. For instance
\begin{verbatim}
\iftrue{\else}\fi
\end{verbatim}
will give an open brace,
as conditionals are handled by expansion. The closing
brace is simply skipped as part of the \gr{false text};
any consequences it has for grouping only come into
play in a later stage of \TeX\ processing.

Undelimited macro arguments are either single tokens
or groups of tokens enclosed in explicit braces.
Thus it is not possible for an explicit open or close brace
to be a macro argument. However, braces can be assigned
with \cs{let}, for instance as in
\begin{verbatim}
\let\bgroup={
\end{verbatim}
This is used in the plain \cs{footnote} macro
(see page~\pageref{footnote:ex}).

%\spoint \csc{\char 123} and \csc{\char 125}
\subsection{Open and closing brace control symbols}
% \csc{\char 123} and \csc{\char 125}}

The control sequences \verb-\{- and \verb-\}- do not really belong
\csidx{\char123}\csidx{\char125}%
in this chapter,  not being concerned with grouping.
They have been defined with \cs{let} as synonyms of
\cs{lbrace} and \cs{rbrace} respectively,
and these control sequences are \cs{delimiter} instructions
(see Chapter~\ref{mathchar}).

The Computer Modern Roman font has no braces, but there are
braces in the typewriter font, and for mathematics 
there are braces of different sizes \ldash and extendable ones \rdash in
the extension font.

\index{braces|)}
%%%% end of input file [group]

%\InputFile:macro
%%%% this is input file [macro]
%\subject[macro]  Macros
\endofchapter
\chapter{Macros}\label{macro}

Macros are \TeX's abbreviation mechanism for sequences of commands
that are needed more than once,
somewhat like procedures in ordinary programming languages.
\TeX's parameter mechanism, however, is quite unusual.
This chapter explains how \TeX\ macros work. It also
treats the commands \cs{let} and~\cs{futurelet}.

\label{cschap:def}\label{cschap:gdef}\label{cschap:edef}\label{cschap:xdef}\label{cschap:csname}\label{cschap:endcsname}\label{cschap:global2}\label{cschap:outer}\label{cschap:long}\label{cschap:let}\label{cschap:futurelet}
\begin{inventory}
\item [\cs{def}] 
      Start a macro definition.

\item [\cs{gdef}] 
      Synonym for \verb-\global\def-.

\item [\cs{edef}] 
      Start a macro definition; 
      the replacement text is expanded at definition time.
      This command is treated also in the next chapter.

\item [\cs{xdef}] 
      Synonym for \verb-\global\edef-.

\item [\cs{csname}] 
      Start forming the name of a control sequence.

\item [\cs{endcsname}] 
      Stop forming the name of a control sequence.

\item [\cs{global}] 
      Make the next definition, arithmetic statement,
      or assignment global.

\item [\cs{outer}] 
      Prefix indicating that the macro being defined 
      can be used on the `outer' level only.

\item [\cs{long}] 
      Prefix indicating that the arguments of the macro being defined
      may contain \cs{par} tokens.

\item [\cs{let}] 
      Define a control sequence to be equivalent to the next token.

\item [\cs{futurelet}] 
      Define a control sequence to be equivalent to
      the token after the next token.

\end{inventory}

\section{Introduction}

A \indexterm{macro}
is basically a sequence of tokens that has
been abbreviated into a control sequence.
Statements starting with (among others) \cs{def}
are called {\italic macro definitions}\alt, and
writing
\begin{verbatim}
\def\abc{\de f\g}
\end{verbatim}
defines the macro \cs{abc},
with the {\italic replacement text\/} \verb>\de f\g>.
Macros can be used in this way to abbreviate
pieces of text or sequences of commands
that have to be given more than once.
Any time that \TeX's expansion processor
encounters the control sequence \cs{abc},
it replaces it by the replacement text.

If a macro should be sensitive to the context
where it is used, it can be defined with parameters.
A~macro \verb+\PickTwo+ defined as
\begin{verbatim}
\def\PickTwo#1#2{(#1,#2)}
\end{verbatim}
has two \emph{parameters}. When it is used, it will scoop up two pieces of
text, the corresponding \emph{arguments},
and reproduces them in parentheses.
For example:

\begin{tabular}{|l|llllll|}
\hline
          &macro              &argument1& argument2& 
    &expansion&\\ \hline
definition&\verb+\def\PickTwo+&\verb+#1+& \verb+#2+& \verb+{+& 
    \verb+(#1,#2)+& \verb+}+\\
use&       \verb+\PickTwo+&1&2&&(1,2)&\\
use&       \verb+\PickTwo+&\verb+{ab}+&\verb+{cd}+&&(ab,cd)&\\
\hline
\end{tabular}

The activity of substituting the replacement text
for a macro and its arguments is called {\italic macro expansion}.

%\point Layout of a macro definition
\section{Layout of a macro definition}

A \indextermsub{macro}{definition} consists of, in sequence,
\begin{enumerate} \item any number of \cs{global},
\cs{long}, and \cs{outer} prefixes,
\item a \gr{def} control sequence, or anything
that has been \cs{let} to one,
\item a control sequence or active character to be defined, 
\item possibly a \gr{parameter text} specifying among other things
how many parameters the macro has, and
\item a replacement text enclosed in explicit character tokens
with category codes 1\index{category!1} and~2\index{category!2},
by default \verb-{- and~\verb-}-
in plain \TeX.
\end{enumerate}
These elements will all be explained in subsequent sections.

After a macro definition is completed, any saved \cs{afterassignment}
token (see section~\ref{sec:afterassignment}) is inserted.

The `expanding' definitions \cs{edef} and \cs{xdef}
are treated in Chapter~\ref{expand}.

%\point Prefixes
\section{Prefixes}

There are three \emph{prefixes}\index{prefixes !macro}
that alter the status of the
macro definition:
\begin{description}
\item [\cs{global}]
If the definition occurs inside a  group, this prefix\csidx{global}
makes the definition global.
This prefix can also be used for assignments other than
macro definitions; in fact,
for macro definitions abbreviations exist obviating the
use of \cs{global}:
\begin{disp}\verb>\gdef\foo...>\quad is equivalent to\quad \verb>\global\def\foo...>
\end{disp} and
\begin{disp}\verb>\xdef\foo...>\quad is equivalent to\quad \verb>\global\edef\foo...>
\end{disp}

If the parameter \cs{globaldefs}
is positive, all assignments are
implicitly global;
if\handbreak \cs{globaldefs} is negative any \cs{global} prefixes are
ignored,
and \cs{gdef} and \cs{xdef} make local definitions
(see Chapter~\ref{group}).

\item [\cs{outer}]
The mechanism of defining an \indextermbus{outer}{macro} is supposed to facilitate
\csidx{outer}%
locating (among other errors) unbalanced braces: an \cs{outer}
macro is supposed
to  appear only in non-embedded contexts.
To be precise, it is not allowed to occur 
\begin{itemize}
\item in macro replacement texts (but it can appear in
    for instance \cs{edef} after 
    \cs{noexpand}, and after \cs{meaning}),
\item in parameter texts,
\item in skipped conditional text,
\item in alignment preambles, and
\item in the \gram{balanced text} of a \cs{message}, \cs{write},
et cetera. \end{itemize}
For certain applications, however, it is inconvenient
that some of the plain macros  are outer, 
in particular macros such as \cs{newskip}. One remedy is to
redefine them, without the `outer' option, which
is done for instance in \LaTeX, but  cleverer tricks are possible.

\item [\cs{long}]
Ordinarily, macro parameters are not supposed to contain
\csidx{long}%
\cs{par} tokens. This restriction is useful (much more so
than the \cs{outer} definitions) in locating
forgotten closing braces. 
For example, \TeX\ will complain about a `runaway argument'
\message{Example on}
in the following sequence:
\begin{verbatim}
\def\a#1{ ... #1 ... }
\a {This sentence should be in braces.

And this is not supposed to be part of the argument
\end{verbatim}
\message{one page}
The empty line generates a \cs{par}, which most of the times
means that a closing brace has been forgotten.

If arguments to a particular macro should be allowed
to contain \cs{par} tokens,  then the macro must be declared
to be \cs{long}. \end{description}

The \cs{ifx} test for equality of tokens 
(see Chapter~\ref{if}) takes prefixes into
account when testing whether two tokens have the same definition.

\begin{comment}
With a little ingenuity it is possible 
for \cs{par} tokens to sneak into macro arguments anyway.
Consider the example
\begin{verbatim}
\def\a#1\par!{ ... }
\a bc\par ef\par!
\end{verbatim}
Here the macro \cs{a} is not \cs{long}, but the argument
is \verb>bc\par ef>, which contains a \cs{par} token.
However,
this is of no importance in general.
\end{comment}

%\point The definition type
\section{The definition type}

There are four \gr{def} control sequences in \TeX:
\csterm{def}, \csterm{gdef}, \csterm{edef}, and \csterm{xdef}.
The control sequence 
\alt
\cs{gdef} is a synonym for \verb>\global\def> and
\cs{xdef} is a synonym for \verb>\global\edef>.
The `expanding definition' \cs{edef} is treated in 
Chapter~\ref{expand}.

The difference between the various types of macro definitions
is only relevant at the time of the definition.
When a macro is called there is no way of telling how
it was defined.

%\point[param:text] The parameter text
\section{The parameter text}
\label{param:text}

Between the control sequence or active character to be defined
and the opening brace of the replacement text, a \gr{parameter
text} can occur, somewhat corresponding to \indexterm{arguments}
in regular programming languages. This specifies whether the macro has 
parameters\index{parameter},
how many, and how they are delimited. 
The \gr{parameter text} cannot contain
explicit braces.

A macro can have at most nine parameters. 
A~parameter is indicated by a parameter token,
consisting of a macro parameter character
(that is, a character of category code~6\index{category!6},
in plain \TeX~\verb=#=) 
followed by a digit~\n1--\n9. 
For instance, \verb>#6>~denotes the sixth parameter of a macro.
Parameter tokens cannot appear outside the context
of a macro definition.

In the parameter text,
parameters must be numbered consecutively, starting at~1.
A~space after a parameter token is significant,
both in the parameter text and the replacement text.

Parameters can be delimited or undelimited; this determines what the
extent of the macro arguments will be. A~parameter
is called undelimited if it is followed immediately
by another parameter in the \gr{parameter text}, so in
\verb+\def\foo#1#2+ the first parameter is undelimited.
A~parameter is also undelimited if it is immediately followed
by the opening brace of the replacement text, as in \verb+\def\foo#1{...}+.
A~parameter is called delimited if it is followed by any other token; 
in \verb+\def\foo#1!#2{...}+ the first parameter is delimited by the
exclamation sign.

The tokens (zero or more) that are substituted for
a parameter when a macro is expanded (or `called')
are called
the `argument' corresponding to that parameter.

%\spoint Undelimited parameters
\subsection{Undelimited parameters}

When a macro with an \indextermbus{undelimited}{parameter}, for instance
a macro \cs{foo} with one parameter
\begin{verbatim}
\def\foo#1{ ... #1 ...}
\end{verbatim}
is expanded, \TeX\ scans ahead (without expanding)
until a non-blank token is found.
If this token is not an explicit \gr{left brace}, 
it is taken to be the argument
corresponding to the parameter. Otherwise a \gr{balanced text}
is absorbed by scanning until the matching explicit
\gr{right brace} has been found.
This balanced text then
constitutes the argument.

An example with three undelimited parameters follows: with
\begin{verbatim}
\def\foo#1#2#3{#1(#2)#3}
\end{verbatim}
the macro call \cs{foo123} gives `\hbox{1(2)3}';
but \hbox{\verb-\foo 1 2 3-} also gives the same result.
In the call
\begin{disp}\cs{foo}\n{\char32 1\char32 2\char 32 3}\end{disp}
the first space is skipped in the input processor of \TeX.
The argument corresponding to the first parameter is then
the~\n1. In order to find the second parameter \TeX\ then
skips all blanks, in this case exactly one. As second
parameter \TeX\ finds then the~\n2. Similarly the third
parameter is~\n3.


In order to pass several tokens as one undelimited argument
one can use braces. With the above definition of \cs{foo}
the call \verb>\foo a{bc}d> gives `\hbox{a(bc)d}'.
When the argument of a macro is a balanced text instead of
a single token, the delimiting braces are not inserted when 
the argument is
inserted in the replacement text.
For example:
\begin{verbatim}
\def\foo#1{\count0=1#1\relax}
\foo{23}
\end{verbatim}
will expand to \verb>\count0=123\relax>,
which assigns the value of 123 to the counter.
On the other hand,  the statement
\begin{verbatim}
\count0=1{23}
\end{verbatim}
would
assign~1 and print~23.

%\spoint Delimited parameters
\subsection{Delimited parameters}

Apart from enclosing it in braces there is another way
to pass a sequence of tokens as a single argument to a macro,
namely by using a \indextermbus{delimited}{parameter}.

Any non-parameter tokens in the \gr{parameter text} occurring
after a macro parameter (that is, after the parameter number
following the parameter character)
act as a delimiter for that parameter. This includes space tokens:
a space after a parameter number is significant.
Delimiting tokens can also occur between the control
sequence being defined and the first parameter token~\verb>#1>.

Character tokens acting as delimiters in the parameter text
have both their character code and
category code stored; the delimiting character tokens of the
actual arguments have to match both.
Category codes of such characters may include some that
can normally only appear in special contexts; for instance, after
the definition
\begin{verbatim}
\def\foo#1_#2^{...}
\end{verbatim}
the macro \cs{foo}
can be used outside math mode.

When looking for the argument corresponding to
a delimited parameter, \TeX\ absorbs all tokens without expansion (but
balancing braces) until the 
(exact sequence of) delimiting tokens is encountered.
The delimiting tokens are not part of the argument;
they are removed from the input stream during the macro call.

%\spoint Examples with delimited arguments
\subsection{Examples with delimited arguments}

As a simple example,
\begin{verbatim}
\def\DoASentence#1#2.{{#1#2.}}
\end{verbatim}
defines a macro with an undelimited first parameter,
and a second parameter delimited by a period.
In the call
\begin{verbatim}
\DoASentence \bf This sentence is the argument.
\end{verbatim}
the arguments are:
\begin{verbatim}
#1<-\bf
#2<-This sentence is the argument
\end{verbatim}
Note that the closing period is not in the argument, but it has
been absorbed; it is no longer in the input stream.

A~commonly used delimiter is \cs{par}:
\begin{verbatim}
\def\section#1. #2\par{\medskip\noindent {\bf#1. #2\par}}
\end{verbatim}
This macro has a first parameter that is delimited by~`\n{.\char32}',
and a second parameter that is delimited by \cs{par}.
The call\message{example on one page}
\begin{verbatim}
\section 2.5. Some title

The text of the section...
\end{verbatim}
will give
\begin{disp}\verb>#1<-2.5>\nl
\verb>#2<-Some title>\n{\char32}\end{disp}
Note that there is a space at the end of the second argument
generated by the line end. If this space is unwanted one might
define
\begin{verbatim}
\def\section#1. #2 \par{...}
\end{verbatim}
with \n{\char32}\cs{par} delimiting the second
argument. This approach, however,
precludes  the user's writing the \cs{par} explicitly:
\begin{verbatim}
\section 2.5 Some title\par
\end{verbatim}
One way out of this dilemma is to write
\verb>#2\unskip> on all places in the definition text
where the trailing space would be unwanted.

Control sequences acting as delimiters need not be defined,
as they are absorbed without expansion. Thus
\begin{verbatim}
\def\control#1\sequence{...}
\end{verbatim}
is a useful
definition, even if \cs{sequence} is undefined.

The importance of category codes in delimited arguments
is shown by the following example:
\begin{verbatim}
\def\a#1 #2.{ ... }
\catcode`\ =12
\a b c
d.
\end{verbatim}
which gives
\begin{verbatim}
\a #1 #2.-> ...
#1<- b c
#2<-d
\end{verbatim}
Explanation: in the macro definition,
the delimiter between parameters 1 and~2 is a space
of category~10\index{category!10}.
When the macro is invoked,
in between \n{b} and \n{c} there is a space
of category~12\index{category!12},
so this does not delimited the arguments;
the first space of  category~10
is the space that is generated by the line end.

For a `real-life' application of matching of category codes,
see the explanation of \cs{newif} in Chapter~\ref{if},
and the example on page~\pageref{ex:jobnumber}.


%\spoint Empty arguments
\subsection{Empty arguments}

If the user specifies a \gr{balanced text} in braces
when \TeX\ expects a macro
argument, that text is used as the argument.
Thus, specifying \verb-{}- will give an argument that is
an empty list of tokens; this is called an `empty argument'.

Empty arguments can also arise from the use of delimited
parameters. For example, after the definition
\begin{verbatim}
\def\mac#1\ro{ ... }
\end{verbatim}
the call
\begin{verbatim}
\mac\ro
\end{verbatim}
will give an empty argument. 

\begin{comment}
However, only
one empty argument can be created this way: 
if the macro had been defined as
\begin{verbatim}
\def\mac#1#2\ro{ ... }
\end{verbatim}
the same call
\begin{verbatim}
\mac\ro \othermacro \stillothermacro
\end{verbatim}
will probably cause a `\n{Runaway argument?}' error message.
Explanation: the first parameter is undelimited, so the corresponding
argument is `\cs{ro}'; after that \TeX\ starts looking for a list
of tokens delimited by~\cs{ro}.
\end{comment}

\subsection{The macro parameter character}

When \TeX's input processor scans a macro definition text, it inserts
a parameter token for any occurrence of a macro
\indextermsub{parameter}{character}\index{character!parameter}
followed by a digit.  In effect, a
parameter token in the replacement text states `insert parameter
number such and such here'.  Two parameter characters in a row are
replaced by a single one.

The latter fact can be used for nested macro definitions.
\label{nest:def}\howto Nested macro definitions\par
Thus
\begin{verbatim}
\def\a{\def\b#1{...}}
\end{verbatim}
gives an error message
because \cs{a} was defined without parameters, and
yet there is a parameter token in its replacement text.

The following
\begin{verbatim}
\def\a#1{\def\b#1{...}}
\end{verbatim}
defines a macro \cs{a} that
defines a macro \cs{b}. However, \cs{b} still does not
have any parameters: the call
\begin{verbatim}
\a z
\end{verbatim}
defines a macro \cs{b} without parameters,
that has to be followed by a~\n z.
Note that this
does not attempt to define a macro \cs{bz}, because the
control sequence \cs{b} has already been formed in \TeX's
input processor when that input line was read.

Finally,
\begin{verbatim}
\def\a{\def\b##1{...}}
\end{verbatim}
defines a macro \cs{b} 
with one parameter.

Let us examine the handling of the parameter character
in some detail.
Consider
\begin{verbatim}
\def\a#1{ .. #1 .. \def\b##1{ ... }}
\end{verbatim}
When this is read as input, the input processor
\begin{itemize}
\item replaces the characters \verb>#1> by \gr{parameter token$_1$}, and
\item replaces the characters \verb>##> by \verb>#>\end{itemize}
A macro call of \cs{a} will then let the input processor scan
\begin{verbatim}
\def\b#1{ ... }
\end{verbatim}
in which the two characters \verb>#1> are
\alt
replaced by a parameter token.

%\spoint Brace delimiting
\subsection{Brace delimiting}

Ordinarily, it is not possible to have left or right
braces in the \gr{parameter text} of a definition.
There is a special mechanism, however, that can make
the last parameter of a macro act as if it is delimited
by an opening brace. 

If the last parameter token
is followed by a parameter character (\verb>#>),
which in turn is followed by the opening brace of the
replacement text, \TeX\ makes the last parameter
be delimited by a beginning-of-group character.
Furthermore, unlike other delimiting tokens in
parameter texts, this opening brace is not
removed from the input stream.

Consider an example.
Suppose we want to have a macro
\cs{every} that can fill token lists as follows:
\begin{verbatim}
\every par{abc} \every display{def}
\end{verbatim}
This macro can be defined as
\begin{verbatim}
\def\every#1#{\csname every#1\endcsname}
\end{verbatim}
In the first call above, the argument corresponding to
the parameter is \n{abc}, so the call 
expands to
\begin{verbatim}
\csname everypar\endcsname{abc}
\end{verbatim}
which gives the desired result.


\section{Construction of control sequences}
\label{cs:name}

The commands \csterm{csname} and \csterm{endcsname} can be used
to construct a control sequence. 
For instance
\begin{verbatim}
\csname hskip\endcsname 5pt
\end{verbatim}
is equivalent to \verb=\hskip5pt=.

During this construction process
all macros and other expandable control sequences
between \cs{csname} and \cs{endcsname}
are expanded as usual, until only unexpandable
character tokens remain. A~variation of the above example,
\begin{verbatim}
\csname \ifhmode h\else v\fi skip\endcsname 5pt
\end{verbatim}
performs an \cs{hskip} or \cs{vskip} depending on the mode.
The final result of the expansion should 
consist of only character tokens, but
their category codes do not matter.
An unexpandable control sequence gives an error here:
\TeX\ will insert an \cs{endcsname} right before it
as an attempt at error recovery.

With \cs{csname} it is possible to construct
control sequences that cannot ordinarily be written,
because the constituent character tokens may have another category
\alt
than~11, letter. This principle can be used to hide
\howto Hide counters from the user\par
inner control sequences of a macro package from the user.
\begin{example}
\begin{verbatim}
\def\newcounter#1{\expandafter\newcount
    \csname #1:counter\endcsname}
\def\stepcounter#1{\expandafter\advance
    \csname #1:counter\endcsname 1\relax}
\end{verbatim}
In the second definition the \cs{expandafter} is superfluous,
but it does no harm, and it is conceptually clearer.
\end{example}

The name of the actual counter created by \cs{newcounter}
contains a colon, so that it takes some effort to write this
control sequence. In effect, the counter
is now hidden from the user, who can only
access it through control sequences such as \cs{stepcounter}.
By the way, the macro \cs{newcount} is defined \cs{outer} in
the plain format, so the above definition of \cs{newcounter}
can only be written after \cs{newcount} has been redefined.

If a control sequence formed with \verb>\csname...\endcsname>
has not been defined
before, its meaning is set to \cs{relax}.
Thus if \verb=\xx= is an undefined control sequence, the
command
\begin{verbatim}
\csname xx\endcsname
\end{verbatim}
will {\em not\/}
give an error message, as it is equivalent to \verb=\relax=.
Moreover, after this execution of the
\verb-\csname...\endcsname- statement, the control sequence
\verb=\xx= is itself equivalent to \cs{relax}, so it
will no longer give an `undefined control sequence' error
(see also page~\pageref{relax:cs}).


%\point Token assignments by \cs{let} and \cs{futurelet}
\section{Token assignments by \protect\cs{let} and \protect\cs{futurelet}}

There are two \gr{let assignment}s in \TeX.
Their syntax is
\begin{disp}\cs{let}\gr{control sequence}\gr{equals}%
     \gr{one optional space}\gr{token}\nl
     \cs{futurelet}\gr{control sequence}\gr{token}\gr{token}
     \end{disp}
In the syntax of a \cs{futurelet} assignment
no optional equals sign appears.

%\spoint[let] \cs{let}
\subsection{\protect\cs{let}}
\label{let}

The primitive command \csterm{let} assigns the current meaning
of a~token to a control sequence or active character.

For instance, in the plain format \cs{endgraf} is defined
as
\begin{verbatim}
\let\endgraf=\par
\end{verbatim}
This enables macro writers to redefine \cs{par}, while
still having the functionality of the primitive \cs{par}
command available. For example,
\begin{verbatim}
\everypar={\bgroup\it\def\par{\endgraf\egroup}}
\end{verbatim}

The case where the \gr{token} to be assigned is not a control
sequence but a character token instead has been treated 
in Chapter~\ref{char}.

%\spoint \cs{futurelet}
\subsection{\protect\cs{futurelet}}

As was explained above, the sequence with \cs{let}
\begin{disp}\cs{let}\gr{control sequence}\gr{token$_1$}\gr{token$_2$}%
       \gr{token$_3$}\gr{token$\cdots$}\end{disp}
assigns (the meaning of) \gr{token$_1$} to the control sequence, 
and the remaining input stream looks like
\begin{disp}\gr{token$_2$}\gr{token$_3$}\gr{token$\cdots$}\end{disp}
That is, the \gr{token$_1$} has disappeared from the stream.

The command \csterm{futurelet} works slightly differently:
given the input stream
\begin{disp}\cs{futurelet}\gr{control sequence}\gr{token$_1$}\gr{token$_2$}%
       \gr{token$_3$}\gr{token$\cdots$}\end{disp}
it assigns (the meaning of) \gr{token$_2$} to the control sequence, 
and the remaining stream looks like
\begin{disp}\gr{token$_1$}\gr{token$_2$}\gr{token$_3$}\gr{token$\cdots$}\end{disp}
That is, neither \gr{token$_1$} nor \gr{token$_2$} has
been lifted from the stream.
However, now \gr{token$_1$}
`knows' what \gr{token$_2$} is, without having had to absorb it
as a macro parameter. See an example below.

If a character token has been \cs{futurelet} to a control
sequence, its category code is fixed.
The subsequent \gr{token$_1$} cannot change
it anymore.

%\point Assorted remarks
\section{Assorted remarks}

%\spoint Active characters
\subsection{Active characters}

A character token of category~13\index{category!13} is called an
\indexterm{active character}, and it
can be defined just like a control sequence.
If the definition of the character appears inside a macro,
the character has to be active at the time of the definition
of that macro.

Consider for example the following definition
(taken from Chapter~\ref{mouth}):
\begin{verbatim}
{\catcode`\^^M=13 %
 \gdef\obeylines{\catcode`\^^M=13 \def^^M{\par}}%
}
\end{verbatim}
The unusual category of the \verb>^^M> character
has to be set during the definition of \cs{obeylines},
otherwise \TeX\ would think that the line ended
after \cs{def}.

\subsection{Macros versus primitives}

The distinction between \indexterm{primitive
  commands}\indexterm{command !primitive} and user macros is not
nearly as important in \TeX\ as it is in other programming
languages.
\begin{itemize}
\item The user can use primitive commands under different names:
     
\begin{verbatim}
\let\StopThisParagraph=\par
\end{verbatim}
\item Names of primitive commands can be used for
      user macros:
\begin{verbatim}
\def\par{\hfill$\bullet$\endgraf}
\end{verbatim}
\item Both user macros and a number of \TeX\ primitives
      are subject to expansion, for instance all conditionals,
      and commands such as \cs{number} and~\cs{jobname}.
\end{itemize}

%\spoint Tail recursion
\subsection{Tail recursion and the \protect\cs{loop} macro}

Macros in \TeX, like procedures in most modern programming
languages, are allowed to be \emph{recursive}\index{recursion}: that is, the 
definition of a macro can contain a call to this same macro,
or to another macro that will call this macro.
Recursive macros tend to clutter up \TeX's memory
if too many `incarnations' of such a macro are active
at the same time. However, \TeX\ is able to prevent this
in one frequently occurring case of recursion: tail recursion.

In order to  appreciate what goes on here, some background
knowledge is needed. When \TeX\ starts executing a macro
it absorbs the parameters, and places an item pointing to
the replacement text on the \indextermsub{input}{stack},
so that the scanner will next be directed to
this replacement. Once it has been processed, the item on the 
input stack can be removed.
However, if the definition text
of a macro contains further macros, this process will be
repeated for them: new items may be placed on the input stack
directing the scanner to other macros
even before the first one has been completed.

In general this `stack build-up' is a necessary evil, but
it can be prevented if the nested macro call is the
{\em last\/} token in the replacement text of the original
macro. After the last token no further tokens need to be
considered, so one might as well clear the top item
from the input stack
before a new one is put there.
This is what \TeX\ does.

The \csterm{loop} macro of plain \TeX\ provides a good illustration
\label{loop:ex}
of this principle. The definition is
\begin{verbatim}
\def\loop#1\repeat{\def\body{#1}\iterate}
\def\iterate{\body \let\next=\iterate
    \else \let\next=\relax\fi \next}
\end{verbatim}
and this macro can be called for example as follows:
\begin{verbatim}
\loop \message{\number\MyCount}
    \advance\MyCount by 1
    \ifnum\MyCount<100 \repeat
\end{verbatim}
The macro \cs{iterate} can call itself and, when it does so,
the recursive call is performed by the last token in the list.
It would have been possible to define \cs{iterate}
as
\begin{verbatim}
\def\iterate{\body \iterate\fi}
\end{verbatim}
but then \TeX\ would not have been able to resolve the recursion
as the call \cs{iterate} is not the last token in the replacement
text of \cs{iterate}. Assigning \verb>\let\next=\iterate>
is here a way to let
the recursive call be the last token in the list.

Another way of resolving tail recursion is to use
\cs{expandafter} (see page~\pageref{after:cond}): in
\begin{verbatim}
\def\iterate{\body \expandafter\iterate\fi}
\end{verbatim}
it removes the \cs{fi} token.
Tail recursion would also be resolved if the last
tokens in the list were arguments for the
recursive macro.

An aside: by defining \cs{iterate} as
\begin{verbatim}
\def\iterate{\let\next\relax 
    \body \let\next\iterate \fi \next}
\end{verbatim}
it becomes possible to write
\begin{verbatim}
\loop ... \if... ... \else ... \repeat
\end{verbatim}

There is further discussion of the \cs{loop} macro in~\cite{Eloop},
and see also the `repeat' package~\cite{Erepeat}.

%\point Macro techniques
\section{Macro techniques}

%\spoint Unknown number of arguments
\subsection{Unknown number of arguments}

In some applications,
\howto  Macros with an undetermined number
of arguments\par
a macro is needed that can have a
number of arguments that is not specified in advance.

Consider the problem of translating a position on a chess board
(for full macros and fonts, see~\cite{chess} and~\cite{Tut}),
given like
\begin{verbatim}
\White(Ke1,Qd1,Na1,e2,f4)
\end{verbatim} 
to a sequence of typesetting instructions
\begin{verbatim}
\WhitePiece{K}{e1} \WhitePiece{Q}{d1} \WhitePiece{N}{a1} 
\WhitePiece{P}{e2} \WhitePiece{P}{f4}
\end{verbatim}
Note that for pawns the `P' is omitted in the list of positions.

The first problem is that the list of pieces 
is of variable length, so we append a terminator piece:
\begin{verbatim}
\def\White(#1){\xWhite#1,xxx,}
\def\endpiece{xxx}
\end{verbatim}
for which we can test.
Next, the macro \cs{xWhite} takes one position from the list,
tests whether it is the terminator, and if not,
subjects it to a test to see whether it is a pawn.
\begin{verbatim}
\def\xWhite#1,{\def\temp{#1}%
    \ifx\temp\endpiece 
    \else \WhitePieceOrPawn#1XY%
          \expandafter\xWhite 
    \fi}
\end{verbatim}
An \cs{expandafter} command is necessary to remove the
\cs{fi} (see page~\pageref{after:cond}), so that 
\cs{xWhite} will get the next position as argument
instead of \cs{fi}.

Positions  are either two or three characters long.
The call to \cs{White\-Piece\-OrPawn}, a four-parameter macro,
appended a terminator string \n{XY}. 
In the case of a pawn, therefore, argument~3 is the character~\n X
and argument~4 is empty; for all other pieces argument~1
is the piece, 2~and~3 are the position, and argument~4 is~\n X.
\begin{verbatim}
\def\WhitePieceOrPawn#1#2#3#4Y{
    \if#3X \WhitePiece{P}{#1#2}%
    \else  \WhitePiece{#1}{#2#3}\fi}
\end{verbatim}

%\spoint Examining the argument
\subsection{Examining the argument}

It may be necessary in some cases to test whether a macro
\howto Examine a macro argument for the presence of some element\par
\howto Apply \cs{uppercase} when the argument has a \cs{footnote}\par
argument contains some element. For a real-life example,
consider the following (see also the \cs{DisplayEquation}
\alt
example on page~\pageref{left:display}).

Suppose the title and author of an article are given as
\begin{verbatim}
\title{An angle trisector}
\author{A.B. Cee\footnote*{Research supported by the
Very Big Company of America}}
\end{verbatim}
with multiple authors
given as
\begin{verbatim}
\author{A.B. Cee\footnote*{Supported by NSF grant 1}
        \and
        X.Y. Zee\footnote{**}{Supported by NATO grant 2}}
\end{verbatim}
Suppose further that the \cs{title} and \cs{author} macros
are defined as
\begin{verbatim}
\def\title#1{\def\TheTitle{#1}}  \def\author#1{\def\TheAuthor{#1}}
\end{verbatim}
which will be used as
\begin{verbatim}
\def\ArticleHeading{ ... \TheTitle ... \TheAuthor ... }
\end{verbatim}

For some journals it is required to
have the authorship and the title of the article in all capitals.
The implementation of this could be
\begin{verbatim}
\def\ArticleCapitalHeading
   { ...
    \uppercase\expandafter{\TheTitle}
     ...
    \uppercase\expandafter{\TheAuthor}
     ...
   }
\end{verbatim}
Now the \cs{expandafter} commands will expand the title and
author into the actual texts, and the \cs{uppercase} commands
will capitalize them. However, for the authors this is wrong,
since the \cs{uppercase} command will also capitalize the
footnote texts.
The problem is then to uppercase only the parts
of the title in between the footnotes.

As a first attempt, let us take the case of one author, and
let the basic call be
\begin{verbatim}
\expandafter\UCnoFootnote\TheAuthor
\end{verbatim}
This expands into
\begin{verbatim}
\UCnoFootnote A.B. Cee\footnote*{Supported ... }
\end{verbatim}
The macro
\begin{verbatim}
\def\UCnoFootnote#1\footnote#2#3{\uppercase{#1}\footnote{#2}{#3}}
\end{verbatim}
will analyse this correctly:
\begin{verbatim}
#1<-A.B. Cee
#2<-*
#3<-Supported ...
\end{verbatim}
However, if there is no footnote, this macro is completely wrong.

As a first refinement we add a footnote ourselves, just to make
sure that one is present:
\begin{verbatim}
\expandafter\UCnoFootnote\TheAuthor\footnote 00
\end{verbatim}
Now we have to test what kind of footnote we find:
\begin{verbatim}
\def\stopper{0}
\def\UCnoFootnote#1\footnote#2#3{\uppercase{#1}\def\tester{#2}%
    \ifx\stopper\tester
    \else\footnote{#2}{#3}\fi}
\end{verbatim}
With \cs{ifx} we test the delimiter footnote sign against the
actual sign encountered. Note that a solution with
\begin{verbatim}
\ifx0#2
\end{verbatim}
would be wrong if the footnote sign consists
of more than one token, for instance~\verb>{**}>.

The macro so far is correct if there was no footnote,
but if there was one it is wrong:
the terminating tokens remain to be disposed of.
They are taken care of in the following version:
\begin{verbatim}
\def\stopper{0}
\def\UCnoFootnote#1\footnote#2#3{\uppercase{#1}\def\tester{#2}%
    \ifx\stopper\tester
    \else\footnote{#2}{#3}\expandafter\UCnoFootnote
    \fi}
\end{verbatim}
A repeated call to \cs{UCnoFootnote} removes the delimiter tokens
(the \cs{expandafter} first removes the \cs{fi}),
and as an added bonus, this macro is also correct for multiple
authors.


%\spoint Optional macro parameters with \cs{futurelet}
\subsection{Optional macro parameters with \protect\cs{futurelet}}

One standard application of \cs{futurelet} is implementing
\howto Macros with optional parameters\par
optional parameters of macros. The general course of action
is as follows:
\begin{verbatim}
\def\Com{\futurelet\testchar\MaybeOptArgCom}
\def\MaybeOptArgCom{\ifx[\testchar \let\next\OptArgCom 
                 \else \let\next\NoOptArgCom \fi \next}
\def\OptArgCom[#1]#2{ ... }\def\NoOptArgCom#1{ ... }
\end{verbatim}
Note that \cs{ifx} is used even though it tests
for a character. The reason is of course that,
if the optional argument is omitted, there might be an
expandable control sequence behind the~\cs{Com}.

The macro \cs{Com} now has one optional and one regular
argument; it can be called as 
\begin{verbatim}
\Com{argument}
\end{verbatim}
or as
\begin{verbatim}
\Com[optional]{argument}
\end{verbatim}
Often the call without the optional argument will insert some
default value:
\begin{disp}\verb>\def\NoOptArgCom#1{\OptArgCom[>%
{\italic default\/}\verb>]{#1}}>\end{disp}
This mechanism is widely used in formats such as \LaTeX\ and
\LamsTeX; see also~\cite{svb:future}.



%\spoint Two-step macros
\subsection{Two-step macros}

Often what looks to the user like one macro is in reality
a two-step process, where one macro will set up conditions,
and a second macro will do the work.

As an example, here is
a macro \cs{PickToEol}\label{pick:eol}
\howto Take an input line as macro argument\par
with an argument that is delimited by the line end.
First we write a macro without arguments that 
changes the category code of the line end, and then
calls the second macro.
\begin{verbatim}
\def\PickToEol{\begingroup\catcode`\^^M=12 \xPickToEol}
\end{verbatim}
The second macro can then take as an argument everything
up to the end of the line:
\begin{verbatim}
\def\xPickToEol#1^^M{ ... #1 ... \endgroup}
\end{verbatim}
There is one problem with this definition: the \verb>^^M> character
should have category~12. We arrive at the following:
\begin{verbatim}
\def\PickToEol{\begingroup\catcode`\^^M=12 \xPickToEol}
{\catcode`\^^M=12 %
 \gdef\xPickToEol#1^^M{ ... #1 ... \endgroup}%
}
\end{verbatim}
where the category code of \verb>^^M> is changed for the
sake of the definition of \cs{xPickToEol}. Note that
the \verb>^^M> in \cs{PickToEol} occurs in a control symbol,
so there the category code  is irrelevant. Therefore that
definition can be outside the group where the category code 
of \verb>^^M> is redefined.


\subsection{A comment environment}

As an application of the above idea of two-step macros,
\howto Comment environment\par
and in order to illustrate tail recursion, here are 
macros for a `comment' environment.

Often it is necessary to remove a part of \TeX\
input temporarily. For this one would like to
write
\begin{verbatim}
\comment
...
\endcomment
\end{verbatim}
The simplest implementation of this, 
\begin{verbatim}
\def\comment#1\endcomment{}
\end{verbatim} 
has a number of weaknesses. For instance,
it cannot cope with outer macros or input that 
does not have balanced braces. Its worst
shortcoming, however, is that it reads the complete
comment text as a macro argument. This limits the size
of the comment to that of \TeX's input buffer.

It would be a better idea to take on the out-commented
text one line at a time. For this we want to write
a recursive macro with a basic structure
\begin{verbatim}
\def\comment#1^^M{ ... \comment }
\end{verbatim}
In order to be able to write this definition at all,
the category code of the line end must be changed; as above
\altt
we will have
\begin{verbatim}
\def\comment{\begingroup \catcode`\^^M=12 \xcomment}
{\catcode`\^^M=12 \endlinechar=-1 %
 \gdef\xcomment#1^^M{ ... \xcomment}
}
\end{verbatim}
Changing the \cs{endlinechar} is merely to 
prevent having to put comment characters at the end
of every line of the definition.

Of course, the process must stop at a certain time.
To this purpose we investigate the line that was
scooped up as macro argument:
\begin{verbatim}
{\catcode`\^^M=12 \endlinechar=-1 %
 \gdef\xcomment#1^^M{\def\test{#1}
    \ifx\test\endcomment \let\next=\endgroup
    \else \let\next=\xcomment \fi
    \next}
}
\end{verbatim}
and we have to define \cs{endcomment}:
\begin{verbatim}
\def\endcomment{\endcomment}
\end{verbatim}
This command will never be executed: it is merely for purposes
of testing whether the end of the environment has been reached.

We may want to comment out text that is not syntactically
correct. Therefore we switch to a
\indexterm{verbatim mode}
when commenting. The following macro is given 
in plain \TeX:
\begin{verbatim}
\def\dospecials{\do\ \do\\\do\{\do\}\do\$\do\&%
  \do\#\do\^\do\^^K\do\_\do\^^A\do\%\do\~}
\end{verbatim}
We use it to define \cs{comment} as follows:
\begin{verbatim}
\def\makeinnocent#1{\catcode`#1=12 }
\def\comment{\begingroup
    \let\do=\makeinnocent \dospecials
    \endlinechar`\^^M \catcode`\^^M=12 \xcomment}
\end{verbatim}
Apart from the possibility mentioned above of commenting
out text that is not syntactically correct, for instance
because of unmatched braces, this solution can handle
outer macros. The former implementation of \cs{xcomment}
would cause a \TeX\ error if one occurred in the comment text.

However, using verbatim mode poses the problem of concluding the 
environment.
\altt
The final line of the comment is now not the control sequence
\cs{endcomment}, but the characters constituting it. We have
to test for these then:
\begin{verbatim}
{\escapechar=-1
 \xdef\endcomment{\string\\endcomment}
}
\end{verbatim}
The sequence \verb>\string\\> gives a backslash.
We could not have used
\begin{verbatim}
\edef\endcomment{\string\endcomment}
\end{verbatim}
because
the letters of the word \n{endcomment} would then have
category code~12, instead of the 11 that the ones on the
last line of the comment will have.

%%%% end of input file [macro]

%\InputFile:expand
%%%% this is input file [expand]
%\subject[expand]  Expansion
\endofchapter

\chapter{Expansion}\label{expand}

\emph{Expansion}\index{expansion} in \TeX\ is rather different from
procedure calls in most programming languages. This chapter treats the
commands connected with expansion, and gives a number of (non-trivial)
examples.

\label{cschap:relax}\label{cschap:expandafter}\label{cschap:noexpand}\label{cschap:afterassignment}\label{cschap:the}
\begin{inventory}
\item [\cs{relax}] 
     Do nothing.


\item [\cs{expandafter}]  
      Take the next two tokens and place the expansion of the
      second after the first.

\item [\cs{noexpand}]   
      Do not expand the next token.


\item [\cs{edef}] 
      Start a macro definition; 
      the replacement text is expanded at definition time.

 
\item [\cs{aftergroup}]  
      Save the next token for insertion after the current group.

\item [\cs{afterassignment}]   
      Save the next token for execution after the next assignment
      or macro definition.


\item [\cs{the}] 
      Expand the value of various quantities in \TeX\ into a string
      of character tokens.

\end{inventory}


%\point Introduction
\section{Introduction}

\TeX's expansion processor accepts a stream of tokens
coming out of the input processor, and its result is
again a stream of tokens, which it feeds to the execution
processor. For the input processor there are two
kinds of tokens: expandable and unexpandable ones.
The latter category is passed untouched, and it contains
largely assignments and typesettable material;
the former category
is expanded, and the result of that expansion is examined anew.

%\point Ordinary expansion
\section{Ordinary expansion}

The following list gives those constructs
that are expanded, unless
expansion is inhibited:
\index{expansion!expandable constructs}
\begin{itemize}
\item macros\label{expand:lijst}
\item conditionals
\item \cs{number}, \cs{romannumeral}
\item \cs{string}, \cs{fontname}, \cs{jobname}, 
      \cs{meaning}, \cs{the}
\item \verb,\csname ... \endcsname,
\item \cs{expandafter}, \cs{noexpand}
\item \cs{topmark}, \cs{botmark}, \cs{firstmark}, 
      \cs{splitfirstmark}, \cs{splitbotmark}
\item \cs{input}, \cs{endinput}
\end{itemize}

This is the list of all instances where
expansion is inhibited:
\begin{itemize}\label{noexp:list}
\item when \TeX\ is reading a token to be defined by
      
\begin{itemize} \item a \gr{let assignment}, that is,
           by \cs{let} or \cs{futurelet},
        \item a \gr{shorthand definition}, that is,
           by \cs{chardef} or \cs{mathchardef}, or a
           \gr{register def}, that is, \cs{countdef},
           \cs{dimendef}, \cs{skipdef}, \cs{muskipdef},
           or~\cs{toksdef},
        \item a \gr{definition}, that is a macro definition
           with \cs{def}, \cs{gdef}, \cs{edef}, or~\cs{xdef},
        \item the \gr{simple assignment}s \cs{read} and \cs{font};
      \end{itemize}
\item when a \gr{parameter text} or macro arguments
      are being read; also when  the replacement text of a 
      control sequence
      being defined by \cs{def}, \cs{gdef}, or \cs{read}
      is being read;
\item when the token list for a \gr{token variable} or
      \cs{uppercase}, \cs{lowercase}, or \cs{write}
      is being read; however, the token list for \cs{write}
      will be  expanded later when it is shipped out;
\item when tokens are being deleted during error recovery;
\item when part of a conditional is being skipped;
\item in two instances when \TeX\ has to know what follows
      
\begin{itemize}\item after a left quote in a context where
         that is used to denote an integer (thus in 
         \verb-\catcode`\a- the \cs{a} is not expanded), or
        \item after a math shift character that begins math mode
         to see whether another math shift character follows (in which case
         a display opens);
        \end{itemize}
\item when an alignment preamble is being scanned; however,
      in this case a~token
      preceded by \cs{span} and the tokens in a \cs{tabskip} 
      assignment are still expanded.
\end{itemize}

%\point Reversing expansion order
\section{Reversing expansion order}

Every once in a while you need to change the normal order of
expansion of tokens. \TeX\ provides several mechanisms for
this. Some of the control sequences in this section are
not strictly concerned with expansion.

%\spoint One step expansion: \cs{expandafter}
\subsection{One step expansion: \protect\cs{expandafter}}

The most obvious tool for reversed expansion order is
\csterm{expandafter}. The sequence
\begin{disp}\cs{expandafter}\gr{token$_1$}\gr{token$_2$}\end{disp}
expands to 
\begin{disp}\gr{token$_1$}\gr{\italic the expansion of token$_2$}
\end{disp}
Note the following.
\begin{itemize} \item If \gr{token$_2$} is a macro, it is replaced
by its replacement text, not by its final expansion.
Thus, if 
\begin{verbatim}
\def\tokentwo{\ifsomecondition this \else that \fi}
\def\tokenone#1{ ... }
\end{verbatim}
the call
\begin{verbatim}
\expandafter\tokenone\tokentwo
\end{verbatim}
will give \cs{ifsomecondition} as the parameter
to \cs{tokenone}:
\begin{verbatim}
\tokenone #1-> ...
#1<-\ifsomecondition
\end{verbatim}
\item If the \cs{tokentwo} is a macro with one or more
parameters, sufficiently many subsequent tokens will be absorbed
to form the replacement text.\end{itemize}

\subsection{Total expansion: \protect\cs{edef}}
\label{expand:edef}

Macros are usually defined by \cs{def}, but for the cases where
one wants the replacement text to reflect current conditions
(as opposed to conditions at the time of the call),
there is an `expanding define', \csterm{edef}, which expands
everything in the replacement text, before assigning it to the
control sequence. 

\begin{example}
\begin{verbatim}
\edef\modedef{This macro was defined in
    `\ifvmode vertical\else \ifmmode math
     \else horizontal\fi\fi' mode}
\end{verbatim}
The mode tests will be executed at definition time, so the 
replacement text will be a single string.

As a more useful example, suppose that in a file that will be
\cs{input} the category code of the~\n@ will be changed.
One could then write
\begin{verbatim}
\edef\restorecat{\catcode`@=\the\catcode`@}
\end{verbatim}
at the start, and 
\begin{verbatim}
\restorecat
\end{verbatim}
at the end. See page~\pageref{store:cat}
for a fully worked-out version of this.
\end{example}

Contrary to the `one step expansion' of
\cs{expandafter}, the expansion inside an \cs{edef} is complete:
it goes on
until only unexpandable character and control sequence
tokens remain.
There are two exceptions to this total expansion:
\begin{itemize} \item any control sequence preceded by \cs{noexpand}
is not expanded, and,
\item if \cs{sometokenlist} is a token list, the expression
\begin{verbatim}
\the\sometokenlist 
\end{verbatim}
is expanded to the contents
of the list, but the contents are not expanded
any further (see Chapter~\ref{token} for examples).\end{itemize}

On certain occasions the \cs{edef} can conveniently be
abused, in the sense that one is not interested in defining
a control sequence, but only in the result of the  expansion.
For example, with the definitions
\alt
\begin{verbatim}
\def\othermacro{\ifnum1>0 {this}\else {that}\fi}
\def\somemacro#1{ ... }
\end{verbatim}
the call
\begin{verbatim}
\expandafter\somemacro\othermacro 
\end{verbatim}
gives the parameter assignment
\begin{verbatim}
#1<-\ifnum
\end{verbatim}
This can be repaired by calling
\begin{verbatim}
\edef\next{\noexpand\somemacro\othermacro}\next
\end{verbatim}
Conditionals are completely expanded inside an \cs{edef},
so the replacement text of \cs{next} will consist of the sequence
\begin{verbatim}
\somemacro{this}
\end{verbatim} 
and a~subsequent call to \cs{next} executes this statement.


%\spoint \cs{afterassignment}
\subsection{\protect\cs{afterassignment}}
\label{sec:afterassignment}

The \csterm{afterassignment}\ command
takes one token and sets it aside for insertion
in the token stream
after the next assignment or macro definition.
If the first assignment is of a~box
 to a box register,
the token will be inserted right after the opening
\alt
brace of the box (see page~\pageref{every:box:assign}).

Only one token can be saved this way; a subsequent token 
saved by \cs{afterassignment} will override the first.

Let us consider an example of the use of \cs{afterassignment}.
It is often desirable to have a macro that will
\begin{itemize} \item assign the argument to some variable, and then
\item do a little calculation, based on the new value
of the variable.\end{itemize}
The following example illustrates the
straightforward approach:
\begin{verbatim}
\def\setfontsize#1{\thefontsize=#1pt\relax
    \baselineskip=1.2\thefontsize\relax}
\setfontsize{10}
\end{verbatim}
A more elegant solution is possible using \cs{afterassignment}:
\begin{verbatim}
\def\setbaselineskip
   {\baselineskip=1.2\thefontsize\relax}
\def\fontsize{\afterassignment\setbaselineskip
    \thefontsize}
\fontsize=10pt
\end{verbatim}
Now the macro looks like an assignment: the equals sign
is even optional. In reality its expansion
ends with a variable to be assigned to. The control sequence
\cs{setbaselineskip} is saved for execution after
the assignment to \cs{thefontsize}.

Examples of \cs{afterassignment} in plain \TeX\ are
the \cs{magnification} and \cs{hglue} macros.
See \cite{Maus} for another creative application of
this command.

\subsection{\protect\cs{aftergroup}}
\label{sec:aftergroup}

Several tokens can be saved for insertion after the current
\csidx{aftergroup}%
group with an 
\begin{disp}\cs{aftergroup}\gr{token}\end{disp} command.
The tokens are inserted after the group in the sequence
the \cs{aftergroup} commands were given in.
The group can be delimited either by implicit or explicit
braces, or by \cs{begingroup} and \cs{endgroup}.

\begin{example}
\begin{verbatim}
{\aftergroup\a \aftergroup\b}
\end{verbatim}
is equivalent to
\begin{verbatim}
\a \b
\end{verbatim}
\end{example}

This command has many applications. One can be found
\alt
in the \cs{textvcenter} macro on page~\pageref{text:vcenter};
another one is provided
by the footnote mechanism of plain \TeX.

The footnote command of plain \TeX\ has the layout
\label{footnote:ex}
\begin{disp}\cs{footnote}\gr{footnote symbol}\lb\gr{footnote text}\rb
\end{disp} which looks like a macro with two arguments.
However, it is undesirable to scoop up the footnote text,
since this precludes for
instance category code changes in the footnote.

What happens in the plain footnote macro is (globally) the following.
\begin{itemize}\item The \cs{footnote} command opens
an insert,
\begin{verbatim}
\def\footnote#1{ ...#1... %treat the footnote sign
    \insert\footins\bgroup
\end{verbatim}
\item In the insert box a group is opened,
and an \cs{aftergroup} command
is given to close off the insert properly:
\begin{verbatim}
    \bgroup\aftergroup\@foot
\end{verbatim}
This command is meant to wind up after the closing brace of
the text that the user typed to end the footnote text;
the opening brace of the user's footnote text must
be removed by 
\begin{verbatim}
    \let\next=}%end of definition \footnote
\end{verbatim}
which assigns the next token, the brace, to \cs{next}.
\item The footnote text is set as ordinary text
in this insert box.
\item After the footnote the command \cs{@foot}
defined by
\begin{verbatim}
\def\@foot{\strut\egroup}
\end{verbatim}
will be executed.\end{itemize}


%\point Preventing expansion
\section{Preventing expansion}

Sometimes it is necessary to prevent expansion in a place
where it normally occurs. For this purpose the control
sequences \csterm{string} and \csterm{noexpand} are available.

The use of \cs{string} is rather limited, since it converts
a control sequence token into a string of characters, with
the value of \cs{escapechar} used for the character of
category code~0\index{category!0}.
It is eminently suitable for use in a 
\cs{write}, in order to output a control sequence name
(see also Chapter~\ref{io}); for another application see
the explanation of \cs{newif} in Chapter~\ref{if}.

All characters resulting from \cs{string} have category
code~12, `other', except for space characters; they receive
code~10. See also Chapter~\ref{char}.

%\spoint \cs{noexpand}
\subsection{\protect\cs{noexpand}}

The \cs{noexpand} command is expandable, and its expansion
is the following token. The meaning of that token is
made temporarily equal to \cs{relax}, so that it cannot
be expanded further.

For \cs{noexpand} the most important application is probably
in \cs{edef} commands (but in write statements it can often
replace \cs{string}). Consider as an example
\begin{verbatim}
 \edef\one{\def\noexpand\two{\the\prevdepth}}
\end{verbatim}
Without the \cs{noexpand}, \TeX\ would try to expand
\cs{two}, thus giving an `undefined control sequence' error.

A  (rather pointless)
illustration of the fact that \cs{noexpand} makes the following
token effectively into a \cs{relax} is
\begin{verbatim}
\def\a{b}
\noexpand\a
\end{verbatim}
This will not produce any output, because the
effect of the \cs{noexpand} is to make the control sequence
\cs{a} temporarily equal to \cs{relax}.

%\spoint \cs{noexpand} and active characters
\subsection{\protect\cs{noexpand} and active characters}

The combination \cs{noexpand}\gr{token} 
is\index{character!active, and \cs{noexpand}}
equivalent to \cs{relax}, even if the token
is an active character. Thus,
\begin{verbatim}
\csname\noexpand~\endcsname
\end{verbatim}
will not be the same as~\verb>\char`\~>.
Instead it will give an error message, because
unexpandable commands \ldash such as \cs{relax} \rdash  are not allowed to appear
in between \cs{csname} and \cs{endcsname}.
The solution is to use \cs{string} instead; see page~\pageref{store:cat}
for an example.

In another context, however, the sequence
\cs{noexpand}\gr{active character} is equivalent
to the character, but in unexpandable form. This is
when the conditionals \cs{if} and \cs{ifcat} are used
(for an explanation of these, see Chapter~\ref{if}).
Compare
\begin{verbatim}
\if\noexpand~\relax % is false
\end{verbatim}
where the character code of the tilde is tested, with
\begin{verbatim}
\def\a{ ... } \if\noexpand\a\relax % is true
\end{verbatim}
where two control sequences are tested.

%\point \cs{relax}
\section{\protect\cs{relax}}

The control sequence \csterm{relax} cannot be expanded, but
when it is executed nothing happens.

This statement sounds a bit paradoxical, so  consider
an example. Let  counters
\begin{verbatim}
\newcount\MyCount
\newcount\MyOtherCount \MyOtherCount=2
\end{verbatim}
be given.
In the assignment
\begin{verbatim}
\MyCount=1\number\MyOtherCount3\relax4
\end{verbatim}
the command \cs{number} is expandable, and \cs{relax} is not.
When \TeX\ constructs the number that is to be assigned
it will expand all commands, either until a non-digit is 
found, or until an unexpandable command is encountered.
Thus it reads the~\n1; it expands the sequence \verb>\number\MyOtherCount>,
which gives~\n2; it reads the~\n3; it sees the \cs{relax}, and
as this is unexpandable it halts. The number to be assigned
is then \n{123}, and the whole call has been expanded into
\begin{verbatim}
\MyCount=123\relax4
\end{verbatim}
Since the \cs{relax} token has no effect when it is executed,
the result of this line is that \n{123} is assigned to
\cs{MyCount}, and the digit 4 is printed.

Another example of how \cs{relax} can be used to indicate
the end of a command\label{fil:l:l}\ is
\begin{verbatim}
\everypar{\hskip 0cm plus 1fil }
\indent Later that day, ... 
\end{verbatim}
This will be misunderstood: \TeX\ will see
\begin{verbatim}
\hskip 0cm plus 1fil L
\end{verbatim}
and \hbox{\n{fil L}} is a~valid,
if bizarre,
way of writing \n{fill} (see Chapter~\ref{gramm}).
One remedy is to write
\begin{verbatim}
\everypar{\hskip 0cm plus 1fil\relax}
\end{verbatim}

%\spoint[relax:cs] \cs{relax} and \cs{csname}
\subsection{\cs{relax} and \cs{csname}}
\label{relax:cs}

If a \verb-\csname ... \endcsname- command forms the name
of a previously undefined control sequence,
that control sequence is made equal to \cs{relax},
and the whole statement is also equivalent to \cs{relax}
(see also page~\pageref{cs:name}).

However,  this assignment of \cs{relax} is
\altt
only local:
\begin{verbatim}
{\xdef\test{\expandafter\noexpand\csname xx\endcsname}}
\test
\end{verbatim}
gives an error message for an
undefined control sequence~\cs{xx}.

Consider as an example the \LaTeX\ environments,
which are delimited by
\begin{verbatim}
\begin{...} ... \end{...}
\end{verbatim}
The begin and end commands are (in essence)
defined as follows:
\begin{verbatim}
\def\begin#1{\begingroup\csname#1\endcsname}
\def\end#1{\csname end#1\endcsname \endgroup}
\end{verbatim}
Thus, for the list environment the commands
\cs{list} and \cs{endlist} are defined, but any
command can be used as an environment name,
even if no corresponding \cs{end...} has been defined.
For instance,
\begin{verbatim}
\begin{it} ... \end{it}
\end{verbatim}
is equivalent to 
\begin{verbatim}
\begingroup\it ... \relax\endgroup
\end{verbatim}
See page~\pageref{begin:end:macros} for the rationale
behind using \cs{begingroup} and \cs{endgroup}
instead of \cs{bgroup} and \cs{egroup}.

%\spoint Preventing expansion with \cs{relax}
\subsection{Preventing expansion with \cs{relax}}

Because \cs{relax} 
cannot be expanded, a control sequence can be prevented
from being expanded (for instance in an \cs{edef} or a \cs{write})
by making it temporarily equal to \cs{relax}:
\begin{verbatim}
{\let\somemacro=\relax \write\outfile{\somemacro}}
\end{verbatim}
will write the string `\cs{somemacro}' to an output file.
It would write the expansion 
of the macro \cs{somemacro} (or give an error message
if the macro is undefined) if the \cs{let} statement
had been omitted.

%\spoint[bump:relax] \TeX\ inserts a \cs{relax}
\subsection{\TeX\ inserts a \cs{relax}}
\label{bump:relax}

\TeX\ itself inserts \cs{relax} on some occasions.
For instance, \cs{relax} is inserted if \TeX\ encounters an
\cs{or}, \cs{else}, or~\cs{fi} while still determining
the extent of the test. 
\begin{example}
\begin{verbatim}
\ifvoid1\else ... \fi
\end{verbatim}
is changed into
\begin{verbatim}
\ifvoid1\relax \else ...\fi
\end{verbatim}
internally.
\end{example}

Similarly, if one of the tests \cs{if}, \cs{ifcat}
is given only one comparand, as in
\begin{verbatim}
\if1\else ...
\end{verbatim}
a \cs{relax} token is inserted. Thus this test
is equivalent to
\begin{verbatim}
\if1\relax\else ...
\end{verbatim}

Another place where \cs{relax} is used is the following.
While a control sequence is being defined in a \gr{shorthand
definition} \ldash that is, a \gr{registerdef} or \cs{chardef}
or \cs{mathchardef} \rdash  its meaning is temporarily made
equal to \cs{relax}. This makes it possible to write
\verb>\chardef\foo=123\foo>.

%\spoint The value of non-macros; \cs{the}
\subsection{The value of non-macros; \cs{the}}

Expansion is a precisely defined activity in \TeX.
\csidx{the}%
The full list of tokens that can be expanded
was given above. 
Other tokens than those in the above list may have an `expansion'
in an informal sense. For instance one may wish to `expand'
the \cs{parindent} into its value, say~\n{20pt}.

Converting  the value of (among others) an
\gr{integer parameter}, a \gr{glue parameter}, 
\gr{dimen parameter} or a \gr{token parameter}
into a string of character tokens is done by the expansion processor.
The command \cs{the}
is expanded whenever expansion is not inhibited,
and it takes the value of various sorts of parameters.
Its result (in most cases)
is a string of tokens of category~12\index{category!12}, except
that spaces have category code~10\index{category!10}.

Here is the list of everything that can be prefixed with \cs{the}.
\begin{description}\item [\gr{parameter} or \gr{register}]
If the parameter or register is of type integer, glue, dimen
or muglue,
its value is given as a string of character tokens;
if it is of type token list (for instance
\cs{everypar} or \cs{toks5}), the result is a string of tokens.
Box registers are excluded here.
\item [\gr{codename}\gr{8-bit number}]
See page~\pageref{codename}.
\item [\gr{special register}]
The integer registers \cs{prevgraf}, \cs{deadcycles}, \cs{insertpenalties}
\cs{inputlineno}, \cs{badness}, \cs{parshape}, \cs{spacefactor}
(only in horizontal mode), or \cs{prevdepth} (only in vertical mode).
The dimension registers \cs{pagetotal}, \cs{pagegoal}, \cs{pagestretch},
\handbreak \cs{pagefilstretch}, \cs{pagefillstretch}, \cs{pagefilllstretch},
\cs{pageshrink}, or \cs{pagedepth}.
\item [Font properties:]
\cs{fontdimen}\gr{parameter number}\gr{font}, 
\cs{skew\-char}\gr{font},
\cs{hy\-phen\-char}\gr{font}.
\item [Last quantities:]
\cs{lastpenalty}, \cs{lastkern}, \cs{lastskip}.
\item [\gr{defined character}]
Any control sequence defined by \cs{chardef} or \cs{mathchardef};
the result is the decimal value.
\end{description}
In some cases \cs{the} can give a control sequence token 
or list of such tokens.
\begin{description}\item [\gr{font}]
The result is the control sequence that stands for the
font.
\item [\gr{token variable}]
Token list registers and \gr{token parameter}s can be prefixed
with \cs{the}; the result is their contents.
\end{description}

Let us consider an example of the use of \cs{the}.
If in a file that is to be \cs{input} the
category code of a character, say the at~sign, is changed,
one could write
\begin{verbatim}
\edef\restorecat{\catcode`@=\the\catcode`@}
\end{verbatim}
and call \cs{restorecat} at the end of the file.
If the category code was~11, \cs{restorecat}
is defined equivalent to
\begin{verbatim}
\catcode`@=11
\end{verbatim}
See page~\pageref{store:cat} for more elaborate macros
for saving and restoring catcodes.


%\point Examples
\section{Examples}

%\spoint Expanding after
\subsection{Expanding after}

The most obvious use of \cs{expandafter} is to reach over
a control sequence:
\begin{verbatim}
\def\stepcounter
    #1{\expandafter\advance\csname 
             #1:counter\endcsname 1\relax}
\stepcounter{foo}
\end{verbatim}
Here the \cs{expandafter} lets the \cs{csname} command form
the control sequence \cs{foo:counter}; after \cs{expandafter}
is finished the statement has reduced to
\begin{verbatim}
\advance\foo:counter 1\relax
\end{verbatim}
It is possible to reach over tokens other than control sequences: in
\begin{verbatim}
\uppercase\expandafter{\romannumeral \year}
\end{verbatim}
it expands  \cs{romannumeral} on the other side of the opening
brace.

You can expand after two control sequences:
\begin{verbatim}
\def\globalstepcounter
    #1{\expandafter\global\expandafter\advance
             \csname #1:counter\endcsname 1\relax}
\end{verbatim}
If you think of \cs{expandafter} as reversing the evaluation
order of {\sl two\/} control sequences, you can reverse
{\sl three\/} by
\begin{verbatim}
\expandafter\expandafter\expandafter\a\expandafter\b\c 
\end{verbatim}
which reaches across the three control sequences
\begin{verbatim}
            \expandafter            \a            \b 
\end{verbatim} 
to expand \cs{c} first.

There is even an unexpected use for \cs{expandafter} in
conditionals;
with
\begin{verbatim}
\def\bold#1{{\bf #1}}
\end{verbatim}
the sequence
\begin{verbatim}
\ifnum1>0 \bold \fi {word}
\end{verbatim}
will not give a boldface `word', but
\begin{verbatim}
\ifnum1>0 \expandafter\bold \fi {word}
\end{verbatim}
will.
The \cs{expandafter} lets \TeX\ see the \cs{fi} and remove it
before it tackles the macro \cs{bold}
(see also page~\pageref{after:cond}).

%\spoint Defining inside an \cs{edef}
\subsection{Defining inside an \cs{edef}}

There is one \TeX\ command that is executed instead of
expanded that is worth pointing out explicitly:
the primitive command \cs{def} (and all other \gr{def} commands)
is not expanded.

Thus the call
\begin{verbatim}
\edef\next{\def\thing{text}}
\end{verbatim}
will give an `undefined
control sequence' for \cs{thing}, even though after
\cs{def} expansion is ordinarily inhibited (see page~\pageref{noexp:list}).
After
\begin{verbatim}
\edef\next{\def\noexpand\thing{text}}
\end{verbatim}
the `meaning' of \cs{next} will be
\begin{verbatim}
macro: \def \thing {text}
\end{verbatim}
The definition
\begin{verbatim}
\edef\next{\def\noexpand\thing{text}\thing}
\end{verbatim}
will again give an `undefined control sequence' for \cs{thing}
(this time on its second occurrence),
as it will only be defined when \cs{next} is called,
not when \cs{next} is defined.


%\spoint[expand:write] Expansion and \cs{write}
\subsection{Expansion and \cs{write}}
\label{expand:write}

The argument token list of \csterm{write} is treated in much
the same way as the replacement text of an \cs{edef};
that is, expandable control sequences and active characters
are completely expanded. Unexpandable control sequences
are treated by \cs{write} as if they are prefixed
by \cs{string}.

Because of the expansion performed by \cs{write},
some care has to be taken when outputting control
sequences with \cs{write}.
Even more complications arise from the fact that 
the expansion of the argument of \cs{write} is only performed
when it is shipped out. Here follows a worked-out
example.

Suppose \cs{somecs} is a macro, and you
want to write the string 
\begin{disp}\verb-\def\othercs-\lb {\italic the expansion of \cs{somecs}}\rb
\end{disp}
to a file.

The first attempt is
\begin{verbatim}
\write\myfile{\def\othercs{\somecs}}
\end{verbatim}
This gives an error `undefined control sequence' for \cs{othercs},
\altt
because the \cs{write} will try to expand that token. 
Note that the \cs{somecs} is also  expanded, 
so that part is right.

The next attempt is
\begin{verbatim}
\write\myfile{\def\noexpand\othercs{\somecs}}
\end{verbatim}
This is almost right, but not quite. The
statement written is
\begin{disp}\verb>\def\othercs>\lb{\italic expansion of \cs{somecs}}\rb\end{disp}
which looks right.

However, writes \ldash and the expansion of their argument \rdash 
are not executed
on the spot, but saved until the part of the page on which
they occur is shipped out (see Chapter~\ref{io}).
So, in the meantime, the value of \cs{somecs} may have
changed. In other words, the value written may not be the
value at the time the \cs{write} command was given.
Somehow, therefore, the current expansion must be
inserted in the write command.

The following is an attempt at repair:
\begin{verbatim}
\edef\act{\write\myfile{\def\noexpand\othercs{\somecs}}}
\act
\end{verbatim} 
Now the  write command will be
\begin{disp}\verb>\write\myfile{\def\othercs{>\italic value of\/
     \verb>\somecs}}>\end{disp}
The \cs{noexpand} prevented the \cs{edef} from expanding
the \cs{othercs}, but after the definition it has disappeared,
so that execution of the write will again give an undefined control
sequence. The final solution is
\begin{verbatim}
\edef\act{\write\myfile
          {\def \noexpand\noexpand \noexpand\othercs{\somecs}}}
\act
\end{verbatim} 
In this case the write command caused by the expansion of \cs{act}
will be
\begin{disp}\verb>\write\myfile{\def\noexpand\othercs>\lb
     {\italic current value of \cs{somecs}}\rb\end{disp}
and the string actually written is
\begin{disp}\verb>\def\othercs>\lb
     {\italic current value of \cs{somecs}}\rb\end{disp}
This mechanism is the basis for cross-referencing
macros in several macro packages.


%\spoint Controlled expansion inside an \cs{edef}
\subsection{Controlled expansion inside an \cs{edef}}

Sometimes you may need an \cs{edef} to evaluate current
\howto Control expansion inside an \cs{edef}\par
conditions, but you want to expand something in the replacement
text only to a certain level. Suppose that
\begin{verbatim}
\def\a{\b} \def\b{c} \def\d{\e} \def\e{f}
\end{verbatim}
is given, and you want to define \cs{g} as \cs{a} expanded
one step, followed by \cs{d} fully expanded. The following
works:
\begin{verbatim}
\edef\g{\expandafter\noexpand\a \d}
\end{verbatim}
Explanation: the \cs{expandafter} reaches over the \cs{noexpand}
to expand \cs{a} one step, after which the 
sequence \verb-\noexpand\b- is left.

This trick comes in handy when you need to
construct a control sequence with \cs{csname} inside 
an \cs{edef}. The following sequence inside an \cs{edef}
\begin{verbatim}
\expandafter\noexpand\csname name\endcsname
\end{verbatim}
will expand exactly to \cs{name}, but not further.
As an example, suppose
\begin{verbatim}
\def\condition{true}
\end{verbatim}
has been given, then
\begin{verbatim}
\edef\setmycondition{\expandafter\noexpand
           \csname mytest\condition\endcsname}
\end{verbatim}
will let \cs{setmycondition} expand to \cs{mytesttrue}.

%\spoint Multiple prevention of expansion
\subsection{Multiple prevention of expansion}

As   was pointed out above, prefixing  a command with
\cs{noexpand} prevents its expansion in commands
such as \cs{edef} and~\cs{write}. However, if a sequence of tokens
passes through more than one expanding command
stronger measures are needed.

The following trick can be used:
in order to protect a command against expansion
it can be prefixed with \csterm{protect}.
During the stages of processing where expansion is
not desired the definition of \cs{protect} is
\begin{verbatim}
\def\protect{\noexpand\protect\noexpand}
\end{verbatim}
Later on, when the command is actually needed,
\cs{protect} is defined as 
\begin{verbatim}
\def\protect{}
\end{verbatim}

Why does this work? The expansion of
\begin{verbatim}
\protect\somecs
\end{verbatim}
is at first
\begin{verbatim}
\noexpand\protect\noexpand\somecs
\end{verbatim}
Inside an \cs{edef} this sequence is expanded further,
and the subsequent expansion is
\begin{verbatim}
\protect\somecs
\end{verbatim}
That is, the expansion is equal to the original sequence.


%\spoint More examples with \cs{relax}
\subsection{More examples with \cs{relax}}

Above, a first example was given in which \cs{relax} served
to prevent \TeX\ from scanning too far.
Here are some more examples, using \cs{relax} to bound
numbers.

After
\begin{verbatim}
\countdef\pageno=0 \pageno=1
\def\Par{\par\penalty200}
\end{verbatim}
the sequence
\begin{verbatim}
\Par\number\pageno
\end{verbatim}
is misunderstood as
\begin{verbatim}
\par\penalty2001
\end{verbatim}
In this case it is sufficient to define
\begin{verbatim}
\def\Par{\par\penalty200 }
\end{verbatim}
as an \gr{optional space} is allowed to follow a number.

Sometimes, however, such a simple escape is not possible.
Consider the definition
\begin{verbatim}
\def\ifequal#1#2{\ifnum#1=#2 1\else 0\fi}
\end{verbatim}
The question is whether the space after \verb-#2-
is necessary, superfluous, or simply wrong.
Calls such as \verb-\ifequal{27}{28}- that compare two
numbers (denotations) will correctly give \n1 or~\n0,
and the space is necessary to prevent misinterpretation.

However, \verb-\ifequal\somecounter\othercounter- will
give \n{\char 32 1} if the counters are equal; in this
case the space could have been dispensed with.
The solution that works in both cases is
\begin{verbatim}
\def\ifequal#1#2{\ifnum#1=#2\relax 1\else 0\fi}
\end{verbatim}
Note that \cs{relax} is not expanded, so
\begin{verbatim}
\edef\foo{1\ifequal\counta\countb}
\end{verbatim}
will define \cs{foo} as either \verb-1\relax1- or~\n{10}.

\subsection{Example: category code saving and restoring}
\label{store:cat}

In many applications it is necessary to change
\howto Save and restore category codes\par
the category code of a certain character during the
execution of some piece of code. If the writer of
that code is also the writer of the surrounding code,
s/he can simply change the category code back and forth.
However, if the surrounding code is by another author,
the value of the category code will have to be stored
and restored.

Thus one would like to write
\begin{verbatim}
\storecat@
... some code ...
\restorecat@
\end{verbatim}
or maybe
\begin{verbatim}
\storecat\%
\end{verbatim}
for characters that
are possibly a comment character (or ignored or invalid).
\alt
The basic idea is to define
\begin{verbatim}
\def\storecat#1{%
    \expandafter\edef\csname restorecat#1\endcsname
        {\catcode`#1=\the\catcode`#1}}
\end{verbatim}
so that, for instance, \verb>\storecat$> will define
the single control sequence `\verb>\restorecat$>'
(one control sequence) as
\begin{verbatim}
\catcode`$=3
\end{verbatim}
The macro \cs{restorecat} can then be implemented as
\begin{verbatim}
\def\restorecat#1{%
    \csname restorecat#1\endcsname}
\end{verbatim}
Unfortunately, things are not so simple.

The problems occur with active characters, because these
are expanded inside the \cs{csname} \verb>...> \cs{endcsname} pairs.
One might be tempted to write \verb>\noexpand#1> everywhere,
but this is wrong. As was explained above, this is essentially
equal to \cs{relax}, which is unexpandable, and will therefore
lead to an error message when it appears between
\cs{csname} and \cs{endcsname}. The proper solution is then
to use \verb>\string#1>. For the case where the argument
was given as a control symbol (for example~\verb>\%>),
the escape character has to be switched off for a while.
 
Here are the complete macros. The \cs{storecat} macro
gives its argument a default category code of~12.
\begin{verbatim}
\newcount\tempcounta % just a temporary
\def\csarg#1#2{\expandafter#1\csname#2\endcsname}
\def\storecat#1%
   {\tempcounta\escapechar \escapechar=-1
    \csarg\edef{restorecat\string#1}%
          {\catcode`\string#1=
                \the\catcode\expandafter`\string#1}%
    \catcode\expandafter`\string#1=12\relax
    \escapechar\tempcounta}
\def\restorecat#1%
   {\tempcounta\escapechar \escapechar=-1
    \csname restorecat\string#1\endcsname
    \escapechar\tempcounta}
\end{verbatim}

%\spoint Combining \cs{aftergroup} and boxes
\subsection{Combining \cs{aftergroup} and boxes}

At times, one wants to construct a box and immediately
after it has been constructed to
do something with it. The \cs{aftergroup} command
can be used to put both the commands creating the box,
and the ones handling it, in one macro.

As an example, here is a macro 
\cs{textvcenter}\label{text:vcenter}\
which defines a variant of the \cs{vcenter} box
\howto \cs{vcenter} outside math mode\par
(see page~\pageref{vcenter}\label{tvcenter})
that can be used outside math mode.
\begin{verbatim}
\def\textvcenter
   {\hbox \bgroup$\everyvbox{\everyvbox{}%
    \aftergroup$\aftergroup\egroup}\vcenter}
\end{verbatim}
The idea is that the macro inserts \verb>\hbox {$>,
and that the matching \verb>$}> gets inserted
by the \cs{aftergroup} commands. In order to get the 
\cs{aftergroup} commands inside the box, an
\cs{everyvbox} command is  used.

This macro can even be used with a \gr{box specification}
(see page~\pageref{box:spec}), for example
\begin{verbatim}
\textvcenter spread 8pt{\hbox{a}\vfil\hbox{b}}
\end{verbatim}
and because it  is really just an \cs{hbox}, it can also
be used in a \cs{setbox} assignment.

\subsection{More expansion}

There is a particular charm to macros that work
purely by expansion. See the articles by
\cite{E2}, \cite{Jeffrey:lists}, and~\cite{Maus2}.

%%%% end of input file [expand]

%\InputFile:ifelsefi
%%%% this is input file [ifelsefi]
%\subject[if] Conditionals
\endofchapter
\chapter{Conditionals}\label{if}

\emph{Conditionals}\index{conditionals}
are an indispensible tool for powerful macros.
\TeX\ has a large repertoire of conditionals for querying
such things as category codes or processing modes.
This chapter gives an inventory of the various conditionals,
and it treats the evaluation of
conditionals in detail.

\label{cschap:if}\label{cschap:ifcat}\label{cschap:ifx}\label{cschap:ifcase}\label{cschap:ifnum}\label{cschap:ifodd}\label{cschap:ifhmode}\label{cschap:ifvmode}\label{cschap:ifmmode}\label{cschap:ifinner}\label{cschap:ifdim}\label{cschap:ifvoid}\label{cschap:ifhbox2}\label{cschap:ifvbox2}\label{cschap:iftrue}\label{cschap:iffalse}\label{cschap:fi}\label{cschap:else}\label{cschap:or}\label{cschap:newif}
\begin{inventory}
\item [\cs{if}] 
      Test equality of character codes. 

\item [\cs{ifcat}] 
      Test equality of category codes. 

\item [\cs{ifx}] 
      Test equality of macro expansion, or equality of character code and
      category code.

\item [\cs{ifcase}] 
      Enumerated case statement.

\item [\cs{ifnum}] 
      Test relations between numbers.

\item [\cs{ifodd}] 
      Test whether a number is odd.

\item [\cs{ifhmode}] 
      Test whether the current mode is (possibly restricted) horizontal mode.

\item [\cs{ifvmode}] 
      Test whether the current mode is (possibly internal) vertical mode.

\item [\cs{ifmmode}] 
     Test whether the current mode is (possibly display) math mode.

\item [\cs{ifinner}] 
      Test whether the current mode is an internal mode.

\item [\cs{ifdim}] 
      Compare two dimensions. 

\item [\cs{ifvoid}] 
      Test whether a box register is empty.

\item [\cs{ifhbox}] 
      Test whether a box register contains a horizontal box.

\item [\cs{ifvbox}] 
      Test whether a box register contains a vertical box. 

\item [\cs{ifeof}] 
      Test for end of input stream or non-existence of file.

\item [\cs{iftrue}] 
      A test that is always true.
\item [\cs{iffalse}] 
      A test that is always false.
\item [\cs{fi}] 
      Closing delimiter for all conditionals.

\item [\cs{else}] 
      Select \gr{false text} of a conditional 
      or default case of \cs{ifcase}.

\item [\cs{or}] 
      Separator for entries of an \cs{ifcase}.

\item [\cs{newif}] 
     Create a new test.

\end{inventory}

%\point The shape of conditionals
\section{The shape of conditionals}

Conditionals in \TeX\ have one of the following two forms
\csterm{else}%
\csterm{fi}%
\begin{disp}\cs{if...}\gr{test tokens}\gr{true text}\cs{fi}\nl
     \cs{if...}\gr{test tokens}\gr{true text}\cs{else}%
     \gr{false text}\cs{fi}\end{disp}
where the \gr{test tokens} are zero or more tokens, depending on
the particular conditional; the \gr{true text} is a series of tokens
to be processed if the test turns out true, and the \gr{false text}
is a series of tokens to be processed if the test turns out false.
Both the \gr{true text} and the \gr{false text} can be empty.

The exact process of how \TeX\ expands conditionals is treated
below.


\section{Character and control sequence tests}
\label{sec:charactertests}

Three tests exist for testing character tokens and
control sequence tokens.


\subsection{\cs{if}}

Equality of character codes can be tested by
\csterm{if}%
\begin{Disp}\cs{if}\gr{token$_1$}\gr{token$_2$}\end{Disp}
In order to allow the tokens to be control sequences,
\TeX\ assigns character code~256 to control sequences,
the lowest positive number that is not the character code of a
character token (remember that the legal character codes
are~0--255).

Thus all control sequences are equal as far as \cs{if} is
concerned, and they are unequal to all character tokens.
As an example, this fact can be used to define
\howto Test whether a token is a control sequence\par
\begin{verbatim}
\def\ifIsControlSequence#1{\if\noexpand#1\relax}
\end{verbatim}
which tests whether a token is a control sequence token
instead of a character token (its result is unpredictable
if the argument is a \verb>{...}> group).

After \cs{if} \TeX\ will expand until two unexpandable
tokens are obtained, so it is necessary to prefix
expandable control sequences and active characters
with \cs{noexpand} when testing them with~\cs{if}.

\begin{example} After
\begin{verbatim}
\catcode`\b=13 \catcode`\c=13 \def b{a} \def c{a} \let\d=a
\end{verbatim}
we find that
\begin{tdisp}
\verb-\if bc- is true, because both \n b and \n c expand to \n a,\nl
\verb-\if\noexpand b\noexpand c- is false, and\nl
\verb-\if b\d- is true because \n{b} expands to the character~\n{a},
   and \cs{d} is an implicit character token~\n{a}.
\end{tdisp}
\end{example}

%\spoint \cs{ifcat}
\subsection{\cs{ifcat}}

The \cs{if} test ignores category codes; these can be tested
\csterm{ifcat}%
by 
\begin{Disp}\cs{ifcat}\gr{token$_1$}\gr{token$_2$}\end{Disp}

This test is a lot like \cs{if}: \TeX\ expands after it
until unexpandable tokens remain. For this test
control sequences
are considered to have category code~16\index{category!16}
(ordinarily, category codes are in the range~0--15), which makes them
all equal to each other, and different from all character
tokens.

%\spoint \cs{ifx}
\subsection{\protect\cs{ifx}}

Equality of tokens is tested in a stronger sense than
\csterm{ifx}%
the above by 
\begin{Disp}\cs{ifx}\gr{token$_1$}\gr{token$_2$}\end{Disp}

\begin{itemize}\item Character tokens are equal for \cs{ifx} if
they have the same character code and category code.
    \item Control sequence tokens are equal if they represent the
same \TeX\ primitive, or have been similarly defined by
\cs{font}, \cs{countdef}, or some such. For example,
\begin{verbatim}
\let\boxhor=\hbox \ifx\boxhor\hbox %is true
\font\a=cmr10 \font\b=cmr10 \ifx\a\b %is true
\end{verbatim}
\item Control sequences are also equal if they are
macros with the same parameter text and replacement text,
and the same status with respect to \cs{outer} and~\cs{long}.
For example,
\begin{verbatim}
\def\a{z} \def\b{z} \def\c1{z} \def\d{\a}
\ifx\a\b %is true
\ifx\a\c %is false
\ifx\a\d %is false
\end{verbatim}
\end{itemize}

Tokens following this test are not expanded.

By way of example of the use of \cs{ifx} consider string testing.
A simple implementation of string testing in \TeX\ is as follows:
\begin{verbatim}
\def\ifEqString#1#2{\def\testa{#1}\def\testb{#2}%
    \ifx\testa\testb}
\end{verbatim}
The two strings are used as the replacement text of two macros,
and equality of these macros is tested.
This is about as efficient as string testing can get:
\TeX\ will traverse the definition texts of the
macros \cs{testa} and \cs{testb}, which has precisely the
right effect.

As another example, one can test whether a control sequence is defined
by\howto Test whether a control sequence is (un)defined\par
\begin{verbatim}
\def\ifUnDefinedCs#1{\expandafter
    \ifx\csname#1\endcsname\relax}
\ifUnDefinedCs{parindent} %is not true
\ifUnDefinedCs{undefined} %is (one hopes) true
\end{verbatim}
This uses the fact that a \verb>\csname...\endcsname> command
is equivalent to \cs{relax} if the control sequence
has not been defined before. Unfortunately, this test also
turns out true if a control sequence has been \cs{let} to
\cs{relax}.


\section{Mode tests}

In order to determine in which of the six modes 
(see Chapter~\ref{hvmode}) \TeX\ 
is currently operating, the tests \csterm{ifhmode},
\csterm{ifvmode}, \csterm{ifmmode}, and~\csterm{ifinner}
are available.

\begin{itemize}\item\cs{ifhmode} is true if \TeX\ is in horizontal mode
or restricted horizontal mode.
\item\cs{ifvmode} is true if \TeX\ is in vertical mode or
internal vertical mode.
\item\cs{ifmmode} is true if \TeX\ is in math mode or display
math mode.\end{itemize}

The \cs{ifinner} test is true if \TeX\ is in any of the three
internal modes: restricted horizontal mode, internal vertical
mode, and non-display math mode.

See also chapter~\ref{chap:hvmode}.

\section{Numerical tests}

Numerical relations between \gr{number}s can be tested
\csterm{ifnum}%
with 
\begin{disp}\cs{ifnum}\gr{number$_1$}\gr{relation}%
\gr{number$_2$}\end{disp}
where the relation is a character \n{<}, \n{=}, or~\n{>},
of category~12\index{category!12}.

Quantities such as glue can be used as a number here
through the conversion to scaled points, and \TeX\
will expand in order to arrive at the two \gr{number}s.

Testing for odd or even numbers can be done with \csterm{ifodd}:
the test
\begin{disp}\cs{ifodd}\gr{number}\end{disp}
is true if the \gr{number} is odd.

%\point Other tests
\section{Other tests}

%\spoint Dimension testing
\subsection{Dimension testing}

Relations between \gr{dimen} values (Chapter~\ref{glue})
can be tested with
\csterm{ifdim} using the same three relations as in \cs{ifnum}.

%\spoint Box tests
\subsection{Box tests}

Contents of box registers (Chapter~\ref{boxes}) can be tested with
\csterm{ifhbox}%
\csterm{ifvbox}%
\csterm{ifvoid}%
\begin{disp}\cs{ifvoid}\gr{8-bit number}\end{disp}
which is true if the register contains no box,
\begin{disp}\cs{ifhbox}\gr{8-bit number}\end{disp}
which is true if the register contains a horizontal box, and
\begin{disp}\cs{ifvbox}\gr{8-bit number}\end{disp}
which is true if the register contains a vertical box.

%\spoint I{/}O tests
\subsection{I{/}O tests}

The status of input streams (Chapter~\ref{io}) can be tested with
\csterm{ifeof}%
the end-of-file test
\cs{ifeof}\gr{number}, which is only false
if the number is in the range 0--15, and the corresponding
stream is open and not fully read. In particular, this test
is true if the file name connected
to this stream (through \cs{openin})
does not correspond to an existing file.
See the example on page~\pageref{ex:eof}.

%\spoint Case statement
\subsection{Case statement}

The \TeX\ case statement is called \cs{ifcase};
\csterm{ifcase}%
\csterm{or}%
its syntax is
\begin{disp}\cs{ifcase}\gr{number}\gr{case$_0$}\cs{or}%
\n{...}\cs{or}\gr{case$_n$}\cs{else}\gr{other cases}\cs{fi}
\end{disp} where for $n$ cases there are $n-1$ \cs{or}
control sequences. Each of the \gr{case$_i$}
parts can be empty,
and the \cs{else}\gr{other cases} part is optional.

%\spoint Special tests
\subsection{Special tests}

The tests \cs{iftrue} and \cs{iffalse} are always
\csterm{iftrue}%
\csterm{iffalse}%
true and false respectively.
They are mainly useful as tools in macros.

For instance, the sequences
\begin{verbatim}
\iftrue{\else}\fi
\end{verbatim}
and
\begin{verbatim}
\iffalse{\else}\fi
\end{verbatim}
yield a left and right
brace respectively, but they have balanced braces, so they
can be used inside a macro replacement text.

The \cs{newif} macro, treated below,
provides another use of \cs{iftrue} and \cs{iffalse}.
On page 260 of \TeXbook\ these control sequences
are also used in an interesting manner.

%\point[newif:def] The \cs{newif} macro
\section{The \protect\cs{newif} macro}
\label{newif:def}

The plain format defines an (outer) macro \csterm{newif} by
which the user can define new conditionals.
If the user defines
\begin{verbatim}
\newif\iffoo
\end{verbatim}
\TeX\ defines three new control sequences,
\cs{footrue} and \cs{foofalse} with which the user can set
the condition, and \cs{iffoo} which tests the `foo' condition.

The macro call \verb-\newif\iffoo- expands to
\begin{verbatim}
\def\footrue{\let\iffoo=\iftrue} \def\foofalse{\let\iffoo=\iffalse}
\foofalse
\end{verbatim}
The actual definition, especially the part that ensures that
the \cs{iffoo} indeed starts with \cs{if}, is a pretty hack.
An explanation follows here.
This uses concepts from Chapters~\ref{macro}
and~\ref{expand}.

The macro \cs{newif} starts as follows:
\begin{verbatim}
\outer\def\newif#1{\count@\escapechar \escapechar\m@ne
\end{verbatim}
This saves the current escape character in \cs{count@}, and
sets the value of \cs{escapechar} to~\n{-1}.
The latter action has the
effect that no escape character is used in the output
of \cs{string}\gr{control sequence}.

An auxiliary macro \verb>\if@> is defined by
\begin{verbatim}
{\uccode`1=`i \uccode`2=`f \uppercase{\gdef\if@12{}}}
\end{verbatim}
Since the uppercase command changes only character codes, and
not category codes, the macro \cs{if@} now has
to be followed by the characters \n{if} of category~12\index{category!12}.
Ordinarily, these characters have category code~11\index{category!11}.
In effect this
macro then eats these two characters, and \TeX\ complains if
they are not present.

Next there is a macro \verb>\@if> defined by
\begin{verbatim}
\def\@if#1#2{\csname\expandafter\if@\string#1#2\endcsname}
\end{verbatim}
which will be called like \verb>\@if\iffoo{true}> and
\verb>\@if\iffoo{false}>. 

Let us examine the call \verb>\@if\iffoo{true}>.
\begin{itemize}\item The \cs{expandafter} reaches over the \verb>\if@>
\mdqon
to expand \cs{string} first. The part \cs{string}""\cs{iffoo}
\mdqoff
expands to \n{iffoo} because the escape character is not printed,
and all characters have category~12.
\item The \verb>\if@> eats the first two characters
\n i$_{12}$\n f$_{12}$ of this.
\item As a result, the final expansion of \verb>\@if\iffoo{true}>
is then
\begin{verbatim}
\csname footrue\endcsname
\end{verbatim}
\end{itemize}

Now we can treat the relevant parts of \cs{newif} itself:
\begin{verbatim}
\expandafter\expandafter\expandafter
   \edef\@if#1{true}{\let\noexpand#1=\noexpand\iftrue}%
\end{verbatim}

The three \cs{expandafter} commands may look intimidating, so let us
take one step at a time.
\begin{itemize}\item One \cs{expandafter} is necessary to reach over the \cs{edef},
such that \verb>\@if> will expand:
\begin{verbatim}
\expandafter\edef\@if\iffoo{true}
\end{verbatim}
gives
\begin{verbatim}
\edef\csname footrue\endcsname
\end{verbatim}
\item Then another \cs{expandafter} is necessary to activate
\altt
the \cs{csname}:
\begin{verbatim}
\expandafter \expandafter \expandafter \edef \@if ...
%   new          old          new
\end{verbatim}
\item This makes the final expansion 
\begin{verbatim}
\edef\footrue{\let\noexpand\iffoo=\noexpand\iftrue}
\end{verbatim}
\end{itemize}

After this follows a similar statement for the \n{false} case:
\begin{verbatim}
   \expandafter\expandafter\expandafter
   \edef\@if#1{false}{\let\noexpand#1=\noexpand\iffalse}%
\end{verbatim}
The conditional starts out false, and the escape character
has to be reset:
\begin{verbatim}
  \@if#1{false}\escapechar\count@} 
\end{verbatim}


\section{Evaluation of conditionals}

\TeX's conditionals behave differently from those
in ordinary programming languages. In many instances
one may not notice the difference, but in certain contexts
it is important to know precisely the \indextermbus{evaluation of}{conditionals} proceeds.

When \TeX\ evaluates a conditional, it first determines
what is to be tested. This in itself may involve some
expansion; as we saw in the previous chapter, 
only after an \cs{ifx} test
does \TeX\  not expand. After all other tests \TeX\ will
expand tokens until the extent of the test and the tokens
to be tested have been determined. On the basis of the outcome
of this test the \gr{true text} and the \gr{false text}
are either expanded or skipped.

For the processing of the parts of the conditional
let us consider some cases separately.
\begin{itemize}
\item \verb>\if... ... \fi> and the result of the test is false.
 After the test \TeX\ will start skipping material
 without expansion, without counting braces, but balancing
 nested conditionals, until a \cs{fi} token is encountered.
 If the \cs{fi} is not found an error message results
 at the end of the file:
 
\begin{disp}\tt Incomplete \cs{if...}; all text was ignored after line \n{...}
 \end{disp} where the line number indicated is that of the line
 where \TeX\ started skipping, that is, where the conditional
 occurred.

\item \verb>\if... \else ... \fi> and the result of the test is false.
 Any material in between the condition and the \cs{else} is skipped
 without expansion, without counting braces, but balancing nested
 conditionals.

 The \cs{fi} token can be the result of expansion; if it never
 turns up \TeX\ will give a diagnostic message
 
\begin{disp}\tt \cs{end} occurred when \cs{if...} on line \n{...}
      was incomplete\end{disp}
 This sort of error is not visible in the output.

 This point plus the previous may jointly be described as follows:
 after a false condition \TeX\ skips until an \cs{else} or \cs{fi}
 is found; any material in between \cs{else} and \cs{fi} is processed.
 
\item \verb>\if... ... \fi> and the result of the test is true.
 \TeX\ will start processing the material following the condition.
 As above, the \cs{fi} token may be inserted by expansion of
 a macro.
 
\item \verb>\if... \else ... \fi> and the result of the test is true.
 Any material following the condition is processed until the \cs{else}
 is found; then \TeX\ skips everything until the matching \cs{fi}
 is found.
 
 This point plus the previous may be described as follows:
 after a true test \TeX\ starts processing material until
 an \cs{else} or \cs{fi} is found; if an \cs{else} is found
 \TeX\ skips until it finds the matching \cs{fi}.
\end{itemize}


%\point Assorted remarks
\section{Assorted remarks}

%\spoint The test gobbles up tokens
\subsection{The test gobbles up tokens}

A common mistake is to write the following:
\begin{verbatim}
\ifnum\x>0\someaction \else\anotheraction \fi
\end{verbatim}
which has the effect that the \verb.\someaction. is expanded,
regardless of whether the test succeeds or not. 
The reason for this is that \TeX\ evaluates the input stream until
it is certain that it has found the arguments to be tested.
In this case it is perfectly possible for the \verb.\someaction.
to yield a digit, so it is expanded. The remedy is to insert
\altt
a space or a \cs{relax} control sequence
after the last digit of the number to be tested.

%\spoint The test wants to gobble up the \cs{else} or \cs{fi}
\subsection{The test wants to gobble up the \cs{else} or \cs{fi}}

The same mechanism that underlies the phenomenon in the previous
point can lead to even more surprising effects if \TeX\
bumps into an \verb.\else., \verb.\or., or \verb.\fi. 
while still busy determining the extent of the test itself.

Recall that \verb.\pageno. is a synomym for \verb.\count0., and
consider the following examples:
\begin{verbatim}
\newcount\nct \nct=1\ifodd\pageno\else 2\fi 1
\end{verbatim}
and
\begin{verbatim}
\newcount\nct \nct=1\ifodd\count0\else 2\fi 1
\end{verbatim}
The first example will assign either 11 or~121 to \cs{nct}, 
but the second one will assign 1 or~121. 
The explanation is that
in cases like the second, where
\altt
an \verb.\else. is encountered while the
test still has not been delimited, a \verb.\relax. is inserted.
In the case that \verb.\count0. is odd the result will thus be \verb.\relax.,
and the example will yield
\begin{verbatim}
\nct=1\relax2
\end{verbatim}
which will assign~1 to \cs{nct}, and print~2.


%\spoint[after:cond] Macros and conditionals; the use of \cs{expandafter}
\subsection{Macros and conditionals; the use of \cs{expandafter}}
\label{after:cond}

Consider the following example:
\begin{verbatim}
\def\bold#1{{\bf #1}} \def\slant#1{{\sl #1}}
\ifnum1>0 \bold \else \slant \fi {some text} ...
\end{verbatim}
This will make not only `some text', 
but {\sl all\/} subsequent text bold.
Also, at the end of the job there will be a notice that
`end occurred inside a group at level~1'.
Switching on \cs{tracingmacros} reveals that the argument
of \verb.\bold. was \verb.\else..
This means that, after expansion of \verb.\bold., 
the input stream looked like
\begin{verbatim}
\ifnum1>0 {\bf \else }\fi {some text} rest of the text 
\end{verbatim}
so the closing brace was skipped as part of the \gram{false text}.
Effectively, then, the resulting stream is
\begin{verbatim}
{\bf {some text} rest of the text
\end{verbatim}
which is unbalanced.

One solution to this sort of problem would be to write
\begin{verbatim}
\ifnum1>0 \let\next=\bold \else \let\next=\slant \fi \next
\end{verbatim}
but  a solution using \cs{expandafter} is also possible:
\begin{verbatim}
\ifnum1>0 \expandafter \bold \else \expandafter \slant \fi
\end{verbatim}
This works, because the \cs{expandafter} commands let \TeX\ determine
the boundaries of the \gram{true text} and the \gram{false text}.

In fact, the second solution may be preferred over the first,
since conditionals are handled by the expansion processor,
and the \cs{let} statements are tackled only by the execution
processor; that is, they are not expandable.
Thus the second solution will (and the first will not)
work, for instance,
inside an~\cs{edef}.

Another example with \cs{expandafter} is the sequence
\begin{verbatim}
\def\get#1\get{ ... }
\expandafter \get \ifodd1 \ifodd3 5\fi \fi \get
\end{verbatim}
This gives
\begin{verbatim}
#1<- \ifodd3 5\fi \fi
\end{verbatim}
and
\begin{verbatim}
\expandafter \get \ifodd2 \ifodd3 5\fi\fi \get
\end{verbatim}
gives
\begin{verbatim}
#1<-
\end{verbatim}
This illustrates again that the result of evaluating a 
conditional is not the final expansion, but the start
of the expansion of the \gr{true text} or \gr{false text},
depending on the outcome of the test.

A detail should be noted: with \cs{expandafter}
it is possible that the \verb.\else. is encountered
before the \gram{true text} has been expanded completely.
This raises the question as to the exact timing of expansion
and skipping.
In the example
\begin{verbatim}
\def\hello{\message{Hello!}}
\ifnum1>0 \expandafter \hello \else \message{goodbye} \bye
\end{verbatim}
the error message caused by the missing \verb.\fi. is given
without \verb.\hello. ever having been expanded.
The conclusion must be that the \gram{false text} is
skipped as soon as it has been located, even if this is at a time
when the \gram{true text} has not been expanded completely.

%\spoint Incorrect matching
\subsection{Incorrect matching}

\TeX's matching of \verb.\if., \verb.\else., and \verb.\fi.
is easily upset. For instance, \TeXbook\ warns you that
you should not say
\begin{verbatim}
 \let\ifabc=\iftrue
\end{verbatim}
inside a 
conditional, because if this text is skipped \TeX\ sees
at least one \verb.\if. to be matched. 

The reason for this is that when \TeX\ is skipping
it recognizes all \cs{if...}, \cs{or}, \cs{else}, and \cs{fi}
tokens, and everything that has been declared a synonym of
such a token by \cs{let}. In \verb>\let\ifabc=\iftrue>
\TeX\ will therefore at least see the \cs{iftrue} as
the opening of a conditional, and, if the current meaning
of \cs{ifabc} was for instance \cs{iffalse}, it will also
be considered as the opening of a conditional statement.

As another example, if
\begin{verbatim}
 \csname if\sometest\endcsname \someaction \fi
\end{verbatim}
is skipped as part of conditional text,
the \verb.\fi. will unintentionally close the
outer conditional.

It does not help to enclose such potentially dangerous
constructs inside a group, because grouping is independent of
conditional structure. Burying such commands inside macros is
the safest approach.

Sometimes another solution is possible, however.
The \cs{loop} macro of plain \TeX\ (see page~\pageref{loop:ex})
is used as
\begin{verbatim}
\loop ... \if ... \repeat
\end{verbatim}
where the \cs{repeat} is not an actually executable
command, but is merely a delimiter:
\begin{verbatim}
\def\loop#1\repeat{ ... }
\end{verbatim}
Therefore,
by declaring
\begin{verbatim}
\let\repeat\fi
\end{verbatim}
the \cs{repeat} balances the \cs{if...} that terminates
the loop, and it becomes possible to have loops in
skipped conditional text.

%\spoint Conditionals and grouping
\subsection{Conditionals and grouping}

It has already been mentioned above that group nesting in \TeX\
is independent of conditional nesting.
The reason for this is that conditionals are handled by the
expansion part of \TeX; in that stage braces are just
unexpandable tokens that require no special treatment.
Grouping is only performed in the later stage of execution
processing.

An example of this independence is now given.
One may write a macro that yields part of
a conditional:
\begin{verbatim}
\def\elsepart{\else \dosomething \fi}
\end{verbatim}
The other way around, the following macros
yield a left brace and a right brace respectively:
\begin{verbatim}
\def\leftbrace{\iftrue{\else}\fi}
\def\rightbrace{\iffalse{\else}\fi}
\end{verbatim}
Note that braces in these definitions are properly nested.

%\spoint A trick
\subsection{A trick}

In some contexts it may be hard to get rid of
\cs{else} or \cs{fi} tokens in a proper
manner. The above approach with \cs{expandafter}
works only if there is a limited number of tokens involved.
In other cases the following trick may provide a way out:
\begin{verbatim}
\def\hop#1\fi{\fi #1}
\end{verbatim}
Using this as
\begin{disp}\verb>\if... \hop >\gr{lots of tokens}\verb>\fi>\end{disp}
will place the tokens outside the conditional.
This is for instance used in~\cite{E2}.

As a further example of this sort of trick,
consider the problem (suggested to me and solved by
Alan Jeffrey) of implementing a conditional 
\verb-\ifLessThan#1#2#3#4-
such that the arguments corresponding to \verb-#3- or
\verb-#4- result, depending on whether \verb-#1- is
less than \verb-#2- or not. 

The problem here is how to get rid of the \cs{else} and the~\cs{fi}.
The \ldash or at least, one \rdash solution is to scoop them up
as delimiters for macros:
\begin{verbatim}
\def\ifLessThan#1#2{\ifnum#1<#2\relax\taketrue \else \takefalse \fi}
\def\takefalse\fi#1#2{\fi#2}
\def\taketrue\else\takefalse\fi#1#2{\fi#1}
\end{verbatim}
Note that \cs{ifLessThan} has only two parameters
(the things to be tested); however, its
result is a macro that
chooses between the next two arguments.

%\spoint More examples of expansion in conditionals
\subsection{More examples of expansion in conditionals}

Above, the macro \cs{ifEqString} was given
\alt
that compares two strings:
\howto Compare two strings\par
\begin{verbatim}
\def\ifEqString#1#2%
   {\def\csa{#1}\def\csb{#2}\ifx\csa\csb }
\end{verbatim}
However, this macro relies on \cs{def},  which is not an
expandable command. If we need a string tester that will
work, for instance, inside an  \cs{edef}, we need some
more ingenuity (this solution was taken from~\cite{E2}).
The basic principle of this solution is to compare the strings
one character at a time. Macro delimiting by \cs{fi} is used;
this was explained above.

First of all, the \cs{ifEqString} call is replaced by a
sequence \verb>\ifAllChars ...\Are ...\TheSame>, and both
strings are delimited by a dollar sign, which is not supposed
to appear in the strings themselves.
\begin{verbatim}
\def\ifEqString
    #1#2{\ifAllChars#1$\Are#2$\TheSame}
\end{verbatim}
The test for equality of characters first determines
whether either string has ended. If both have ended, the original
strings were equal; if only one has ended, they were of unequal
length, hence unequal. If neither string has ended, we test
whether the first characters are equal, and if so, we make a recursive
call to test the remainder of the string.
\begin{verbatim}
\def\ifAllChars#1#2\Are#3#4\TheSame
   {\if#1$\if#3$\say{true}%
          \else \say{false}\fi
    \else \if#1#3\ifRest#2\TheSame#4\else
                 \say{false}\fi\fi}
\def\ifRest#1\TheSame#2\else#3\fi\fi
   {\fi\fi \ifAllChars#1\Are#2\TheSame}
\end{verbatim}
The \cs{say} macro is supposed to  give \cs{iftrue} for
\verb>\say{true}> and \cs{iffalse} for \verb>\say{false}>.
Observing that all  calls to this macro occur two conditionals deep,
we use the `hop' trick explained above as follows.
\begin{verbatim}
\def\say#1#2\fi\fi
   {\fi\fi\csname if#1\endcsname}
\end{verbatim}

Similar to the above example, let us write a macro
that will test lexicographic (`dictionary') precedence
of two strings:
\howto Compare two strings lexicographically\par
\begin{verbatim}
\let\ex=\expandafter
\def\ifbefore
    #1#2{\ifallchars#1$\are#2$\before}
\def\ifallchars#1#2\are#3#4\before
   {\if#1$\say{true\ex}\else
     \if#3$\say{false\ex\ex\ex}\else
      \ifnum`#1>`#3 \say{false%
         \ex\ex\ex\ex\ex\ex\ex}\else
       \ifnum`#1<`#3 \say{true%
         \ex\ex\ex\ex\ex\ex\ex
         \ex\ex\ex\ex\ex\ex\ex\ex}\else
       \ifrest#2\before#4\fi\fi\fi\fi}
\def\ifrest#1\before#2\fi\fi\fi\fi
   {\fi\fi\fi\fi
    \ifallchars#1\are#2\before}
\def\say#1{\csname if#1\endcsname}
\end{verbatim}
In this macro a slightly
different implementation of \cs{say} is used.

Simplified, a call to \cs{ifbefore} will eventually lead to a situation
that looks (in the `true' case) like
\begin{verbatim}
\ifbefore{...}{...}
       \if... %% some comparison that turns out true
          \csname iftrue\expandafter\endcsname
       \else .... \fi
   ... %% commands for the `before' case
\else
   ... %% commands for the `not-before' case
\fi
\end{verbatim}
When the comparison has turned out true, \TeX\ will start processing
the \gr{true text}, and make a mental note to remove any
\verb>\else ... \fi> part once an \cs{else} token is seen.
Thus, the sequence
\begin{verbatim}
\csname iftrue\expandafter\endcsname \else ... \fi
\end{verbatim}
is replaced by
\begin{verbatim}
 \csname iftrue\endcsname
\end{verbatim}
as the \cs{else} is seen while \TeX\ is still processing
\verb>\csname...\endcsname>.

Calls to \cs{say} occur inside nested conditionals, so
the number of \cs{expandafter} commands necessary may be 
\alt
larger than~1: for level two it is~3, for level three
it is~7, and for level~4 it is 15. Slightly more compact
implementations of this macro do exist.

%%%% end of input file [ifelsefi]

%\InputFile:token
%%%% this is input file [token]
%\subject[token] Token Lists
\endofchapter
\chapter{Token Lists}\label{token}

\TeX\ has only one type of data structure: the \indextermsub{token}{list}
There are token list registers that are available to the user,
and \TeX\ has some special token lists: the \cs{every...}
variables, \cs{errhelp}, and \cs{output}.

\label{cschap:toks}\label{cschap:toksdef}\label{cschap:newtoks}
\begin{inventory}
\item [\cs{toks}] 
      Prefix for a token list register.

\item [\cs{toksdef}] 
      Define a control sequence to be a synonym for
      a~\cs{toks} register.

\item [\cs{newtoks}] 
      Macro that allocates a token list register.

\end{inventory}

%\point Token lists
\section{Token lists}

Token lists are the only type of data structure that \TeX\ knows.
They can contain character tokens and control sequence tokens.
Spaces in a token list are significant.
The only operations on token lists are assignment and
unpacking.

\TeX\ has 256 token list registers \verb|\toks|$nnn$ that can be
allocated using the macro \verb|\newtoks|, or explicitly
assigned by \cs{toksdef}; see below.

%\point Use of token lists
\section{Use of token lists}

Token lists are  assigned by a \gr{variable assignment},
which is in this case takes one of the forms
\begin{disp}\gr{token variable}\gr{equals}\gr{general text}\nl
     \gr{token variable}\gr{equals}\gr{filler}\gr{token variable}\end{disp}
Here a \gr{token variable} is an explicit \cs{toks}$nnn$
register, something that has been defined to such a register
by \cs{toksdef} (probably hidden in \cs{newtoks}),
or one of the special \gr{token parameter}
lists below.
A~\gr{general text} has an explicit closing brace, but the
open brace can be implicit.

Examples of token lists are (the first two lines are equivalent):
\begin{verbatim}
\toks0=\bgroup \a \b cd}
\toks0={\a \b cd}
\toks1=\toks2
\end{verbatim}

Unpacking a token list is done by the command \cs{the}:
the expansion of \cs{the}\gr{token variable} is the 
sequence of tokens that was in the token list.

Token lists have a special behaviour in \cs{edef}:
when prefixed by \verb|\the| they are unpacked, 
but the resulting tokens
are not evaluated further. Thus
\begin{verbatim}
\toks0={\a \b} \edef\SomeCs{\the\toks0}
\end{verbatim}
gives
\begin{verbatim}
\SomeCs: macro:-> \a \b
\end{verbatim}
This is in contrast to what happens ordinarily in an~\cs{edef};
see page~\pageref{expand:edef}.


%\point \gr{token parameter}
\section{\gr{token parameter}}

There are in \TeX\ a number of token lists that are automatically
inserted at certain points. These \gr{token parameter}s are
the following:
\begin{description} \item [\cs{output}]
   this token list is inserted
   whenever \TeX\ decides it has sufficient material for a page,
   or when the user forces activation by a penalty~$\leq-10\,000$
   in vertical mode
   (see Chapter~\ref{output});
\item [\cs{everypar}]
   is inserted when \TeX\ switches from external or internal
   vertical mode to unrestricted horizontal mode 
   (see Chapter~\ref{par:start});
\item [\cs{everymath}]
   is inserted after a single math-shift character that starts
   a formula;
\item [\cs{everydisplay}]
   is inserted after a double math-shift character that starts
   a display formula;
\item [\cs{everyhbox}]
   is inserted when an \cs{hbox} begins (see Chapter~\ref{boxes});
\item [\cs{everyvbox}]
   is inserted when a vertical box begins (see Chapter~\ref{boxes});
\item [\cs{everyjob}]
   is inserted when a job begins (see Chapter~\ref{run});
\item [\cs{everycr}]
   is inserted in alignments after \cs{cr} or a non-redundant
   \cs{crcr} (see Chapter~\ref{align});
\item [\cs{errhelp}]
   contains tokens to supplement an \cs{errmessage}
    (see Chapter~\ref{error}).
\end{description}

A \gr{token parameter} behaves the same as an explicit \cs{toks}$nnn$
list, or a quantity defined by \cs{toksdef}.

%\point Token list registers
\section{Token list registers}

Token lists can be stored in \csterm{toks} registers:
\begin{Disp}\cs{toks}\gr{8-bit number}\end{Disp}
which is a \gr{token variable}.
Synonyms for token list registers can be made by the \gr{registerdef}
command \csterm{toksdef} in a \gr{shorthand definition}:
\begin{Disp}\cs{toksdef}\gr{control sequence}\gr{equals}\gr{8-bit number}
\end{Disp} A control sequence defined this way is called
a \gr{toksdef token}, and this is also a token variable
(the remaining third kind of token variable is
the \gr{token parameter}).

The plain \TeX\ macro \csterm{newtoks} uses \cs{toksdef} to 
allocate unused token list registers. This macro is \cs{outer}.

\section{Examples}

Token lists are probably among the least obvious components
of \TeX: most \TeX\ users will never find occasion  for their use,
but format designers and other macro writers
can find interesting applications.
Following are some examples of the sorts of things that can be
done with token lists.

%\spoint Operations on token lists: stack macros
\subsection{Operations on token lists: stack macros}

The number of primitive operations available for token lists is
\howto Stack macros\par
rather limited: assignment and unpacking. However, these are
sufficient to implement other operations such as appending.

Let us say we have allocated a token register
\begin{verbatim}
\newtoks\list \list={\c} 
\end{verbatim}
and we want to add tokens to it,
\alt
using the syntax
\begin{verbatim}
\Prepend \a \b (to:)\list
\end{verbatim}
such that
\begin{verbatim}
\showthe\list
\end{verbatim}
gives
\begin{verbatim}
> \a \b \c .
\end{verbatim}
For this the original list has to be unpacked, and 
\alt
the new tokens followed by the old contents have to assigned
again to the register. Unpacking can be done with \cs{the}
inside an \cs{edef}, so we arrive at the following macro:
\begin{verbatim}
\def\Prepend#1(to:)#2{\toks0={#1}%
    \edef\act{\noexpand#2={\the\toks0 \the#2}}%
    \act}
\end{verbatim}
Note that the tokens that are to be added are first packed
\alt
into a temporary token list, which is then again unpacked
inside the \cs{edef}. Including them directly would have
led to their expansion.

Next we want to use token lists as a sort of stack:
we want a `pop' operation that removes the first element
from the list. Specifically,
\begin{verbatim}
\Pop\list(into:)\first
\show\first \showthe\list
\end{verbatim}
should give
\begin{verbatim}
> \first=macro:
->\a .
\end{verbatim}
and for the remaining list
\begin{verbatim}
 
> \b \c .
\end{verbatim}

Here we make creative use of delimited and undelimited
parameters. With an \cs{edef} we unpack the list,
and the auxiliary macro \cs{SplitOff} scoops up the elements
as one undelimited argument, the first element, and one
delimited argument, the rest of the elements.
\begin{verbatim}
\def\Pop#1(into:)#2{%
    \edef\act{\noexpand\SplitOff\the#1%
              (head:)\noexpand#2(tail:)\noexpand#1}%
    \act}
\def\SplitOff#1#2(head:)#3(tail:)#4{\def#3{#1}#4={#2}}
\end{verbatim}

\subsection{Executing token lists}

The \cs{the} operation for unpacking token lists was used above
only inside an \cs{edef}. Used on its own it has the effect
of feeding the tokens of the list to \TeX's expansion mechanism.
If the tokens have been added to the list in a uniform syntax,
this gives rise to some interesting possibilities.

Imagine that we are implementing the bookkeeping of external
files for a format. Such external files can be used for
table of contents, list of figures, et cetera.
If the presence
of such objects is under the control of the user, we need some
general routines for opening and closing files, and keeping
track of what files we have opened at the user's request.

Here only  some routines for bookkeeping will be described.
Let us say there is a list of auxiliary files, and an auxiliary
counter:
\begin{verbatim}
\newtoks\auxlist \newcount\auxcount
\end{verbatim}
First of all there must be an operation to add auxiliary files:
\begin{verbatim}
\def\NewAuxFile#1{\AddToAuxList{#1}%
    % plus other actions
    }
\def\AddToAuxList#1{\let\\=\relax
    \edef\act{\noexpand\auxlist={\the\auxlist \\{#1}}}%
    \act}
\end{verbatim}
This adds the name to the list in a uniform format:
\begin{verbatim}
\NewAuxFile{toc} \NewAuxFile{lof}
\showthe\auxlist
> \\{toc}\\{lof}.
\end{verbatim}
using the control sequence \verb>\\> which is left undefined.

Now this control sequence can be used for instance to
count the number of elements in the list:
\begin{verbatim}
\def\ComputeLengthOfAuxList{\auxcount=0 
    \def\\##1{\advance\auxcount1\relax}%
    \the\auxlist}
\ComputeLengthOfAuxList \showthe\auxcount
> 2.
\end{verbatim}
Another use of this structure is the following:
at the end of the job we can now close all auxiliary
files at once, by
\begin{verbatim}
\def\CloseAuxFiles{\def\\##1{\CloseAuxFile{##1}}%
    \the\auxlist}
\def\CloseAuxFile#1{\message{closing file: #1. }%
    % plus other actions
    }
\CloseAuxFiles
\end{verbatim}
which gives the output
\begin{verbatim}
closing file: toc.  closing file: lof. 
\end{verbatim}

% 
%%% \begin{comment}
%%% 
%%% % %\spoint Dynamic macro definition
%%% % \subsection{Dynamic macro definition}
%%% 
%%% % Unpacking token lists inside an \cs{edef} can be put to a
%%% % rather ambitious use: dynamic definition of macros.
%%% % Consider a simple example.
%%% % \altt
%%% % We set ourselves the goal of letting
%%% % the user define macros, without ever having to use \cs{def}.
%%% % The syntax for this could look like
%%% \begin{verbatim}
%%% % \startdefinition
%%% % \do:this
%%% % \do:that
%%% % \define:MyMacro
%%% % \end{verbatim}
%%% % such that \verb>\show\MyMacro> gives
%%% \begin{verbatim}
%%% % > \MyMacro=macro:
%%% % ->\this \that .
%%% % \end{verbatim}
%%% % An implementation of this uses a token list to collect
%%% % the commands that the user specifies:
%%% \begin{verbatim}
%%% % \newtoks\actionlist
%%% % \end{verbatim}
%%% % The first command is easy:
%%% \begin{verbatim}
%%% % \def\startdefinition{\actionlist{}}
%%% % \end{verbatim}
%%% % Now the \cs{do} command has to hang control sequences
%%% % in the \cs{actionlist}:
%%% \begin{verbatim}
%%% % \def\do:#1 {%
%%% %     \edef\act{\noexpand\appendaction
%%% %               \expandafter\noexpand\csname#1\endcsname}%
%%% %     \act}
%%% % \end{verbatim}
%%% % The \cs{edef} is used solely to form the actual control sequence.
%%% % The next macro uses \cs{edef} to unpack the \cs{actionlist} so far:
%%% %
%%% \begin{verbatim}
%%% % \def\appendaction#1{%
%%% %     \edef\act{\noexpand\actionlist=
%%% %                  {\the\actionlist \noexpand#1}}%
%%% %     \act}
%%% % \end{verbatim}
%%% % Finally, definition of the user macro also needs an \cs{edef}.
%%% % Some \cs{expandafter} trickery is necessary here to form
%%% % the control sequence of the user macro:
%%% \begin{verbatim}
%%% % \def\define:#1 {%
%%% %     \expandafter\edef\csname#1\endcsname{\the\actionlist}}
%%% % \end{verbatim}
%%% 
%%% % Of course, this is a very simple, rather pointless, example.
%%% % However, it illustrates an important principle of how
%%% % token lists can be used to implement another syntax level
%%% % in \TeX\ (see~\cite{EL}). This principle underlies the
%%% % \indexterm{Lollipop}
%%% % `Lollipop' format that was used to typeset this book.
%%% 
%%% % \end{comment}
%%% %%%% end of input file [token]

\endofchapter
\chapter{Baseline Distances}\label{baseline}

%\hbox{}\vfil\vfil\hbox{}
Lines of text are in most cases not of equal height or depth.
Therefore \TeX\ adds interline glue to keep baselines at a uniform
distance from one another. 
This chapter treats the computation of such
interline glue.

\label{cschap:baselineskip}\label{cschap:lineskiplimit}\label{cschap:lineskip}\label{cschap:prevdepth}\label{cschap:nointerlineskip}\label{cschap:offinterlineskip}\label{cschap:openup}
\begin{inventory}

\item [\cs{baselineskip}] 
      The `ideal' baseline distance between neighbouring 
      boxes on a vertical list. Plain \TeX\ default:~\n{12pt}.

\item [\cs{lineskiplimit}] 
      Distance to be maintained between the bottom and top of 
      neighbouring boxes on a vertical list.
      Plain \TeX\ default:~\n{0pt}.

\item [\cs{lineskip}]  
      Glue added if the distance between bottom
      and top of neighbouring boxes 
      is less than \cs{lineskiplimit}.
      Plain \TeX\ default:~\n{1pt}.

\item [\cs{prevdepth}]  
      Depth of the last box added to a vertical list as it is 
      perceived by \TeX.

\item [\cs{nointerlineskip}]
      Macro to prevent interline glue insertion once.

\item [\cs{offinterlineskip}]
      Macro to prevent interline glue globally
      henceforth.

\item [\cs{openup}]
      Increase \cs{baselineskip}, \cs{lineskip}, 
      and \cs{lineskiplimit} by specified amount.

\end{inventory}


\section{Interline glue}

\input figs17
%\message{fig17 missing}

\TeX\ tries to keep a certain distance between the reference
points of boxes that are added to a vertical list
by inserting \indextermbus{interline}{glue}.
In particular it tries to keep a constant \indextermsub{baseline}{distance}
between lines of ordinary text: the \csterm{baselineskip}. Actually,
the \cs{baselineskip} is  a \gr{glue}, so line distances can
stretch or shrink. However, the natural sizes,
as well as the stretch and the shrink, are the same
between all lines.

When boxes, whether they are lines of a paragraph or explicit boxes,
are appended to a vertical list, glue 
is added usually so that the depth of the preceding box
and the height of the current one add up to the \cs{baselineskip}.
This has the effect of keeping the reference points 
of subsequent lines at regular intervals.

%\eject

\begin{quotation}
  \figone
\end{quotation}
%\begin{disp}\leavevmode\hbox{}\nl\figone\end{disp}

However, this process can bring the bottom and top of two
subsequent boxes to be less than \cs{lineskiplimit} apart:
\begin{quotation}
  \figtwo
\end{quotation}
%\begin{disp}\leavevmode\figtwo\end{disp}

In that case, \cs{lineskip} glue is added:
\begin{quotation}
  \figthree
\end{quotation}
%\begin{disp}\leavevmode\figthree\end{disp}
Note that this will usually increase the distance
between the baselines of the boxes to more than the
\cs{baselineskip}.

The exact process is this:
\begin{itemize}
\item if \cs{prevdepth} is \n{-1000pt} or less,
no glue is added, otherwise
\item \TeX\ calculates the distance between the bottom of the previous box
and the top of the current one as the natural width of the
\cs{baselineskip} minus \cs{prev\-depth} (the
depth of the last box) and minus the height of the current box;
\item if this distance is at least \csterm{lineskiplimit}, 
glue is added with the calculated distance as natural size,
and with the stretch and shrink of the \cs{baselineskip},
\item otherwise \csterm{lineskip} glue is added.
\item \csterm{prevdepth} is set to the depth of the
current item.
\end{itemize}

There are two exceptional  situations:
no interline glue is added before and after a rule,
and the \cs{prevdepth} is not updated by an \cs{unvbox}
or \cs{unvcopy} command. After a rule interline glue
is prevented by a value of \n{-1000pt} of the \cs{prevdepth}.

The above process is carried out, irrespective of what extra
glue may have been inserted in between the boxes.
Thus a skip in between boxes in vertical mode will not
affect the distance calculated from the baseline distances,
and therefore also not the amount of baselineskip glue.
The same holds for glue added with \cs{vadjust} inside
a paragraph.

\begin{example}
\begin{verbatim}
\baselineskip=10pt \lineskiplimit=2pt \lineskip=2pt
\setbox0=\vbox{\hbox{\vrule depth4pt}
               \hbox{\vrule height 3pt}}
\showbox0
\end{verbatim}
gives
\begin{verbatim}
\box0=
\vbox(10.0+0.0)x0.4
.\hbox(0.0+4.0)x0.4
..\rule(*+4.0)x0.4
.\glue(\baselineskip) 3.0
.\hbox(3.0+0.0)x0.4
..\rule(3.0+*)x0
\end{verbatim}
Bringing the boxes to within  \cs{lineskiplimit}
of each other, that is
\begin{verbatim}
\setbox0\vbox{\hbox{\vrule depth4pt}
              \hbox{\vrule height 5pt}}
\showbox0
\end{verbatim}
gives
\begin{verbatim}
\box0=
\vbox(11.0+0.0)x0.4
.\hbox(0.0+4.0)x0.4
..\rule(*+4.0)x0.4
.\glue(\lineskip) 2.0
.\hbox(5.0+0.0)x0.4
..\rule(5.0+*)x0.4
\end{verbatim}
where \cs{lineskip} glue has been inserted
instead of the usual \cs{baselineskip} glue.
\end{example}

The plain \TeX\ default values are
\begin{verbatim}
\lineskiplimit=0pt lineskip=1pt
\end{verbatim}
so, when boxes start to touch each other, they are
moved one point apart.

\section{The perceived depth of boxes}

The decision process for interline glue uses \csterm{prevdepth}
as the perceived depth of the preceding box on the vertical
list. The \cs{prevdepth} parameter can be used only in
vertical mode.

The \cs{prevdepth} is set to the depth of boxes added to the
vertical list, but it is not affected by \cs{unvbox}
or \cs{unvcopy}. After an \cs{hrule} it is set to
\n{-1000pt} to prevent interline glue before the next box.

At the beginning of a vertical list \cs{prevdepth}
is set to \n{-1000pt}, except in an \cs{halign}
and \cs{noalign} code contained therein, where it
is carried over from the surrounding list.
At the end of the alignment the value of \cs{prevdepth}
set by the last alignment row is carried to the outer list.

In order to prevent interline glue just once, all that
is needed is to alter the \cs{prevdepth}.
\csterm{nointerlineskip}%
\begin{verbatim}
\def\nointerlineskip{\prevdepth=-1000pt}
\end{verbatim}

The \csterm{offinterlineskip} macro is much more drastic:
it prevents {\sl all\/} interline glue from the moment
of its call onwards, or, if it is used inside a paragraph,
from the start of that paragraph.
Its definition is
\begin{verbatim}
\baselineskip=-1000pt \lineskip=0pt 
\lineskiplimit\maxdimen
\end{verbatim}
where the second line is the essential one: it
causes \TeX\ to add \cs{lineskip} glue (which is zero)
always. 
Settings for \cs{baselineskip} do not matter any more then.

The \cs{offinterlineskip} macro has an important application
in alignments (see Chapter~\ref{align}).

By setting
\begin{verbatim}
\lineskiplimit=-\maxdimen
\end{verbatim}
you can force \TeX\  to apply the \cs{baselineskip}
always, regardless of whether this would bring boxes too close
together or, indeed, if this would make them overlap.

%\point Terminology
\section{Terminology}

In hot metal typesetting, all letters of a particular font
were on a `body' of the same
size. Thus every line of type had the same height and depth, and
the resulting distance between the baselines would be some suitable
value for that type. If for some reason this distance should
be larger (see~\cite{White:line} for a discussion of this),
strips of lead would be inserted. The extra distance was
called the `leading' (pronounced `ledding').

With phototypesetting, when the baseline distance was sometimes
called the `film transport', this terminology blurred, and the
term `leading' was also used for the baseline distance. Some of this
confusion is also present in \TeX: the parameter \cs{baselineskip}
specifies the baseline distance, but in the trace output
(see the examples above) the glue inserted to make the 
baseline distance equal to \cs{baselineskip} is called
\cs{baselineskip}.

%\point Additional remarks
\section{Additional remarks}

In general, for documents longer than one page it is desirable
to have the same baseline distance throughout. However,
for one-page documents you may add stretchability to the
baselineskip, for instance if the text has to be flush bottom.

Increasing the distance between just one pair of lines
can be done with \cs{vadjust}. The argument of this
command is vertical material that\vadjust{\kern2pt}
will be inserted in the
vertical list right after the line where this command was given.
The second line of this paragraph, for instance, 
contains the command \verb-\vadjust{\kern2pt}-.

The amount of leading cannot be changed in the middle of
a paragraph, because the value for \cs{baselineskip}
that is used is the one that
is current when the paragraph is finally broken and
added to the main vertical list. The same holds
for the \cs{lineskip} and \cs{lineskiplimit}.

The plain \TeX\ macro \csterm{openup}
increases the \cs{baselineskip}, \cs{lineskip}, and
\cs{lineskiplimit} by the amount of the argument
to the macro. In effect, this increases line distances
by this amount regardless of whether they are governed
by \cs{baselineskip} or \cs{lineskip}.


%%%% end of input file [baseline]

%\InputFile:par
%%%% this is input file [par]
%\subject[par:start] Paragraph Start
\endofchapter
\chapter{Paragraph Start}\label{par:start}
\index{paragraph!start|(}

At the start of a paragraph \TeX\ inserts a vertical skip
as a separation from the preceding paragraph, and a horizontal
skip as an indentation for the current paragraph.
This chapter explains the exact sequence
of actions,
and it discusses how \TeX's decisions can be altered.

\label{cschap:indent}\label{cschap:noindent}\label{cschap:parskip}\label{cschap:parindent}\label{cschap:everypar}\label{cschap:leavevmode}
\begin{inventory}
\item [\cs{indent}] 
      Switch to horizontal mode and insert a box of width \cs{parindent}.

\item [\cs{noindent}]  
      Switch to horizontal mode with an empty horizontal list.

\item [\cs{parskip}] 
      Amount of glue added to 
      the surrounding vertical list when a paragraph starts.
      Plain \TeX\ default:~\n{0pt plus 1pt}.

\item [\cs{parindent}]  
      Size of the indentation box added in front of a paragraph.
      Plain \TeX\ default:~\n{20pt}.

\item [\cs{everypar}] 
      Token list inserted in front of paragraph text; 

\item [\cs{leavevmode}] 
      Macro to switch to horizontal mode if necessary.

\end{inventory}


%\point When does a paragraph start
\section{When does a paragraph start}

\TeX\ starts a paragraph whenever it switches from 
vertical mode to (unrestricted) horizontal mode. This switch can
be effected by one of the commands
\cs{indent} and 
\cs{noindent}, for example
\begin{verbatim}
{\bf And now~\dots}
\vskip3pt
\noindent It's~\dots
\end{verbatim}
or by any \gram{horizontal command}.
Horizontal commands include characters, in-line formulas,
and horizontal skips, but not boxes.
Consider the following examples.
\alt
The character `I' is a horizontal command:
\begin{verbatim}
\vskip3pt
It's~\dots
\end{verbatim}
A single \n\$ is a horizontal command:
\begin{verbatim}
$x$ is supposed~\dots
\end{verbatim} 
The control sequence \cs{hskip} is a horizontal command:
\begin{verbatim}
\hskip .5\hsize Long indentation~\dots
\end{verbatim}
The full list of horizontal commands is given on
page~\pageref{h:com:list}.

Upon recognizing a horizontal command in vertical mode,
\TeX\ will perform an \cs{indent} command (and all the actions
associated with it; see below), 
and after that it will reexamine the horizontal command,
this time executing it.



%\point What happens when a paragraph starts
\section{What happens when a paragraph starts}

The \csterm{indent} and \csterm{noindent}  commands 
cause a paragraph to be started.
An~\cs{indent} command can either be placed explicitly by
the user or a macro, or it can be inserted by \TeX\ when
a \gr{horizontal command} occurs in vertical mode;
a~\cs{noindent} command can only be placed explicitly.

After  either command is encountered,
\csterm{parskip} glue is appended to the surrounding vertical
list
unless \TeX\ is in internal vertical mode 
and that list is empty
(for example, at the start of a \cs{vbox} or \cs{vtop}).
\TeX\ then switches to unrestricted horizontal mode
with an empty horizontal list.
In the case of \cs{indent} (which may be inserted
implicitly) an empty \cs{hbox} of width
\csterm{parindent}%
\cs{parindent} is placed at the start of the horizontal list; 
after \cs{noindent} no indentation
box is inserted. 

The contents of the \csterm{everypar} \gr{token parameter}
are then inserted into the input (see some applications below).
After that,
the page builder is exercised (see Chapter~\ref{page:break}).
Note that this happens in horizontal mode: this is to 
move the \cs{parskip} glue to the current page.

If an \cs{indent} command is given while \TeX\ is already in
horizontal mode, the indentation box is inserted just the same.
This is not very useful.

%\point Assorted remarks
\section{Assorted remarks}

%\spoint Starting a paragraph with a box
\subsection{Starting a paragraph with a box}

An \cs{hbox} does not imply horizontal mode, so 
an attempt to start a paragraph with a box, for instance
\begin{verbatim}
\hbox to 0cm{\hss$\bullet$\hskip1em}Text ....
\end{verbatim}
will make the text following the box
wind up one line below the box.
It is necessary to switch to horizontal mode
explicitly, using for instance \cs{noindent} or
\cs{leavevmode}. 
The latter is defined using \cs{unhbox},
which is a horizontal command.

%\spoint Starting a paragraph with a group
\subsection{Starting a paragraph with a group}

If the first \gram{horizontal command} of a paragraph
is enclosed in braces, the \cs{everypar} is evaluated
inside the group. This may give unexpected results.
Consider this example:
\begin{verbatim}
\everypar={\setbox0=\vbox\bgroup\def\par{\egroup}}
{\bf Start} a paragraph ... \par
\end{verbatim}
The \gr{horizontal command} starting the paragraph is the
character~`S', so when \cs{everypar} has been inserted
the input is essentially
\begin{verbatim}
{\bf \indent\setbox0=\vbox\bgroup
    \def\par{\egroup}Start} a paragraph ... \par
\end{verbatim}
which is equivalent to
\begin{verbatim}
{\bf \setbox0=\vbox{Start} a paragraph ... \par
\end{verbatim}
The effect of this is rather different from what was intended.
\alt
Also, \TeX\ will probably end the job inside a group.

%\point Examples
\section{Examples}

%\spoint Stretchable indentation 
\subsection{Stretchable indentation }

Considering that \cs{parindent} is a \gram{dimen}, not a \gram{glue},
it is not possible to declare
\begin{verbatim}
\parindent=1cm plus 1fil
\end{verbatim}
in order to get
a variable indentation at the start of a paragraph.
This problem may be solved by putting
\begin{verbatim}
\everypar={\nobreak\hskip 1cm plus 1fil\relax}
\end{verbatim}
The \cs{nobreak} serves to prevent (in rare cases) a line break
at the stretchable glue.

%\spoint Suppressing indentation
\subsection{Suppressing indentation}

Inserting 
\verb.{\setbox0=\lastbox}. in the horizontal list
at the beginning of the paragraph
removes the indentation:
indentation consists of a box, which is available through
\cs{lastbox}. Assigning it effectively removes it from the list.

However, this command sequence
has to be inserted at a moment when \TeX\ has
already switched to horizontal mode, so explicit insertion
of these commands in front of the first \gram{horizontal
command} of the paragraph does not work. 
The moment of insertion of the \cs{everypar} tokens 
is a better candidate: specifying
\begin{verbatim}
\everypar={{\setbox0=\lastbox}}
\end{verbatim}
leads to unindented paragraphs, even if \cs{parindent} is 
not zero. 


%\spoint[indent:scheme] An indentation scheme
\subsection{An indentation scheme}
\label{indent:scheme}

The above idea of letting the indentation box be removed
\howto Control indentation systematically\par
by \cs{everypar} can be put to use in a systematic approach
to indentation, where two conditionals
\begin{verbatim}
\newif\ifNeedIndent %as a rule
\newif\ifneedindent %special cases
\end{verbatim}
control whether paragraphs should indent as a rule, and
whether in special cases indentation is needed.
This section is taken from~\cite{E3}.

We take a fixed \cs{everypar}:
\begin{verbatim}
\everypar={\ControlledIndentation}
\end{verbatim}
which executes in some cases the macro \cs{RemoveIndentation}
\begin{verbatim}
\def\RemoveIndentation{{\setbox0=\lastbox}}
\end{verbatim}
The implementation of \cs{ControlledIndentation} is:
\begin{verbatim}
\def\ControlledIndentation
   {\ifNeedIndent \ifneedindent 
                  \else \RemoveIndentation\needindenttrue \fi
    \else \ifneedindent \needindentfalse
          \else         \RemoveIndentation
    \fi   \fi}
\end{verbatim}
In order to regulate indentation for a whole document,
the user now once specifies, for instance,
\begin{verbatim}
\NeedIndenttrue
\end{verbatim}
to indicate that, in principle,
all paragraphs should indent.
Macros such as \cs{section} can then prevent
indentation in individual cases:
\begin{verbatim}
\def\section#1{ ... \needindentfalse}
\end{verbatim}
    

%\spoint[skip:scheme] A paragraph skip scheme
\subsection{A paragraph skip scheme}
\label{skip:scheme}

The use of \cs{everypar} to control indentation,
\howto Control vertical white space systematically\par
as was sketched above, can be extended to the
paragraph skip.

A visible white space between paragraphs can be
created by the \cs{parskip} parameter, but, once this
parameter has been set to some value, it is difficult
to prevent paragraph skip in certain places elegantly.
Usually, white space above and below environments
and section headings should be specifiable independently
of the paragraph skip. This section sketches an
approach where \cs{parskip} is set to zero directly
above and below certain constructs, while the \cs{everypar}
is used to restore former values. This section is
taken from~\cite{E4}.

First of all, here are two tools. The control sequence
\cs{csarg} will be used only inside other macros;
a typical call will look like
\begin{verbatim}
\csarg\vskip{#1Parskip}
\end{verbatim}
Here is the definition:
\begin{verbatim}
\def\csarg#1#2{\expandafter#1\csname#2\endcsname}
\end{verbatim}
Next follows a generalization of \cs{vskip}: the macro
\cs{vspace} will not place its argument if the previous glue item
is larger; otherwise it will eliminate the preceding
glue, and place its argument.\begin{verbatim}
\newskip\tempskipa
\def\vspace#1{\tempskipa=#1\relax
    \ifvmode \ifdim\tempskipa<\lastskip 
             \else \vskip-\lastskip \vskip\tempskipa \fi
    \else    \vskip\tempskipa \fi}
\end{verbatim}
    
Now assume that any construct \n{foo} 
with surrounding white space
starts and ends with macro calls \verb>\StartEnvironment{foo}> and
\verb>\EndEnvironment{foo}> respectively.
Furthermore, assume that to this environment there correspond
three glue registers:
the \cs{fooStartskip} (glue
above the environment), \cs{fooParskip} (the paragraph skip
inside the environment), and the \cs{fooEndskip} (glue below
the environment).

For restoring the value of the paragraph skip
a conditional and a glue register are needed:
\begin{verbatim}
\newskip\TempParskip \newif\ifParskipNeedsRestoring
\end{verbatim}
The basic sequence for the
starting and ending macros for the environments is then
\begin{verbatim}
\TempParskip=\parskip\parskip=0cm\relax
\ParskipNeedsRestoringtrue
\end{verbatim}

The implementations can now be given as:
\begin{verbatim}
\def\StartEnvironment#1{\csarg\vspace{#1Startskip}
    
\begingroup % make changes local
    \csarg\TempParskip{#1Parskip} \parskip=0cm\relax
    \ParskipNeedsRestoringtrue}
\def\EndEnvironment#1{\csarg\vspace{#1Endskip}
    \endgroup % restore global values
    \ifParskipNeedsRestoring
    \else \TempParskip=\parskip \parskip=0cm\relax
          \ParskipNeedsRestoringtrue
    \fi}
\end{verbatim}
The \cs{EndEnvironment} macro needs a little comment:
if an environment is used inside another one, and
it occurs before the first paragraph in that environment,
the value of the paragraph skip for the outer environment
has already been saved. Therefore no further actions are
required in that case.

Note that both macros start with a vertical skip. This prevents
the \cs{begingroup} and \cs{endgroup} statements from
occurring in a paragraph. 

We now come to the main point: if necessary, the
\cs{everypar} will restore the value of the paragraph skip.
\begin{verbatim}
\everypar={\ControlledIndentation\ControlledParskip}
\def\ControlledParskip
   {\ifParskipNeedsRestoring
       \parskip=\TempParskip \ParskipNeedsRestoringfalse
    \fi}
\end{verbatim}

\index{paragraph!start|)}
\endofchapter

\chapter{Paragraph End}\label{par:end}
\index{paragraph!end|(}

\TeX's mechanism for ending a paragraph is ingenious and effective.
This chapter explains the mechanism, the role of \cs{par} in it,
and it gives a number of practical remarks.

\label{cschap:par}\label{cschap:endgraf}\label{cschap:parfillskip}
\begin{inventory}
\item [\cs{par}] 
      Finish off a paragraph and go into vertical mode.

\item [\cs{endgraf}]
      Synonym for \cs{par}: \verb>\let\endgraf=\par>

\item [\cs{parfillskip}] 
      Glue that is placed between the last          
      element of the paragraph and the line end.
      Plain \TeX\ default:~\n{0pt plus 1fil}.
\end{inventory}

\section{The way paragraphs end}

A paragraph is terminated by the primitive \cs{par} command, 
which can
be explicitly typed by the user (or inserted by
a macro expansion):
\begin{verbatim}
... last words.\par
A new paragraph ...
\end{verbatim}
It can be implicitly generated in the input processor of \TeX\
by an empty line (see Chapter~\ref{mouth}):
\begin{verbatim}
... last words.

A new paragraph ...
\end{verbatim} 
The \cs{par} can be inserted because a \gr{vertical command}
occurred in unrestricted horizontal mode:
\begin{verbatim}
... last words.\vskip6pt
A new paragraph ...
\end{verbatim}
Also, a paragraph ends if a closing brace is found
in horizontal mode inside \cs{vbox}, \cs{insert}, or \cs{output}.

After the \cs{par} command \TeX\ goes into vertical mode
and exercises the page builder (see page~\pageref{par:page:build}).
If the \cs{par} was inserted because a vertical command occurred in
horizontal mode, the vertical command is then examined anew.
The \cs{par} does not insert any vertical
glue or penalties itself. A~\cs{par} command also clears
the paragraph shape parameters (see Chapter~\ref{par:shape}).

%\spoint The \cs{par} command and the \cs{par} token
\subsection{The \cs{par} command and the \cs{par} token}

It is important to distinguish between the \cs{par} token
and the primitive \cs{par} command that is the initial meaning of
that token. The \cs{par} token is inserted when the input
processor sees an empty
line, or when the execution processor finds a \gram{vertical command}
in horizontal mode;
the \cs{par} command is what actually closes off a paragraph.
Decoupling the token and the command is an important tool
for special effects in paragraphs (see some examples in
Chapters \ref{boxes} and~\ref{rules}).


%\spoint Paragraph filling: \cs{parfillskip}
\subsection{Paragraph filling: \cs{parfillskip}}

After the last element of the paragraph \TeX\ implicitly inserts
the equivalent of
\csterm{parfillskip}%
\begin{verbatim}
\unskip \penalty10000 \hskip\parfillskip
\end{verbatim}
The \cs{unskip} serves to remove any spurious glue at the
paragraph end, such as the space generated by the
line end if the \cs{par} was inserted by the input processor.
For example:\message{check unsplit paragraph example}
\begin{verbatim}
end.

\noindent Begin
\end{verbatim}
results in the tokens
\begin{disp}\n{end.\char32}\cs{par} \n{Begin}\end{disp}
With the sequence inserted by the \cs{par} this becomes
\begin{disp}\n{end.\char32}\verb>\unskip\penalty10000\hskip ...>\end{disp}
which in turn gives
\begin{disp}\verb>end.\penalty ...>\end{disp}

The \cs{parfillskip} is in plain \TeX\ first-order infinite
(\n{0pt plus 1fil}),
so ending a paragraph with \verb.\hfil$\bullet$\par.
will give a bullet halfway between the last word and the
line end; with \verb.\hfill$\bullet$\par. it will be
flush right.


%\point Assorted remarks
\section{Assorted remarks}

%\spoint Ending a paragraph and a group at the same time
\subsection{Ending a paragraph and a group at the same time}

If a paragraph is set in a group, 
it may be necessary to ensure that the \cs{par} ending
the paragraph occurs inside the group.
The parameters influencing the typesetting of the paragraph,
such as the \cs{leftskip} and the \cs{baselineskip},
are only looked at when the paragraph is finished.
Thus finishing off a paragraph with 
\begin{verbatim}
... last words.}\par
\end{verbatim}
causes the values to be used
that prevail outside the group, instead of those inside.

Better ways to end the paragraph are
\begin{verbatim}
... last words.\par}
\end{verbatim}
or
\begin{verbatim}
... last words.\medskip}
\end{verbatim}
In the second example the vertical command \cs{medskip}
causes the \cs{par} token to be inserted.

%\spoint Ending a paragraph with \cs{hfill}\cs{break}
\subsection{Ending a paragraph with \cs{hfill}\cs{break}}

The sequence \verb.\hfill\break. is a way to force
a `newline' inside a paragraph. If you end a paragraph
with this, however, you will probably
get an \verb-Underfull \hbox- error.
Surprisingly, the underfull box is not the broken line
\ldash after all, that one was filled \rdash 
but a completely empty box following it (actually, it
does contain the \cs{leftskip} and \cs{rightskip}). 

What happens?
The paragraph ends with
\begin{verbatim}
\hfill\break\par
\end{verbatim}
which turns into 
\begin{verbatim}
\hfill\break\unskip\nobreak\hskip\parfillskip
\end{verbatim}
The \cs{unskip} finds no preceding glue, so the \cs{break}
is followed by a penalty item and a glue item, both of
which disappear after the line break has been chosen at the
\cs{break}.
However, \TeX\ has already decided that there should be an extra
line, that is, an \verb.\hbox to \hsize.. And there is nothing
\alt
to fill it with, so an underfull box results.

%\spoint Ending a paragraph with a rule
\subsection{Ending a paragraph with a rule}

See page~\pageref{par:leaders:end} for paragraphs ending with 
rule leaders instead of the default \cs{parfillskip}
white space.

%\spoint No page breaks in between paragraphs
\subsection{No page breaks in between paragraphs}

The \cs{par} command does not insert any glue in the
\howto Prevent page breaks in between paragraphs\par
vertical list, so
in the sequence
\begin{verbatim}
 ... last words.\par \nobreak \medskip 
\noindent First words ...
\end{verbatim}
no page breaks will occur  between the paragraphs.
The vertical list generated is
\begin{verbatim}
\hbox(6.94444+0.0)x ...       % last line of paragraph
\penalty 10000                % \nobreak
\glue 6.0 plus 2.0 minus 2.0  % \medskip
\glue(\parskip) 0.0 plus 1.0  % \parskip
\glue(\baselineskip) 5.05556  % interline glue
\hbox(6.94444+0.0)x ...       % first line of paragraph
\end{verbatim}
\TeX\ will not break this vertical list above the \cs{medskip},
because the penalty value prohibits it; it will not break
at any other place, because it can only break at glue if
that glue is preceded by a  non-discardable item.

%\spoint Finite \cs{parfillskip}
\subsection{Finite \cs{parfillskip}}

In plain \TeX, \cs{parfillskip} has a (first-order) infinite
stretch component. All other glue in the last line of a 
paragraph will then be set at natural width.
If the \cs{parfillskip} has only finite (or possibly zero)
stretch, other glue will be stretched or shrunk.
A display formula in a paragraph with such a last line
will be surrounded by \cs{abovedisplayskip} and \cs{belowdisplayskip},
even if \cs{abovedisplayshortskip} glue would be in order.

The reason for this is that glue setting is slightly
machine-dependent, and any such processes should be kept
out of \TeX's global decisions.

%\spoint A precaution for paragraphs that do not indent
\subsection{A precaution for paragraphs that do not indent}

If you are setting a text with both the paragraph indentation
and the white space between paragraphs zero, you run the risk
that the start of a new paragraph may be indiscernible when
the last line of the previous paragraph ends almost
or completely flush right.
A~sensible precaution for this is to set the \cs{parfillskip}
to, for instance
\begin{verbatim}
 \parfillskip=1cm plus 1fil
\end{verbatim}
instead of the usual \n{0cm~plus~1fil}.

On the other hand, you may let yourself be convinced by
\cite{Tsch} that paragraphs should always indent.


\index{paragraph!end|)}
\endofchapter

\chapter{Paragraph Shape}\label{par:shape}
\index{paragraph!shape|(}

This chapter discusses the parameters and commands that influence the
shape of a paragraph.

\label{cschap:parindent2}\label{cschap:hsize2}\label{cschap:leftskip}\label{cschap:rightskip}\label{cschap:hangindent}\label{cschap:hangafter}\label{cschap:parshape}
\begin{inventory}
\item [\cs{parindent}] 
      Width of the indentation box added in front of a paragraph.
      Plain \TeX\ default:~\n{20pt}.

\item [\cs{hsize}] 
      Line width used for typesetting a paragraph.
      Plain \TeX\ default:~\n{6.5in}.

\item [\cs{leftskip}] 
      Glue that is placed to the left of all lines of a paragraph.


\item [\cs{rightskip}] 
      Glue that is placed to the right of all lines of a paragraph.


\item [\cs{hangindent}]   
      If positive, this indicates indentation from the left margin; 
      if negative, this is the negative of the indentation 
      from the right margin.

\item [\cs{hangafter}]   
      If positive, this denotes the number of lines 
      before indenting starts; 
      if negative, the absolute value of this is the number 
      of indented lines starting with the first line of the paragraph.
      Default:~\n1.

\item [\cs{parshape}]
      Command for general paragraph shapes.

\end{inventory}


\begin{figure}[ht]
  \input figs20
\end{figure}

%\point The width of text lines
\section{The width of text lines}
\index{line!width|(}

When \TeX\ has finished absorbing a paragraph, 
it has formed a horizontal list, starting with an indentation
box, and ending with \cs{parfillskip} glue.
This list is then broken into lines of length \cs{hsize}.
\csterm{hsize}%
\csterm{leftskip}%
\csterm{rightskip}%
Each line of a paragraph is padded left and right with
certain amounts of glue, the \cs{leftskip} and \cs{rightskip},
which are taken into account in reaching \cs{hsize}.

The values of \cs{leftskip} and \cs{rightskip} are taken 
into account in the line-breaking algorithm.
Thus the main point about the \csterm{raggedright} 
macro in plain \TeX\ and the \LaTeX\ `flushleft'
environment is that they
set the \cs{rightskip} to  zero plus some stretch.

The commands \cs{parshape} and \cs{hangindent}
also affect line width. They work by altering the
\cs{hsize} and afterwards shifting the boxes 
containing the lines.

\index{line!width|)}

%\point Shape parameters
\section{Shape parameters}

%\spoint Hanging indentation
\subsection{Hanging indentation}
\index{indentation!hanging|(}

\message{twolines?}
A simple, and frequently occurring, paragraph shape is that
\csterm{hangafter}%
\csterm{hangindent}%
with a number of starting or trailing lines indented.
\TeX\ can realize such shapes using two parameters:
\cs{hangafter} and \cs{hangindent}.
Both can assume positive and negative values.

The \cs{hangindent} controls the amount of indentation:
\begin{itemize}\item \cs{hangindent}${}>0$: the paragraph
is indented at the left margin by this amount.
\item\cs{hangindent}${}<0$: the paragraph is indented
at the right margin by the absolute value of this amount.
\end{itemize}
\def\exnul{\leftskip=0pt \rightskip=0pt \relax}
For example (assume \cs{parindent=0pt}),
\begin{disp}\leavevmode\message{Check parshape example!}%
\hbox{%\Distance:verbatimwhiteleft=0pt
$\vcenter{\hsize=3.5in \snugbox{\begin{verbatim}
 a a a a a a a a a a a a ...

 \hangindent=10pt
 a a a a a a a a a a a a ...

 \hangindent=-10pt
 a a a a a a a a a a a a ...
\end{verbatim}
}}$\quad gives\quad\quad
$\vcenter{\parindent0pt \setbox0\hbox{a a a a a}\hsize\wd0
 \leftskip=0pt %\parskip6pt 
 a a a a a a a a a a a a \dots\par%\vskip\baselineskip
 \hangindent=10pt
 a a a a a a a a a a a a \dots\par%\vskip\baselineskip
 \hangindent=-10pt
 a a a a a a a a a a a a \dots\par}$
}\end{disp}
The default value of \cs{hangindent} is~\n{0pt}.

The \cs{hangafter} parameter determines the number of
lines that is indented:
\begin{itemize}\item \cs{hangafter}${}\geq0$: 
after this number of lines the rest of the lines will be
indented; in other words, this many lines from the
start of the paragraph will not be indented.
\item \cs{hangafter}${}<0$: the absolute value of this
is the number of lines that will be indented starting
at the beginning of the paragraph.\end{itemize}
For example,
\message{check left align}
\begin{disp}\leavevmode\hbox{%\Distance:verbatimwhiteleft=0pt
$\vcenter{\hsize=3.5in \snugbox{\begin{verbatim}
 a a a a a a a a a a a a ...

 \hangindent=10pt \hangafter=2
 a a a a a a a a a a a a ...

 \hangindent=10pt \hangafter=-2
 a a a a a a a a a a a a ...
\end{verbatim}
}}$\quad looks like\quad\quad
$\vcenter{\parindent0pt \setbox0\hbox{a a a a a}\hsize\wd0
 \leftskip=0pt %\parskip6pt
 a a a a a a a a a a a a \dots\par%\vskip\baselineskip
 \hangindent=10pt \hangafter=2
 a a a a a a a a a a a a \dots\par%\vskip\baselineskip
 \hangindent=10pt \hangafter=-2
 a a a a a a a a a a a a \dots\par}$
}\end{disp}
The default value for \cs{hangafter} is~\n1.

With both parameters having the possibility to
be positive and negative,
four ways of hanging indentation result. See below
for hanging indentation into the margin (`outdent').

Hanging indentation is implemented as follows.
The amount of hanging indentation is subtracted
from the \cs{hsize} for the lines that indent;
after the paragraph has been broken into horizontal
boxes, the lines that should indent on the left are
shifted right.

Regular indentation of size \cs{parindent} is not
influenced by hanging indentation. Thus you should
start a paragraph with hanging indentation 
explicitly by~\cs{noindent} if the extra
indentation is unwanted.

The default values of \cs{hangindent} and \cs{hangafter} are
restored after every \cs{par} command.

\index{indentation!hanging|)}

\subsection{General paragraph shapes: \cs{parshape}}

Quite general paragraph shapes can be implemented
using \csterm{parshape}. With this command line lengths and indentation
for the first $n$ lines
of a paragraph can be specified. Thus this command
takes $2n+1$ parameters: the number of lines $n$, followed
by $n$ pairs of an indentation   and a line length.
\begin{disp} \cs{parshape}\gr{equals}
    $n$ $i_1$ $\ell_1$ $\ldots$ $i_n$ $\ell_n$\end{disp}
The   specification for the last line is repeated if the
paragraph following has more than $n$ lines. If there are fewer
than $n$ lines the remaining specifications are ignored.
The default value is (naturally) \cs{parshape${}={}$0}.

A \cs{parshape} command takes precedence over a \cs{hangindent}
if both have been specified. 
%Regular \cs{parindent} indentation
%is suppressed if \cs{parshape} is in effect.
Regular \cs{parindent}, \cs{leftskip}, 
and \cs{rightskip} are still obeyed if \cs{parshape} is in effect.

The \cs{parshape} parameter is, like \cs{hangindent}, \cs{hangafter},
and \cs{looseness} (see Chapter~\ref{line:break}),
cleared after a \cs{par}
command. Since every empty line generates a \cs{par} token,
one should not leave an empty line
between a paragraph shape (or hanging indentation)
declaration and the following paragraph.

The control sequence
\alt
\cs{parshape} is an \gr{internal integer}:
its value is the number of lines $n$ with which
it was set.

%\point Assorted remarks
\section{Assorted remarks}

%\spoint Centred last lines
\subsection{Centred last lines}

Equal stretch and shrink amounts for the \cs{leftskip} and 
\cs{rightskip}
give centred texts, in the sense that each line is
centred. 
For proper centring of the first
and last lines of a paragraph the \cs{parindent} and
\cs{parfillskip} have to be made zero.
However, the margins are ragged.

A surprising application of \cs{leftskip} and \cs{rightskip}
\mdqon
\howto Centre the first/""last line of a paragraph\par
\mdqoff
leads to paragraphs with flush margins and a centred
last line.
\begin{verbatim}
\leftskip=0cm plus 0.5fil \rightskip=0cm plus -0.5fil
\parfillskip=0cm plus 1fil
\end{verbatim}

For all lines of a paragraph but the 
last one the stretch components
add up to zero so the \cs{leftskip} and \cs{rightskip}
inserted are zero.
On the last line the \cs{parfillskip} adds \hbox{\n{plus 1fil}}
of stretch; therefore there is a total of
\hbox{\n{plus 0.5fil}} of stretch at both the left and right
end of the line.

It would have been incorrect to specify
\begin{verbatim}
\leftskip=0cm plus 0.5fil \rightskip=0cm minus 0.5fil
\end{verbatim}
\TeX\ gives an error about this: it complains about
`infinite shrinkage'.

Centring not only the last line, but also the
first line of a paragraph can be done by
the parameter settings
\begin{verbatim}
\parindent=0pt \everypar{\hskip 0pt plus -1fil}
\leftskip=0pt plus .5fil
\rightskip=0pt plus -.5fil
\end{verbatim}
This time a horizontal skip inserted by \cs{everypar}
combines with the \cs{leftskip} to give the same
amount of stretchability on both sides of the
first line of the paragraph.

%\spoint Indenting into the margin
\subsection{Indenting into the margin}

Suppose you want a hanging indent of \n{1cm} {\sl into\/}
\howto Indent into the margin\par
the left margin after the first two lines of a paragraph. 
Specifying \verb/\hangindent=-1cm/ will give
a hanging indentation of one centimetre from the {\sl right\/}
margin, so another approach is necessary. The following does the
job:
\begin{verbatim}
\leftskip=-1cm \hangindent=1cm \hangafter=-2
\end{verbatim}
The only problem with this is that
the leftskip needs to be reset after the paragraph.
Suitable redefinition of \cs{par} removes this objection:
\begin{verbatim}
\def\hangintomargin{\bgroup
    \leftskip=-1cm \hangindent=1cm \hangafter=-2
    \def\par{\endgraf\egroup}}
\end{verbatim}
The redefinition of \cs{par} is here local to the paragraph that
should be outdented.

Another, elegant, solution uses \cs{parshape}:
\begin{verbatim}
 
\dimen0=\hsize \advance\dimen0 by 1cm
\parshape=3        % three lines:
    0cm\hsize      % first  line specification
    0cm\hsize      % second line specification
    -1cm\dimen0    % third  line specification
\end{verbatim}

%\spoint Hang a paragraph from an object
\subsection{Hang a paragraph from an object}

The \LaTeX\ format has a macro, \cs{@hangfrom}, to have
\howto Hang a paragraph from an object\par
one paragraph of text hanging from some object, usually a box
or a short line of text. 

\begingroup
\medskip
\def\hangobject{Example \ }
\setbox0=\hbox{\hangobject}
\hangindent \wd0 \noindent \box0
This paragraph is an example of the \cs{hangfrom} macro
defined below.
In the \LaTeX\ document
styles, the \cs{@hangfrom} macro (which is similar to this)
is used for multi-line section headings.\par
\endgroup

Consider then the macro \cs{hangfrom}:
\begin{verbatim}
\def\hangfrom#1{\def\hangobject{#1}\setbox0=\hbox{\hangobject}%
    \hangindent \wd0 \noindent \box0 \ignorespaces}
\end{verbatim}
Because of the default \cs{hangafter=1}, this 
will produce one line of width \cs{hsize}, after which the
rest of the paragraph will be left indented  by the width of the
\cs{box0}. (Note that using \cs{hangobject} instead of \cs{box0}
is incorrect, as that sets the glue in the hangobject a second time.)

%\spoint Another approach to hanging indentation
\subsection{Another approach to hanging indentation}

Hanging indentation can also be attained by a combination
of shifting the left margin and outdenting.
Itemized lists can for instance be implemented in this manner:
\begin{verbatim}
\newdimen\listindent
\def\itemize{\begingroup
    \advance\leftskip by \listindent
    \parindent=-\listindent}
\def\stopitemize{\par\endgroup}
\def\item#1{\par\leavevmode
    \hbox to \listindent{#1\hfil}\ignorespaces
    }
\end{verbatim}
If an item should encompass more than one paragraph, the
implementation could be
\begin{verbatim}
\newdimen\listindent \newdimen\listparindent
\def\itemize{\begingroup
    \advance\leftskip by \listindent
    \parindent=\listparindent}
\def\stopitemize{\par\endgroup}
\def\item#1{\par\noindent
    \hbox to 0cm{\kern-\listindent #1\hfil}\ignorespaces
    }
\end{verbatim}

\begin{example}
\begin{verbatim}
\itemize\item{1.}First item\par
Is two paragraphs long.
\item{2.}Second item.\stopitemize
\end{verbatim}
gives
\begin{disp}
\def\itemize{\begingroup
    \advance\leftskip by \parindent
    \parindent=1em\relax}
\def\stopitemize{\par\endgroup}
\def\item#1{\par\noindent
    \hbox to 0cm{\kern-\parindent #1\hfil}\ignorespaces
    }
\itemize\item{1.}First item\par
Is two paragraphs long.
\item{2.}Second item.\stopitemize
\end{disp}
\end{example}

%\spoint Hanging indentation versus \cs{leftskip} shifting
\subsection{Hanging indentation versus \cs{leftskip} shifting}

From the above examples it would seem that
hanging indentation and modifying the \cs{leftskip} and \cs{rightskip}
are interchangeable. They are, but only to a certain   extent.
\altt

Setting \cs{leftskip} to some positive value for a paragraph
means that the \cs{hsize} stays the same, but every line
starts with a glue item. Hanging indentation, on the other hand,
is implemented by decreasing the \cs{hsize} value for the
lines that hang, and shifting the finished 
horizontal boxes horizontally in the surrounding vertical list.

The difference between the two approaches becomes visible
mainly in the fact that display formulas are not shifted
when the \cs{leftskip} is altered.
See Chapter~\ref{rules} for an example showing how leaders
are affected by margin shifting.

%\spoint More examples
\subsection{More examples}

Some more examples of paragraph shapes (effected by
various means) can be found in~\cite{E1}. One example
from that article appears on page~\pageref{varioset}.

\index{paragraph!shape|)}
\endofchapter

\chapter{Line Breaking}\label{line:break}
\index{paragraph!breaking into lines|(}
\index{line breaking|(}

This chapter treats line breaking and the concept of `badness' that \TeX\
uses to decide how to break a paragraph into lines,
or where to break a page.
The various penalties contributing to the cost of line breaking
are treated here, as is hyphenation.
Page breaking is treated in Chapter~\ref{page:break}.

\label{cschap:penalty}\label{cschap:linepenalty}\label{cschap:hyphenpenalty}\label{cschap:exhyphenpenalty}\label{cschap:adjdemerits}\label{cschap:doublehyphendemerits}\label{cschap:finalhyphendemerits}\label{cschap:allowbreak}\label{cschap:pretolerance}\label{cschap:tolerance}\label{cschap:emergencystretch}\label{cschap:looseness}\label{cschap:prevgraf}\label{cschap:discretionary}\label{cschap:-}\label{cschap:hyphenchar2}\label{cschap:defaulthyphenchar2}\label{cschap:uchyph}\label{cschap:lefthyphenmin}\label{cschap:righthyphenmin}\label{cschap:patterns}\label{cschap:hyphenation}\label{cschap:language}\label{cschap:setlanguage}
\begin{inventory} 
\item [\cs{penalty}] 
      Specify desirability of not breaking at this point.

\item [\cs{linepenalty}] 
      Penalty value associated with each line break. 
      Plain \TeX\ default:~\n{10}.

\item [\cs{hyphenpenalty}] 
      Penalty associated with break at a discretionary item
      in the general case. 
      Plain \TeX\ default:~\n{50}.

\item [\cs{exhyphenpenalty}] 
      Penalty for breaking a horizontal line at a discretionary
      item in the special case where the prebreak text is empty. 
      Plain \TeX\ default:~\n{50}.

\item [\cs{adjdemerits}] 
      Penalty for adjacent visually incompatible lines. 
      Plain \TeX\ default:~\n{10$\,$000}.

\item [\cs{doublehyphendemerits}] 
      Penalty for consecutive lines ending with a hyphen. 
      Plain \TeX\ default: \n{10$\,$000}.

\item [\cs{finalhyphendemerits}] 
      Penalty added when the penultimate line of a 
      paragraph ends with a hyphen. 
      Plain \TeX\ default:~\n{5000}.

\item [\cs{allowbreak}] 
      Macro for creating a breakpoint by inserting a
      \cs{penalty0}.

\item [\cs{pretolerance}] 
      Tolerance value for a paragraph without hyphenation. 
      Plain \TeX\ default:~\n{100}.

\item [\cs{tolerance}] 
      Tolerance value for lines in a paragraph with hyphenation. 
      Plain \TeX\ default:~\n{200}.

\item [\cs{emergencystretch}]
      (\TeX3 only)
      Assumed extra stretchability in lines of a paragraph.

\item [\cs{looseness}]  
      Number of lines by which this paragraph has to be made longer 
      than it would be ideally.

\item [\cs{prevgraf}]  
      The number of lines in the paragraph last
      added to the vertical list.

\item [\cs{discretionary}] 
      Specify the way a character sequence is split up at a line break.

\item [\cs{-}] 
      Discretionary hyphen; this is
      equivalent to \verb|\discretionary{-}{}{}|.

\item [\cs{hyphenchar}] 
      Number of the hyphen character of a font.

\item [\cs{defaulthyphenchar}] 
      Value of \cs{hyphenchar} when a font is loaded.
      Plain \TeX\ default:~\n{`\cs{-}}.

\item [\cs{uchyph}] 
      Positive to allow hyphenation of words starting with a capital 
      letter.
      Plain \TeX\ default:~\n{1}.

\item [\cs{lefthyphenmin}] 
      (\TeX3 only)
      Minimal number of characters before a hyphenation.
      Plain \TeX\ default:~\n{2}.

\item [\cs{righthyphenmin}] 
      (\TeX3 only)
      Minimum number of characters after a hyphenation.
      Plain \TeX\ default:~\n{3}.

\item [\cs{patterns}] 
      Define a list of hyphenation patterns for the current
      value of \cs{language};  allowed only in \IniTeX.

\item [\cs{hyphenation}] 
      Define hyphenation exceptions for the current value of \cs{language}.

\item [\cs{language}] 
      Choose a set of hyphenation patterns and exceptions.

\item [\cs{setlanguage}] 
      Reset the current language.

\end{inventory}


%\point Paragraph break cost calculation
\section{Paragraph break cost calculation}

A paragraph is broken such that the amount $d$ of {\em demerits\/}
associated with breaking it is minimized. 
The total amount of demerits for a paragraph is the sum
of those for the individual lines, plus possibly some extra
penalties. Considering a paragraph as a whole instead of 
breaking it on a line-by-line basis can lead to better
line breaking: \TeX\ can choose to take a slightly less beautiful
line in the beginning of the paragraph in order to avoid
bigger trouble later on.

For each line demerits are calculated from the {\em badness\/}~$b$
of stretching or shrinking the line to the break, and
the {\em penalty\/}~$p$ associated with the break.
The badness is not allowed to exceed a certain prescribed
tolerance.

In addition to the demerits for breaking individual lines,
\TeX\ assigns demerits for the way lines combine; see below.

The
\mdqon
implementation of \TeX's paragraph"-breaking algorithm
\mdqoff
is explained in~\cite{K:break}.

%\spoint Badness
\subsection{Badness}

From the ratio between the stretch or shrink present in a 
line, and the actual stretch or shrink taken,
the \emph{badness}\index{badness! and line breaking}\index{line breaking!badness}
of breaking a line at a certain point is calculated.
This badness is an important
factor in the process of line breaking.
See page~\pageref{bad:form} for the formula for badness.

In this chapter
badness will only be discussed in the context of line breaking. 
Badness is also computed when a vertical list is stretched
or shrunk (see Chapter~\ref{page:break}). 

The following terminology is used to describe badness:
\begin{description} \item [tight (3)]
is any line that has shrunk with a badness~$b\geq13$, 
that is, by using at least one-half of its amount of shrink
(see page~\pageref{bad:form} for the computation).
\item [decent (2)]
is any line with a badness~$b\leq12$.
\item [loose (1)]
is any line that has stretched with a badness~$b\geq13$, 
that is, by using at least one-half of its amount of stretch.
\item [very loose (0)]
is any line that has stretched with a badness~$b\geq100$, 
that is, by using its full amount of stretch or more. Recall
that glue can stretch, but not shrink more than its
allowed amount.
\end{description}
The numbering is used in trace output (Chapter~\ref{trace}), and
it is also used in the following definition:
if the classifications of two adjacent lines differ by more than~1,
the lines are said to be {\em visually incompatible\/}.
See below for the \cs{adjdemerits} parameter associated with this.

Overfull horizontal and vertical
boxes are passed unnoticed if their excess width
or height is less than \cs{hfuzz} or \cs{vfuzz} respectively; 
they are not reported if the badness is less than
\cs{hbadness} or \cs{vbadness} (see Chapter~\ref{boxes}).

%\spoint Penalties and other break locations
\subsection{Penalties and other break locations}
 
Line breaks can occur at the following \emph{breakpoints}
\index{lists !horizontal! breakpoints in}
in horizontal lists:
\begin{enumerate} \item At a penalty. The penalty value is the
`aesthetic cost' of breaking the line at that place.
Negative penalties are considered as bonuses.
A~penalty of $10\,000$ or more inhibits, and a penalty
of $-10\,000$ or less forces, a~break.

Putting more than one penalty
in a row is equivalent to putting just the one with the
minimal value, because that one is the best candidate for line breaking.

Penalties in horizontal mode are inserted by the user (or a
user macro). The only exception is the \cs{nobreak}
inserted before the \cs{parfillskip} glue.

\item At a glue, if it is not part of a math formula, and
if it is preceded by a non-discardable item (see Chapter~\ref{hvmode}).
There is no penalty associated with breaking at glue.

The condition about the non-discardable precursor is necessary, 
because otherwise breaking in between  two pieces of glue would
be possible, which would cause ragged edges to the paragraph.

\item At a kern, if it is not part of a math formula
and if it is followed by  glue.
There is no penalty associated with breaking at a~kern.

\item At a math-off, if that is followed by glue.
Since math-off
(and math-on) act as kerns (see Chapter~\ref{math}),
this is very much like the previous case.
There is no penalty associated with breaking at a~math-off.

\item At a discretionary break. The penalty
is the \cs{hyphenpenalty} or the \cs{exhyphenpenalty}.
This is treated below.
\end{enumerate}

Any discardable material following the break \ldash glue, kerns,
\mdqon
math-on/""off and penalties \rdash  is discarded. If one considers
\mdqoff
a line break at glue (kern, math-on{/}off) to occur at the
front end of the glue item, this implies that that piece
of glue disappears in the break.

%\spoint Demerits
\subsection{Demerits}

From the badness of a line and the penalty, if any, the demerits
of the line are calculated. Let $l$ be the value of
\csterm{linepenalty}, $b$~the badness of the line,
$p$~the penalty at the break; then the \indexterm{demerits}~$d$
are given by
\begin{disp}$\displaystyle d=\cases{(l+b)^2+p^2&if $0\leq p<10\,000$\cr
           (l+b)^2-p^2&if $-10\,000<p<0$\cr
           (l+b)^2    &if $p\leq-10\,000$\cr}$\end{disp}

Both this formula and the one for the badness are described
\alt
in \cite{K:break} as `quite arbitrary',
but they have been shown  to lead to
good results in practice.

The demerits for a paragraph are the sum of the demerits for
the lines, plus 
\begin{itemize}
\item the \csterm{adjdemerits} for any two
      adjacent lines that are not visually compatible (see above),
\item \csterm{doublehyphendemerits} for any two
      consecutive lines ending with a hyphen, and the
\item \csterm{finalhyphendemerits}
      if the penultimate line of a paragraph
   ends with a hyphen.\end{itemize}

At the start of a paragraph \TeX\ acts as if
there was a preceding line which was `decent'.
Therefore \cs{adjdemerits} will be added if the first
line is `very loose'. Also, the last line
of a paragraph is ordinarily also `decent'
\ldash all spaces are set at natural width
owing to the infinite stretch in the \cs{parfillskip} \rdash 
so \cs{adjdemerits} are added if
the preceding line is `very loose'.

Note that the penalties at which a line break
is chosen weigh about as heavily as the badness of
the line, so they can be relatively small.
However, the three extra demerit parameters
have to be of the order of the square of 
penalties and badnesses to weigh equally heavily.

%\spoint The number of lines of a paragraph
\subsection{The number of lines of a paragraph}

After a paragraph has been completed (or partially
completed prior to a display), the variable \csterm{prevgraf}
records the number of lines in the paragraph.
By assigning to this variable \ldash and
because this is a \gr{special integer}
such an assignment is automatically global \rdash 
\TeX's decision processes can be influenced.
This may be useful in combination with hanging indentation
or \cs{parshape} specifications (see Chapter~\ref{par:shape}).

\mdqon
Some direct influence of the line"-breaking process
\mdqoff
on the resulting number of lines exists. One factor
is the \cs{linepenalty} which is included in the demerits 
of each line. By increasing the line penalty \TeX\ can be
made to minimize the number of lines in a paragraph.

Deviations from the optimal number of lines, that is, the
number of lines stemming from the optimal way of breaking a
paragraph into lines, can be forced by the user by means
of the \csterm{looseness} parameter. This parameter, which is
reset every time the shape parameters 
are cleared (see Chapter~\ref{par:shape}), 
indicates by how many lines the current
paragraph should be made longer than is optimal. A~negative
value of \cs{looseness} will attempt to make the paragraph shorter
by a number of lines that is the absolute value of the parameter.

\TeX\ will still observe the values
of \cs{pretolerance} and \cs{tolerance} (see below)
when lengthening or shortening a paragraph under influence
of \cs{looseness}.
Therefore,
\TeX\ will only lengthen or shorten a paragraph for as far
as is possible without exceeding these parameters.


%\spoint[between:lines] Between the lines
\subsection{Between the lines}
\label{between:lines}

\TeX's
paragraph mechanism packages lines into horizontal boxes
that are appended to the surrounding vertical list.
The resulting sequence of vertical items is then a 
repeating sequence of
\begin{itemize}\item a box containing a line of text,
\item possibly migrated vertical material (see page~\pageref{migrate}),
\item a penalty item reflecting the cost of a page break
      at that point, which is normally the \cs{interlinepenalty}
      (see Chapter~\ref{page:break}), and
\item interline glue, which is calculated automatically
      on basis of the \cs{prevdepth} (see Chapter~\ref{baseline}).
\end{itemize}

%\point The process of breaking
\section{The process of breaking}

\TeX\ tries to break paragraphs in such a way that 
the badness of each line does not exceed a certain tolerance.
If there exists more than one solution to this, the one with
the fewest demerits is taken.

By setting \csterm{tracingparagraphs} to a positive value,
\TeX\ can be made to report the calculations of the
paragraph mechanism in the log file. Some implementations of \TeX\
may have this option disabled to make \TeX\ run faster.

%\spoint Three passes
\subsection{Three passes}

First an attempt is made to split the paragraph into lines
without hyphenating, that is, without inserting discretionary
hyphens. This attempt succeeds if none of the
lines has a badness exceeding \csterm{pretolerance}.

Otherwise, a second pass is made, inserting discretionaries
and using \csterm{tolerance}.
\handbreak If \cs{pretolerance} is negative, the first pass is omitted.

\TeX\ can be made to make a third pass if the first and
second pass fail.
If \csterm{emergencystretch} is a positive dimension, 
\TeX\ will assume this much extra stretchability 
in each line when badness and demerits are calculated.
Thus solutions that only slightly exceeded the given
tolerances will now become feasible.
However, no glue of size \cs{emergencystretch} is
actually present, so underfull box messages
may still occur.

%\spoint Tolerance values
\subsection{Tolerance values}

How much
trouble \TeX\ will have typesetting a piece of text
depends partly on the tolerance value.
Therefore it is sensible to have some idea of
what badness values mean in visual terms.

For lines that are stretched, the badness is
100 times the cube of the stretch ratio.
A~badness of 800 thus means that the stretch ratio
is~2.
If the space is,
\alt
as in the ten-point Computer Modern Font,
\begin{verbatim}
3.33pt plus 1.67pt minus 1.11pt
\end{verbatim}
a badness of 800 means that spaces have been stretched to
\begin{disp} \n{3.33pt}${}+2\times{}$\n{1.67pt}${}={}$\n{6.66pt}\end{disp}
that is, to exactly double their natural size.
It is up to you to decide whether this is too large.

%\point Discretionaries
\section{Discretionaries}

A \indexterm{discretionary item} \verb-\discretionary{..}{..}{..}-
\csterm{discretionary}%
marks a place where a word can be broken.
Each of the three arguments is a \gr{general text}
(see Chapter~\ref{gramm}):
they are, in sequence,
\begin{itemize} \item the {\em pre-break\/} text, which is appended
to the part of the word before the break,
\item the {\em post-break\/} text, which is prepended to the part
of the word after the break, and
\item the {\em no-break\/} text, which is used if the word
is not broken at the discretionary item.\end{itemize}
For example: \verb>ab\discretionary{g}{h}{cd}ef>
is the word \hbox{\n{abcdef}}, but it can be hyphenated
\alt
with \n{abg} before the break and \n{hef} after.
Note that there is no automatic hyphen character.

All three texts may contain any sorts of tokens,
but any primitive commands and macros
should expand to boxes, kerns, and characters.

%\spoint Hyphens and discretionaries
\subsection{Hyphens and discretionaries}

Internally, \TeX\ inserts the equivalent of
\csterm{hyphenchar}%
\begin{verbatim}
\discretionary{\char\hyphenchar\font}{}{}
\end{verbatim}
at every place where a word can be broken. This causes a \indextermbus{hyphen}{character} to be placed before any break.
No
such discretionary is inserted if \verb>\hyphenchar\font>
is not in the range 0--255, or if its position in the
font is not filled.
When a font is loaded, its \cs{hyphenchar} value
is set to \csterm{defaulthyphenchar}. The \cs{hyphenchar}
value can be changed after this.

In plain \TeX\ the \cs{defaulthyphenchar} has the value~\verb>`\->, so
for all fonts character~45 (the \ascii\ hyphen character)
is the hyphen sign, unless
it is specified otherwise.

The primitive command \verb|\-| (called a `discretionary hyphen')
\csterm{-}\indexterm{discretionary hyphen}
is equivalent to the above
\handbreak
\verb|\discretionary{\char\hyphenchar\font}{}{}|.
Breaking at such a discretionary, whether inserted implicitly
by \TeX\ or explicitly by the user, has
a cost of \csterm{hyphenpenalty}.


In unrestricted horizontal mode an empty discretionary
\cs{disc\-re\-tio\-na\-ry}\verb-{}{}{}-
is automatically inserted after characters
whose character code is the \cs{hyphenchar} value 
of the font, thus enabling hyphenation at that point.
The penalty for breaking a line at
such a discretionary with an empty pre-break text
is \csterm{exhyphenpenalty}, that is, the `explicit hyphen' penalty.

If a word contains
discretionary breaks, for instance 
because of  explicit hyphen characters,
\TeX\ will not consider it for further hyphenation.
People have solved the ensuing problems by tricks
such as
\howto Enable hyphenation of a word containing a hyphen\par
\begin{verbatim}
\def\={\penalty10000 \hskip0pt -\penalty0 \hskip0pt\relax}
... integro\=differential equations...
\end{verbatim}
The skips before and after the hyphen lead \TeX\ into
treating the first and second half of the 
compound expression as separate words; the penalty
before the first skip inhibits breaking before the hyphen.

%\spoint Examples of discretionaries
\subsection{Examples of discretionaries}

\emph{Languages}\index{languages} such as German or Dutch have words
that change spelling when hyphenated (German: `\hbox{backen}' becomes
`\hbox{bak-ken}'; Dutch: `\hbox{autootje}' becomes
`\hbox{auto-tje}'). This problem can be solved with \TeX's
discretionaries.

For instance, for German (this is inspired by~\cite{Partl}):
\begin{verbatim}
\catcode`\"=\active
\def"#1{\ifx#1k\discretionary{k-}{k}{ck}\fi}
\end{verbatim}
which enables the user to write \verb>ba"ken>.

In Dutch there is a further problem which allows a nice
systematic solution. Umlaut characters (`trema' is the
Dutch term) should often
disappear in a break, for instance `\hbox{na"apen}'
hyphenates as `\hbox{na-apen}', and `\hbox{onbe"invloedbaar}'
hyphenates as `\hbox{onbe-invloedbaar}'. A solution
(inspired by~\cite{Babel}) is
\begin{verbatim}
\catcode`\"=\active
\def"#1{\ifx#1i\discretionary{-}{i}{\"\i}%
        \else  \discretionary{-}{#1}{\"#1}\fi}
\end{verbatim}
which enables the user to type \verb>na"apen> and
\verb>onbe"invloedbaar>.

%\point Hyphenation
\section{Hyphenation}

\TeX's \indexterm{hyphenation}
algorithm uses a list of patterns to
determine at what places a word that is a candidate for
hyphenation can be broken.
Those aspects of hyphenation connected with these
patterns are
treated in appendix~H of \TeXbook;
the method of generating hyphenation patterns automatically
is described in~\cite{Liang}. People have been known
to generate lists of patterns by hand; 
see for instance~\cite{Vas:add}. Such hand-generated lists
may be superior to automatically generated lists.

Here it will mainly be described how \TeX\ declares a word to
be a candidate for hyphenation. The  problem here is
how to cope with punctuation and things such as quotation marks
that can be attached to a word. Also, {\em implicit kerns\/},
that is, kerns inserted because of font information,
must  be handled properly.

%\spoint Start of a word
\subsection{Start of a word}

\TeX\ starts at glue items (if they are not in math mode)
looking for a {\em starting letter\/} of a word:
a character with non-zero \cs{lccode}, or a ligature starting
\mdqon
with such a character (upper/""lowercase codes are explained
\mdqoff
on page~\pageref{uc/lc}).
Looking for this starting letter,
\TeX\ bypasses any implicit kerns, and
characters with zero \cs{lccode} (this includes,
for instance, punctuation and quotation marks), 
or ligatures starting with
such a character.

If no suitable starting letter turns up, that is, if
something is found that is not a character or ligature,
\TeX\ skips to the next glue, and starts this algorithm anew.
Otherwise a trial word is collected consisting of
all following characters with non-zero \cs{lccode}
from the same font as the starting letter, or ligatures consisting
completely of such characters. Implicit kerns are allowed
between the characters and ligatures.

If the starting letter is from a font for which the value
of \cs{hyphenchar} is invalid, or for which this character
does not exist, hyphenation is abandoned for this word.
If the starting letter is an uppercase letter (that is,
it is not equal to its own \cs{lccode}), \TeX\ will
abandon hyphenation unless \csterm{uchyph} is positive.
The default value for this parameter is~1  in
plain \TeX,
implying that capitalized words are subject to hyphenation.

%\spoint End of a word
\subsection{End of a word}


Following the trial word can be characters (from another
font, or with zero \cs{lccode}), ligatures or implicit kerns.
After these items, if any, must follow
\begin{itemize}\item glue or an explicit kern,
\item a penalty,
\item a whatsit, or
\item a \cs{mark}, \cs{insert}, or \cs{vadjust} item.
\end{itemize}
In particular, the word will not be hyphenated if it is
followed by a 
\begin{itemize}\item box, \item rule, \item math
formula, or \item discretionary item.\end{itemize}

Since discretionaries are inserted after the \cs{hyphenchar}
of the font, occurrence of this character inhibits further
hyphenation. Also, placement of accents is implemented using
explicit kerns (see Chapter~\ref{char}), so any \cs{accent}
command is considered to be the end of a word, and inhibits
hyphenation of the word.

%\spoint \TeX2 versus \TeX3
\subsection{\TeX2 versus \TeX3}

There is a noticeable difference in the treatment of
\thecstoidxsub{TeX}{version 3}
hyphenated fragments between \TeX2 and \TeX3. 
\TeX2 insists that the part before the break should be
at least two characters, and the part after the break three
characters, long.
Typographically this is a sound decision: this way
there are no two-character pieces of a word stranded at the
end or beginning of the line. Both before and after the break
there are at least three characters.

In \TeX3 two integer parameters have been introduced to control
the length of these fragments:
\csterm{lefthyphenmin} and \csterm{righthyphenmin}. These are
set to 2 and~3 respectively in the plain format for \TeX3.
If the sum of these two is 63 or more, all hyphenation is
suppressed.

Another addition in \TeX3,
the possibility to have several sets of hyphenation patterns,
is treated below.

%\spoint Patterns and exceptions
\subsection{Patterns and exceptions}

The statements 
\begin{disp}\cs{patterns}\gr{general text}\nl
\csterm{hyphenation}\gr{general text}\end{disp}
are \gr{hyphenation assignment}s, which are
\gr{global assignment}s.
The \csterm{patterns} command, which specifies a list
of hyphenation patterns, is allowed only in \IniTeX\
(see Chapter~\ref{TeXcomm}),
and all patterns must be specified before the first
paragraph is typeset.

Hyphenation exceptions can be specified at any time
\howto Specify exceptional hyphenations\par
with statements such as
\begin{verbatim}
\hyphenation{oxy-mo-ron gar-goyle}
\end{verbatim}
which specify locations where a word may be hyphenated.
Subsequent \cs{hyphenation} statements are cumulative.

In \TeX3 these statements are taken to hold for the
language that is the current value of the \cs{language}
parameter.

%\point Switching hyphenation patterns
\section{Switching hyphenation patterns}

When typesetting paragraphs, \TeX\ (version~3) can use several
\alt
sets of patterns and hyphenation exceptions, for at most 256
\emph{languages}\index{language}. 

If a \cs{patterns} or \cs{hyphenation}
command is given (see above), \TeX\ stores the patterns or exceptions
under the current value of the \csterm{language} parameter.
The \cs{patterns} command is only allowed in \IniTeX, and
patterns must be specified before any typesetting is done.
Hyphenation exceptions, however, can
be specified cumulatively, and not only in \IniTeX.

In addition to the \cs{language} parameter, which can be set by the
user, \TeX\ has internally a \indextermbus{current}{language}. This is
set to zero at the start of every paragraph.  For every character that
is added to a paragraph the current language is compared with the
value of \cs{language}, and if they differ a whatsit element is added
to the horizontal list, resetting the current language to the value of
\cs{language}.

At the start of a paragraph, this whatsit is inserted
\altt
after the \cs{everypar} tokens, but \cs{lastbox}
can still access the indentation box.

As an example, suppose that a format has been created such that
language~0 is English, and language~1 is Dutch. English hyphenations
will then be used if the user does not specify otherwise;
if a job starts with
\begin{verbatim}
\language=1
\end{verbatim}
the whole document
will be set using Dutch hyphenations, because \TeX\ will insert
a command changing the current language at the start of
every paragraph. For example:
\begin{verbatim}
\language=1
T...
\end{verbatim}
gives
\begin{verbatim}
.\hbox(0.0+0.0)x20.0           % indentation
.\setlanguage1 (hyphenmin 2,3) % language whatsit
.\tenrm T                      % start of text
\end{verbatim}

The whatsit can be inserted explicitly, without changing
the value of \csterm{language}, by specifying
\begin{disp}\cs{setlanguage}\gr{number}\end{disp}
However, this will hardly ever be needed.
One case where it may be necessary is when the contents of
a horizontal box are unboxed to a paragraph: inside the box no
whatsits are added automatically, since inside such a box
no hyphenation can take place.
See page~\pageref{wide:vbox} for another problem with text
in horizontal boxes.

%%%% end of input file [par]

%\InputFile:space
%%%% this is input file [space]
%\subject[space] Spacing
\endofchapter
\index{paragraph!breaking into lines|)}
\index{line breaking|)}

\chapter{Spacing}\label{space}
\index{spacing|(}

The usual interword space in \TeX\ is specified in the
font information, but the user can override this.
This chapter explains the rules by which
\TeX\ calculates interword space.

\label{cschap:char322}\label{cschap:spaceskip}\label{cschap:xspaceskip}\label{cschap:spacefactor}\label{cschap:sfcode}\label{cschap:frenchspacing}\label{cschap:nonfrenchspacing}
\begin{inventory}

\item [\cs{\char32}]
      Control space.
      Insert the same amount of space as a space token would
      if \cs{spacefactor}${}=1000$.

\item [\cs{spaceskip}] 
      Interword glue if non-zero.

\item [\cs{xspaceskip}] 
      Interword glue if non-zero and \cs{spacefactor}${}\geq2000$.

\item [\cs{spacefactor}] 
      1000 times the ratio by which the stretch (shrink) component of the
      interword glue should be multiplied (divided).

\item [\cs{sfcode}] 
      Value for \cs{spacefactor} associated with a character.

\item [\cs{frenchspacing}]
      Macro to switch off extra space after punctuation.

\item [\cs{nonfrenchspacing}]
      Macro to switch on extra space after punctuation.

\end{inventory}


\section{Introduction}

In between words in a text, \TeX\ inserts space. This space has a
natural component, plus stretch and shrink to make justified
(right-aligned) text possible. Now, in certain styles of typesetting,
there is more space after punctuation. This chapter discusses the
mechanism that \TeX\ uses to realize such effect.

Here is the general idea:
\begin{itemize}
\item After every character token, the \cs{spacefactor} quantity is
  updated with the space factor code of that character.
\item When space is inserted, its natural size can be augmented
  (if \cs{spacefactor}${}\geq2000$), and in general its stretch is
  multiplied, and its shrink divided, by \cs{spacefactor}${}/1000$.
\item There are further rules, for instance so that in \n{...word.)
  And...} the space is modified according to the period, not the
  closing parenthesis.
\end{itemize}

%\point Automatic interword space
\section{Automatic interword space}


For every space token in horizontal mode the interword glue
of the current font
is inserted, with stretch and shrink components, all
determined by \cs{fontdimen} parameters.
To be specific, font dimension~2 is the normal interword space,
dimension~3 is the amount of stretch of the interword
space, and 4~is the amount of shrink. Font dimension
7 is called the `extra space'; see below (the list
of all the font dimensions appears on page~\pageref{font:dims}). 

Ordinarily all spaces between words (in one font) would be treated the
same. To allow for differently sized spaces \ldash for instance a
typeset equivalent of the double spacing after punctuation in
typewritten documents \rdash \TeX\ associates with each character a
so-called \indextermsub{space}{factor}.

When a character is added to the current horizontal list,
the space factor code (\csterm{sfcode})
of that character
is assigned to the space factor \csterm{spacefactor}.
There are two exceptions to this rule:
\begin{itemize}
\item When the space factor code is zero, the \cs{spacefactor} does
  not change. This mechanism allows space factors to persist through
  parentheses and such; see section~\ref{sec:sf-through-paren}.
\item When the space factor code of the last character is ${>}1000$
  and the current space factor is ${<}1000$, the space factor
  becomes~1000. This mechanism prevents elongated spaces after
  initials; see section~\ref{sec:sf-punct}.
\end{itemize}
The maximum space factor  is~$32\,767$.

The stretch component of the interword space is
multiplied by the space factor divided by 1000;
the shrink component is divided by this factor. 
The extra space (font dimension~7) is
added to the natural component of the
interword space when the space factor is~${}\geq2000$.

%\point User interword space
\section{User interword space}

The user can override the interword space contained in
the \cs{fontdimen} parameters
by setting the
\csterm{spaceskip} and the \csterm{xspaceskip} to non-zero values.
If \cs{spaceskip} is non-zero, it is taken instead
of the normal interword space
(\cs{fontdimen2} plus \cs{fontdimen3} minus \cs{fontdimen4}), but
a non-zero \cs{xspaceskip} is used as interword space if
the space factor is~${}\geq2000$.

If the \cs{spaceskip} is used,
its stretch and shrink components are
multiplied and divided respectively by \cs{spacefactor}$/1000$.

Note that, if \cs{spaceskip} and \cs{xspaceskip} are
defined in terms of \n{em}, they change with the font.

\begin{example} Let the following macros be given:
\begin{verbatim}
\def\a.{\vrule height10pt width4pt\spacefactor=1000\relax}
\def\b.{\vrule height10pt width4pt\spacefactor=3000\relax}
\def\c{\vrule height10pt width4pt\relax}
\end{verbatim} 
    then

%\begin{disp}\leavevmode\PopIndentLevel

\hbox{%
$\vcenter{\hsize=3in \snugbox{%
\begin{verbatim}
\vbox{
\fontdimen2\font=4pt % normal space 
\fontdimen7\font=3pt % extra space
\a. \b. \c\par
% zero extra space
\fontdimen7\font=0pt
\a. \b. \c\par
% set \spaceskip for normal space
\spaceskip=2\fontdimen2\font
\a. \b. \c\par
% set \xspaceskip
\xspaceskip=2pt
\a. \b. \c\par
}
\end{verbatim}
}}$%
%
\quad\quad gives\quad\quad
%
\message{Check snug and drop!}%
$\vcenter{\snugbox{\parindent0pt\parskip=0pt
\def\a.{\vrule height10pt width4pt\spacefactor=1000\relax}
\def\b.{\vrule height10pt width4pt\spacefactor=3000\relax}
\def\c{\vrule height10pt width4pt\relax}
\leavevmode\strut\par\hbox{}\hbox{}
% set the normal space and extra space
\fontdimen2\font=4pt \fontdimen7\font=3pt
\a. \b. \c\par \vskip2\baselineskip
% zero extra space
\fontdimen7\font=0pt
\a. \b. \c\par \vskip2\baselineskip
% set \spaceskip for normal space
\spaceskip=2\fontdimen2\font
\a. \b. \c\par \vskip2\baselineskip
% set \xspaceskip
\xspaceskip=2pt
\a. \b. \c\par \leavevmode\strut
}}$%
%
}
%\end{disp}

In all of these lines the glue is set at natural width. In the first
line the high space factor value after \cs{b} causes the extra
space \cs{fontdimen7} to be added. If this is zero (second line), the
only difference between space factor values is the stretch/shrink
ratio. In the third line the \cs{spaceskip} is taken
for all space factor values. If the \cs{xspaceskip} is nonzero,
it is taken (fourth line) instead of the \cs{spaceskip}
for the high value of the space factor.
\end{example}

%\point[tie] Control space and tie
\section{Control space and tie}
\label{tie}

The control character \csc{\char32}, \indextermbus{control}{space}
is a horizontal command
which inserts a space,  \csterm{\char32}
acting as if the current space factor is~1000.
However, it does not affect the value of \cs{spacefactor}.

Control space has two main uses. First, it is convenient to use after
a control sequence: \verb+\TeX\ is fun!+
Secondly, it can be used after abbreviations when \cs{nonfrenchspacing}
(see below) is in effect. For example:
\begin{verbatim}
\hbox spread 9pt{\nonfrenchspacing
      The Reverend Dr. Drofnats}
\end{verbatim}
gives
\begin{disp} \hbadness=10000 \leavevmode
\hbox spread 9pt{\nonfrenchspacing
      The Reverend Dr. Drofnats}\end{disp}
while
\begin{verbatim}
\hbox spread 9pt{\nonfrenchspacing
      The Reverend Dr.\ Drofnats}
\end{verbatim}
gives
\begin{disp} \hbadness=10000 \leavevmode
\hbox spread 9pt{\nonfrenchspacing
      The Reverend Dr.\ Drofnats}\end{disp}
(The \n{spread 9pt} is used to make the effect more visible.)

The active character (in the plain format) tilde or
\indexterm{tie},~\n{\char126},
\term ~@\char126\par
uses control space: it is defined as
\begin{verbatim}
\catcode`\~=\active
\def~{\penalty10000\ }
\end{verbatim}
Such an active tilde is called a `tie'; it inserts an ordinary 
amount of space, and prohibits breaking at this space.


%\point More on the space factor
\section{More on the space factor}

%\spoint Space factor assignments
\subsection{Space factor assignments}

The space factor of a particular character 
is contained in its \indexterm{spacefactor code}
and can be assigned as
\csterm{sfcode}%
\begin{disp}\cs{sfcode}\gr{8-bit number}\gr{equals}\gr{number}\end{disp}

\IniTeX\ assigns a space factor code of 1000 to all characters
\label{ini:sf}%
except uppercase characters; they get a space factor code of~999.
The plain format then assigns space factor codes greater than
1000 to various punctuation symbols, for instance
\verb-\sfcode`\.=3000-, which triples the stretch and shrink
after a full stop. Also, for all space factor values $\geq2000$
the extra space is added; see above.

%\spoint Punctuation
\subsection{Punctuation}
\label{sec:sf-punct}

Because the space factor cannot jump from a value below 1000
to one above, a punctuation symbol after an uppercase
character will not have the effect on the interword space
that punctuation after a lowercase character has.

\begin{example}
\begin{verbatim}
a% \sfcode`a=1000, space factor becomes 1000
.% \sfcode`.=3000, spacefactor becomes 3000
 % subsequent spaces will be increased.

A% \sfcode`A=999,  space factor becomes 999
.% \sfcode`.=3000, space factor becomes 1000
 % subsequent spaces will not be increased.
\end{verbatim}
\end{example}

Thus, initials
are not mistaken for sentence ends.
If an uppercase character does end a sentence, for instance
\begin{verbatim}
... and NASA.
\end{verbatim}
there are several solutions:
\begin{verbatim}
... NASA\spacefactor=1000.
\end{verbatim}
or
\begin{verbatim}
... NASA\hbox{}.
\end{verbatim}
which abuses the fact that after
a box the space factor is set to~1000.
The \LaTeX\ macro \cs{@} is equivalent to the first
possibility.

In the plain format two macros are defined that switch between
\csterm{frenchspacing}%
\csterm{nonfrenchspacing}%
uniform interword spacing, \indexterm{frenchspacing},
and extra space after punctuation, which is more an American custom.
The macro \cs{frenchspacing} sets the space factor code
of all punctuation to~1000; the macro \cs{nonfrenchspacing}
sets it to values greater than~1000.

Here are the actual definitions from \n{plain.tex}:
\begin{verbatim}
\def\frenchspacing{\sfcode`\.\@m \sfcode`\?\@m 
  \sfcode`\!\@m \sfcode`\:\@m 
  \sfcode`\;\@m \sfcode`\,\@m}
\def\nonfrenchspacing{\sfcode`\.3000 \sfcode`\?3000
  \sfcode`\!3000 \sfcode`\:2000
  \sfcode`\;1500 \sfcode`\,1250 }
\end{verbatim}
where
\begin{verbatim}
\mathchardef\@m=1000
\end{verbatim}
is given in the plain format.

French spacing is a somewhat controversial issue:
\TeXbook\ acts as if non-French spacing
is standard practice in printing, but for instance in~\cite{Hart}
one finds `The space of the line should be used after
all points in normal text'.
Extra space after punctuation 
may be considered a `typewriter habit', but this is
not entirely true. It used to be a lot more common
than it is nowadays, and there are rational arguments
against it: the full stop (point, period) at the end of a
sentence, where extra punctuation is most visible, 
is rather small, so it carries some extra visual space 
 of its own above it. This book does not use extra space
after punctuation.

%\spoint Other non-letters
\subsection{Other non-letters}
\label{sec:sf-through-paren}

The zero value of the space factor code makes
characters that are not a letter and not punctuation
`transparent' for the space factor.

\message{check break after Example}
\begin{example}
\begin{verbatim}
a% \sfcode`a=1000, space factor becomes 1000
.% \sfcode`.=3000, spacefactor becomes 3000
 % subsequent spaces will be increased.

a% \sfcode`a=1000, space factor becomes 1000
.% \sfcode`.=3000, space factor becomes 3000
)% \sfcode`)=0,    space factor stays 3000
 % subsequent spaces will be increased.
\end{verbatim}
\end{example}

%\spoint Other influences on the space factor
\subsection{Other influences on the space factor}

The space factor is 1000 when \TeX\ starts forming a
horizontal list, in particular after \cs{indent}, \cs{noindent},
and directly after a display. It is also 1000 after
a \cs{vrule}, an accent, or a \gr{box} (in horizontal mode), but
it is not influenced by \cs{unhbox} or \cs{unhcopy}
commands.

In the first column of a \cs{valign} the space factor of
the surrounding horizontal list is carried over; similarly,
after a vertical alignment the space factor is set to the
value reached in the last column.

%%%% end of input file [space]

%\InputFile:math
%%%% this is input file [math]
%\subject[mathchar] Characters in Math Mode
\endofchapter
\index{spacing|)}

\chapter{Characters in Math Mode}\label{mathchar}

In math mode every character specifies by its
\cs{mathcode} what position of
a font to access, among other things.
For delimiters this story is a bit
more complicated. This chapter explains the concept
of math codes, and shows how \TeX\ implements variable
size delimiters.
 
\label{cschap:mathcode}\label{cschap:mathchar}\label{cschap:mathchardef}\label{cschap:delcode}\label{cschap:delimiter}\label{cschap:delimiterfactor}\label{cschap:delimitershortfall}\label{cschap:nulldelimiterspace}\label{cschap:left}\label{cschap:right}\label{cschap:big}\label{cschap:Big}\label{cschap:bigg}\label{cschap:Bigg}\label{cschap:bigl}\label{cschap:bigm}\label{cschap:bigr}\label{cschap:radical}\label{cschap:mathaccent}\label{cschap:skewchar}\label{cschap:defaultskewchar}\label{cschap:skew}\label{cschap:widehat}\label{cschap:widetilde}
\begin{inventory}
\item [\cs{mathcode}] 
      Code of a character determining its treatment in math mode.

\item [\cs{mathchar}] 
      Explicit denotation of a mathematical character.

\item [\cs{mathchardef}] 
      Define a control sequence to be a synonym for
      a~math character code.

\item [\cs{delcode}] 
      Code specifying how a character should be used as delimiter.

\item [\cs{delimiter}] 
      Explicit denotation of a delimiter.

\item [\cs{delimiterfactor}] 
      1000 times the fraction of a delimited formula that should be
      covered by a delimiter.
      Plain \TeX\ default:~\n{901}

\item [\cs{delimitershortfall}] 
      Size of the part of a delimited formula that is allowed 
      to go uncovered by a delimiter.
      Plain \TeX\ default:~\n{5pt}

\item [\cs{nulldelimiterspace}] 
      Width taken for empty delimiters. 
      Plain \TeX\ default:~\n{1.2pt}

\item [\cs{left}] 
      Use the following character as an open delimiter.

\item [\cs{right}] 
      Use the following character as a closing delimiter.

\item [\cs{big}] 
      One line high delimiter.

\item [\cs{Big}] 
      One and a half line high delimiter.

\item [\cs{bigg}] 
      Two lines high delimiter.

\item [\cs{Bigg}] 
      Two and a half lines high delimiter.

\item [\cs{bigl {\MainFont etc.}}]
      Left delimiters.

\item [\cs{bigm {\MainFont etc.}}]
      Delimiters used as binary relations.

\item [\cs{bigr {\MainFont etc.}}]
      Right delimiters.

\item [\cs{radical}] 
      Command for setting things such as root signs.

\item [\cs{mathaccent}] 
      Place an accent in math mode.

\item [\cs{skewchar}] 
      Font position of an after-placed accent.

\item [\cs{defaultskewchar}] 
      Value of \cs{skewchar} when a font is loaded.

\item [\cs{skew}] 
      Macro to shift accents on top of characters explicitly.

\item [\cs{widehat}]
      Hat accent that can
      accommodate wide expressions.

\item [\cs{widetilde}]
      Tilde accent that can
      accommodate wide expressions.

\end{inventory}

\section{Mathematical characters}
\index{math characters|(}

Each of the 256 permissible character codes has
an associated \csterm{mathcode}, which can be assigned by
\begin{disp}\cs{mathcode}\gr{8-bit number}\gr{equals}\gr{15-bit number}\end{disp}
When processing in math mode, \TeX\ replaces all characters of
categories 11 and~12, and \cs{char} and \cs{chardef} characters,
by their associated mathcode.

The  15-bit math code is most conveniently denoted hexadecimally
as \verb-"xyzz-, where
\begin{disp}
\n x${}\leq7$ is the class (see page~\pageref{math:class}),\nl
\n y is the font family number \alt
(see Chapter~\ref{mathfont}), and \nl
\n{zz} is the position of the character in the font.\end{disp}

Math codes can also be specified directly by 
\csterm{mathchar}%
\csterm{mathchardef}%
a \gr{math character}, which can be\label{math:character}
\begin{itemize}\item\cs{mathchar}\gr{15-bit number}; 
\item \gr{mathchardef token}, a control sequence that was defined by
\begin{disp}\cs{mathchardef}\gr{control sequence}\gr{equals}\gr{15-bit number}
\end{disp}
 or
\item a delimiter command\alt
\begin{disp}\cs{delimiter}\gr{27-bit number}\end{disp}
 where the last 12 bits
are discarded.\end{itemize}
The commands \cs{mathchar} and \cs{mathchardef}
are analogous to \cs{char} and \cs{char\-def} in text mode.
Delimiters are treated below.
A~\gr{mathchardef token} 
can be used as a \gr{number}, even outside math mode.

In \IniTeX\ all letters receive \cs{mathcode} \verb-"71zz- and
all digits receive \verb-"70zz-, where \verb-"zz- is the 
hexadecimal position of the character in the font.
Thus, letters are initially from family~1
(math italic in plain \TeX), and digits are from family~0
(roman).
For all other characters, \IniTeX\ assigns
\begin{disp}\cs{mathcode}$\,x=x$,\end{disp}
thereby placing them also in family~0.

If the mathcode is \verb-"8000-,
\label{mcode:8000}the smallest integer that is
not a \gr{15-bit number}, the character is treated as an active
character with the original character code. Plain \TeX\
assigns a \cs{mathcode} of \verb-"8000- to the space, underscore and prime.

\index{math characters|)}

\section{Delimiters}
\index{delimiters|(}

After \csterm{left} and \csterm{right}
commands \TeX\ looks for a delimiter. A~delimiter
is either an explicit \cs{delimiter} command (or a
macro abbreviation for it), or a character with a non-zero
delimiter code.

The \cs{left} and \cs{right} commands
implicitly delimit a group, which is considered as a subformula.
Since the enclosed formula can
be arbitrarily large, the quest for the proper delimiter is
a complicated story of looking at variants in two different
fonts, linked chains of variants in a font, and building
extendable delimiters from repeatable pieces.

The fact that a group enclosed in \verb>\left...\right> is
treated as an independent subformula implies that a
sub- or superscript at the start of this formula is
not considered to belong to the delimiter. 
For example, \TeX\ acts as if 
\verb>\left(_2> is equivalent to \verb>\left({}_2>.
(A~subscript after a \cs{right} delimiter is positioned
with respect to that delimiter.)

%\spoint[delcodes] Delimiter codes 
\subsection{Delimiter codes }
\label{delcodes}

To each character code there corresponds a 
\csterm{delcode}%
\indexterm{delimiter code},
assigned by
\begin{disp}\cs{delcode}\gr{8-bit number}\gr{equals}%
          \gr{24-bit number}\end{disp}
A delimiter code thus consists of six hexadecimal digits
\verb-"uvvxyy-, where
\begin{disp}
\n{uvv} is the small variant of the delimiter, and\nl
\n{xyy} is the large variant;\nl
\n u, \n x are the font families of the variants, and\nl
\n{vv}, \n{yy} are the locations in those fonts.\end{disp}
Delimiter codes are used after \cs{left} and \cs{right}
commands.
\IniTeX\ sets all delimiter codes to~$-1$,
except\label{ini:del}
\verb-\delcode`.=0-, which makes the period an empty delimiter.
In plain \TeX\ delimiters have typically \n{u}${}=2$ and~\n{x}${}=3$,
that is, first family~2 is tried, and if no big
enough delimiter turns up family~3 is tried.


%\spoint Explicit \cs{delimiter} commands
\subsection{Explicit \cs{delimiter} commands}

Delimiters can also be denoted 
\csterm{delimiter}%
explicitly by a \gr{27-bit number},
\begin{verbatim}
\delimiter"tuvvxyy
\end{verbatim}
where \n{uvvxyy} are the small and large variant of the
delimiter as above;
the extra digit \n{t} (which is~$<8$) denotes the class
(see page~\pageref{math:class}).
For instance, the \cs{langle} macro is defined as
\begin{verbatim}
\def\langle{\delimiter "426830A }
\end{verbatim}
which means it belongs to class~4, opening. Similarly,
\cs{rangle} is of class~5, closing; and \cs{uparrow} is of class~3,
relation.

After \cs{left} and \cs{right} \ldash that is, when \TeX\
is looking for a delimiter \rdash  the class digit is ignored;
otherwise \ldash when \TeX\ is not looking for a delimiter \rdash 
the rightmost three digits are ignored, and the
four remaining digits are treated as a~\cs{mathchar}; see above.

%\spoint[successor] Finding a delimiter; successors
\subsection{Finding a delimiter; successors}
\label{successor}
\index{delimiter!size|(}

Typesetting a delimiter is a somewhat involved affair.
First \TeX\ determines the size $y$ of the formula to be covered,
which is twice the maximum of the height and depth of the
formula. Thus the formula may not look optimal if
it is not centred itself.

The size of the delimiter should be at least 
\csterm{delimiterfactor}${}\times y/1000$ and at least 
$y-{}$\csterm{delimitershortfall}.
\TeX\ then tries first the small variant, and if that one
is not satisfactory (or if the \n{uvv} part of the delimiter
is~\n{000}) it tries the large variant. If trying the large variant
does not meet with success, \TeX\ takes the largest delimiter
encountered in this search; if no delimiter at all was found
(which can happen if  the \n{xyy} part is
\altt
also~\n{000}),
an empty box of width~\csterm{nulldelimiterspace} is taken.

Investigating a variant means, in sequence,
\begin{itemize}
\item if the current style (see page~\pageref{math:styles}) is
  scriptscriptstyle, the \cs{scriptscriptfont} of the family is tried;
\item if the current style is scriptstyle or smaller, the
  \cs{scriptfont} of the family is tried;
\item otherwise the \cs{textfont} of the family is tried.
\end{itemize}
The plain format puts the \verb-cmex10- font  in all three
styles of family~3.

Looking for a delimiter at a certain position in a certain font
means
\begin{itemize}
\item if the character is large enough, accept it;
\item if the character is
  \emph{extendable}\index{character!extendable}, accept it;
\item otherwise, if the character has a \indexterm{successor}, that
  is, it is part of a chain of increasingly bigger delimiters in the
  same font, try the successor.
\end{itemize}
Information about
successors and extensibility of a delimiter is coded in the font
metric file of the font.  An extendable character has a top, a bottom,
possibly a mid piece, and a piece which is repeated directly below the
top piece, and directly above the bottom piece if there is a mid
piece.

\index{delimiter!size|)}

\subsection{\cs{big}, \cs{Big}, \cs{bigg}, and \cs{Bigg}
delimiter macros}

In order to be able to use a delimiter outside the 
\verb-\left...\right- context, or to specify a delimiter of
a different size than \TeX\ would have chosen,
four macros for `big' delimiters exist: \cs{big},
\cs{Big}, \cs{bigg}, and \cs{Bigg}. These can be used with
anything that can follow \cs{left} or \cs{right}.

Twelve further macros (for instance \cs{bigl}, \cs{bigm},
\csterm{big \rm etc.}%
and~\cs{bigr}) force such delimiters in the context of
an opening symbol, a binary relation, and a closing symbol
respectively:
\begin{verbatim}
\def\bigl{\mathopen\big}
\def\bigm{\mathrel\big} \def\bigr{\mathclose\big}
\end{verbatim}

The `big' macros themselves put the requested delimiter and
a null delimiter around an empty vertical box:
\begin{verbatim}
\def\big#1{{\nulldelimiterspace=0pt \mathsurround=0pt
            \hbox{$\left#1\vbox to 8.5pt{}\right.$}}}
\end{verbatim}
As an approximate measure,
the \n{Big} delimiters are one and a half times as large (11.5pt) as
\n{big} delimiters; \n{bigg} ones are twice (14.5pt), and \n{Bigg}
ones are two and a half times as large (17.5pt).

\index{delimiters|)}

\section{Radicals}

A \indexterm{radical} is a compound of a left delimiter and an overlined
math expression.
\csterm{radical}%
The overlined expression is set in the
cramped version of the surrounding style
\alt
(see page~\pageref{math:styles}).

In the plain format and the Computer Modern
math fonts there is only one radical: the square root
construct
\begin{verbatim}
\def\sqrt{\radical"270370 }
\end{verbatim}
The control sequence \cs{radical} is followed by a \gr{24-bit number}
which specifies a small and a large variant of the left delimiter
as was explained above. Joining the delimiter and the rule
is done by letting the delimiter have a large depth, and a height
which is equal to the desired rule thickness. The rule can then
be placed on the current baseline. After the delimiter and the
ruled expression have been joined the whole is shifted 
vertically to achieve the usual vertical centring 
(see Chapter~\ref{math}).

%\point Math accents
\section{Math accents}
\index{accents in math mode|(}

Accents in math mode are specified by
\csterm{mathaccent}%
\begin{disp}\cs{mathaccent}\gr{15-bit number}\gr{math field}\end{disp}
Representing the 15-bit number as \verb>"xyzz>,
only the family~\n{y} and the character position~\n{zz}
are used: an accented expression acts as \cs{mathord} expression
(see Chapter~\ref{math}).

In math mode whole expressions can be accented,
\alt
whereas in text mode only characters can be accented.
Thus in math mode accents can be stacked. However, the top
accent may (or, more likely, will) not be properly positioned
horizontally. Therefore the plain format has a macro \csterm{skew}
that effectively shifts the top accent. Its definition is
\begin{verbatim}
\def\skew#1#2#3{{#2{#3\mkern#1mu}\mkern-#1mu}{}}
\end{verbatim}
and it is used for instance like
\begin{verbatim}
$\skew4\hat{\hat x}$
\end{verbatim}
\message{skew thing.}
%which gives~{\font\tmp=cmmi10 $\textfont\VMIfam=\tmp\skew4\hat{\hat x}$}.
which gives~{$\skew4\hat{\hat x}$}.

For the correct positioning of accents over single characters
the symbol and extension font have a \csterm{skewchar}:
this is the largest accent that adds to the width of an
accented character. Positioning of any accent
is based on the width of the character to be accented,
followed by the skew character. 

The skew characters of the Computer Modern
math italic and symbol fonts are character \n{\hex7F},
\alt
`$\mathchar"7F$',\message{skew characters}
and \n{\hex30}, `$\mathchar"30$', respectively. The \csterm{defaultskewchar}
value is assigned to the \cs{skewchar} when a font is loaded.
In plain \TeX\ this is~\n{-1}, so fonts ordinarily have no
\cs{skewchar}.

Math accents can adapt themselves to the size of the accented
expression: \TeX\ will look for a successor of an accent
in the same way that it looks for a successor of a delimiter.
In the Computer Modern math fonts this mechanism is used in
\csterm{widehat}%
\csterm{widetilde}%
the \cs{widehat} and \cs{widetilde} macros.
For example,
\begin{disp}\verb>\widehat x>, \verb>\widehat{xy}>, \verb>\widehat{xyz}>
\end{disp} give
\begin{disp}$\widehat x$, $\widehat{xy}$, $\widehat{xyz}$
\end{disp} respectively.

\index{accents in math mode|)}

\endofchapter
\chapter{Fonts in Formulas}\label{mathfont}

For math typesetting a single current font is not sufficient, as it
is for text typesetting. Instead \TeX\ uses several font families,
and each family can contain three fonts. This chapter
explains how font families are organized, and how \TeX\ determines
from what families characters should be taken.

\label{cschap:fam}\label{cschap:newfam}\label{cschap:textfont}\label{cschap:scriptfont}\label{cschap:scriptscriptfont}
\begin{inventory}

\item [\cs{fam}]  
      The number of the current font family.

\item [\cs{newfam}]  
      Allocate a new math font family.

\item [\cs{textfont}]   
      Access the textstyle font of a family.\alt

\item [\cs{scriptfont}]  
      Access the scriptstyle font of a family.\alt

\item [\cs{scriptscriptfont}]  
      Access the scriptscriptstyle font of a family.\alt

\end{inventory}

%\point Determining the font of a character in math mode
\section{Determining the font of a character in math mode}

The characters in math formulas can be taken from several
different fonts (or better, \indexterm{font families}) without any user
commands. For instance, in plain \TeX\ math formulas use
the roman font, the math italic font, 
the symbol font and the math extension font.

In order to determine from which font a character is to be
taken, \TeX\ considers for each character in a formula its
\cs{mathcode} (this is treated in Chapter~\ref{mathchar}).
A~\cs{mathcode} is a 15-bit number of the form
\verb."xyzz., where the hex digits
have the following meaning:
\begin{disp}
\n x:~class,\nl
\n y:~family,\nl
\n{zz}:~position in font.\end{disp}

In general only the family determines from what font 
a character is to be taken.
The class of a math character is mostly used to
control spacing and other aspects of  typesetting.
Typical classes include `relation', `operator', `delimiter';
see section~\ref{math:class} for details.

Class~7 is special in this respect: 
it is called `variable family'. 
If a character has a \cs{mathcode} of the form \verb."7yzz.
it is taken from family \n{y},
unless the parameter \cs{fam} has a value in the range 0--15;
then it is taken from family~\cs{fam}.


%\point Initial family settings
\section{Initial family settings}

Both lowercase and uppercase letters
are defined by \IniTeX\ to have math codes \verb>"71zz>,
\label{ini:fam}%
which means that they are of variable family, initially from
family~1.
As \TeX\ sets \verb.fam=-1., that is,
an invalid value, when a formula starts, 
characters are indeed taken from
family~1, which in plain \TeX\ is math italic.

Digits have math code \verb>"70zz> so they are initially from
family~0, in plain \TeX\ the roman font. 
All other character codes have a mathcode
assigned by \IniTeX\ as
\begin{disp}\cs{mathcode}$\,x=x$\end{disp} which puts them in class~0,
ordinary, and family~0, roman in plain \TeX.

In plain \TeX, commands such as \cs{sl} then set both a font and
a family:
\begin{verbatim}
\def\sl{\fam\slfam\tensl}
\end{verbatim}
so putting \cs{sl} in a formula will cause all letters, digits,
and  uppercase Greek characters, to change to
slanted style.

In most cases, any font can be assigned to any family, but
two families in \TeX\ have a special meaning: these are
families 2 and~3.
For instance, their number of \cs{fontdimen} parameters
is different from the usual~7. Family~2 needs 22 parameters,
and family~3 needs~13. These parameters have all a very
specialized meaning for positioning in math typesetting. 
Their meaning is explained below, but for the full story
the reader is referred to appendix~G of \TeXbook.


%\point Family definition
\section{Family definition}

\TeX\ can access 16 families of fonts in math mode;
font families have numbers 0--15. 
The number of the
current family is recorded in the parameter~\csterm{fam}.

The macro \csterm{newfam} gives the number of an unused family.
This number is assigned using \cs{chardef} to the control sequence.


Each font family can have a font meant for text style, script style,
and scriptscript style. Below it is explained how \TeX\
determines in what style a (sub-) formula is to be typeset.

Fonts are assigned to a family
\csterm{textfont}%
\csterm{scriptfont}%
\csterm{scriptscriptfont}%
as follows:
\begin{verbatim}
\newfam\MyFam
\textfont\MyFam=\tfont \scriptfont\MyFam=\sfont
\scriptscriptfont\MyFam=\ssfont
\end{verbatim}
for the text, script, and scriptscript fonts of a family.
In general it is not necessary to fill all three members
of a family (but it is for family~3). 
If \TeX\ needs a character from a family member
that has not been filled,
it uses the \cs{nullfont} instead,
a~primitive font that has no characters (nor a \n{.tfm} file).


%\point Some specific font changes
\section{Some specific font changes}

%\spoint Change the font of ordinary characters and uppercase Greek
\subsection{Change the font of ordinary characters and uppercase Greek}

All letters and the uppercase Greek characters are
by default in plain \TeX\ of class~7,
variable family, so changing \cs{fam} will change the font
from which they are taken.
For example
\begin{verbatim}
{\fam=9 x}
\end{verbatim}
 gives an \n{x} from family~9.

Uppercase Greek characters are defined by
\cs{mathchardef} statements in the plain format as \verb>"70zz>,
that is, variable family, initially roman.
Therefore, uppercase Greek character also change with the family.

%\spoint Change uppercase Greek independent of text font
\subsection{Change uppercase Greek independent of text font}

In the Computer Modern font layout, uppercase Greek letters
are part of the roman font; see page~\pageref{cmr:table}.
\alt
Therefore, introducing another
text font (with another layout)
will change the uppercase Greek characters
(or even make them disappear).
One way of remedying this is by introducing a new family in
which the \n{cmr} font, which contains the uppercase Greek,
resides.
The control sequences accessing these characters then have
to be redefined:
\begin{verbatim}
\newfam\Kgreek 
\textfont\Kgreek=cmr10 ...
\def\hex#1{\ifcase#10\or 1\or 2\or 3\or 4\or 5\or 6\or
    7\or 8\or 9\or A\or B\or C\or D\or E\or F\fi}
\mathchardef\Gamma="0\hex\Kgreek00 % was: "0100
\mathchardef\Beta ="0\hex\Kgreek01 % was: "0101
\mathchardef\Gamma ...
\end{verbatim}
Note, by the way,
the absence of a either a space or a \cs{relax} token after
\n{\#1} in the definition of \cs{hex}. This implies that this
macro can only be called with an argument that is a 
control sequence.

%\spoint Change the font of lowercase Greek 
\subsection{Change the font of lowercase Greek
       and mathematical symbols}

Lowercase Greek characters have math code
\verb>"01zz>, meaning they are always from the math italic family. 
In order to change this one might redefine them,
\mdqon
for instance \verb.\mathchardef.""\verb.\alpha="710B., 
\mdqoff
to make them variable family.
This is not done in plain \TeX, because the Computer Modern
roman font does not
have Greek lowercase, although it does have the uppercase characters.

Another way is to redefine them like \verb.\mathchardef\alpha="0n0B.
where \n{n} is the (hexadecimal) number of a family
compatible with math italic, containing for instance a bold
math italic font.


%\point Assorted remarks
\section{Assorted remarks}

%\spoint New fonts in formulas
\subsection{New fonts in formulas}

There are two ways to access a font inside mathematics.
\howto Change fonts in a math formula\par
\mdqon
After \cs{font}""\cs{newfont=....} it is not possible to get
\mdqoff
the `a' of the new font by \verb-$...{\newfont a}...$-
because \TeX\ does not look at the current font in math mode.
What does work is
\begin{verbatim}
$ ... \hbox{\newfont a} ...$
\end{verbatim}
but this precludes the use of the new font in script and 
scriptscript styles.

The proper solution  takes a bit more work:
\begin{verbatim}
\font\newtextfont=... 
\font\newscriptfont=... \font\newsscriptfont=...
\newfam\newfontfam
\textfont\newfontfam=\newtextfont
\scriptfont\newfontfam=\newscriptfont
\scriptscriptfont\newfontfam=\newsscriptfont
\def\newfont{\newtextfont \fam=\newfontfam}
\end{verbatim}
after which the font can be used as
\begin{verbatim}
$... {\newfont a_{b_c}} ...$
\end{verbatim}
in all three styles.

%\spoint Evaluating the families
\subsection{Evaluating the families}

\TeX\ will only look at what is actually in the \cs{textfont}
et cetera of the various families at the end of the whole
formula. Switching fonts in the families is thus not possible
inside a single formula.
The number of 16 families may therefore turn out to be restrictive
for some applications.


%\subject[math] Mathematics Typesetting
\endofchapter
\chapter{Mathematics Typesetting}\label{math}

\TeX\ has two math modes, display and non-display, and
four styles, display, text, script, and scriptscript style, and
\altt
every object in math mode belongs to one of eight classes.
This chapter treats these concepts.


\label{cschap:everymath}\label{cschap:displaystyle}\label{cschap:textstyle}\label{cschap:scriptstyle}\label{cschap:scriptscriptstyle}\label{cschap:mathchoice}\label{cschap:mathord}\label{cschap:mathop}\label{cschap:mathbin}\label{cschap:mathrel}\label{cschap:mathopen}\label{cschap:mathclose}\label{cschap:mathpunct}\label{cschap:mathinner}\label{cschap:mathaccent2}\label{cschap:vcenter}\label{cschap:limits}\label{cschap:nolimits}\label{cschap:displaylimits}\label{cschap:scriptspace}\label{cschap:nonscript}\label{cschap:mkern}\label{cschap:mskip}\label{cschap:muskip}\label{cschap:muskipdef}\label{cschap:newmuskip}\label{cschap:thinmuskip}\label{cschap:medmuskip}\label{cschap:thickmuskip}\label{cschap:mathsurround}\label{cschap:over}\label{cschap:atop}\label{cschap:above}\label{cschap:overwithdelims}\label{cschap:atopwithdelims}\label{cschap:abovewithdelims}\label{cschap:underline}\label{cschap:overline}\label{cschap:relpenalty}\label{cschap:binoppenalty}
\begin{inventory}
\item [\cs{everymath}] 
      Token list inserted at the start of a non-display formula.

\item [\cs{everydisplay}]
      Token list inserted at the start of a display formula.

\item [\cs{displaystyle}] 
      Select the display style of mathematics typesetting.

\item [\cs{textstyle}] 
      Select the text style of mathematics typesetting.

\item [\cs{scriptstyle}] 
      Select the script style of mathematics typesetting.

\item [\cs{scriptscriptstyle}] 
      Select the scriptscript style of mathematics typesetting.

\item [\cs{mathchoice}] 
      Give four variants of a formula for the four styles
      of mathematics typesetting.

\item [\cs{mathord}] 
      Let the following character or subformula function 
      as an ordinary object.

\item [\cs{mathop}] 
      Let the following character or subformula function 
      as a large operator.

\item [\cs{mathbin}] 
      Let the following character or subformula function 
      as a binary operation.

\item [\cs{mathrel}] 
      Let the following character or subformula function as a relation.

\item [\cs{mathopen}] 
      Let the following character or subformula function 
      as a opening symbol.

\item [\cs{mathclose}] 
      Let the following character or subformula function
      as a closing symbol.

\item [\cs{mathpunct}] 
      Let the following character or subformula function 
      as a punctuation symbol.

\item [\cs{mathinner}] 
      Let the following character or subformula function 
      as an inner formula.

\item [\cs{mathaccent}] 
      Place an accent in math mode.

\item [\cs{vcenter}] 
      Construct a vertical box, vertically centred
      on the math axis.

\item [\cs{limits}] 
      Place limits over and under a large operator.

\item [\cs{nolimits}] 
      Place limits of a large operator as subscript and 
      superscript expressions.

\item [\cs{displaylimits}] 
      Restore default placement for limits.

\item [\cs{scriptspace}] 
      Extra space after subscripts and superscripts.
      Plain \TeX\ default:~\n{0.5pt}

\item [\cs{nonscript}] 
      Cancel the next glue item if it occurs in 
      scriptstyle or scriptscriptstyle.

\item [\cs{mkern}] 
      Insert a kern measured in mu units.

\item [\cs{mskip}] 
      Insert glue measured in mu units.

\item [\cs{muskip}] 
      Prefix for skips measured in mu units. 

\item [\cs{muskipdef}] 
      Define a control sequence to be a synonym for
      a~\cs{muskip} register.

\item [\cs{newmuskip}] 
      Allocate a new muskip register.

\item [\cs{thinmuskip}] 
      Small amount of mu glue.

\item [\cs{medmuskip}] 
      Medium amount of mu glue.

\item [\cs{thickmuskip}] 
      Large amount of mu glue. 

\item [\cs{mathsurround}] 
      Kern amount placed before and after in-line formulas.

\item [\cs{over}]
      Fraction.

\item [\cs{atop}]
      Place objects over one another.

\item [\cs{above}]
      Fraction with specified bar width. 

\item [\cs{overwithdelims}]
      Fraction with delimiters.

\item [\cs{atopwithdelims}]
      Place objects over one another with delimiters.

\item [\cs{abovewithdelims}]
      Generalized fraction with delimiters.

\item [\cs{underline}] 
      Underline the following \gr{math symbol} or group.

\item [\cs{overline}] 
      Overline the following \gr{math symbol} or group.


\item [\cs{relpenalty}] 
      Penalty for breaking after a binary relation
      not enclosed in a subformula.
      Plain \TeX\ default:~\n{500}

\item [\cs{binoppenalty}] 
      Penalty for breaking after a binary operator not enclosed in
      a subformula.
      Plain \TeX\ default:~\n{700}

\item [\cs{allowbreak}] 
      Macro for creating a breakpoint.

\end{inventory}

%\point[math:modes] Math modes
\section{Math modes}
\label{math:modes}

\TeX\ changes to \indexterm{math mode} when it encounters a
\indexterm{math shift character}, category~3\index{category!3},
in the input. After such
an opening math shift it investigates (without expansion) the next
token to see whether this is another math shift.  In the latter case
\TeX\ starts processing in \indextermbus{display}{math mode} until a
closing double math shift is encountered:
\begin{disp}\verb> .. $$ >{\italic displayed formula}\verb> $$ ..>\end{disp}
Otherwise it starts processing an in-line formula in
\indextermbus{non-display}{math mode}:
\begin{disp}\verb> .. $ >{\italic in-line formula}\verb> $ ..>\end{disp}
The single math shift character is a \gr{horizontal command}.

Exception: displays are not possible in restricted horizontal
mode, so inside an \cs{hbox} the sequence
\verb>$$> is an empty math formula and
not the start of a displayed formula.

Associated with the two math modes are two \gr{token parameter}
registers (see also Chapter~\ref{token}):
at the start of an in-line formula the \csterm{everymath} tokens
are inserted; at the start of a displayed formula the
\cs{everydisplay} tokens are inserted.
Display math is treated further in the next chapter.

Math modes can be tested for: \cs{ifmmode} is true
in display and non-display math mode, and \cs{ifinner}
is true in non-display mode, but not in display mode.

%\point[math:styles] Styles in math mode
\section{Styles in math mode}
\label{math:styles}

Math formulas are set in any of eight 
\indexterm{math styles}:
\begin{description} 
\item [D] display style,
\item [T] text style,
\item [S] script style,
\item [SS] scriptscript style,
\end{description}
and the four \emph{cramped}\indexterm{cramped styles}
variants $D'$, $T'$, $S'$, $SS'$ of
these. The cramped styles differ mainly in the
fact that superscripts are not raised as far as in
the original styles.

%\spoint Superscripts and subscripts
\subsection{Superscripts and subscripts}
\index{superscript|(}\index{subscript|(}

\TeX\ can typeset a symbol or group
as a superscript (or subscript) to the preceding
symbol or group, if that preceding item
does not already have a superscript
(subscript). Superscripts (subscripts) are specified by
the syntax
\begin{disp}\gr{superscript}\gr{math field}\end{disp}
or 
\begin{disp}\gr{subscript}\gr{math field}\end{disp}
where a \gr{superscript} (\gr{subscript}) is either a character
of category~7\index{category!7} (8\index{category!8}),
or a control sequence \cs{let} to such
a character.
The plain format has the control
\csidx{\char94}%
\csidx{\char95}%
sequences
\begin{verbatim}
\let\sp=^ \let\sb=_
\end{verbatim}
as implicit superscript
and subscript characters.

Specifying a superscript (subscript) expression as the first
item in an empty math list is equivalent to specifying
it as the superscript (subscript) of an empty expression.
For instance, 
\begin{disp}
\verb>$^{...}>\quad is equivalent to\quad \verb>${}^{...}>\end{disp}

For \TeX's internal calculations, superscript and subscript
expressions are made wider by \csterm{scriptspace};
the value of this in plain \TeX\ is~\n{0.5pt}.

\index{superscript|)}\index{subscript|)}

\subsection{Choice of styles}

Ordering the four styles $D$, $T$, $S$, and~$SS$, and
considering the other four as mere variants, the
style rules for math mode are as follows:
\begin{itemize}\item In any style superscripts and subscripts
are taken from the next smaller style. Exception:
in display style they are taken in script style.
\item Subscripts are always in the cramped variant of
the style; superscripts are only cramped if the original
style was cramped.
\item In an \verb-{..\over..}- formula in any style
the numerator and denominator are taken from the next
smaller style.
\item The denominator is always in cramped style;
the numerator is only in cramped style if the original
style was cramped.
\item Formulas under a \cs{sqrt} or \cs{overline}
are in cramped style.\end{itemize}

Styles can be forced by the explicit commands
\alt
\csterm{displaystyle}%
\csterm{textstyle}%
\csterm{scriptstyle}%
\csterm{scriptscriptstyle}%
\cs{displaystyle}, \cs{textstyle}, \cs{scriptstyle},
and \cs{scriptscriptstyle}.


In display style and text style the \cs{textfont} of the
current family is used, 
in scriptstyle the \cs{scriptfont} is used, and in
\alt
scriptscriptstyle the \cs{scriptscriptfont} is used.

The primitive command
\csterm{mathchoice}%
\begin{disp}\cs{mathchoice}\lb {\it D\/\rb\lb T\/\rb\lb S\/\rb\lb SS\/}\rb
\end{disp}
lets the user specify four variants of a formula for the
four styles. 
\TeX\ constructs all four and inserts the appropriate one.

%\point[math:class] Classes of mathematical objects
\section{Classes of mathematical objects}
\label{math:class}

Objects in math mode belong to one of eight \indexterm{math
  classes}. Depending on the class the object may be surrounded by
some amount of white space, or treated specially in some way. Commands
exist to force symbols, or sequences of symbols, to act as belonging
to a certain class.  In the hexadecimal representation \verb>"xyzz>
the class is the \gr{3-bit number}~\n x.

This is the list of classes and commands that force those
classes. The examples are from the plain format 
(see the tables starting at page~\pageref{math:sym:tables}).
\begin{enumerate} \message{set a counter here!}%\SetCounter:item=-1
\item {\em ordinary\/}: lowercase Greek characters and those symbols
      that are `just symbols'; 
      the command \csterm{mathord} forces this class.
\item {\em large operator\/}: integral and sum signs,
      and `big' objects such as \cs{bigcap} or\handbreak \cs{bigotimes};
      the command \csterm{mathop} forces this class.
      Characters that are
      large operators are centred vertically, and they
      may behave differently in display style from in the
      other styles; see below.
\item {\em binary operation\/}: plus and minus,
      and things such as \cs{cap} or \cs{otimes};
      the command \csterm{mathbin} forces this class.
\item {\em relation\/} (also called {\em binary relation\/}): 
      equals, less than, and greater than signs, subset and
      superset, perpendicular, parallel;
      the command \csterm{mathrel} forces this class.
\item {\em opening symbol\/}: opening brace, bracket, parenthesis, angle,
 \altt
      floor, ceiling;
      the command \csterm{mathopen} forces this class.
\item {\em closing symbol\/}: closing brace, bracket, parenthesis, angle,
 \altt
      floor, ceiling;
      the command \csterm{mathclose} forces this class.
\item {\em punctuation\/}: most punctuation marks, but
      \n:~is a relation, the \cs{colon} is a punctuation colon;
      the command \csterm{mathpunct} forces this class.
\item {\em variable family\/}: symbols in this class change font
      with the \cs{fam} parameter; in plain \TeX\ uppercase
      Greek letters and ordinary letters and digits are
      in this class.
\end{enumerate}

There is one further class: the {\em inner\/} subformulas.
No characters can be assigned to this class, but characters and
subformulas can be forced into it by \csterm{mathinner}.
The \gr{generalized fraction}s and \verb-\left...\right- groups
are inner formulas. Inner formulas are surrounded
by some white space; see the table below.

Other subformulas than those that are inner are treated as
ordinary symbols. In particular, subformulas enclosed in
braces are ordinary: \verb-$a+b$- looks like `$a\mathop+b$', but
\message{Check a+b look}%
\verb-$a{+}b$- looks like~`$a{+}b$'. Note, however, that
in \verb-${a+b}$- the whole subformula is treated as an
ordinary symbol, not its components; 
therefore the result is~`${a+b}$'.

%\point Large operators and their limits
\section{Large operators and their limits}

The large operators in the Computer Modern fonts come in
two sizes: one for text style and one for display style.
Control sequences such as \cs{sum} are simply defined by
\cs{mathchardef} to correspond to a position in a font:
\begin{verbatim}
\mathchardef\sum="1350
\end{verbatim}
but if the
current style is display style, \TeX\ looks to see whether
that character has a successor in the font.

Large operators in text style behave as if they are followed
\csterm{limits}%
\csterm{nolimits}%
by \cs{nolimits}, which places the limits as sub/superscript
expressions after the
operator:
\begin{disp}$\sum_{k=1}^\infty$\end{disp}
In display style they behave as if they are followed by
\cs{limits}, which places the limits over and under
the operator:
\begin{disp}$\displaystyle\sum_{k=1}^\infty$\end{disp}
The successor mechanism (see page~\pageref{successor})
\alt
lets \TeX\ take a larger variant
of the delimiter here.

The integral sign has been defined in plain \TeX\ as
\begin{verbatim}
\mathchardef\intop="1352 \def\int{\intop\nolimits}
\end{verbatim}
which places the limits after the operator, even in display style:
\begin{disp}$\displaystyle\int_0^\infty e^{-x^2}\,dx=\sqrt{\pi}/2$
\end{disp}

With \verb-\limits\nolimits- or \verb-\nolimits\limits- the
last specification has precedence; the default placement
can be restored by \csterm{displaylimits}. For instance,
\begin{verbatim}
$ ... \sum\limits\displaylimits ... $
\end{verbatim}
is equivalent to
\begin{verbatim}
$ ... \sum ... $
\end{verbatim}
and 
\begin{verbatim}
$$ ... \sum\nolimits\displaylimits ... $$
\end{verbatim}
is equivalent to
\begin{verbatim}
$$ ... \sum ... $$
\end{verbatim}

%\point Vertical centering: \cs{vcenter}
\section{Vertical centring: \protect\cs{vcenter}}

Each formula has an \emph{axis}\index{formula!axis
  of}\index{formula!centring of}, which is for an in-line formula
about half the x-height of the surrounding text; the exact value is
the \cs{fontdimen22} of the font in family~2, the symbol font, in the
current style.

The bar line in fractions is placed on the axis; large
operators, delimiters and \cs{vcenter} boxes are centred on it.

A \csterm{vcenter}\label{vcenter}
box is a vertical box that is arranged
so that it is centred on the math axis.
It is possible to give a \n{spread} or \n{to}
specification with a \cs{vcenter} box.

The \cs{vcenter} box is allowed only in math mode, and
it does not behave like other boxes; for instance, it can
not be stored in a box register. It does not qualify as
a~\gr{box}. See page~\pageref{tvcenter} for a macro that
repairs this.

%\point[muglue] Mathematical spacing: \n{mu} glue
\section{Mathematical spacing: \n{mu} glue}
\label{muglue}
\index{math spacing|(}

Spacing around mathematical objects is measured in \indexterm{math units}:
multiples of a~\n{mu}. A~\n{mu} is $1/18$th part of \cs{fontdimen6} of
the font in family~2 in the current style, the \indexterm{quad} value of the
symbol font.

%\spoint Classification of \n{mu} glue
\subsection{Classification of \n{mu} glue}

The user can specify \n{mu} spacing by \cs{mkern} or~\cs{mskip},
\csterm{mkern}%
\csterm{mskip}%
but most \indexterm{mu glue} is inserted automatically by \TeX,
based on the classes to which objects belong (see above).
First, here are some rules of thumb describing the global 
behaviour.

\begin{itemize} \item A \cs{thickmuskip} (default value in plain
\TeX: \n{5mu plus 5mu})
\csterm{thickmuskip}%
is inserted around (binary) relations, except where these are
preceded or followed by other relations or punctuation, and
except if they follow an open, or precede a close symbol.
\item A \csterm{medmuskip} (default value in plain
\TeX: \n{4mu plus 2mu minus 4mu}) 
is put around binary operators.
\item A \csterm{thinmuskip} 
(default value in plain \TeX: \n{3mu}) follows after
punctuation, and is put around inner objects, except where these
are followed by a close or preceded by an open symbol, and
except if the other object is a large operator or a
binary relation.
\item No \n{mu} glue is inserted after an open or before a close
symbol except where the latter is preceded by punctuation;
no \n{mu} glue is inserted also before punctuation, except where
the preceding object is punctuation or an inner object.
\end{itemize} 

The following table gives the complete definition of mu glue
between math objects.
\begin{disp}\leavevmode
\vbox{\offinterlineskip
    \halign{#\enspace\hfil&#\enspace\hfil\vrule
           &&\hfil\enspace#\hfil\strut\cr
    \omit\hfil&\omit\hfil& 0:& 1:& 2:& 3:& 4:& 5:& 6:\cr
    \omit\hfil&\omit\hfil&\hfill Ord&\hfill Op&\hfill Bin&\hfill Rel&
                  \hfill Open&\hfill Close&\hfill Punct&\hfill Inner\cr
    \omit\hfil&\omit\hfil&\multispan8\hrulefill\cr
    0:&Ord&    0&  1&(2)&(3)&  0&  0&  0&(1)\cr
    1:&Op&     1&  1&  *&(3)&  0&  0&  0&(1)\cr
    2:&Bin&  (2)&(2)&  *&  *&(2)&  *&  *&(2)\cr
    3:&Rel&  (3)&(3)&  *&  0&(2)&  *&  *&(2)\cr
    4:&Open&   0&  0&  *&  0&  0&  0&  0&  0\cr
    5:&Close&  0&  1&(2)&(3)&  0&  0&  0&(1)\cr
    6:&Punct&(1)&(1)&  *&(1)&(1)&(1)&(1)&(1)\cr
      &Inner&(1)&  1&(2)&(3)&(1)&  0&(1)&(1)\cr
%    \omit\hfil&\omit\hfil&\multispan8\hrulefil\cr
}}
\end{disp}

where the symbols have the following meanings:
\begin{itemize}\item 0, no space; 1, thin space; 2, medium space;
     3, thick space;
\item $(\cdot)$, insert only in text and display
     mode, not in script or scriptscript mode;
\item    cases * cannot occur, because a Bin object is converted
    to Ord if it is the first in the list, preceded by
    Bin, Op, Open, Punct, Rel, or followed by Close,
    Punct, and Rel; also, a Rel is converted to Ord when
    \alt
    it is followed by Close or Punct.
\end{itemize}

Stretchable \n{mu} glue is set according to the same rules that
govern ordinary glue. However, only \n{mu} glue on the outer
level can be stretched or shrunk; any \n{mu} glue enclosed
in  a~group is set at natural width.

%\spoint Muskip registers
\subsection{Muskip registers}

Like ordinary glue, \n{mu} glue can be stored in registers,
\csterm{muskip}%
\csterm{muskipdef}%
\csterm{newmuskip}%
the \cs{muskip} registers,
of which there are 256 in \TeX. 
The registers are denoted by
\begin{disp}\cs{muskip}\gr{8-bit number}\end{disp}
and they can be assigned to a control sequence by
\begin{disp}\cs{muskipdef}\gr{control sequence}\gr{equals}\gr{8-bit number}
\end{disp}
and there is a macro that allocates unused registers:
\begin{disp}\cs{newmuskip}\gr{control sequence}\end{disp}
Arithmetic for mu glue exists as for glue; see
Chapter~\ref{glue}.

%\spoint Other spaces in math mode
\subsection{Other spaces in math mode}

In math mode space tokens are ignored; however,
the math code of the space character is \verb-"8000-
in plain \TeX,
so if its category is made `letter' or `other character', it
will behave like an active character in math mode.
See also page~\pageref{mcode:8000}.

Admissible glue in math mode is of type~\gr{mathematical skip},
which is either a \gr{horizontal skip} (see Chapter~\ref{hvmode}) 
or~\cs{mskip}\gr{muglue}. Leaders in math mode can be specified
with a \gr{horizontal skip}.

A glue item preceded by \csterm{nonscript}
is cancelled if it occurs in scriptstyle or scriptscriptstyle.

Control space functions in math mode
\alt
as it does in horizontal mode.

In-line formulas are surrounded by kerns of size
\csterm{mathsurround}, the so-called `math-on' and
`math-off' items. Line breaking can occur at the front of
the math-off kern if it is followed by glue.

\index{math spacing|)}

\section{Generalized fractions}

Fraction-like objects can be set with six primitive commands
of type \gr{generalized fraction}.
Each of these \indexterm{generalized fractions}
takes the preceding and the following subformulas
and puts them over one another, if necessary with a fraction
bar and with delimiters.
\begin{description} \item [\cs{over}]
  is the ordinary fraction;\csidx{over}
  the bar thickness is \cs{fontdimen8}
   of the extension font: 
   
\begin{disp}\verb>$\pi\over2$>\quad gives\quad `$\pi\over2$'\message{pi over 2}\end{disp}
\item [\cs{atop}]
   is equivalent to a fraction with zero bar thickness:\csidx{atop}
   
\begin{disp}\verb>$\pi\atop2$>\quad gives\quad `$\pi\atop2$'\end{disp}
\item [\cs{above}\gr{dimen}]
   specifies the thickness
   of the bar line explicitly:\csidx{above}
   
\begin{disp}\verb>$\pi\above 1pt 2$>\quad gives\quad `$\pi\above 1pt 2$'\end{disp}
\end{description} 

To each of these three there corresponds a \cs{...withdelims} variant
\csterm{overwithdelims}%
\csterm{atopwithdelims}%
\csterm{abovewithdelims}%
that lets the user specify delimiters for the expression.
For example, the most general command, in terms of which
all five others could have been defined, is
\begin{disp}\cs{abovewithdelims}\gr{delim$_1$}\gr{delim$_2$}\gr{dimen}.
\end{disp}
Delimiters in these generalized fractions do not grow with the
enclosed expression: in display mode a delimiter is taken
which is at least \cs{fontdimen20} high, otherwise
\alt
it has to be
at least \cs{fontdimen21} high.
These dimensions are taken
from the font in family~2, the symbol font, in the current style.

The control sequences \cs{over}, \cs{atop}, and \cs{above}
are primitives, although they could have been defined
as \cs{...withdelims..}, that is, with two null delimiters.
Because of these implied surrounding null delimiters,
there is a kern of size \cs{nulldelimiterspace} before and after
these simple generalized fractions. 

%\point Underlining, overlining
\section{Underlining, overlining}

The primitive commands \csterm{underline} and \csterm{overline} take a 
\gr{math field} argument, that is, a \gr{math symbol} or
a group, and draw a line under or over it.
The result is an `Under' or `Over' atom, which
is appended to the current math list.
The line thickness is font dimension~8 of the extension font,
which also determines the clearance between the line and
the \gr{math field}.

Various other \cs{over...} and \cs{under...} commands exist
in plain \TeX;
these are all macros
that use the \TeX\ \cs{halign} command.

%\point Line breaking in math formulas
\section{Line breaking in math formulas}

In-line formulas can be broken after relations and binary operators.
\csterm{relpenalty}%
\csterm{binoppenaly}%
The respective \emph{penalties}\index{penalties in math mode}
are the \cs{relpenalty} 
and the~\cs{binoppenalty}. However, \TeX\ will only break
after such symbols if they are not enclosed in braces.
Other \emph{breakpoints}\index{breakpoints in math lists}
can be created with~\cs{allowbreak},
\csterm{allowbreak}%
which is an abbreviation for~\cs{penalty0}.

Unlike in horizontal or vertical mode where putting two penalties
in a row is equivalent to just placing the smallest one,
in math mode a penalty placed at a break point \ldash that is,
after a relation or binary operator \rdash  will effectively
replace the old penalty by the new one.

%\point[fam23:fontdims] Font dimensions of families 2 and 3
\section{Font dimensions of families 2 and 3}
\label{fam23:fontdims}

If a font is used in text mode, \TeX\ will look at its
first 7 \cs{fontdimen} parameters
(see page~\pageref{font:dims}), for instance to
control spacing.
In math, however, more font dimensions are needed.
\TeX\ will look at the first 22 parameters of the
fonts in family~2, and the first 13 of the fonts in
family~3, to control various
aspects of math typesetting. The next two subsections
have been quoted loosely from~\cite{BB:ISO}.

%\spoint Symbol font attributes
\subsection{Symbol font attributes}
\index{symbol font|(}

Attributes of the font in family~2, the \emph{symbol font},
mainly specify the
initial vertical positioning
of parts of fractions, subscripts, superscripts, et cetera.
The position determined by applying these
attributes may be further modified because of other
conditions, for example the presence of a fraction bar.

One text font dimension, number~6,
the quad, determines the size of mu glue;
see above.

Fraction numerator attributes: minimum shift up, from
the main baseline, of the baseline of the numerator
of a generalized fraction,
\begin{enumerate} \setcounter{enumi}{7} %\message{set a counter here!}%\SetCounter:item=7
\item num1:
 for display style,
\item num2:
 for text style or smaller if a fraction bar is present,
\item num3:
 for text style or smaller if no fraction bar is present.
\end{enumerate}

Fraction denominator attributes: minimum shift down, from
the main baseline, of the baseline of the denominator
of a generalized fraction,
\begin{enumerate} \setcounter{enumi}{10} %\message{set a counter here!}%\SetCounter:item=10
\item denom1:
for display style,
\item denom2:
for text style or smaller.
\end{enumerate}

Superscript attributes: minimum shift up, from the main baseline,
of the baseline of a superscript,
\begin{enumerate} \setcounter{enumi}{12} %\message{set a counter here!}%\SetCounter:item=12
\item sup1:
for display style,
\item sup2:
for text style or smaller, non-cramped,
\item sup3:
for text style or smaller, cramped.
\end{enumerate}

Subscript attributes: minimum shift down, from the main baseline,
of the baseline of a subscript,
\begin{enumerate} \setcounter{enumi}{15} %\message{set a counter here!}%\SetCounter:item=15
\item sub1:
when no superscript is present,
\item sub2:
when a superscript is present.
\end{enumerate}

Script adjustment attributes: for use only with non-glyph,
that is, composite, objects.
\begin{enumerate} \setcounter{enumi}{17} %\message{set a counter here!}%\SetCounter:item=17
\item sup\_drop:
maximum distance of superscript baseline below top of nucleus
\item sub\_drop:
minimum distance of subscript baseline below bottom of nucleus.
\end{enumerate}

Delimiter span attributes: height plus depth of delimiter enclosing
a generalized fraction,
\begin{enumerate} \setcounter{enumi}{19} %\message{set a counter here!}%\SetCounter:item=19
\item delim1:
in display style,
\item delim2:
in text style or smaller.
\end{enumerate}

A parameter with many uses, the height of the math axis,
\begin{enumerate} \setcounter{enumi}{21} %\message{set a counter here!}%\SetCounter:item=21
\item axis\_height:
the height above the baseline
of the fraction bar, and the centre of large delimiters
and most operators and relations. This position is
used in vertical centring operations.
\end{enumerate}

\index{symbol font|)}

\subsection{Extension font attributes}

\index{extension font|(}

Attributes of the font in family~3, the \emph{extension font},
mostly specify
the way the limits of large operators are set.

The first parameter, number 8, default\_rule\_thickness,
serves many purposes. It
is the thickness of the rule used for overlines,
underlines, radical extenders (square root), 
and fraction bars. Various clearances are  also specified
in terms of this dimension: between the fraction bar and
the numerator and denominator, between an object and
the rule drawn by an underline, overline, or radical,
and between the bottom of superscripts and top of subscripts.

Minimum clearances around large operators are as follows:
\begin{enumerate} \setcounter{enumi}{8} %\message{set a counter here!}%\SetCounter:item=8
\item big\_op\_spacing1:
minimum clearance between baseline of upper limit and top
of large operator; see below.
\item big\_op\_spacing2:
minimum clearance between bottom of large operator and top of 
lower limit.
\item big\_op\_spacing3:
minimum clearance between baseline of
upper limit and top of large operator,
taking into account depth of upper limit; see below.
\item big\_op\_spacing4:
minimum clearance between bottom of large operator and top of lower
limit, taking into account height of lower limit; see below.
\item big\_op\_spacing5:
clearance above upper limit or below lower limit of a large operator.
\end{enumerate}
The resulting clearance above an operator is the maximum
of parameter~7, and parameter~11 minus the depth of the
upper limit.
The resulting clearance below an operator is the maximum
of parameter~10, and parameter~12 minus the height of the
lower limit.

\index{extension font|)}

\subsection{Example: subscript lowering}

The location of a subscript depends on whether there
\alt
\howto Adjust subscript lowering\par
is a superscript; for instance
\begin{disp} $X_1+Y^2_1=1$\end{disp}
If you would rather have that look like
\begin{disp} $\global\tempdima=\fontdimen16\textfont2\relax
       \global\tempdimb=\fontdimen17\textfont2\relax
       \fontdimen16\textfont2=3pt \fontdimen17\textfont2=3pt
       X_1+Y^2_1=1$,$\fontdimen16\textfont2=\tempdima\relax
                     \fontdimen17\textfont2=\tempdimb\relax$
\end{disp}
\message{check lowering}
it suffices to specify
\begin{verbatim}
\fontdimen16\textfont2=3pt \fontdimen17\textfont2=3pt
\end{verbatim}
which makes the subscript drop equal in both cases.
Since font dimension assignments are global, you 
have to specify this only once in your document.

%\subject[displaymath] Display Math
\endofchapter
\chapter{Display Math}\label{displaymath}

Displayed formulas are set on a line of their own, usually
somewhere in a paragraph. This chapter explains
how surrounding white space (both above/below and to the
left/right) is calculated.

\label{cschap:abovedisplayskip}\label{cschap:belowdisplayskip}\label{cschap:abovedisplayshortskip}\label{cschap:belowdisplayshortskip}\label{cschap:predisplaypenalty}\label{cschap:postdisplaypenalty}\label{cschap:displayindent}\label{cschap:displaywidth}\label{cschap:predisplaysize}\label{cschap:everydisplay}\label{cschap:eqno}\label{cschap:leqno}
\begin{inventory}
\item [\cs{abovedisplayskip \cs{belowdisplayskip}}]
\mdqon
      Glue above/""below a display.
\mdqoff
      Plain \TeX\ default:~\n{12pt plus 3pt minus 9pt}

\item [\cs{abovedisplayshortskip \cs{belowdisplayshortskip}}]
\mdqon
      Glue above/""below a display if the line preceding the display 
\mdqoff
      was short.
      Plain \TeX\ defaults:~\n{0pt plus 3pt} and
      \n{7pt plus 3pt minus 4pt} respectively.

\item [\cs{predisplaypenalty \cs{postdisplaypenalty}}]
\mdqon
      Penalty placed in the vertical list above/""below a display.
\mdqoff
      Plain \TeX\ defaults:~\n{10$\,$000} and~\n{0}
      respectively.

\item [\cs{displayindent}] 
      Distance by which the box, in which the display 
      is centred, is indented owing to hanging indentation.

\item [\cs{displaywidth}] 
      Width of the box in which the display is centred.

\item [\cs{predisplaysize}] 
      Effective width of the line preceding the display.

\item [\cs{everydisplay}] 
      Token list inserted at the start of a display.

\item [\cs{eqno}] 
      Place a right equation number in a display formula.

\item [\cs{leqno}] 
      Place a left equation number in a display formula.

\end{inventory}

%\point Displays
\section{Displays}

\TeX\ starts building a \indexterm{display math} formula
when it encounters two
math shift characters (characters of category~3\index{category!3},
\verb>$>~in plain \TeX) in a row.
Another such pair (possibly followed
\alt by one optional space) indicates the end of the display.

Math shift is a \gr{horizontal command}, but displays are only
allowed in unrestricted horizontal mode
(\verb>$$>~is an empty math formula in restricted horizontal mode).
Displays themselves, however, are started in the
surrounding (possibly internal) vertical mode in order to calculate
quantities such as~\cs{prevgraf}; the result of the display is
appended to the vertical list.

The part of the paragraph above the display is broken into
lines as an independent paragraph (but \cs{prevgraf} is
carried over; see below), and the remainder of the
paragraph is set, starting with an empty list and \cs{spacefactor}
equal to~1000. 
The \cs{everypar} tokens are not inserted for the part of the
paragraph after the display, nor is \cs{parskip} glue inserted.
 
Right at the beginning of the display the \csterm{everydisplay}
token list is inserted (but after the calculation of
\cs{displayindent}, \cs{displaywidth}, and \cs{predisplaysize}).
See page~\pageref{left:display} for an example of the use
of \cs{everydisplay}.

The page builder is exercised
before the display 
(but after the \cs{everydisplay} tokens have been inserted),
and after the display finishes.

The `display style' of math typesetting was treated in 
Chapter~\ref{mathfont}.

%\point Displays in paragraphs
\section{Displays in paragraphs}

Positioning of a display in a paragraph may be influenced
by hanging indentation or a \cs{parshape} specification.
For this, \TeX\ uses the \cs{prevgraf} parameter
(see Chapter~\ref{par:shape}), and
acts as if the display is three lines deep.

If $n$ is the value of \cs{prevgraf} when the display starts
\ldash so there are $n$ lines of text above the display \rdash 
\cs{prevgraf} is set to to $n+3$ when the paragraph resumes.
The display occupies, as it were, lines $n+1$, $n+2$, and~$n+3$.
The shift and line width for the display are those
that would hold for line~$n+2$.

The shift for the display is recorded in \cs{displayindent};
\csterm{displayindent}%
\csterm{displaywidth}%
the line width is recorded in\handbreak \cs{displaywidth}. These parameters
(and the \cs{predisplaysize} explained below)
are set immediately after the \verb>$$> has been scanned.
Usually they are equal to zero and \cs{hsize} respectively.
The user can change the values of these parameters; 
\TeX\ will use the
values that hold after the math list 
of the display has been processed.

A~display is vertical material, and therefore
not influenced by settings of \cs{leftskip} and \cs{rightskip}.

%\point Vertical material around displays
\section{Vertical material around displays}

A display is preceded in the vertical list by
\begin{itemize}\item a penalty of size \cs{predisplaypenalty}
   \csterm{predisplaypenalty}%
\csterm{abovedisplayskip}%
   \csterm{abovedisplayshortskip}%
(plain \TeX\ default~$10\,000$), and
\item glue of size \cs{abovedisplayskip} 
or \cs{abovedisplayshortskip}; this glue is omitted in
cases where a~\cs{leqno} equation number is set on
a line of its own (see below).\end{itemize}
A display is followed by 
\begin{itemize}\item a penalty of size \cs{postdisplaypenalty}
   \csterm{postdisplaypenalty}%
\csterm{belowdisplayskip}%
   \csterm{belowdisplayshortskip}%
(default~0), and possibly
\item glue of size \cs{belowdisplayskip} or 
\cs{belowdisplayshortskip}; this glue is omitted in cases
where an~\cs{eqno} equation number is set on a line of
its own (see below).\end{itemize}

The `short' variants of the glue are taken if
there is no \cs{leqno} left equation number, and if
the last line of the paragraph above the display is
short enough for the display to be raised a bit without
coming too close to that line.
In order to decide this, the effective width of the
preceding line is saved in \csterm{predisplaysize}.
This value is calculated immediately after the opening \verb>$$>
of the display has
been scanned, together with the \cs{displaywidth}
and \cs{displayindent} explained above.

Remembering that the part of the paragraph above the display
has already been broken into lines, the following method
for finding the effective width of the last line ensues.
\TeX\ takes the last box of the list, which is a horizontal
box containing the last line, and locates the right edge
of the last box in it. The \cs{predisplaysize} is then
the place of that rightmost edge, plus any amount by which
the last line was shifted, plus two ems in the current font.

There are two exceptions to this. The \cs{predisplaysize}
is taken to be $-$\cs{maxdimen} if there was no previous line,
that is,
the display started the paragraph, or it followed another display;
\cs{predisplaysize} is taken to be \cs{maxdimen}
if the glue in the last line was not set at its natural width,
which may happen if the \cs{parfillskip} contained only finite
stretch. The reason for the last clause is that glue
\mdqon
setting is slightly \emph{machine"-dependent}\index{machine dependence},
and such dependences
\mdqoff
should be kept out of \TeX's global decision processes.

%\point Glue setting of the display math list
\section{Glue setting of the display math list}

The display has to fit in \cs{displaywidth}, 
but in addition to the formula there
may be an equation number. The minimum separation
between the formula and the equation number should
be one em in the symbol font, that is,
\mdqon
\cs{font\-dimen\-6}""\cs{textfont2}.
\mdqoff

If the formula plus any equation number
and separation fit into \cs{displaywidth},
the glue in the formula is set at its natural width. 
If it does not fit,
but the formula contains enough shrink, it is shrunk.
Otherwise \TeX\ puts any equation number
on a line of its own, and the glue in the formula is
set to fit it in \cs{displaywidth}.
With the equation
number on a separate line the formula may now very well fit in the
display width; however,
if it was a very long formula the box in which it is
set may still be overfull. \TeX\ nevers breaks a displayed
formula.

%\point Centring the display formula: displacement
\section{Centring the display formula: displacement}

Based on the width of the box containing the formula \ldash which
may not really `contain' it; it may be overfull \rdash 
\TeX\ tries to centre the formula in the \cs{displaywidth},
that is, without taking the equation number into account.
Initially, a displacement is calculated that is 
half the difference between \cs{displaywidth} and the
width of the formula box.

However, if there is an equation number that will not
be put on a separate line and the displacement is less than
twice the width of the equation number, a new displacement
is calculated. This new displacement is zero if the formula
started with glue; otherwise it is such that the
formula box is centred in the space left by the equation
number.

If there was no equation number, or if the equation number
will be put on a separate line, the formula box
is now placed, shifted right by \cs{displayindent} plus
the displacement calculated above.

\section{Equation numbers}

\index{equation numbering|(}

The user can specify a equation number for a display
by ending it with 
\csterm{eqno}%
\csterm{leqno}%
\begin{Disp}\cs{eqno}\gr{math mode material}\verb>$$>\end{Disp}
for an equation number placed on the right, or
\begin{Disp}\cs{leqno}\gr{math mode material}\verb>$$>\end{Disp}
for an equation number placed on the left.

%\spoint Ordinary equation numbers
\subsection{Ordinary equation numbers}

Above it was described how \TeX\ calculates a displacement
from the display formula and the equation number, if this
is to be put on the same line as the formula.

If the equation number was a  \cs{leqno} number,
\TeX\ places a box containing
\begin{itemize}\item the equation number,
\item a kern with the size of the displacement calculated, and
\item the formula.\end{itemize}
This box is shifted right by \cs{displayindent}.

If the equation number was an \cs{eqno} number,
\TeX\ places a box containing
\begin{itemize}\item the formula,
\item a kern with the size of the displacement calculated, and
\item the equation number.\end{itemize}
This box is shifted right by \cs{displayindent} plus
the displacement calculated.

%\spoint The equation number on a separate line
\subsection{The equation number on a separate line}

Since displayed formulas may become rather big, \TeX\ can decide
(as was described above)
that any equation number should be placed on a line of its own.
A~left-placed equation number is then to be placed above the
display, in a box that is shifted right by \cs{displayindent};
a right-placed equation number will be placed below the display,
in a box that is shifted to the right 
by \cs{displayindent} plus \cs{displaywidth} minus the width of
the equation number box.

In both cases a penalty of $10\,000$ is placed between the equation
number box and the formula.

\TeX\ does not put extra glue above a left-placed
equation number or below
a right-placed equation number; \TeX\ here relies on
the baselineskip mechanism.

\index{equation numbering|)}

\section{Non-centred displays}
\label{left:display}

As a default, \TeX\ will center displays.
In order to get \indextermbus{non-centred}{displays} some
macro trickery is needed. 

One approach would
be to write a macro \cs{DisplayEquation}
that would basically look like
\begin{verbatim}
\def\DisplayEquation#1{%
    \par \vskip\abovedisplayskip
    \hbox{\kern\parindent$\displaystyle#1$}
    \vskip\belowdisplayskip \noindent}
\end{verbatim}
but it would be nicer if one could just write
\begin{verbatim}
$$ ... \eqno ... $$
\end{verbatim}
\mdqon
and having this come out as a left"-aligning display.
\mdqoff

Using the \cs{everydisplay} token list, the above
idea can be realized. The basic idea is to write
\begin{verbatim}
\everydisplay{\IndentedDisplay}
\def\IndentedDisplay#1$${ ...
\end{verbatim}
so that the macro \cs{IndentedDisplay}
will receive the formula, including any equation number.
The first step is now to extract an equation number
if it is present. This makes creative use of delimited
macro parameters.
\begin{verbatim}
\def\ExtractEqNo#1\eqno#2\eqno#3\relax
   {\def\Equation{#1}\def\EqNo{#2}}
\def\IndentedDisplay#1$${%
    \ExtractEqNo#1\eqno\eqno\relax
\end{verbatim}
Next the equation should be set in the available
space \cs{displaywidth}:
\begin{verbatim}
    \hbox to \displaywidth
        {\kern\parindent
         $\displaystyle\Equation$\hfil$\EqNo$}$$
    }
\end{verbatim}
Note that the macro ends in the closing \verb>$$>
to balance the opening dollars that caused
insertion of the \cs{everydisplay} tokens.
This also means that the box containing the
displayed material will automatically be
surrounded by \cs{abovedisplayskip} and
\cs{belowdisplayskip} glue.
There is no need to use \cs{displayindent} anywhere
in this macro, because \TeX\ itself will shift the
display appropriately.

% 
\endofchapter
\chapter{Alignment}\label{align}
\index{alignments|(}

\TeX\ provides a general alignment mechanism for making \indexterm{tables}.

\label{cschap:halign}\label{cschap:valign}\label{cschap:omit}\label{cschap:span}\label{cschap:multispan}\label{cschap:tabskip}\label{cschap:noalign}\label{cschap:cr}\label{cschap:crcr}\label{cschap:everycr}\label{cschap:centering}\label{cschap:hideskip}\label{cschap:hidewidth}
\begin{inventory}
\item [\cs{halign}] 
      Horizontal alignment.

\item [\cs{valign}] 
      Vertical alignment.   

\item [\cs{omit}] 
      Omit the template for one alignment entry.

\item [\cs{span}] 
      Join two adjacent alignment entries.

\item [\cs{multispan}] 
      Macro to join a number of adjacent alignment entries.

\item [\cs{tabskip}] 
      Amount of glue in between columns (rows)
      of an \cs{halign} (\cs{valign}). 

\item [\cs{noalign}] 
      Specify  vertical (horizontal)
      material   to be placed in between rows (columns) of
      an \cs{halign} (\cs{valign}).

\item [\cs{cr}]
      Terminate an alignment line.

\item [\cs{crcr}] 
      Terminate an alignment line if it has 
      not already been terminated by~\cs{cr}.

\item [\cs{everycr}] 
      Token list inserted after every \cs{cr} or non-redundant 
      \cs{crcr}.

\item [\cs{centering}] 
      Glue register in plain \TeX\ for centring
      \cs{eqalign} and \cs{eqalignno}.
      Value: \n{0pt plus 1000pt minus 1000pt}

\item [\cs{hideskip}]
      Glue register in plain \TeX\ to make alignment entries invisible.
      Value: \n{-1000pt plus 1fill}

\item [\cs{hidewidth}]
      Macro to make preceding or following entry invisible.

\end{inventory}

%\point Introduction
\section{Introduction}

\TeX\ has a sophisticated alignment mechanism, based on 
templates, with one template entry per column or row.
The templates may contain any common elements
of the table entries, and in general they contain
instructions for typesetting the entries.
\TeX\ first calculates widths (for \cs{halign}) or heights
(for \cs{valign}) of all entries;
then it typesets the whole alignment using in each column (row)
the maximum width (height) of entries in that column (row).

%\point Horizontal and vertical alignment
\section{Horizontal and vertical alignment}

The two alignment commands in \TeX\ are
\csterm{halign}%
\csterm{valign}%
\begin{disp}\cs{halign}\gr{box specification}\lb\gr{alignment material}\rb
\end{disp} for horizontal alignment of columns, and
\begin{disp}\cs{valign}\gr{box specification}\lb\gr{alignment material}\rb
\end{disp} for vertical alignment of rows.
\cs{halign} is a \gr{vertical command}, and
\cs{valign} is a \gr{horizontal command}.

The braces induce a new level of grouping; they can be
implicit.

The discussion below will mostly focus on horizontal
alignments, but, replacing `column' by `row' and vice versa,
it applies to vertical alignments too.

%\spoint Horizontal alignments: \cs{halign}
\subsection{Horizontal alignments: \cs{halign}}

A \indexterm{horizontal alignment}
yields a list of horizontal boxes, the rows,
which are placed on the surrounding vertical list.
The page builder is exercised after the alignment rows have been
added to  the vertical list.
The value of \cs{prevdepth} that holds before the alignment
is used for the baselineskip of the first row,
and after the alignment \cs{prevdepth} is set to a value based
on the last row.

Each entry is processed in a group of its own,
in restricted horizontal mode.

A special type of horizontal alignment exists: the
\indexterm{display alignment},
specified as
\begin{disp}\n{\$\$}\gr{assignments}\cs{halign}\gr{box specification}\lb\n{...}\rb
     \gr{assignments}\n{\$\$}\end{disp}
Such an alignment is shifted by \cs{displayindent} (see
Chapter~\ref{displaymath}) and surrounded by\handbreak
\cs{abovedisplayskip} and \cs{belowdisplayskip} glue.

%\spoint Vertical alignments: \cs{valign}
\subsection{Vertical alignments: \cs{valign}}

A \indexterm{vertical alignment}
can be considered as a  `rotated' horizontal alignments:
they are placed on the surrounding horizontal lists,
and yield a row of columns. The \cs{spacefactor} value
is treated the same way as the \cs{prevdepth} for horizontal
alignments: the value current before the alignment is used
for the first column, and the value reached after the last column
is used after the alignment. In between columns the \cs{spacefactor}
value is~1000.

Each entry is in a group of its own, and it is processed
in internal vertical mode.

%\spoint Material between the lines: \cs{noalign}
\subsection{Material between the lines: \cs{noalign}}

Material that has to be contained in the alignment, but 
should not be treated as an entry or series of entries,
\csterm{noalign}%
can be given by 
\begin{disp}\cs{noalign}\gr{filler}\lb\gr{vertical mode material}\rb
\end{disp} for horizontal alignments, and
\begin{disp}\cs{noalign}\gr{filler}\lb\gr{horizontal mode material}\rb
\end{disp} for vertical alignments.

Examples are
\begin{verbatim}
\noalign{\hrule}
\end{verbatim}
for drawing a horizontal rule
between two lines of an \cs{halign},
and
\begin{verbatim}
\noalign{\penalty100}
\end{verbatim}
for discouraging a page break (or line break) in
between two rows (columns) of an \cs{halign} (\cs{valign}).

%\spoint Size of the alignment
\subsection{Size of the alignment}

The \gr{box specification} can be used to give the alignment
a predetermined size: for instance
\begin{verbatim}
\halign to \hsize{ ... }
\end{verbatim}
Glue contained in the entries of the alignment has no role in this;
any stretch or
shrink required is taken from the \cs{tabskip} glue.
This is explained below.

%\point The preamble
\section{The preamble}

Each line in an alignment is terminated by \cs{cr};
the first line is called the {\it template line}.
It is of the form
\begin{disp}\n{$u_1$\#$v_1$\&...\&$u_n$\#$v_n$}\cs{cr}\end{disp}
where each $u_i$, $v_i$ is a (possibly empty) arbitrary sequence
of tokens, and the template entries are separated by
the 
\indexterm{alignment tab}
character (\n\&~in plain \TeX),
that is, any character of category~4\index{category!4}.

A $u_i$\n\#$v_i$ sequence is the template that will be
used for the $i\,$th column: whatever sequence $\alpha_i$
the user specifies
as the entry for that column will be inserted at the
parameter character. The sequence $u_i\alpha_iv_i$ is
then processed to obtain the actual entry for the $i\,$th
column on the current line. See below for more details.

The length $n$ of the template line need
not be equal to the actual number of columns in the alignment:
the template is used only for as many items as are specified
on a line. Consider as an example
\begin{verbatim}
\halign{a#&b#&c#\cr 1&2\cr 1\cr}
\end{verbatim}
which has a three-item template, but the rows have only
one or two items. The output of this is
\begin{disp}\leavevmode\vbox{\halign{a#&b#&c#\cr 1&2\cr 1\cr}}\end{disp}

%\spoint Infinite preambles
\subsection{Infinite preambles}

For the case where the number of columns is not known in advance,
for instance if the alignment is to be used in a macro where
the user will specify the columns, it is possible to
specify that a trailing piece of the
preamble can be repeated arbitrarily many times.
By preceding it with \n\&, an entry can be marked as the
start of this repeatable part of the preamble.
See the example of \cs{matrix} below.

When the whole preamble is to be repeated, there will be
an alignment tab character at the start of the first entry:
\begin{verbatim}
\halign{& ... & ... \cr ... }
\end{verbatim}
If a starting portion of the preamble is to be exempted from
repetition, a double alignment tab will occur:
\begin{verbatim}
\halign{ ... & ... & ... && ... & ... \cr ... }
\end{verbatim}

The  repeatable part need not be used an integral
number of times. The alignment rows can end at any time;
the rest of the preamble is then not used.

%\spoint Brace counting in preambles
\subsection{Brace counting in preambles}

Alignments may appear inside alignments, so \TeX\ uses the
following rule to determine to which alignment
an \n\&  or \cs{cr} control sequence belongs:
\begin{disp} All tab characters and \cs{cr} tokens of an alignment
      should be on the same level of grouping.\end{disp}
From this it follows that tab characters and \cs{cr} tokens
can appear inside an entry if they are nested in braces.
This makes it possible to have nested alignments.

%\spoint Expansion in the preamble
\subsection{Expansion in the preamble}

All tokens in the preamble \ldash apart from the tab characters \rdash
are stored for insertion in the entries of the alignment,
but a token preceded by \csterm{span} is expanded while
the preamble is scanned. See below for the function of
\cs{span} in the rest of the alignment.

%\spoint \cs{tabskip}
\subsection{\cs{tabskip}}

Entries in an alignment are set to take the width of the
largest element in their column.
Glue for separating columns can be specified by assigning
to \csterm{tabskip}.
\altt
\TeX\ inserts this glue in
between each pair of columns, and before the first and after the
last column.

The value of \cs{tabskip} that holds outside the alignment is
used before the first column, and after all subsequent columns,
unless the preamble contains assignments to \cs{tabskip}.
Any assignment to \cs{tabskip} is executed while \TeX\ is scanning
the preamble; the value that holds when a tab character is
reached will be used at that place in each row, and after all subsequent
columns, unless further assignments occur.
The value of \cs{tabskip} that holds when \cs{cr} is reached
is used after the last column.

Assignments to \cs{tabskip} in the preamble are local to the
alignment, but not to the entry where they are given.
These assignments are ordinary glue assignments:
they remove any optional trailing space.

As an example, in the following table there is no tabskip
glue before the first and after the last column; 
in between all columns there is stretchable tabskip.
\begin{verbatim}
\tabskip=0pt \halign to \hsize{
    \vrule#\tabskip=0pt plus 1fil\strut& 
    \hfil#\hfil& \vrule#& \hfil#\hfil& \vrule#& \hfil#\hfil& 
    \tabskip=0pt\vrule#\cr
  \noalign{\hrule}
    &\multispan5\hfil Just a table\hfil&\cr
  \noalign{\hrule}
    &one&&two&&three&\cr &a&&b&&c&\cr
  \noalign{\hrule}
    }
\end{verbatim}
The result of this is
\begin{disp}\PopListLevel
\leavevmode\message{single indent and sufficient vertical}%
\hbox{\leftskip0pt \rightskip0pt
 \vbox{\offinterlineskip
\tabskip=0pt \halign to \hsize{\strut
    \vrule#\tabskip=0pt plus 1fil\strut&
    \hfil#\hfil& \vrule#& \hfil#\hfil& 
                 \vrule#& \hfil#\hfil&
    \tabskip=0pt\vrule#\cr
    \noalign{\hrule}
    &\multispan5\hfil Just a table\hfil&\cr
    \noalign{\hrule}
    &one&&two&&three&\cr
    &a&&b&&c&\cr
    \noalign{\hrule}
    }}}\end{disp}
All of the vertical rules
of the table are in a separate column. This is the only way
to get the space around the items to stretch.

%\point The alignment
\section{The alignment}

After the template line any number of lines terminated by \cs{cr}
can follow. \TeX\ reads all of these lines, processing the
entries in order to find the maximal width (height) in 
each column (row).
Because all entries are kept in memory,
long tables can overflow \TeX's main memory.
For such tables it is better to write a special-purpose macro.

%\spoint Reading an entry
\subsection{Reading an entry}

Entries in an alignment are composed of the 
constant $u$ and $v$ parts
of the template, and the variable $\alpha$ part.
Basically \TeX\ forms the sequence of tokens $u\alpha v$
and processes this. However, there are two special cases
where \TeX\ has to expand before it forms this sequence.

Above, the \cs{noalign} command was described. 
Since this requires a different treatment from other
alignment entries, 
\TeX\ expands, after it has read a \cs{cr},
the first token of the first $\alpha$ string
of the next line to
see whether that is or expands to \cs{noalign}.
Similarly, for all entries
in a line the first token is expanded to see
whether it is or expands to \cs{omit}. This control sequence
will be described below.

Entries starting with an \cs{if...} conditional, or a macro
expanding to one, may be misinterpreted owing to this
premature expansion. For example,
\begin{verbatim}
\halign{$#$\cr \ifmmode a\else b\fi\cr}
\end{verbatim}
will give
\begin{disp}\leavevmode
     \vbox{\halign{$#$\cr \ifmmode a\else b\fi\cr}}\end{disp}
because the conditional is evaluated before math mode has been set up.
The solution is, as in many other cases, to insert a 
\cs{relax} control sequence to stop the expansion.
Here the \cs{relax} has to be inserted at the start of the 
alignment entry.

If neither \cs{noalign} nor \cs{omit} (see below) is found,
\TeX\ will process an input stream composed
of  the $u$ part, the $\alpha$ tokens
(which are delimited by either \n\& or \cs{span}, see below),
and the $v$ part.

Entries are delimited by \n\&, \cs{span}, or \cs{cr}, but
only if such a token occurs on the same level of grouping.
This makes it possible to have an alignment as an entry of
another alignment.

%\spoint Alternate specifications: \cs{omit}
\subsection{Alternate specifications: \cs{omit}}

The template line will rarely be sufficient to describe
all lines of the alignment. For lines where items should be
set differently the command \csterm{omit} exists:
if the first token in an entry is (or expands to) \cs{omit}
the trivial template \n\# is used instead of
what the template line specifies.

\begin{example} The following alignment uses the same template for
all columns, but in the second column an \cs{omit} command
is given.
\begin{verbatim}
\tabskip=1em
\halign{&$<#>$\cr a&\omit (b)&c \cr}
\end{verbatim}
The output of this is
\begin{disp}\leavevmode\vbox{\tabskip=1em
\halign{&$<#>$\cr a&\omit (b)&c \cr}}
\end{disp}
\end{example}

%\spoint Spanning across multiple columns: \cs{span}
\subsection{Spanning across multiple columns: \cs{span}}

Sometimes it is desirable to have material spanning several
columns. The most obvious example is that of a heading above
a table. For this \TeX\ provides the \cs{span} command.

Entries are delimited either by \n\&, by \cs{cr}, or by \csterm{span}.
In the last case \TeX\ will omit the tabskip glue that
would normally follow the entry thus delimited, and
it will typeset the material just read plus the following
entry in the joint space available.

As an example,
\begin{verbatim}
\tabskip=1em
\halign{&#\cr a&b&c&d\cr a&\hrulefill\span\hrulefill&d\cr}
\end{verbatim}
gives
\begin{disp}\leavevmode\vbox{\tabskip=1em
\halign{&#\cr a&b&c&d\cr a&\hrulefill\span\hrulefill&d\cr}}
\end{disp} Note that there is no tabskip glue in between the
two spanned columns, but there is tabskip glue before the
\alt
first column and after the last.

Using the \cs{omit} command this same alignment could
have been generated as
\begin{verbatim}
\halign{&#\cr a&b&c&d\cr a&\hrulefill\span\omit&d\cr}
\end{verbatim}

The \cs{span}\cs{omit} combination is used in the
plain \TeX\ macro
\cs{multispan}: for instance
\begin{disp}\cs{multispan4}\quad gives\quad \verb>\omit\span\omit\span\omit\span\omit>
\end{disp} which spans across three tabs, and removes the templates
of four entries. 
Repeating the above example once again:
\begin{verbatim}
\halign{&#\cr a&b&c&d\cr a&\multispan2\hrulefill&d\cr}
\end{verbatim}
The argument of \cs{multispan} is a single token,
not a number,
so in order to span more than 9 columns the argument
should be enclosed in braces, for instance \verb>\multispan{12}>.
\alt
Furthermore, a space after a single-digit argument
will wind up in the output.

For a `low budget' solution to spanning columns plain \TeX\ has the
macro \csterm{hidewidth}, defined by 
\begin{verbatim}
\newskip\hideskip \hideskip=-1000pt plus 1fill 
\def\hidewidth{\hskip\hideskip}
\end{verbatim}
Putting \cs{hidewidth} at the beginning or end of an alignment entry
will make its width zero, with the material in the entry
sticking out to the left or right respectively.


%\spoint Rules in alignments
\subsection{Rules in alignments}
\index{alignments!rules in|(}

Horizontal rules inside a horizontal alignment will mostly
\howto Draw rules in an alignment\par
be across the width of the alignment. The easiest way
to attain this is to use
\begin{verbatim}
\noalign{\hrule}
\end{verbatim}
lines inside the alignment. If the alignment is contained
in a vertical box, lines above and below the alignment
can be specified with
\begin{verbatim}
\vbox{\hrule \halign{...} \hrule}
\end{verbatim}
The most general way to get horizontal lines in an alignment
is to use
\csterm{multispan}%
\begin{disp}\cs{multispan}$\,n$\cs{hrulefill}\end{disp}
which can be used to underline arbitrary adjacent columns.

Vertical rules in alignments take some more care.
Since a horizontal alignment breaks up into
horizontal boxes that will be placed on a vertical list,
\TeX\ will insert baselineskip glue in between the rows
of the alignment. If vertical rules in adjacent rows
are to abut, it is necessary to prevent baselineskip glue,
for instance by the \cs{offinterlineskip} macro.

In order to ensure that rows will still be properly spaced
it is then necessary to place a {\italic strut\/}
somewhere in the preamble.
A~strut is an invisible object with a certain height
and depth. Putting that in the preamble guarantees that
every line will have at least that height and depth.
In the plain format \csterm{strut} is
defined statically as
\begin{verbatim}
\vrule height8.5pt depth3.5pt width0pt
\end{verbatim}
so this must be changed when other fonts or sizes are used.

It is a good idea to use a whole column for a~vertical
rule, that is, to write
\begin{verbatim}
\vrule#&
\end{verbatim}
in the preamble and
to leave the corresponding entry in the alignment empty.
Omitting the vertical rule can then be done by specifying \cs{omit},
and the size of the rule can be specified explicitly by
putting, for instance,
\hbox{\n{height 15pt}} in the entry instead of leaving
it empty. Of course, tabskip glue will now be specified to the
left and right of the rule, so some extra tabskip assignments
may be needed in the preamble.

\index{alignments!rules in|)}

\subsection{End of a line: \cs{cr} and \cs{crcr}}

All lines in an alignment are terminated by the \csterm{cr} control 
sequence, including the last line. 
\TeX\ is not  able to infer from
a closing brace in the $\alpha$~part that the
alignment has ended, because an unmatched
closing brace is perfectly valid in
an alignment entry;  it may match an opening brace in
the $u$~part of the corresponding preamble entry.

\TeX\ has a primitive command \csterm{crcr} that is equivalent
to \cs{cr}, but it has no effect if it immediately follows
a~\cs{cr}.
Consider as an example the definition in plain \TeX\
of \csterm{cases}:
\begin{verbatim}
\def\cases#1{%
    \left\{\,\vcenter{\normalbaselines\m@th
              \ialign{ $##\hfil$& \quad##\hfil \crcr #1\crcr}}%
    \right.}
\end{verbatim}
Because of the \cs{crcr} after the user argument \verb.#1.,
the following two applications of this macro
\begin{disp}\verb>\cases{1&2\cr 3&4}>\quad and\quad \verb>\cases{1&2\cr 3&4\cr}>\end{disp}
both work. In the first case the \cs{crcr} in the macro
definition ends the last line;
in the second case the user's \cs{cr} ends the line,
and the \cs{crcr} is redundant.

After \cs{cr} and after a non-redundant \cs{crcr} the
\gr{token parameter} \csterm{everycr} is inserted.
This includes the \cs{cr} terminating the template line.

%\point Example: math alignments
\section{Example: math alignments}

The plain format has several alignment macros that function
in math mode. One example is \csterm{matrix}, defined by
\begin{verbatim}
\def\matrix#1{\null\,\vcenter{\normalbaselines\m@th
    \ialign{\hfil$##$\hfil && \quad\hfil$##$\hfil\crcr
            \mathstrut\crcr
         \noalign{\kern-\baselineskip}
            #1\crcr
            \mathstrut\crcr
         \noalign{\kern-\baselineskip}}}\,}
\end{verbatim}
This uses a repeating (starting with~\verb>&&>) second preamble entry;
each entry is centred by an \cs{hfil} before and after it,
and there is a \cs{quad} of space in between columns.
Tabskip glue was not used for this, because there should not
be any glue preceding or following the matrix.

The combination of a \cs{mathstrut} and \verb>\kern-\baselineskip>
above and below the matrix increases the vertical size
such that two matrices with the same number of rows will have
the same height and depth, which would not otherwise be the case
if one of them had subscripts in the last row, but the other
not. The \cs{mathstrut} causes interline glue to be inserted 
and, because it has a size equal to \cs{baselineskip},
the negative kern will effectively leave only the interline glue,
thereby buffering any differences in the first and last line.
Only to a certain point, of course: objects bigger than the
opening brace will still result in a different height or depth of the 
matrix.

Another, more
complicated, example of an alignment for math mode is \cs{eq\-alignno}.
\csterm{eqalignno}%
\csterm{centering}%
\begin{verbatim}
\def\eqalignno#1{\begin{disp}l@y \tabskip\centering
  \halign to\displaywidth{
    \hfil$\@lign\displaystyle{##}$%    -- first column
        \tabskip\z@skip
    &$\@lign\displaystyle{{}##}$\hfil% -- second column
        \tabskip\centering
    &\llap{$\@lign##$}%                -- third column
        \tabskip\z@skip\crcr   %   end of the preamble
       #1\crcr}}
\end{verbatim}
Firstly, the tabskip is set to zero after the equation
number, so this number is set flush with the right margin.
Since it is placed by \cs{llap}, its effective width
is zero. Secondly, the tabskip between the
first and second columns is also zero, and the tabskip
before the first column and after the second  is
\alt
\cs{centering}, which is \n{0pt plus 1000pt minus 1000pt},
so the first column and second  are jointly centred
in the \cs{hsize}. Note that, because of the
\n{minus 1000pt}, these two columns will happily go
outside the left and right margins, overwriting any
equation numbers.

\index{alignments|)}

\chapter{Page Shape}\label{page:shape}

This chapter treats some of the parameters that 
determine the size of the page and how it appears on paper.

\label{cschap:topskip}\label{cschap:hoffset}\label{cschap:voffset}\label{cschap:vsize2}\label{cschap:maxdepth}\label{cschap:splitmaxdepth2}
\begin{inventory}
\item [\cs{topskip}] 
      Minimum distance between the top of the page box
      and the baseline of the first box on the page. 
      Plain \TeX\ default:~\n{10pt}

\item [\cs{hoffset \cs{voffset}}]
\mdqon
      Distance by which the page is shifted right/""down 
\mdqoff
      with respect to the reference point.

\item [\cs{vsize}] 
      Height of the page box.
      Plain \TeX\ default:~\n{8.9in}

\item [\cs{maxdepth}] 
      Maximum depth of the page box.
      Plain \TeX\ default:~\n{4pt}

\item [\cs{splitmaxdepth}] 
      Maximum depth of a box split off by a \cs{vsplit} operation. 
      Plain \TeX\ default: \cs{maxdimen}

\end{inventory}

%\point The reference point for global positioning
\section{The reference point for global positioning}

The \indexterm{page positioning} on the paper is governed by
a \TeX\ convention, to which output device drivers
must adhere, that the top left point of the page is
one inch from the page edges. Unfortunately this
may lead to lots of trouble, for instance if a printer
(or the page description language it uses)
takes, say, the {\em lower\/} left corner as the
reference point, and is factory set to US paper sizes,
but is used with European standard A4 paper.

The page is shifted on the paper if one assigns non-zero
values to \csterm{hoffset} or \csterm{voffset}: positive values
shift to the right and down respectively.

%\point \cs{topskip}
\section{\protect\cs{topskip}}

The \csterm{topskip} ensures to a certain point
that the first baseline of a page
will be at the same location from page to page,
even if font sizes
are switched between pages or if the first line has
no ascenders.

Before the first box on each page some glue is inserted.
This glue has the same stretch and shrink as \cs{topskip}, but
the natural size is the natural size of \cs{topskip}
minus the height of the first box, or zero if this
would be negative.

Plain \TeX\ sets \cs{topskip} to {\tt 10pt}.
Thus the top lines of pages will have their baselines
at the same place if
the top portion of the characters is ten point or less.
For the Computer Modern fonts this condition is satisfied
if the font size is less than (about) 13~points; 
for larger fonts
the baseline of the top line will drop.

The height of the page box for a page containing only
text (and assuming a zero \cs{parskip})
will be the \cs{topskip} plus a number of times
the \cs{baselineskip}. Thus one can define a macro
to compute the \cs{vsize} from the number of lines
on a page:
\howto Specify page height in lines\par
\begin{verbatim}
\def\HeightInLines#1{\count@=#1\relax
    \advance\count@ by -1\relax
    \vsize=\baselineskip
    \multiply\vsize by \count@
    \advance\vsize by \topskip}
\end{verbatim}
Calculating the \cs{vsize} this way will prevent
underfull boxes for text-only pages.

In cases where the page does not start with a line of text
(for instance a rule), the topskip may give unwanted effects.
To prevent these, start the page with
\begin{verbatim}
\hbox{}\kern-\topskip
\end{verbatim}
followed by what you wanted on top. 

Analogous to the \cs{topskip}, there is a \cs{splittopskip}
for pages generated by a \cs{vsplit} operation; see
the next chapter.

%\point Page height and depth
\section{Page height and depth}

\index{page!height|(}
\index{page!depth|(}

\TeX\ tries to build pages as a \cs{vbox} of height \csterm{vsize};
\alt
see also \cs{pagegoal} in the next chapter.

If the last item on a page has an excessive depth,
that page would be noticeably longer than other pages.
To prevent this phenomenon \TeX\ uses \csterm{maxdepth} as
the maximum depth of the page box. If adding an item to the
page would make the depth exceed this quantity, then the
reference point of the page is moved down to make the depth
exactly \cs{maxdepth}.

The `raggedbottom' effect is obtained in plain \TeX\
\csterm{raggedbottom}%
by giving the \cs{topskip} some finite stretchability:
\hbox{\n{10pt plus 60pt}}.
Thus the natural height of box~255 can vary when it reaches
the output routine.
Pages are then shipped out (more or less) as
\begin{verbatim}
\dimen0=\dp255 \unvbox255
\ifraggedbottom \kern-\dimen0 \vfil \fi 
\end{verbatim}
The \cs{vfil} causes the topskip to be set at natural
width, so the effect is one of a fixed top line  and a
variable bottom line of the page.

Before \cs{box255} is unboxed in the plain \TeX\ output routine,
\cs{boxmaxdepth} is set to \cs{maxdepth}
so that this box will made under the same assumptions
that the page builder used when putting together \cs{box255}.

The depth of box split off by a \cs{vsplit} operation
is controlled by the \cs{splitmaxdepth} parameter.

\index{page!height|)}
\index{page!depth|)}

\endofchapter
\chapter{Page Breaking}\label{page:break}

\index{page!breaking|(}

This chapter treats the `page builder': the part of \TeX\
that decides where to break the main vertical list into pages.
The page builder operates before the output routine,
and it hands its result in \cs{box255} to the output routine.

\label{cschap:vsplit2}\label{cschap:splittopskip}\label{cschap:pagegoal}\label{cschap:pagetotal}\label{cschap:pagedepth}\label{cschap:pagestretch}\label{cschap:pagefilstretch}\label{cschap:pagefillstretch}\label{cschap:pagefilllstretch}\label{cschap:pageshrink}\label{cschap:outputpenalty}\label{cschap:interlinepenalty}\label{cschap:clubpenalty}\label{cschap:widowpenalty}\label{cschap:displaywidowpenalty}\label{cschap:brokenpenalty}\label{cschap:penalty2}\label{cschap:lastpenalty}\label{cschap:unpenalty}
\begin{inventory}
\item [\cs{vsplit}]
      Split of a top part of a box. This is comparable
      with page breaking.

\item [\cs{splittopskip}] 
      Minimum distance between the top of what remains after a
      \cs{vsplit} operation, and the first item in that box.
      Plain \TeX\ default:~\n{10pt}

\item [\cs{pagegoal}] 
      Goal height of the page box. This starts at \cs{vsize},
      and is diminished by heights of insertion items.

\item [\cs{pagetotal}] 
      Accumulated natural height of the current page.

\item [\cs{pagedepth}] 
      Depth of the current page.

\item [\cs{pagestretch}] 
      Accumulated zeroth-order stretch of the current page.

\item [\cs{pagefilstretch}] 
      Accumulated first-order stretch of the current page.

\item [\cs{pagefillstretch}] 
      Accumulated second-order stretch of the current page.

\item [\cs{pagefilllstretch}] 
      Accumulated third-order stretch of the current page.

\item [\cs{pageshrink}] 
      Accumulated shrink of the current page.

\item [\cs{outputpenalty}]   
      Value of the penalty at the current page break,
      or $10\,000$ if the break was not at a penalty.

\item [\cs{interlinepenalty}] 
      Penalty for breaking a page between lines of a paragraph. 
      Plain \TeX\ default:~\n{0}

\item [\cs{clubpenalty}] 
      Additional penalty for breaking a page after 
      the first line of a paragraph. 
      Plain \TeX\ default:~\n{150}

\item [\cs{widowpenalty}] 
      Additional penalty for breaking a page before 
      the last line of a paragraph. 
      Plain \TeX\ default:~\n{150}

\item [\cs{displaywidowpenalty}] 
      Additional penalty for breaking a page before the last line 
      above a display formula. 
      Plain \TeX\ default:~\n{50}

\item [\cs{brokenpenalty}] 
      Additional penalty for breaking a page after a hyphenated line. 
      Plain \TeX\ default:~\n{100}

\item [\cs{penalty}]
      Place a penalty on the current list.
\item [\cs{lastpenalty}]
      If the last item on the list was a penalty, the value of this.
\item [\cs{unpenalty}]
      Remove the last item of the current list if this
      was a penalty.

\end{inventory}

%\point The current page and the recent contributions
\section{The current page and the recent contributions}

The main vertical list of \TeX\ is divided in two parts:
the \indexterm{current page} and the list of \indexterm{recent contributions}.
Any material that is added to the main vertical list is
appended to the recent contributions; the act of moving
the recent contributions to the current page is known
as exercising the \indextermsub{page}{builder}.

Every time something is moved to the current page, \TeX\
computes the cost of breaking the page at that point.
If it decides that it is past the optimal point,
the current page up to 
\altt 
the best break so far
is put in \cs{box255} and the remainder of
the current page is moved back on top of the recent contributions.
If the page is broken at a penalty,
\label{break:penalty}%
that value is recorded in \cs{outputpenalty}, and 
a penalty of size $10\,000$ is placed on top of the
recent contributions; otherwise, \csterm{outputpenalty}
is set to~$10\,000$.

If the current page is empty, discardable items that are moved
from the recent contributions are discarded. This is the mechanism
that lets glue disappear after a page break and at the top of
the first page. When the first non-discardable item is moved
to the current page, the \cs{topskip} glue is inserted;
see the previous chapter.

The workings of the page builder can be made visible by
setting \cs{tracingpages} to some positive value
(see Chapter~\ref{trace}).

%\point Activating the page builder
\section{Activating the page builder}

The page builder comes into play  in the
following circumstances.
\begin{itemize}\item Around paragraphs: after the \cs{everypar}
    tokens have been inserted, and after the paragraph has been
    added to the vertical list. See the end of this chapter for 
    an example.
\item Around display formulas: after the \cs{everydisplay}
    tokens have been inserted, and after the display has been
    added to the list.
\item After \cs{par} commands, boxes, insertions, 
    and explicit penalties in vertical mode. 
\item After an output routine has ended. \end{itemize}
In these places the page builder moves the recent
contributions to the current page. Note that \TeX\ need not be 
in vertical mode when the page builder is exercised.
In horizontal mode, activating the page builder
serves to move preceding vertical glue (for example, \cs{parskip},
\cs{abovedisplayskip}) to the page.

The \cs{end} command \ldash which is only allowed in
external vertical mode \rdash  terminates a \TeX\ job, but only if the
main vertical list is empty and \cs{deadcycles}${}=0$.
If this is not the case the combination
\label{end:play}%
\begin{disp}\verb>\hbox{}\vfill\penalty>$-2^{30}$\end{disp}
is appended, which forces the output routine to act.

%\point Page length bookkeeping
\section{Page length bookkeeping}

\index{page!length|(}

The height and depth of the page box that reaches the output
routine are determined by \cs{vsize}, \cs{topskip},
and~\cs{maxdepth} as described in the previous chapter.
\TeX\ places the \cs{topskip} glue
when the first box is placed on the current page; the
\cs{vsize} and \cs{maxdepth} are read when the first
box or insertion occurs on the page. Any subsequent changes to these
parameters will not be noticeable until the next page or,
more strictly, until after the output routine has been called.

After the first box, rule, or insertion on the current page
the \cs{vsize} is recorded in \cs{pagegoal},
and its value is not looked at  until \cs{output}
has been active.
Changing \cs{pagegoal} does have an effect on the current
page.
When the page is empty,
the pagegoal is \cs{maxdimen}, and \cs{pagetotal} is zero.

Accumulated dimensions and stretch are available in
the parameters \cs{pagetotal}, \cs{pagedepth}, 
\cs{pagestretch}, \cs{pagefilstretch}, \cs{pagefillstretch}, 
\cs{pageshrink},
and \cs{pagefilllstretch}.
\csterm{pagetotal}%
\csterm{pagedepth}%
\csterm{pagestretch}%
\csterm{pagefilstretch}%
\csterm{pagefillstretch}%
\csterm{pageshrink}%
\csterm{pagefilllstretch}%
They are set by the page builder. The stretch and
shrink parameters are updated every time glue is added
to the page. The depth parameter becomes zero
if the last item was kern or glue.

These parameters are \gr{special dimen}s; an assignment
to any of them is an \gr{intimate assignment},
and it is automatically global.

\index{page!length|)}

\section{Breakpoints}

%\spoint Possible breakpoints
\subsection{Possible breakpoints}

Page breaks can occur at the same kind of locations where
line breaks can occur. The \indexterm{breakpoints in vertical lists}
are:
\begin{itemize}\item at glue that is preceded by a non-discardable
item;\item at a kern that is immediately followed by glue;
\item at a penalty.\end{itemize}
\TeX\ inserts interline glue and various sorts of
interline penalties when the lines of a paragraph are
added to the vertical list, so there will usually be 
sufficient breakpoints on the page.

%\spoint Breakpoint penalties
\subsection{Breakpoint penalties}

If \TeX\ decides to break a page at a penalty item, this
penalty will, most of the time, be one that
has been inserted automatically
between the lines of a paragraph.

If the last item on a list (not necessarily a vertical list)
\alt
is a penalty, the value of this is recorded
in the parameter \csterm{lastpenalty}. If the item is other than
a penalty, this parameter has the value zero.
The last penalty of a list can be removed with the command
\csterm{unpenalty}. See Section~\ref{varioset} for an example.
\message{Spoint ref varioset}

Here is a list of the \indextermsub{penalties}{in vertical mode}:
\begin{inventory}
\item [\cs{interlinepenalty}] 
      Penalty for breaking a page between lines of a paragraph. 
      In plain \TeX\ this is zero, so no penalty is added in
      between lines. \TeX\ can then find a valid breakpoint at the
      \cs{baselineskip} glue.

\item [\cs{clubpenalty}] 
      Extra penalty for breaking a page after the first line of a paragraph. 
      In plain \TeX\ this is~\n{150}.
      This amount, and the following penalties, are 
      added to the \cs{interlinepenalty}, and
      a penalty of the resulting size is inserted after the
      \cs{hbox} containing the first line of a paragraph
      instead of the \cs{interlinepenalty}.

\item [\cs{widowpenalty}] 
      Extra penalty for breaking a page before the last line of a paragraph. 
      In plain \TeX\ this is~\n{150}.

\item [\cs{displaywidowpenalty}] 
      Extra penalty for breaking a page before the last line 
      above a display formula. The default value in plain \TeX\
      is~\n{50}.

\item [\cs{brokenpenalty}] 
      Extra penalty for breaking a page after a hyphenated line. 
      The default value in plain \TeX\ is~\n{100}.
\end{inventory}
If the resulting penalty is zero, it is not placed.

Penalties can also be inserted by the user. For instance,
the plain format has macros to encourage (possibly, force)
or prohibit page breaks%
\csterm{penalty}%
:
\begin{verbatim}
\def\break{\penalty-10000 }        % force break
\def\nobreak{\penalty10000 }       % prohibit break
\def\goodbreak{\par\penalty-500 }  % encourage page break
\end{verbatim}
Also, \verb>\vadjust{\penalty ... }> is a way of getting
penalties in the vertical list. This can be used to
discourage or encourage page breaking after a certain
line of a paragraph.

%\spoint Breakpoint computation
\subsection{Breakpoint computation}

\index{breakpoints!computation of|(}
\advance\rightskip by 5.5cm

Whenever an item is moved to the current page, \TeX\
\vadjust{\advance\hsize by -5.5cm
  \hbox to \hsize{\hfil\rlap{\hskip.4cm\vtop to 0pt
  {\kern-2\baselineskip
    \SansSerif %\pointSize:8 \Style:roman
   \parindent0pt \offinterlineskip
   \def\tbox#1{\hbox{\quad\quad #1%
               \vrule height 10pt depth3pt width0cm }}
   \hbox
   {\vrule width\lw \kern-\lw
    \vbox{\hsize=5cm
          \hrule height\lw \ \vskip0cm
           \kern40pt
           \tbox{underfull page}
           \tbox{$b=10\,000$}
           \kern40pt
          \hrule height\lw
           \kern8pt
           \tbox{feasible breakpoints}
           \tbox{$b<10\,000$}
           \kern8pt
          \hrule height\lw
           \kern8pt
           \tbox{overfull page}
           \tbox{$b=\infty$}
           \kern3pt
           \tbox{.\vrule height3.5pt depth1pt width0cm}
           \tbox{.\vrule height3.5pt depth1pt width0cm}
           \tbox{.\vrule height3.5pt depth1pt width0cm}
           \kern8pt
           }%
    \kern-\lw \vrule width\lw}%
    \vss}}}}
computes the penalty $p$ and the badness $b$ associated with
breaking the page at that place. From the penalty and
the badness the cost $c$ of breaking is computed.

The place of least cost is remembered, and when
the cost is infinite, that is, the page is overfull, or
when the penalty is $p\leq-10\,000$, the current page is broken
at  the (last remembered) place of least cost. 
The broken-off piece is then
put in \cs{box255} and the output routine token list
is inserted. Box 255 is always given a height of \cs{vsize},
regardless of how much material it has.

The badness calculation is based on the amount of stretching
or shrinking that is necessary to fit the page in
a box with height \cs{vsize} 
and maximum depth \cs{maxdepth}. This calculation is
the same as for line breaking (see Chapter~\ref{glue}).
Badness is a value $0\leq b\leq 10\,000$, except when
pages are overfull; then~$b=\infty$.

\advance\rightskip by -5.5cm

Some penalties are implicitly inserted by \TeX, 
for instance the \cs{interlinepenalty}
which is put in between every pair of lines of a paragraph.
Other penalties can
be explicitly inserted by the user or a user macro. 
A~penalty
value $p\geq10\,000$ inhibits breaking; a penalty
$p\leq-10\,000$ (in external vertical mode)
\alt
forces a page break, and immediately
activates the output routine. 

Cost calculation proceeds as follows:
\begin{enumerate} \item When a penalty is so low that it forces
a page break and immediate invocation of the output routine, 
but the page is not overfull, that is
\begin{disp}$b<\infty\quad\hbox{and}\quad p\leq-10\,000$\end{disp}
the cost is equal to the penalty:~$c=p$.

\item When penalties do not force anything, and the page is not
overfull, that is
\begin{disp}$b<\infty\quad\hbox{and}\quad |p|<10\,000$\end{disp}
the cost is~$c=b+p$.

\item For pages that are very bad, that is
\begin{disp}$b=10\,000\quad\hbox{and}\quad |p|<10\,000$\end{disp}
the cost is~$c=10\,000$.

\item An overfull page, that is
\begin{disp}$b=\infty\quad\hbox{and}\quad p<10\,000$\end{disp}
gives infinite cost:~$c=\infty$.
In this case \TeX\ decides that the optimal break point
must have occurred earlier, and it invokes the output routine.
Values of \cs{insertpenalties} (see Chapter~\ref{insert})
that exceed $10\,000$
also give infinite cost.
\end{enumerate}

The fact that a penalty $p\leq-10\,000$ activates
the output routine is used extensively
in the \LaTeX\ output routine: 
the excess $\mathopen|p\mathclose|-10\,000$ is
a code indicating the reason for calling the output routine;
see also the second example in the next chapter.

\index{breakpoints!computation of|)}

\section{\protect\cs{vsplit}}
\label{vsplit}

The page-breaking operation is available to the user
through the \csterm{vsplit} operation.

\begin{example}
\begin{verbatim}
\setbox1 = \vsplit2 to \dimen3
\end{verbatim}
assigns to box~1 the top part of size \cs{dimen3}
of box~2. This material is actually removed from box~2.
Compare this with splitting off a chunk of size \cs{vsize}
from the current page.
\end{example}

The extracted
result of 
\begin{disp}\cs{vsplit}\gr{8-bit number}\n{to}\gr{dimen}
\end{disp} is a box with the following properties.
\begin{itemize} \item Height equal to the specified \gr{dimen}; \TeX\ will
      go through the original box register (which must contain
      a vertical box) to find the best breakpoint. This may
      result in an underfull box.
\item Depth at most \csterm{splitmaxdepth}; this is analogous to
            the \cs{maxdepth} for the page box, rather than the \cs{boxmaxdepth}
      that holds for any box.
\item A first and last mark in the \cs{splitfirstmark} and 
      \cs{splitbotmark} registers.
\end{itemize}

The remainder of the \cs{vsplit} operation is a box where
\begin{itemize} \item all discardables have been removed
      from the top;
\item glue of size \csterm{splittopskip} has been inserted on top;
      if the box being split was box~255, it
      already had \cs{topskip} glue on top;
\item its depth has been forced to be at most \cs{splitmaxdepth}.
\end{itemize}

The bottom of the original box is always a valid breakpoint
for the \cs{vsplit} operation. If this breakpoint is taken,
the remainder box register is void. The extracted box
can be empty; it is only void if the original box
was void, or not a vertical box.

Typically, the \cs{vsplit} operation is used to split off part
of \cs{box255}. By setting \cs{splitmaxdepth} equal to \cs{boxmaxdepth}
the result is something that could have been made by \TeX's page
builder. After pruning the top of \cs{box255}, the 
mark registers \cs{firstmark} and \cs{botmark} contain the first
and last marks on the remainder of box~255.
See the next chapter for more information on marks.

%\point Examples of page breaking
\section{Examples of page breaking}

%\spoint Filling up a page
\subsection{Filling up a page}

Suppose a certain vertical box is too large
to fit on the remainder of the page.
Then
\begin{verbatim}
\vfil\vbox{ ... }
\end{verbatim}
is the wrong way
to fill up the page and push the box to the next.
\TeX\ can only break at the start of the glue, and
the \cs{vfil} is discarded after the break: the result
is an underfull, or at least horribly stretched, page.
On the other hand,
\begin{verbatim}
\vfil\penalty0 % or any other value
\vbox{ ... }
\end{verbatim}
is the correct way: \TeX\ will break
at the penalty, and the page will be filled.

%\spoint Determining the breakpoint
\subsection{Determining the breakpoint}

In the following examples the \cs{vsplit} operation is
used, which has the same
mechanism as page breaking.

Let the macros and
parameter settings
\begin{verbatim}
\offinterlineskip \showboxdepth=1
\def\High{\hbox{\vrule height5pt}}
\def\HighAndDeep{\hbox{\vrule height2.5pt depth2.5pt}}
\end{verbatim}
be given.

First let us consider
an example where a vertical list is simply stretched
in order to reach a break point.
\begin{verbatim}
\splitmaxdepth=4pt 
\setbox1=\vbox{\High \vfil \HighAndDeep}
\setbox2=\vsplit1 to 9pt
\end{verbatim}
gives
\begin{verbatim}
> \box2=
\vbox(9.0+2.5)x0.4, glue set 1.5fil
.\hbox(5.0+0.0)x0.4 []
.\glue 0.0 plus 1.0fil
.\glue(\lineskip) 0.0
.\hbox(2.5+2.5)x0.4 []
\end{verbatim}
The two boxes together have a height of \n{7.5pt},
so the glue has to stretch~\n{1.5pt}.

Next, we decrease the allowed depth of the resulting list.
\begin{verbatim}
\splitmaxdepth=2pt 
\setbox1=\vbox{\High \vfil \HighAndDeep}
\setbox2=\vsplit1 to 9pt
\end{verbatim}
gives
\begin{verbatim}
> \box2=
\vbox(9.0+2.0)x0.4, glue set 1.0fil
.\hbox(5.0+0.0)x0.4 []
.\glue 0.0 plus 1.0fil
.\glue(\lineskip) 0.0
.\hbox(2.5+2.5)x0.4 []
\end{verbatim}
The reference point is moved down half a point,
and the stretch is correspondingly diminished,
\alt
but this motion cannot lead to a larger dimension
than was specified.

As an example of this,
\alt
consider the sequence
\begin{verbatim}
\splitmaxdepth=3pt 
\setbox1=\vbox{\High \kern1.5pt \HighAndDeep}
\setbox2=\vsplit1 to 9pt
\end{verbatim}
This gives a box exactly 9 points high and 2.5 points deep.
Setting \verb>\splitmaxdepth=2pt> does not increase
the height by half a point; instead, an underfull box
results because an earlier break is taken.

Sometimes the timing of actions is important.
\TeX\ first locates a breakpoint that will lead
to the requested height, then checks whether accommodating
the \cs{maxdepth} or \cs{splitmaxdepth} will not
violate that height.

Consider an example of this timing:
\alt
in
\begin{verbatim}
\splitmaxdepth=4pt 
\setbox1=\vbox{\High \vfil \HighAndDeep}
\setbox2=\vsplit1 to 7pt
\end{verbatim}
the result is {\italic not\/} a box
of 7 points high and 3 points deep. Instead,
\begin{verbatim}
> \box2=
\vbox(7.0+0.0)x0.4
.\hbox(5.0+0.0)x0.4 []
\end{verbatim}
which is an underfull box.

%\spoint[par:page:build] The page builder after a paragraph
\subsection{The page builder after a paragraph}
\label{par:page:build}

After a paragraph, the page builder moves material
to the current page, but it does not decide whether a breakpoint
has been found yet.

\begin{example}
\begin{verbatim}
\output{\interrupt \plainoutput}% show when you're active
\def\nl{\hfil\break}\vsize=22pt % make pages of two lines
a\nl b\nl c\par \showlists      % make a 3-line paragraph
\end{verbatim}
will report
\begin{verbatim}
### current page:
[...]
total height 34.0
 goal height 22.0
prevdepth 0.0, prevgraf 3 lines
\end{verbatim}
Even though more than enough
material has been gathered, \cs{output} is only invoked
when the next paragraph starts: typing a \n d gives
\begin{verbatim}
! Undefined control sequence.
<output> {\interrupt 
                     \plainoutput }
<to be read again> 
                   d
\end{verbatim}
when \cs{output} is inserted after \cs{everypar}.
\end{example}

\index{page!breaking|)}

\endofchapter
\chapter{Output Routines}\label{output}
\index{output routine|(}

The final stages of page processing are performed by the
output routine. The page builder cuts off a certain portion
of the main vertical list and hands it to the output routine
in \cs{box255}. This chapter treats the commands and parameters
that pertain to the output routine, and it explains how
output routines can receive information through marks.

\label{cschap:output}\label{cschap:shipout}\label{cschap:mark}\label{cschap:topmark}\label{cschap:botmark}\label{cschap:firstmark}\label{cschap:splitbotmark}\label{cschap:splitfirstmark}\label{cschap:deadcycles}\label{cschap:maxdeadcycles}\label{cschap:outputpenalty2}
\begin{inventory}
\item [\cs{output}] 
      Token list with instructions for shipping out pages.

\item [\cs{shipout}] 
      Ship a box to the \n{dvi} file.


\item [\cs{mark}] 
      Specify a mark text.

\item [\cs{topmark}] 
      The last mark on the previous page.

\item [\cs{botmark}] 
      The last mark on the current page.

\item [\cs{firstmark}] 
      The first mark on the current page.

\item [\cs{splitbotmark}] 
      The last mark on a split-off page.

\item [\cs{splitfirstmark}] 
      The first mark on a split-off page.

\item [\cs{deadcycles}] 
      Counter that keeps track of how many times 
      the output routine has been called without a \cs{shipout} 
      taking place. 

\item [\cs{maxdeadcycles}] 
      The maximum number of times that the output routine is allowed to
      be called without a \cs{shipout} occurring.

\item [\cs{outputpenalty}] 
      Value of the penalty at the current page break,
 \alt
      or $10\,000$ if the break was not at a penalty.

\end{inventory}


%\point The \cs{output} token list
\section{The \protect\cs{output} token list}

Common parlance has it that
`the output routine is called' when \TeX\ has found a place
to break the main vertical list.
Actually, \cs{output} is not a macro but a token list that
is inserted into \TeX's command stream.

Insertion of the \cs{output} token list happens
\csterm{output}%
inside a group that is implicitly opened.
Also, \TeX\ enters internal vertical mode.
Because of the group, non-local assignments 
(to the page number, for instance)
have to be prefixed with \cs{global}.
The vertical mode implies that during the workings of the 
output routine 
spaces are mostly harmless.

The \cs{output} token list belongs
to the class of the
\gr{token parameter}s. These behave the same as
\cs{toks}$nnn$ token lists; see Chapter~\ref{token}.
Assigning an output routine can therefore take the following
forms:
\begin{disp}\cs{output}\gr{equals}\gr{general text}\quad
or\quad 
\cs{output}\gr{equals}\gr{filler}\gr{token variable}
\end{disp}


%\point[output255] Output and \cs{box255}
\section{Output and \protect\cs{box255}}
\label{output255}

\TeX's page builder breaks the current page at the optimal point,
and stores everything above that in \cs{box255};
then, the \cs{output} tokens are inserted into the input stream.
Any remaining material on the main vertical list
is pushed back to the recent 
contributions. 
If the page is broken at a penalty,
\alt
that value is recorded in \cs{outputpenalty}, and 
a penalty of size $10\,000$ is placed on top of the
recent contributions; otherwise, \cs{outputpenalty}
is set to~$10\,000$.
When the output routine is finished, \cs{box255} is
supposed to be empty.
If it is not, \TeX\ gives an error message.

Usually, the output routine will take the pagebox,
\csterm{shipout}%
\mdqon
append a headline and/""or footline,
\mdqoff
maybe merge in some insertions such as footnotes,
and ship the page to the \n{dvi} file:
\begin{verbatim}
\output={\setbox255=\vbox
           {\someheadline 
            \vbox to \vsize{\unvbox255 \unvbox\footins}
            \somefootline}
         \shipout\box255}
\end{verbatim}
When box 255 reaches the output routine, its height has
been set to \cs{vsize}.
However, the material in it can have considerably
smaller height.
Thus, the above output routine may lead to underfull boxes.
This can be remedied with a \cs{vfil}.

The output routine is under no obligation to
\csterm{deadcycles}%
do anything useful with \cs{box255}; it can empty it, or
unbox it to let \TeX\ have another go at finding a page
break. The number of times
that the output routing  postpones the \cs{shipout}
is recorded in \cs{deadcycles}: this parameter is set to~0
by \cs{shipout}, and increased by~1 just before 
every \cs{output}.

When  the number of dead cycles reaches
\csterm{maxdeadcycles}, \TeX\ gives an error message,
and performs the default output routine
\begin{verbatim}
\shipout\box255
\end{verbatim}
instead of the routine it was about
to start.
The \LaTeX\ format has a much higher value for \cs{maxdeadcycles}
than plain \TeX, because the output routine in \LaTeX\
is often called for
intermediate handling of floats and marginal notes.

The \cs{shipout} command can send any \gr{box} to the \n{dvi} file;
this need not be box 255, or even a box
containing the current page.
It does not have to be called inside the output routine, either.

If the output routine produces any material, for instance
by calling
\begin{verbatim}
\unvbox255
\end{verbatim}
this is put on top
of the recent contributions.

After the output routine finishes, the page builder is
activated. In particular, because the current page
has been emptied, the \cs{vsize} is read again.
Changes made to this parameter inside the output
routine (using \cs{global}) will therefore take effect.

%\point Marks
\section{Marks}

Information can be passed to the output routine through the
\csterm{mark}%
mechanism of \indexterm{marks}. The user can specify a token list
with 
\begin{disp}\cs{mark}\lb\gr{mark text}\rb\end{disp}
which is put in a mark item on the current vertical list.
The mark text is subject to expansion as in \cs{edef}.

If the mark is given in horizontal mode it migrates to
the surrounding vertical lists like an insertion item
(see page~\pageref{migrate});
however, if this is not the external vertical list, the
output routine will not find the mark.

Marks are the main mechanism through which the output routine
can obtain information about the contents of the currently
broken-off page, in particular its top and bottom.
\TeX\ sets three variables:
\begin{description} 
\item [\cs{botmark}]
   the last mark occurring on the current page;\csidx{botmark}
\item [\cs{firstmark}]
   the first mark occurring on the current page;\csidx{firstmark}
\item [\cs{topmark}]
   the last mark of the previous page,\csidx{topmark}
   that is, the value of \cs{botmark}
   on the previous page.
\end{description}
If no marks have occurred yet, all three are empty;
if no marks occurred on the current page, 
all three mark variables are equal
to the \cs{botmark} of the previous page.

For boxes generated by a \cs{vsplit} command (see previous chapter),
the \cs{splitbotmark} and \cs{splitfirstmark}
\csterm{splitbotmark}%
\csterm{splitfirstmark}%
contain the marks of the split-off part; \cs{firstmark}
and \cs{botmark} reflect the state of what remains in the register.

\begin{example} Marks can be used to get a section heading into
\howto Do tricks with headlines\par
the headline or footline of the page.
\begin{verbatim}
\def\section#1{ ... \mark{#1} ... }
\def\rightheadline{\hbox to \hsize
   {\headlinefont \botmark\hfil\pagenumber}}
\def\leftheadline{\hbox to \hsize
   {\headlinefont \pagenumber\hfil\firstmark}}
\end{verbatim}
This places the title of the first section that starts on a 
left page in the left headline, and the title of the last section
that starts on the right page in the right headline.
Placing the headlines on the page is the job of the output routine;
see below.

It is important that no page breaks can occur in between the
mark and the box that places the title:
\begin{verbatim}
\def\section#1{ ...
    \penalty\beforesectionpenalty
    \mark{#1}
    \hbox{ ... #1 ...}
    \nobreak
    \vskip\aftersectionskip
    \noindent}
\end{verbatim}
\end{example}

Let us consider
another example with headlines: often a page looks better if
the headline is omitted on pages where a chapter starts.
This can be implemented as follows:
\begin{verbatim}
\def\endofchapter
\chapter#1{ ... \def\chtitle{#1}\mark{1}\mark{0} ... }
\def\theheadline{\expandafter\ifx\firstmark1 
    \else \chapheadline \fi}
\end{verbatim}
Only on the page where a chapter starts will the mark be~1,
and on all other pages a headline is placed.

%\point Assorted remarks
\section{Assorted remarks}

%\spoint Hazards in non-trivial output routines
\subsection{Hazards in non-trivial output routines}

If the final call to the output routine does not
perform a \cs{shipout}, \TeX\ will call the output
routine endlessly, since a run will only stop if both
the vertical list is empty, and \cs{deadcycles}
is zero. The output routine can set \cs{deadcycles}
to zero to prevent this.

%\spoint Page numbering
\subsection{Page numbering}
\index{page!numbering|(}

The page number is not an intrinsic property of the output
routine; in plain \TeX\ it is  the value of \cs{count0}.
The output routine is responsible for increasing the
page number when a shipout of a  page occurs.

Apart from   \cs{count0}, counter registers~1--9 are also used
for page identification: at shipout \TeX\ writes the values
of these ten counters to the \n{dvi} file (see Chapter~\ref{TeXcomm}).
Terminal and log file output display only the non-zero counters,
and the zero counters for which a non-zero counter with
a higher number exists, that is, if \cs{count0}${}=1$ and
\cs{count3}${}=5$ are the only non-zero counters, the
displayed list of counters is~\n{[1.0.0.5]}.

\index{page!numbering|)}

\subsection{Headlines and footlines in plain \TeX}

Plain \TeX\ has token lists \cs{headline} and
\cs{footline}; these are used in the macros
\handbreak \cs{makeheadline} and \cs{makefootline}.
The page is shipped out as (more or less)
\begin{verbatim}
\vbox{\makeheadline\pagebody\makefootline}
\end{verbatim}

Both headline and footline are inserted inside a \cs{line}.
For non-standard headers and footers it is easier to
redefine the macros \cs{makeheadline} and \cs{makefootline}
than to tinker with the token lists.

%\spoint Example: no widow lines
\subsection{Example: no widow lines}

Suppose that one does not want to allow widow lines,
but pages have in general no stretch or shrink,
for instance because they only contain plain text.
A~solution would be to increase the page length
by one line if a page turns out to be broken
at a widow line.

\TeX's output routine can perform this sort of
trick: if the \cs{widowpenalty} is set to
some recognizable value, the output routine
can see by the \cs{outputpenalty} if a widow
line occurred. In that case, the output routine
can temporarily increase the \cs{vsize}, and
let the page builder have another go at
finding a break point.

Here is the skeleton of such an output routine.
No headers or footers are provided for.
\begin{verbatim}
\newif\ifLargePage \widowpenalty=147
\newdimen\oldvsize \oldvsize=\vsize
\output={
    \ifLargePage \shipout\box255 
          \global\LargePagefalse
          \global\vsize=\oldvsize
    \else \ifnum \outputpenalty=\widowpenalty
             \global\LargePagetrue
             \global\advance\vsize\baselineskip
             \unvbox255 \penalty\outputpenalty
          \else  \shipout\box255
    \fi   \fi}
\end{verbatim}
The test \cs{ifLargePage} is set to true by the
output routine if the \cs{outputpenalty}
equals the \cs{widowpenalty}. The page box
is then \cs{unvbox}$\,$ed, so that the page builder
will tackle the same material once more.

%\spoint Example: no indentation top of page
\subsection{Example: no indentation top of page}

Some  output routines can be classified
\howto Prevent indentation on top of page\par
as abuse of the output routine mechanism.
The output routine in this section is a good example of this.

It is imaginable that one wishes paragraphs not to indent
if they start at the top of a page. (There are plenty of objections
to this layout, but occasionally it is used.)
This problem can be solved using the output routine to
investigate whether the page is still empty and, if so,
to give a signal that a paragraph should not indent.

Note that we cannot use the fact here
that the page builder comes into play after
the insertion of \cs{everypar}: even if we could
force the output routine to be activated here,
there is no way for it to remove the indentation box.

The solution given here lets the \cs{everypar}
terminate the paragraph immediately
with
\begin{verbatim}
\par\penalty-\specialpenalty
\end{verbatim}
which activates the output routine.
Seeing whether the pagebox is empty (after removing
the empty line and any \cs{parskip} glue),
the output routine then can set a switch
signalling whether the retry of the paragraph
should indent.

There are some minor matters in the following
routines, the sense of which is left
for the reader to ponder.
\begin{verbatim}
\mathchardef\specialpenalty=10001
\newif\ifPreventSwitch 
\newbox\testbox 
\topskip=10pt

\everypar{\begingroup \par
    \penalty-\specialpenalty
    \everypar{\endgroup}\parskip0pt 
    \ifPreventSwitch \noindent \else \indent \fi
    \global\PreventSwitchfalse
    }
\output{
    \ifnum\outputpenalty=-\specialpenalty 
        \setbox\testbox\vbox{\unvbox255 
                  {\setbox0=\lastbox}\unskip}
        \ifdim\ht\testbox=0pt \global\PreventSwitchtrue
        \else \topskip=0pt \unvbox\testbox \fi
    \else \shipout\box255 \global\advance\pageno1 \fi}
\end{verbatim}


%\spoint More examples of output routines
\subsection{More examples of output routines}

A large number of examples of output routines
can be found in~\cite{Sal1} and~\cite{Sal2}.

\index{output routine|)}

\endofchapter
\chapter{Insertions}\label{insert}

Insertions are \TeX's way of handling floating information.
\TeX's page builder calculates what insertions and how many
of them will fit on the page; these insertion items are then
placed in insertion boxes which are to be handled by the
output routine.


\label{cschap:insert}\label{cschap:newinsert}\label{cschap:insertpenalties}\label{cschap:floatingpenalty}\label{cschap:holdinginserts}\label{cschap:footins}\label{cschap:topins}\label{cschap:topinsert}\label{cschap:pageinsert}\label{cschap:midinsert}\label{cschap:endinsert}
\begin{inventory}
\item [\cs{insert}] 
      Start an insertion item.

\item [\cs{newinsert}] 
      Allocate a new insertion class. 

\item [\cs{insertpenalties}] 
      Total of penalties for split insertions.
      Inside the output routine, the number of held-over insertions.

\item [\cs{floatingpenalty}] 
      Penalty added when an insertion is split.

\item [\cs{holdinginserts}]
      (\TeX3 only)
      If this is positive, insertions are not placed in their boxes 
      at output time.

\item [\cs{footins}]
      Number of the footnote insertion class in plain \TeX.

\item [\cs{topins}]
      Number of the top insertion class.

\item [\cs{topinsert}]
      Plain \TeX\ macro to start a top insert.

\item [\cs{pageinsert}]
      Plain \TeX\ macro to start an insert that will take
      up a whole page.

\item [\cs{midinsert}]
      Plain \TeX\ macro that places its argument if there is space,
      and converts it into a top insert otherwise.

\item [\cs{endinsert}]
      Plain \TeX\ macro to wind up an insertion item
      that started with \cs{topinsert}, \cs{midinsert},
      or \cs{pageinsert}.

\end{inventory}


%\point Insertion items
\section{Insertion items}

Floating information, items that are not bound to a fixed
place in the vertical list, such as figures or footnotes,
are handled by \indexterm{insertions}.
The treatment of insertions is a strange interplay between the
user, \TeX's internal workings, and the output routine.
First the user specifies an insertion, which is
a certain amount of vertical material; 
then \TeX's page builder decides what insertions should go
on the current page and puts these insertions in insertion boxes;
finally, the output routine has to do something with these boxes.

An insertion item looks like
\csterm{insert}%
\begin{disp}\cs{insert}\gr{8-bit number}\lb\gr{vertical mode material}\rb
\end{disp} where the 8-bit number should not be~255,
because \cs{box255} is used by \TeX\ for passing the page to the output
routine.

The braces around the vertical mode material in an insertion
item can be implicit; they imply a new level of grouping.
The vertical mode material is processed in internal
vertical mode.

Values of \cs{splittopskip}, \cs{splitmaxdepth}, 
and \cs{floatingpenalty} are relevant for split insertions
(see below); the values that are current just before
the end of the group are used.

Insertion items can appear in vertical mode, horizontal
mode, and math mode. For the latter two modes they have to
migrate to the surrounding vertical list
(see page~\pageref{migrate}).
After an insertion item is put on  the vertical list the
page builder is exercised.


%\point Insertion class declaration
\section{Insertion class declaration}

In the plain format 
the  number for a new insertion class
is allocated by \csterm{newinsert}:
\begin{verbatim}
\newinsert\myinsert % new insertion class
\end{verbatim}
which uses \cs{chardef} to assign a number to the control
sequence.

Insertion classes are allocated numbering from 254 downward.
As box~255 is used for output, this allocation scheme leaves
\cs{skip255}, \cs{dimen255}, and \cs{count255}
free for scratch use.

%\point Insertion parameters
\section{Insertion parameters}

For each insertion class~$n$ four registers are allocated:
\begin{itemize}
\item \cs{box}$\,n$ When the output routine is active this
 box contains the insertion items of class~$n$ that should
 be placed on the current page.
\item \cs{dimen}$\,n$ This is the maximum space allotted for
 insertions of class~$n$ per page. If this amount would
 be exceeded \TeX\ will split insertions.
\item \cs{skip}$\,n$ Glue of this size is added the first
 time an insertion item of class~$n$ is added to the
 current page. This is useful for such phenomena as a rule
 separating the footnotes from the text of the page.
\item \cs{count}$\,n$ Each insertion item is a vertical list,
 so it has a certain height. However, the effective height,
 the amount of influence it has on the text height of the
 page, may differ from this real height.
 The value of \cs{count}$\,n$
 is then 1000 times the factor by which the height should
 be multiplied to obtain the effective height.
 
 Consider the following examples:
 
\begin{itemize}\item Marginal notes do not affect
 the text height, so the factor should be~0. \item Footnotes
 set in double column mode affect the page by half of their height:
 the count value should by~500. \item Conversely, footnotes
 set at page width underneath a page in double column mode
 affect both columns, so \ldash provided that the double column mode
 is implemented by applying \cs{vsplit} to a double-height column \rdash 
 the count value should be~2000.\end{itemize}
\end{itemize}

%\point Moving insertion items from the contributions list
\section{Moving insertion items from the contributions list}

The most complicated issue with insertions is the algorithm
that adds insertion items to the main vertical list,
and calculates breakpoints if necessary.

\TeX\ never changes the \cs{vsize}, but it diminishes the
\csterm{pagegoal} by the (effective) heights of the insertion
items that will appear before a page break. Thus the output
routine will receive a \cs{box255} that has height \cs{pagegoal},
not necessarily \cs{vsize}.

\begin{enumerate}
\item When the first insertion of a certain class $n$ occurs
  on the current page \TeX\ has to account for the quantity
  \cs{skip}$\,n$. This step is executed only if no earlier
  insertion item of this class occurs on the vertical list
  \ldash this includes insertions that were split \rdash  but \cs{box}$\,n$
  need not be empty at this time.
  
  If \cs{box}$\,n$ is not empty, its height plus depth is multiplied
  by \cs{count}$\,n/1000$ and the result is subtracted
  from \cs{pagegoal}. Then the \cs{pagegoal} is diminished
  by the natural component of \cs{skip}$\,n$. Any stretch and
  shrink of \cs{skip}$\,n$ are incorporated in \cs{pagestretch}
  and \cs{pageshrink} respectively.
\item If there is a split insertion of class $n$ on the page
  \ldash this case and the previous step in the algorithm are
  mutually exclusive \rdash  the \csterm{floatingpenalty} is added to
  \csterm{insertpenalties}. A~split insertion is an insertion item
  for which a breakpoint has been calculated as it will not
  fit on the current page in its entirety. Thus the insertion
  currently under consideration will certainly not wind up 
  on the current page.
\item After the preliminary action of the two previous points
  \TeX\ will place the actual insertion item on the main vertical
  list, at the end of the current contributions.
  First it will check whether the item will fit without being split.
  
  There are two conditions to be checked:
\begin{itemize}\item
  adding the insertion item (plus all previous insertions of that class)
  to \cs{box}$\,n$ should not let
  the height plus depth of that box exceed \cs{dimen}$\,n$, and
  \item either the effective height of the insertion is negative, or
  \cs{pagetotal} plus \cs{pagedepth} minus \cs{pageshrink}
  plus the effective size of the insertion should be less than
  \cs{pagegoal}.\end{itemize}
  If these conditions are satisfied, \cs{pagegoal} is diminished
  by the effective size of the insertion item, that is,
  by the height plus depth, multiplied by \cs{count}$n/1000$.

\item Insertions that fail on one of the two conditions in the
  previous step of the algorithm will be considered for splitting.
  \TeX\ will calculate the size of the maximal portion to 
  be split off the insertion item, such that
  
\begin{enumerate}\item adding this portion
  together with earlier insertions of this class to \cs{box}$\,n$
  will not let the size of the box exceed \cs{dimen}$\,n$,
  and \item the effective size of this portion,
  added to \cs{pagetotal} plus \cs{pagedepth}, will not
  exceed \cs{pagegoal}. Note that \cs{pageshrink} is not taken
  into account this time, as it was in the previous step.
  \end{enumerate}
  
  Once this maximal size to be split off has been determined,
  \TeX\ locates the least-cost breakpoint in the current 
  insertion item that will result in a box with a  height
  that is equal to this maximal size. The penalty associated
  with this breakpoint is added to \cs{insertpenalties},
  and \cs{pagegoal} is diminished by the effective height plus
  depth of the box to be split off the insertion item.

\end{enumerate}



%\point Insertions in the output routine
\section{Insertions in the output routine}

When the output routine comes into action \ldash more precisely:
when \TeX\ starts processing the tokens in the \cs{output}
token list \rdash  all insertions that should be placed on the
current page have been put in their boxes, and
it is the responsibility of the output routine
to put them somewhere in the box that is going to be shipped out.

\begin{example} The plain \TeX\ output routine
handles top inserts and footnotes by packaging the following
sequence:
\begin{verbatim}
\ifvoid\topins \else \unvbox\topins \fi
\pagebody
\ifvoid\footins \else \unvbox\footins \fi
\end{verbatim}
Unboxing the insertion boxes makes the glue on various parts
of the page stretch or shrink in a uniform manner.
\end{example}

With \TeX3 the insertion mechanism has been extended slightly:
\handbreak \csterm{holdinginserts}%
\thecstoidxsub{TeX}{version 3}
the parameter \cs{holdinginserts} can be used to specify that
insertions should not yet be placed in their boxes.
This is very useful if the output routine wants to
recalculate the \cs{vsize}, or if the output routine
is called to do other intermediate calculations instead of
ejecting a page.

During the output routine the parameter
\csterm{insertpenalties} holds the number of insertion items that
are being held over for the next page.
In the plain \TeX\ output routine this is used after the
last page:
\begin{verbatim}
\def\dosupereject{\ifnum\insertpenalties>0 
    % something is being held over
  \line{}\kern-\topskip\nobreak\vfill\supereject\fi}
\end{verbatim}

%\point Plain \TeX\ insertions
\section{Plain \TeX\ insertions}

The plain \TeX\ format has only two insertion classes:
the footnotes and the top inserts.
The macro \csterm{pageinsert} generates
top inserts that are stretched to be exactly \cs{vsize} high.
The \csterm{midinsert} macro tests whether the vertical material
specified by the user fits on the page; if so, it is placed
there; if not, it is converted to a top insert.

Footnotes are allowed to be split, but once one has been
split no further footnotes should appear on the current
page. This effect is attained by setting 
\begin{verbatim}
\floatingpenalty=20000
\end{verbatim} 
The \cs{floatingpenalty} is added to \cs{insertpenalties}
if an insertion follows a split insertion of the same 
class. However, \cs{floatingpenalty}${}>10\,000$ has infinite
cost, so \TeX\ will take an earlier breakpoint for
splitting off the page from the vertical list.

Top inserts essentially contain only a vertical box
which holds whatever the user specified. Thus such an insert
cannot be split. However, the \csterm{endinsert} macro
puts a \cs{penalty100} on top of the box, so the
insertion can be split with an empty part before the split.
The effect is that the whole insertion is carried over to
the next page. As the \cs{floatingpenalty} for top inserts
is zero, arbitrarily many of these inserts can be moved forward
until there is a page with sufficient space.

Further examples of insertion macros can be found
in~\cite{Sal3}.

%\message{Maybe spaceleft example?}

%%%% end of input file [page]

%\InputFile:io
%%%% this is input file [io]
%\subject[io]  File Input and Output
\endofchapter
\chapter{File Input and Output}\label{io}
\index{I/O!file|(}

This chapter discusses the various ways in which \TeX\ can read from
and write to external \indexterm{files}.

\label{cschap:input}\label{cschap:endinput}\label{cschap:inputlineno}\label{cschap:message}\label{cschap:write}\label{cschap:read}\label{cschap:newread}\label{cschap:newwrite}\label{cschap:openin}\label{cschap:closein}\label{cschap:openout}\label{cschap:closeout}\label{cschap:ifeof}\label{cschap:immediate}\label{cschap:newlinechar}
\begin{inventory}
\item [\cs{input}] 
      Read a specified file as \TeX\ input. 

\item [\cs{endinput}] 
      Terminate inputting the current file after the current line.

\item [\cs{pausing}] 
      Specify that \TeX\ should pause after each line that is 
      read from a file.

\item [\cs{inputlineno}] 
      Number of the current input line.

\item [\cs{message}]
      Write a message to the terminal.

\item [\cs{write}]   
      Write a \gr{general text} to the terminal or to a file. 

\item [\cs{read}]   
      Read a line from a stream into a control sequence.

\item [\cs{newread \cs{newwrite}}]
\mdqon
      Macro for allocating a new input/""output stream.
\mdqoff

\item [\cs{openin \cs{closein}}]
      Open/close an input stream.

\item [\cs{openout \cs{closeout}}]
      Open/close an output stream.

\item [\cs{ifeof}]  
      Test whether a file has been fully read, or does not exist.

\item [\cs{immediate}]  
      Prefix to have output operations executed right away.

\item [\cs{escapechar}] 
      Number of the character that is  used 
      when control sequences are being converted
      into character tokens.
      \IniTeX\ default:~92.

\item [\cs{newlinechar}] 
      Number of the character that triggers a new line in
      \cs{write} and \cs{message} statements.

\end{inventory}


%\point Including files: \cs{input} and \cs{endinput}
\section{Including files: \protect\cs{input} and \protect\cs{endinput}}

Large documents can be segmented in \TeX\ by putting 
\csterm{input}%
parts in separate \indexterm{input files}, and loading these with \cs{input}
into the master file. The exact syntax for
file names is implementation dependent; most of the
time a \n{.tex} file extension is assumed if no explicit
extension is given.
File names can be delimited with a space or with \cs{relax}.
The \cs{input} command is expandable.

If \TeX\ encounters in an input file the
\csterm{endinput} statement, it acts as if the file
ends after the line on which the statement occurs.
Any statements on the same line as \cs{endinput} are
still executed.
The \cs{endinput} statement is expandable.

%\point File I{/}O
\section{File I{/}O}

\TeX\ supports input and output streams for reading and writing
\altt
files one line at a time.

%\spoint Opening and closing streams
\subsection{Opening and closing streams}

\TeX\ supports up to 16 simultaneous input and 16 output 
\indexterm{streams}.
The plain \TeX\ macros
\csterm{newread} and \csterm{newwrite} give the number of an unused
stream. This number is assigned by a \cs{chardef} command. 
Input streams are completely independent of output
streams. 

Input streams are opened by
\csterm{openin}%
\begin{disp}\cs{openin}\gr{4-bit number}\gr{equals}\gr{filename}\end{disp}
and closed by 
\csterm{closein}%
\begin{disp}\cs{closein}\gr{4-bit number}\end{disp}

Output streams are opened by
\csterm{openout}%
\begin{disp}\cs{openout}\gr{4-bit number}\gr{equals}\gr{filename}\end{disp}
and closed by
\csterm{closeout}%
\begin{disp}\cs{closeout}\gr{4-bit number}\end{disp}

If an output file does not yet exist, it is created
by \cs{openout}; if it did exist, an \cs{openout} will
cause it to be overwritten.

The output operations \cs{openout}, \cs{closeout},
and \cs{write} can all three be prefixed by \cs{immediate};
see below.

%\spoint Input with \cs{read}
\subsection{Input with \cs{read}}

In addition to the \cs{input} command, which reads a whole
file, \TeX\ has the \csterm{read} operation, which
reads one line from a file (or from the user terminal).
The syntax of the read command is
  
\begin{disp}\cs{read}\gr{number}\n{to}\gr{control sequence}\end{disp}
The effect of this statement is that one input line
is read from the designated stream, and the control
sequence is defined as a macro without parameters, having
that line as replacement text.

If the input line is not balanced with respect to braces,
\TeX\ will read more than one line, continuing for as long
as is necessary to get a balanced token list.
\TeX\ implicitly appends an empty line to each input stream,
\alt
so the last \cs{read} operation on a stream will always
yield a single \cs{par} token.

Read operations from any stream outside the range 0--15 \ldash or 
streams not associated with an open file, or on which the file
end has been reached \rdash 
read from the terminal. If the stream number is positive
the user is prompted with the name of the control sequence
being defined by the \cs{read} statement.

\begin{example}
\begin{verbatim}
\read16 to \data
\end{verbatim}
displays a prompt
\begin{verbatim}
\data=
\end{verbatim}
and typing `my name'
in response makes the read statement equivalent
to
\begin{verbatim}
\def\data{my name }
\end{verbatim}
The space at the end of the input derives from the line end;
to prevent this one could write
\begin{verbatim}
{\endlinechar=-1 \global\read16 to \data}
\end{verbatim}
\end{example}

%\spoint Output with \cs{write}
\subsection{Output with \cs{write}}

\TeX's \csterm{write} command
%
\begin{disp}\cs{write}\gr{number}\gr{general text}\end{disp}
%
writes a balanced token list to a file which has been opened
by \cs{openout}, to the log file, or to the terminal.

Write operations to a stream outside 0--15 \ldash or to a
stream that is not associated with an open file \rdash  go to the log file;
if the stream number is positive they 
go to the terminal as well as to the log file.

The token list argument of \cs{write}, defined as
  
\begin{disp}\gr{general text} $\longrightarrow$ \gr{filler}%
       \lb\gr{balanced text}\gr{right brace}\end{disp}
can have an implicit opening brace.
This argument is expanded as if it were the replacement
text of an \cs{edef}, so, for instance, 
any macros and conditionals appearing are expanded.
No commands are executed, however. 
This expansion occurs
at the time of shipping out; see below.
Until that time the argument token list is stored
in a whatsit item on the current list.
See further Chapter~\ref{expand} for
a discussion of expansion during writing.

A control sequence output by \cs{write} (or \cs{message})
is represented with  a trailing space, and using
character number \cs{escapechar}
for the escape character.
The \IniTeX\ default for this is~92,
the code for the backslash.
The trailing space can be prevented by prefixing the control
sequence with \cs{string}.

%\point Whatsits
\section{Whatsits}

\index{whatsits|(}

There is an essential difference 
in execution between input and output:
operations concerning output 
(\cs{openout}, \cs{closeout}, \cs{write})
are  executed \emph{asynchronously}\index{I/O!asynchronous}.
That is, instead of being done immediately they are saved until
the box in which they appear is shipped out 
to the \n{dvi} file.

Writes and the other two output operations are placed
in `whatsit' items on whichever list is currently being built.
The actual operation occurs when the part of the page
that has the item is shipped out to the \n{dvi} file.
This delayed output is made necessary by \TeX's
asynchronous output routine behaviour.
See a worked-out example on page~\pageref{expand:write}.

An \verb.\immediate\write. \ldash or any other \csterm{immediate} output
operation \rdash  is executed on the spot, and
does not place a whatsit item on the current list.

The argument of a \cs{special} command
(see page~\pageref{special}) is also placed in a whatsit.

Whatsit items in leader boxes are ignored.

\index{whatsits|)}

\section{Assorted remarks}

%\spoint Inspecting input
\subsection{Inspecting input}

\TeX\ records the current line number in the current input file
in the \gr{internal integer} parameter \csterm{inputlineno}
(in \TeX3).

If the parameter \csterm{pausing} is positive, \TeX\ shows
every line that is input on the terminal \emph{screen}\index{I/O!screen},
and gives the user the opportunity
to insert commands. These can for instance be \cs{show} commands.
Inserted commands are treated as if they were directly
in the source file: it is for instance not necessary
to prefix them with~`i', as would be necessary when 
\TeX\ pauses for an error.

%\spoint Testing for existence of files
\subsection{Testing for existence of files}

\TeX\ is not the friendliest of systems when you
\howto Test whether a file exists\par
ask it to input a non-existing file. Therefore the following
sequence of commands can be used to prevent trouble\label{ex:eof}:
\begin{verbatim}
\newread\instream \openin\instream= fname.tex
\ifeof\instream \message{File 'fname' does not exist!}
\else \closein\instream \input fname.tex
\fi
\end{verbatim}
Here an input stream is opened with the given file name.
The end-of-file test is also true
if an input stream does not correspond to a physical file,
so if this conditional is not true,
the file exists and an \cs{input} command can safely be given.

%\spoint Timing problems
\subsection{Timing problems}

The synchronization between write operations on the
one hand, and opening/closing operations
of files on the other hand,
can be a crucial point. Auxiliary files, such as are
used by various formats to implement cross-references,
are a good illustration of this.

Suppose that during a run of \TeX\ the auxiliary file is written, and
\howto Input a file that was created in the same run of \TeX\par
at the end of the run it has to be input again for a variety
of purposes (such as seeing whether references have changed).
An \cs{input} command is executed right away, so
the file must have been closed with an \verb=\immediate\closeout=.
However, now it becomes possible that the file is closed 
before all writes to it have been performed.
The following sequence remedies this:
\begin{verbatim}
\par\vfil\penalty -10000 \immediate\closeout\auxfile
\end{verbatim}
The first three commands activate the output routine
in order to close off the last page,
so all writes will indeed have been performed before the
file is closed.

%\spoint \cs{message} versus \cs{immediate}\cs{write}16
\subsection{\cs{message} versus \cs{immediate}\cs{write}16}

Messages to the user can be given using 
\csterm{message}\gr{general text}, which writes to the  terminal.
Messages are appended to one another; 
the line is wrapped when the line
length (a~\TeX\ compile-time constant) has been reached.
In \TeX\ version2,
a~maximum of 1000 characters is written per message;
this is not a compile-time constant, but is hard-wired 
into the \TeX\ program.

Each message given with \verb=\immediate\write=
starts on a new line; the user can force a new line
in the message by including the character with
number~\csterm{newlinechar}. This parameter also works
in \cs{message}. The plain \TeX\ default for \cs{newlinechar} is~\n{-1}; 
the \LaTeX\ default of~\n{10} allows you to write 
\verb+\message{two^^Jlines}+.

%\spoint Write inside a vertical box
\subsection{Write inside a vertical box}

Since a write operation winds up on the vertical list in a whatsit,
issuing one at the start of a \cs{vtop} 
will probably influence the height of that box
(see  Chapter~\ref{boxes}). As an example,
\begin{verbatim}
have the \vtop{\write\terminal{Hello!}\hbox{more text}}
dangling from
\end{verbatim}
will have the~\vtop{\write-1{vtop gezien}\hbox{more text}}~dangling 
from the baseline (and when this book is \TeX ed the
message `Hello!' appears on the screen).

%\spoint Expansion and spaces in \cs{write} and \cs{message}
\subsection{Expansion and spaces in \cs{write} and \cs{message}}

Both \cs{write} and \cs{message} expand their argument
as if it were the replacement text of an \cs{edef}.
Therefore
\begin{verbatim}
\def\a{b}\message{\a}
\end{verbatim}
will
write out~`\n b'.

Unexpandable control sequences are displayed with a trailing
space (and prefixed with the \cs{escapechar}):
\begin{verbatim}
\message{\hbox\vbox!}
\end{verbatim}
will write out
`\verb>\hbox \vbox !>'. Undefined control sequences give an error here.

Expandable control sequences can be written out with some
care:
\begin{verbatim}
\message{\noexpand\ifx}
\message{\string\ifx}
{\let\ifx\relax \message{\ifx}}
\end{verbatim}
all write out `\verb>\ifx>'.

Note, however, that spaces after expandable control sequences
are removed in the input processor, which goes into state~$S$
after a control sequence. Therefore
\begin{verbatim}
\def\a{b}\def\c{d}
\message{\a \c}
\end{verbatim}
writes out `\n{bd}'.
Inserting a space can be done as follows:
\begin{verbatim}
\def\space{ } % in plain TeX
\message{\a\space\c}
\end{verbatim}
displays `\n{b d}'.
Note that
\begin{verbatim}
 
\message{\a{ }\c}
\end{verbatim}
does not work: it displays `\verb=b{ }d='
since braces are unexpandable character tokens.

\index{I/O!file|)}
\endofchapter
\chapter{Allocation}\label{alloc}

\TeX\ has registers of a number of types. For some of these,
explicit commands exist to define a synonym for a certain register;
for all of them macros exist in the plain format
to allocate an unused register. This chapter treats
the synonym and allocation commands, and discusses
some guidelines for macro writers regarding allocation.

\begin{inventory}
\item [\cs{countdef}] 
      Define a synonym for a \cs{count} register.
\item [\cs{dimendef}]
      Define a synonym for a \cs{dimen} register.
\item [\cs{muskipdef}]
      Define a synonym for a \cs{muskip} register.
\item [\cs{skipdef}] 
      Define a synonym for a \cs{skip} register.
\item [\cs{toksdef}] 
      Define a synonym for a \cs{toks} register.
\item [\cs{newbox}]
      Allocate an unused \cs{box} register.
\item [\cs{newcount}]
      Allocate an unused \cs{count} register.
\item [\cs{newdimen}]
      Allocate an unused \cs{dimen} register.
\item [\cs{newfam}]
      Allocate an unused math family.
\item [\cs{newinsert}]
      Allocate an unused insertion class.
\item [\cs{newlanguage}]
      (\TeX3 only)
      Allocate a new language number.
\item [\cs{newmuskip}]
      Allocate an unused \cs{muskip} register.
\item [\cs{newskip}]
      Allocate an unused \cs{skip} register.
\item [\cs{newtoks}]
      Allocate an unused \cs{toks} register.
\item [\cs{newread}]
      Allocate an unused input stream.
\item [\cs{newwrite}]
      Allocate an unused output stream.
\end{inventory}

%\point Allocation commands
\section{Allocation commands}
\index{registers!allocation of|(}

In plain \TeX, \cs{new...} macros are defined for
allocation of registers.
The registers of \TeX\ fall into two classes that are 
allocated in different ways. This is treated below.

The \csterm{newlanguage} macro of plain \TeX\ 
does not allocate any register. Instead it merely assigns
a number, starting from~0.
\TeX\ (version~3) can have at most 256 different
sets of hyphenation patterns.

The \cs{new...} macros of plain \TeX\ are defined to be
\cs{outer} (see Chapter~\ref{macro} for a precise explanation),
which precludes use of the allocation macros in other macros.
Therefore the \LaTeX\ format redefines these macros
without the \cs{outer} prefix.

%\spoint \cs{count}, \cs{dimen}, \cs{skip}, \cs{muskip}, \cs{toks}
\subsection{\cs{count}, \cs{dimen}, \cs{skip}, \cs{muskip}, \cs{toks}}

For these registers there exists a \gr{registerdef} command,
for instance \cs{countdef}, to couple a specific register
to a control sequence:
\begin{Disp}\gr{registerdef}\gr{control 
    sequence}\gr{equals}\gr{8-bit number}\end{Disp}

After the definition
\begin{verbatim}
\countdef\MyCount=42
\end{verbatim}
the allocated register can be used as
\begin{verbatim}
\MyCount=314
\end{verbatim}
or
\begin{verbatim}
\vskip\MyCount\baselineskip
\end{verbatim}

The \gr{registerdef} commands are used in plain \TeX\ macros
\cs{newcount} et cetera that allocate an unused register;
after
\begin{verbatim}
\newcount\MyCount
\end{verbatim}
\cs{MyCount} can be used
exactly as in the above two examples.

%\spoint \cs{box}, \cs{fam}, \cs{write}, \cs{read}, \cs{insert}
\subsection{\cs{box}, \cs{fam}, \cs{write}, \cs{read}, \cs{insert}}

For these registers there exists no  \gr{registerdef} command in \TeX,
so \cs{chardef} is used to allocate box registers
in the corresponding plain \TeX\ macros \cs{newbox}, for instance.

The fact that \cs{chardef} is used implies that the
defined control sequence does not stand for the register itself,
but only for its number. Thus after
\begin{verbatim}
\newbox\MyBox
\end{verbatim}
it is necessary to write
\begin{verbatim}
\box\MyBox
\end{verbatim} 
Leaving out the \cs{box} means that the character
in the current font with number
\cs{MyBox} is typeset. The \cs{chardef} command
is treated further in Chapter~\ref{char}.

\index{registers!allocation of|)}

\section{Ground rules for macro writers}

The \cs{new...} macros of plain \TeX\ have been designed
to form a foundation for macro packages, such that
several of such packages can operate without collisions
in the same run of \TeX. In appendix~B of \TeXbook\
Knuth formulates some ground rules that macro writers should
adhere to.
\begin{enumerate}
\item The \cs{new...} macros do not allocate registers
with numbers~0--9. These can therefore be used as `scratch'
registers. However, as any macro family can use them,
no assumption can be made about the permanency of their
contents. Results that are to be passed from one call to
another should reside in specifically allocated registers.

Note that count registers 0--9 are used for page identification
in the \n{dvi} file (see Chapter~\ref{TeXcomm}), so no global assignments
to these should be made.

\item \cs{count255}, \cs{dimen255}, and \cs{skip255} are
also available. This is because inserts are
allocated from 254 downward  and, together with an insertion box,
a count, dimen, and skip register, 
all with the same number, are allocated.
Since \cs{box255} is used by the output routine 
(see Chapter~\ref{output}),
the count, dimen, and skip with number~255 are freely available.

\item Assignments to scratch registers~0, 2, 4, 6, 8, and~255
should be local; assignments to registers~1, 3, 5, 7,~9
should be \cs{global} (with the exception of the \cs{count}
registers). This guideline prevents `save
stack build-up' (see Chapter~\ref{error}).

\item Any register can be used inside a group, as \TeX's
grouping mechanism will restore its value outside
the group. There are two conditions on this use of
a register:
no global assignments should be made to it, and 
it must not be possible that other macros may be
activated in that group that perform global assignments
to that register.

\item Registers that are used over longer periods of time,
or that have to survive in between calls of different
macros, should be allocated by \cs{new...}.
\end{enumerate}


%%%% end of input file [alloc]

%\InputFile:run
%%%% this is input file [run]
%\subject[run] Running \TeX
\endofchapter
\chapter{Running \TeX}\label{run}

This chapter treats the run modes of \TeX, and some
other commands associated with the job being processed.

\label{cschap:everyjob}\label{cschap:jobname}\label{cschap:end}\label{cschap:bye}\label{cschap:pausing}\label{cschap:errorstopmode}\label{cschap:scrollmode}\label{cschap:nonstopmode}\label{cschap:batchmode}
\begin{inventory}
\item [\cs{everyjob}] 
      Token list that is inserted at the start of each new job.

\item [\cs{jobname}] 
      Name of the main \TeX\ file being processed.

\item [\cs{end}] 
      Command to finish off a run of \TeX.

\item [\cs{bye}]
      Plain \TeX\ macro to force the final output.

\item [\cs{pausing}] 
      Specify that \TeX\ should pause after each line that is 
      read from a file.

\item [\cs{errorstopmode}]
      \TeX\ will ask for user input on the occurrence of an error.

\item [\cs{scrollmode}] 
      \TeX\ fixes errors itself,
      but will ask the user for missing files.

\item [\cs{nonstopmode}] 
      \TeX\ fixes errors itself, 
      and performs an emergency stop on serious errors 
      such as missing input files.

\item [\cs{batchmode}] 
      \TeX\ fixes errors itself 
      and performs an emergency stop on serious errors 
      such as missing input files,
      but no terminal output is generated.

\end{inventory}

%\point Jobs
\section{Jobs}
\index{job|(}

\TeX\ associates with each run a name for the file
being processed: the \csterm{jobname}. If \TeX\ is run
interactively
\ldash meaning that it has been invoked without a file argument,
and the user types commands \rdash
the jobname is \n{texput}.

The \cs{jobname} can be used to generate
the names of auxiliary files to be read or
written during the run. For instance, for a file \n{story.tex}
the \cs{jobname} is \n{story}, and writing
\begin{verbatim}
\openout\Auxiliary=\jobname.aux
\openout\TableOfContents=\jobname.toc
\end{verbatim}
will create the files \n{story.aux} and \n{story.toc}.

%\spoint Start of the job
\subsection{Start of the job}

\TeX\ starts each job by inserting the \csterm{everyjob} token
list into the command stream.
Setting this variable during a run of \TeX\ has no use,
but a format can use it to identify itself to the user.
If a
format fills the token list, the commands therein are automatically
executed when \TeX\ is run using that format.

%\spoint End of the job
\subsection{End of the job}

A \TeX\ job is terminated by the \csterm{end} command. This
may involve first forcing the output routine to process any
remaining material (see Chapter~\ref{page:break}).
If the end of job occurs inside a group
\TeX\ will give a diagnostic
message. The \cs{end} command is not allowed in internal
vertical mode, because this would be inside a vertical box.

Usually some sugar coating of the \cs{end} command is necessary.
For instance the plain \TeX\ macro \csterm{bye} is defined
as
\begin{verbatim}
\def\bye{\par\vfill\supereject\end}
\end{verbatim}
where the \cs{supereject} takes care of any leftover insertions.

%\spoint The log file
\subsection{The log file}

For each run \TeX\ creates a \indexterm{log file}. Usually this will be
a file with as name the value of \cs{jobname}, and the
extension \n{.log}. Other extensions such as \n{.lis}
are used by some implementations.
This log file contains all information that
is displayed on the screen during the run of \TeX, but
it will display some information more elaborately, and it
can contain statistics that are usually not displayed on
the screen. If the parameter \cs{tracingonline}
has a positive value, all the log file information will be
shown on the screen.

Overfull and underfull boxes are reported on the terminal
screen, and they are dumped using the parameters
\cs{showboxdepth} and \cs{showboxbreadth} in the log file
(see Chapter~\ref{trace}). These parameters are also used
for box dumps caused by the \cs{showbox} command, and
for the dump of boxes written by \cs{shipout} 
if \cs{tracingoutput} is set to a positive value.

Statistics generated by commands such as \cs{tracingparagraphs}
will be written to the log file; if \cs{tracingonline} is positive
they will also be shown on the screen.

Output operations to a stream that is not open, or to a
stream with a number that is not in the range 0--15,
go to the log file. If the stream number is positive,
they also go to the terminal.

\index{job|)}

\section{Run modes}
\index{run modes|(}

By default, \TeX\ goes into \cs{errorstopmode} if an error occurs:
\csterm{errorstopmode}%
it stops and asks for input from the user. Some implementations 
 have a way of forcing \TeX\ into errorstopmode
when the user interrupts \TeX, so that
the internal state of \TeX\ can be inspected (and altered). 
See page~\pageref{interaction} for ways to switch the run
mode when \TeX\ has been interrupted.

Often, \TeX\ can
fix an error itself if the user asks \TeX\ just to continue
(usually by hitting the return key),
but sometimes (for instance in alignments)
it may take a while before \TeX\ is on the
right track again (and sometimes it never is).
In such cases the user may want to
turn on \csterm{scrollmode},
which instructs \TeX\ to fix as best it can any
occurring error without confirmation from the user.
This is usually done by typing `s' when \TeX\ asks
for input.

In \cs{scrollmode}, \TeX\ also does not ask for input
after \cs{show...} commands.
\alt
However, some errors, such as a file that could not be
found for \cs{input}, are not so easily remedied, so
the user will still be asked for input.

With \csterm{nonstopmode} \TeX\ will scroll through errors and,
in the case of the kind of error that cannot be recovered from,
it will make an emergency stop, aborting the run.
Also \TeX\ will abort the run if a \cs{read} is attempted
from the terminal.
The \csterm{batchmode} differs only from nonstopmode in that
it gives messages only to the log file, not to the terminal.

\index{run modes|)}
\endofchapter
\chapter{\TeX\ and the Outside World}\label{TeXcomm}

This
chapter treats those commands that bear relevance to
\n{dvi} files and formats. It gives some global information
about \IniTeX, font and format files,
Computer Modern typefaces, and \web.

\label{cschap:dump}\label{cschap:special}\label{cschap:mag}\label{cschap:year}\label{cschap:month}\label{cschap:day}\label{cschap:time}\label{cschap:fmtname}\label{cschap:fmtversion}
\begin{inventory}
\item [\cs{dump}] 
      Dump a format file; possible only in \IniTeX, 
      not allowed inside a group.

\item [\cs{special}] 
      Write a \gr{balanced text} to the \n{dvi} file.

\item [\cs{mag}] 
      1000 times the magnification of the document.

\item [\cs{year}] 
      The year of the current job.

\item [\cs{month}] 
      The month of the current job.

\item [\cs{day}] 
      The day of the current job.

\item [\cs{time}]
      Number of minutes after midnight that the current job started.

\item [\cs{fmtname}] 
      Macro containing the name of the format dumped.

\item [\cs{fmtversion}] 
      Macro containing the version of the format dumped.

\end{inventory}


%\point \TeX, \IniTeX, \VirTeX
\section{\TeX, \IniTeX, \VirTeX}

In the terminology established in {\italic \TeX: the Program},
\cite{Knuth:TeXprogram},
\thecstoidx TeX\par
\thecstoidx IniTeX\par
\thecstoidx VirTeX\par
\TeX\ programs come in three flavours.
\IniTeX\ is a version of \TeX\ that can generate formats;
\VirTeX\ is a production version without preloaded format,
and \TeX\ is a production version with 
preloaded (plain) format. Unfortunately, this terminology is
not adhered to in general. A~lot of systems do not use preloaded
formats (the procedure for making them may be impossible on
some operating systems),
and call the `virgin \TeX' simply \TeX.
This manual also follows that convention.

%\spoint Formats: loading
\subsection{Formats: loading}

A \indexterm{format file}
(usually with extension~\n{.fmt})
is a compact dump of \TeX's internal structures.
Loading a format file takes a considerably shorter time than
would be needed for
loading the font information and the macros that
constitute the format.

Both \TeX\ and \IniTeX\ can load a format; the user specifies
this by putting the name on the command line 
\begin{verbatim}
% tex &plain 
\end{verbatim}
or at the \n{**} prompt
\begin{verbatim}
% tex
This is TeX. Version ....
** &plain 
\end{verbatim}
preceded by an ampersand (for UNIX, this should be \verb>\&> on
the command line). An input file name can follow the
format name in both places.

\IniTeX\ does not need a format,
but if no format is specified for (Vir)\TeX, it will try to
load the plain format, and halt if that cannot be found.

%\spoint Formats: dumping
\subsection{Formats: dumping}

\IniTeX\ is the only version of \TeX\ 
that can dump a format, since it is
the only version of \TeX\ that has
the command~\csterm{dump},
which causes the internal structures
to be dumped as a format.
It is also the only version of \TeX\ that has the command
\cs{patterns}, which
is needed to specify a list of hyphenation
patterns.

Dumping is not allowed inside a group, that is
\begin{verbatim}
{ ... \dump }
\end{verbatim}
is not allowed. This restriction
prevents difficulties with \TeX's save stack.
After the \cs{dump} command \TeX\ gives an elaborate listing of
its internal state, and of the font names associated with
fonts that have been loaded and ends the job.

An interesting possibility arises from the fact that
\IniTeX\ can both load and dump a format.
Suppose you have written a set of macros that build
on top of plain \TeX, \n{superplain.tex}.
You could then call
\begin{verbatim}
% initex &plain superplain
*\dump
\end{verbatim}
and get a format file \n{superplain.fmt} that
has all of plain, and all of your macros.

%\spoint Formats: preloading
\subsection{Formats: preloading}

On some systems it is possible to interrupt a running program,
and save its `core image' such that this can be started as
an independent program. 
The executable made from the
core image of a \TeX\ program interrupted after it has loaded 
a format is called a \TeX\ program with preloaded format.
The idea behind preloaded formats is
that interrupting \TeX\ after it has loaded a format, and making
this program available to the user, 
saves in each run the time for loading the format.
In the good old days when computers were quite a bit slower
this procedure made sense.
Nowadays, it does not seem so necessary.
Besides, dumping a core image may not always be possible.

%\spoint The knowledge of \IniTeX
\subsection{The knowledge of \IniTeX}

If no format has been loaded, \IniTeX\ knows very little.
For instance, it has no open/close group characters.
However, it can not be completely devoid of knowledge
lest there be no way to define anything.

Here is the extent of its knowledge.
\begin{itemize} \mathsurround=1.5pt
%\flushright:no
\item \verb>\catcode`\\=0>, \verb>\escapechar=`\\>
      (see page~\pageref{ini:esc}).
\item \verb>\catcode`\^^M=5>, \verb>\endlinechar=`\^^M>
      (see page~\pageref{ini:eol}).
\item \verb>\catcode`\ =10>
      (see page~\pageref{ini:sp}).
\item \verb>\catcode`\%=14>
      (see page~\pageref{ini:comm}).
\item \verb>\catcode`\^^?=15>
      (see page~\pageref{ini:invalid}).
\item \cs{catcode}$x$\n{=11} for $x={}$\n{`a..`z,`A..`Z}
      (see page~\pageref{ini:let}).
\item \cs{catcode}$x$\n{=12} for all other character codes\nl
      (see page~\pageref{ini:other}).
\item \cs{sfcode}$x$=\n{999} for $x={}$\n{`A..`Z},
      \cs{sfcode}$x$\n{=1000} for all other characters
      (see page~\pageref{ini:sf}).
\item \verb>\lccode`a..`z,`A..`Z=`a..`z>, \verb>\uccode`a..`z,`A..`Z=`A..`Z>,
      \cs{lccode}$x$\n{=0}, \cs{uccode}$x$\n{=0} for all other characters
      (see page~\pageref{ini:uclc}).
\item \verb>\delcode`.=0>, \cs{delcode}$x$\n{=-1} for all other characters
      (see page~\pageref{ini:del}).
\item \cs{mathcode}$x$\n{="!7100}${}+x$ for all lowercase and uppercase
      letters, \cs{mathcode}$x$\n{="!7000}${}+x$ for all digits,
      \cs{mathcode}$x$\n=$x$ for all other characters
      (see page~\pageref{ini:fam}).
\item \cs{tolerance=10000}, \cs{mag=1000},
      \cs{maxdeadcycles=25}.
\end{itemize}

%\spoint Memory sizes of \TeX\ and \IniTeX
\subsection{Memory sizes of \TeX\ and \IniTeX}

The main memory size of \TeX\ and \IniTeX\ is controlled by
four constants in the source code:
\n{mem\_bot}, \n{mem\_top}, \n{mem\_min}, and~\n{mem\_max}.
For Ini\TeX's memory \n{mem\_bot${}={}$mem\_min} 
and \n{mem\_top${}={}$mem\_max};
for \TeX\ \n{mem\_bot} and \n{mem\_top} record the main memory
size of the Ini\TeX\ used to dump the format.
Thus versions of \TeX\ and \IniTeX\ have to be adapted
to each other in this    respect.

\TeX's own main memory can be bigger than that of the
corresponding \IniTeX: in general
\n{mem\_min${}\leq{}$mem\_bot} and \n{mem\_top${}\leq{}$mem\_max}.

For \IniTeX\ a smaller main memory can suffice, 
as this program is typically
not meant to do real typesetting. 
There may even be a real need for the main memory
to be smaller, because \IniTeX\ needs a lot of auxiliary
storage for initialization and for building the
hyphenation table.


%\point More about formats
\section{More about formats}

%\spoint Compatibility
\subsection{Compatibility}

\TeX\ has a curious error message: `Fatal format error: I'm stymied',
which is given if \TeX\ tries to load a format that was made
with an incompatible version of \IniTeX. See the point
above about memory sizes, and Chapter~\ref{error} for
the hash size (parameters \n{hash\_size} and \n{hash\_prime})
and the hyphenation exception dictionary (parameter \n{hyph\_size}).

%\spoint Preloaded fonts
\subsection{Preloaded fonts}

During a run of \TeX\ the only information needed about fonts
is the data that is found in the \n{tfm} files (see below).
Since a run of \TeX, especially if the input contains math material,
can easily access 30--40 fonts, the disk access for
all the \n{tfm} files can become significant.
Therefore the plain format and \LaTeX\ load these 
metrics files in \IniTeX. A~\TeX\ version using such a format
does not need to load any \n{tfm} files.

On the other hand, if a format has the possibility of accessing
a range of typefaces, it may be advantageous to have metrics
files loaded on demand during the actual run of \TeX.

%\spoint The plain format
\subsection{The plain format}

The first format written for \TeX, and the basis for all
later ones,
is the plain format, described in \TeXbook.
It is a mixture of 
\begin{itemize}
\item definitions and macros one simply cannot live without
such as the initial \cs{catcode} assignments, 
all of the math delimiter definitions,
and the \cs{new...} macros;
\item constructs that are useful, but for which \LaTeX\ 
and other packages use
a different implementation, such as the tabbing environment; and
\item some macros that are insufficient for any but the
simplest applications: \cs{item} and \cs{beginsection}
are in this category.
\end{itemize}

It is the first category which Knuth meant to serve as a
foundation for future macro packages, so that they
can live peacefully together (see  Chapter~\ref{alloc}).
This idea is reflected in the fact that the name `plain'
is not capitalized: it is the basic set of macros.

%\spoint The \LaTeX\ format
\subsection{The \LaTeX\ format}

The \LaTeX\ format\thecstoidx LaTeX\par,
written by  Leslie Lamport of Digital Equipment Corporation
and described in~\cite{Lamport:LaTeX},
was released around 1985.
The \LaTeX\ format, using its own version
of \n{plain.tex} (called \n{lplain.tex}),
is not compatible with plain \TeX;
a~number of plain macros are not available. Still, it contains
large parts of the plain format (even when they overlap with
its own constructs).

\LaTeX\ is a powerful format with facilities such as
marginal notes, floating objects, cross referencing,
and automatic table of contents generation.
Its main drawback is that the `style files' which
define the actual layout are quite hard to write
(although \LaTeX\ is in the process of a major revision,
in which this problem will be tackled;
see \cite{Frank} and~\cite{Frank2}).
As a result,
people have had at their disposal mostly the styles
written by Leslie Lamport, the layout of which is
rather idiosyncratic. See~\cite{BEP} for a successful
attempt to replace these styles.

%\spoint Mathematical formats
\subsection{Mathematical formats}

There are two formats with extensive facilities for
mathematics typesetting:
\AmsTeX~\cite{Ams}
(which originated at the American Mathematical Society)
and \LamsTeX~\cite{Lams}.
The first of these includes more facilities than plain \TeX\
or \LaTeX\ for typesetting mathematics, but it lacks
features such as automatic numbering and cross-referencing,
available in \LaTeX, for instance. \LamsTeX, then, is the
synthesis of \AmsTeX\ and \LaTeX. Also it includes
still more features for mathematics, such as complicated
tables and commutative diagrams.

%\spoint Other formats
\subsection{Other formats}

Other formats than the above exist:
for instance, \n{Phyzzx}~\cite{Phyzzx}, \n{TeXsis}~\cite{TeXsis}, 
Macro \TeX~\cite{Amy}, \n{eplain}~\cite{Berry},
and \n{\TeX T1}~\cite{TeXT1}.
Typically, such formats provide the facilities of \LaTeX, but
try to be more easily adaptable by the user.
Also, in general they
have been written with the intention of being an
add-on product to the plain format.

This book was, in its incarnation published by Addison-Wesley,
also written in an `other format':
the \indexterm{Lollipop}  format.
This format does not contain user macros, but the
tools with which a style designer can program them; see~\cite{EL}.
The current version of this book is written in \LaTeX.

%\point The \n{dvi} file
\section{The \n{dvi} file}

The \n{dvi} file (this term stands for `device independent')
\term \n{dvi} file\par
contains the output of a \TeX\ run: it
contains compactly dumped representations of boxes that
\mdqon
have been sent there by \cs{shipout}""\gr{box}. The act
\mdqoff
of shipping out usually occurs inside the output routine,
but this is not necessarily so.

%\spoint The \n{dvi} file format
\subsection{The \n{dvi} file format}

A \n{dvi} file is a byte-oriented file,
consisting of a preamble, a postamble,
and a list of pages.

Access for subsequent software to a completed \n{dvi} file
is strictly sequential in nature:
the pages are stored as a backwards linked list. This
means that only two ways of accessing are possible:
\begin{itemize} \item given the start of a page, the next can be
found by reading until an end-of-page code is encountered, and
\item starting at the end of the file pages can be read
backwards at higher speed, as each beginning-of-page code
contains the byte position of the previous one.\end{itemize}

The preamble and postamble contain
\begin{itemize}\item the magnification of the document (see below),
\item the unit of measurement used for the document, and
\item possibly a comment string.\end{itemize}
The postamble contains in addition a list of the font definitions
that appear on the pages of the file.

Neither the preamble nor the postamble 
of the file contains a
table of byte positions of pages.
The full definition of the \n{dvi} file format can be found
in~\cite{Knuth:TeXprogram}.

%\spoint Page identification
\subsection{Page identification}

Whenever a \cs{shipout} occurs, \TeX\ also writes the
values of counters  0--9 to the \n{dvi} file and the terminal.
Ordinarily, only counter~0, the page number, is used, and the
other counters are zero. Those zeros are not output to the
terminal. The other counters can be used to indicate further
structure in the document. Log output shows the non-zero
counters and the zero counters in between.


%\spoint Magnification 
\subsection{Magnification }

The \indexterm{magnification}
of a document can be indicated by the \gr{integer}
\csterm{mag}, which specifies 1000 times the magnification
ratio.

The \n{dvi} file contains the value of \cs{mag} for the
document in its preamble and postamble. 
If no {\tt true} dimensions are used
the \n{dvi} file will look the same as when no magnification
would have been used, except for the \cs{mag} value in the
preamble and the postamble.

Whenever a {\tt true} dimension is used it is divided
by the value of \cs{mag}, so that the final output will have
the dimension as prescribed by the user.
The \cs{mag} parameter cannot be changed after a
\n{true} dimension has been used, or after the first
page has been shipped to the \n{dvi} file.

Plain \TeX\ has the \csterm{magnification} macro for
globally sizing the document, without changing
 the physical size of the page:
\begin{verbatim}
\def\magnification{\afterassignment\m@g\count@}
\def\m@g{\mag\count@
  \hsize6.5truein\vsize8.9truein\dimen\footins8truein}
\end{verbatim}
The explanation for this is
as follows: the command \cs{m@g} is saved with an \cs{afterassignment}
command, and the magnification value (which is 1000 times the
actual magnification factor) is assigned to \cs{count@}.
After this assignment, the macro \cs{m@g} assigns
the magnification value to \cs{mag}, and the horizontal
and vertical size are reset to 
their original values {\tt 6.5truein} and {\tt 8.9truein}.
The \cs{footins} is also reset.

%\point[special] Specials
\section{Specials}
\label{special}

\mdqon
\TeX\ is to a large degree machine"-independent, but it still needs
\mdqoff
a hook for machine-dependent extensions. This is 
done through \indexterm{specials}. The
\csterm{special} command writes a \gr{balanced text}
to the \n{dvi} file which \TeX\ does not interpret like other token lists:
it assumes that the printer driver knows what to do with it.
The \cs{special} command is not supposed to change the
$x$ and $y$ position on the page, so that the implementation
of \TeX\ remains independent of the actual 
\indexterm{device driver}
that handles the \cs{special}.

The most popular application of specials is probably the
inclusion of graphic material, written in some
page description language, such as \indexterm{PostScript}.
The size of the graphics can usually be determined from
the file containing it (in the case of encapsulated
PostScript through
the `bounding box' data), so \TeX\ can leave space for
such material.

%\point Time
\section{Time}

\TeX\ has four parameters, \csterm{year}, \csterm{month}, \csterm{day}, and
\csterm{time}, that tell
the \indexterm{time} and \indexterm{date}
when the current job started.
After this, the parameters are not updated.
The user can change them without this having any effect.

All four parameters are integers; the \cs{time} parameter
gives the number of minutes since midnight that the current
job started.

%\point Fonts
\section{Fonts}

Font information is split in the \TeX\ system into
the metric information (how high, wide, and deep is a character),
and the actual description of the characters in a font.
\TeX, the formatter, needs only the metric information;
printer drivers and screen previewers need the character
descriptions. With this approach it is for instance possible
for \TeX\ to use with relative ease the resident fonts of
a printer.

%\spoint Font metrics
\subsection{Font metrics}

The metric information of \TeX's fonts is stored in \n{tfm}
files, which stands for `\TeX\ \indexterm{font metrics}' files.
Metrics files contain the following information
(see \cite{Knuth:TeXprogram} for the full definition):
\begin{itemize}\item the design size of a font;
\item the values for the \cs{fontdimen} parameters
(see Chapter~\ref{font});
\item the height, depth, width, and italic correction
     of individual characters;
\item kerning tables;
\item ligature tables;
\item information regarding successors and extensions
of math characters (see Chapter~\ref{mathchar}).
\end{itemize}
Metrics files use a packed format, but they can be converted
to and from a readable format by the auxiliary programs
\n{tftopl} and \n{pltotf} (see~\cite{K:Fuchs}). 
Here \n{pl} stands for `property list',
a term deriving from the programming language Lisp.
Files in \n{pl} format are just text, so they can easily be edited;
after conversion
they can then again be used as \n{tfm} files.

%\spoint[virtual:fonts] Virtual fonts
\subsection{Virtual fonts}
\label{virtual:fonts}

With 
\indexterm{virtual fonts}
(see~\cite{K:virt}) it is possible that
what looks like one font to \TeX\ resides in more than
one physical font file. Also, virtual fonts can be used
to change in effect the internal organization of font files.

For  \TeX\ itself, the
presence of virtual fonts makes no difference: everything
is still based on \n{tfm} files containing metric
information. However, the screen or printer driver that displays
the resulting \n{dvi} file on the screen or on a printer
will search for files with extension \n{.vf} to determine
how characters are to be interpreted.
The \n{vf} file can, for instance, instruct the driver
to interpret a character as a certain position
in a certain font file, to interpret a character as more
than one position (a~way of forming accented characters),
or to include \cs{special} information (for
instance to set gray levels).

Readable variants of \n{vf} files have extension \n{vpl},
analogous to the \n{pl} files for the \n{tfm} files; see above.
Conversion between \n{vf} and \n{vpl} files can be
performed with the \n{vftovp} and \n{vptovf} programs.

However, because virtual fonts are a matter for
\indexterm{device drivers}, no more details will be given in this book.

%\spoint Font files
\subsection{Font files}
\index{font files|(}

Character descriptions are stored in three types of files.
\begin{description} \item [gf]
    Generic Font files.
This is the file type that the Metafont program generates.
There are not many previewers or printer drivers that use
this type of file directly.
\item [pxl]
    Pixel files. The \n{pxl} format is a pure bitmap format.
Thus it is easy to generate \n{pxl} files from, for instance,
scanner images.

This format should be superseded by the \n{pk} format.
Pixel files can become rather big,
as their size grows quadratically in the size of the characters.

\item [pk]
    Packed files. Pixel files can be packed by a form of run-length
encoding: instead of storing the complete bitmap only the
starting positions and lengths of `runs' of black and white
pixels are stored. This makes the size of \n{pk} files
approximately linear in the size of the characters.
However, a previewer or printer driver using a packed font file
has to unpack it before it is   able to use it.
\end{description}

The following conversion programs exist:
\n{gftopxl}, \n{gftopk}, \n{pktopxl}, \n{pxltopk}.

\index{font files|)}

\subsection{Computer Modern}

The only family of typefaces that comes with \TeX\ 
in the standard distribution is the 
\indexterm{Computer Modern}
family of typefaces. This is an adaptation (using the terminology
of~\cite{S}) by Donald Knuth of the Monotype Modern~8A typeface
that was used for the first volume of his {\italic Art of Computer
Programming\/} series. The `modern faces' all derive from the
types that were cut between 1780 and 1800 by Firmin Didot in
France, Giambattista Bodoni in Italy, and Justus Erich Walbaum
in Germany. After the first two, these types are also called
`Didone' types. This name was coined in the Vox classification
of types \cite{Vox}. Ultimately, the inspiration for the Didone
types is the `Romain du Roi', the type that was designed by
Nicolas Jaugeon around 1692 for the French Imprimerie Royale.

Didone types are characterized by a strong vertical orientation,
and thin hairlines. The vertical accent is strengthened by the
fact that the insides of curves are flattened.
The result is a clear and brilliant page, provided that the
printing is done carefully and on good quality paper.
\message{Reference format}
However, they are quite vulnerable; \cite{Up}
compares them to the distinguished but fragile furniture
from the same period, saying one is afraid to use either,
`for both seem in danger of breaking in pieces'.
With the current proliferation of low resolution (around
300 dot per inch) printers, the Computer Modern is
a somewhat unfortunate choice.

Recently, Donald Knuth has developed 
a new typeface (or rather, a subfamily of typefaces)
by changing parameters
in the Computer Modern family. The result is a so-called
`Egyptian' typeface: Computer Concrete \cite{K:cc}.
The name derives from the
fact that it was intended primarily for the book {\italic Concrete
Mathematics}. Egyptian typefaces (they fall under the `M\'ecanes'
in the Vox classification, meaning constructed,
not derived from written letters) have a very uniform line width
and square serifs. They do not have anything to do with Egypt;
such types happened to be popular in the first half of the nineteenth
century when Egyptology was developing and popular.

%\point \TeX\ and web
\section{\TeX\ and web}

The \TeX\ program is written in \web, a programming language
\thecstoidx web\par
that can be considered as a subset of \indexterm{Pascal}, augmented with
a preprocessor.

\TeX\ makes no use of some features of Pascal, in order to
facilitate porting to Pascal systems other than the one
it was originally designed for, and even to enable automatic
translation to other programming languages such as~C.
For instance, it does not use the Pascal \n{With} construct.
Also, procedures do not have output parameters; apart from
writing to global variables, the only way
values are returned is through
\n{Function} values.

Actually, \web\ is more than a superset of a subset of Pascal
(and to be more precise, it can also be used with other
programming languages);
it is a `system of structured documentation'. This means that
the \web\ programmer writes pieces of program code,
interspersed with their documentation, in one file. 
This idea of `literate programming' was
introduced  in~\cite{K:literate};
for more information, see~\cite{Sewell}.

Two auxiliary programs,
Tangle and Weave, can then be used to strip the documentation
and convert \web\ into regular Pascal, or to convert the
\web\ file into a \TeX\ file that will typeset the program 
and documentation.

Portability of \web\ programs is achieved by the `change file'
mechanism. A~change file is a list of changes to be made to
the \web\ file; a~bit like a stream editor script.
These changes can comprise both adaptations of the \web\ file
to the particular Pascal compiler that will be used, and
bug fixes to \TeX. Thus the \n{TeX.web} file need never be edited.


%\point The \TeX\ Users Group
\section{The \TeX\ Users Group}

\TeX\ users have joined into several users groups
over the last decade. Many national or language
users groups exist, and a lot of them publish newsletters.
The oldest of all \TeX\ users groups is simply called
that: the \TeX\ Users Group, or \indexterm{TUG},
and its journal is called \indexterm{TUGboat}.
You can reach them at
\begin{disp} \TeX\ Users Group\nl P.O. Box 2311\nl
     Portland, OR 97208-2311, USA
\end{disp}
 or electronically at \n{office@tug.org} on the Internet.


%%%% end of input file [run]

%\InputFile:trace
%%%% this is input file [trace]
%\subject[trace]  Tracing
\endofchapter
\chapter{Tracing}\label{trace}
\index{tracing|(}

\TeX's workings are often quite different from what
the programmer expected, so there are ways to discover how \TeX\
arrived at the result it did. The \cs{tracing...} 
commands all write
\indexterm{statistics}
information of a certain kind to the log file 
(and to the terminal if \cs{tracingonline} is positive),
and a number of \cs{show...} commands can be used to ask the
current status or value of various items of \TeX.

In the following list, only \cs{show} and \cs{showthe}
display their output on the terminal by default,
other \cs{show...} and \cs{tracing...} commands
write to the log file. They will write in addition to
the terminal if \cs{tracingonline} is positive.

\label{cschap:meaning}\label{cschap:show}\label{cschap:showthe}\label{cschap:showbox}\label{cschap:tracingoutput}\label{cschap:showboxdepth}\label{cschap:showboxbreadth}\label{cschap:tracingparagraphs}\label{cschap:tracingpages}\label{cschap:tracinglostchars}\label{cschap:tracingrestores}\label{cschap:tracingstats}\label{cschap:tracingcommands}\label{cschap:tracingmacros}\label{cschap:tracingonline}
\begin{inventory}
\item [\cs{meaning}] 
      Give the meaning of a control sequence as a string of characters.

\item [\cs{show}] 
      Display the meaning of a control sequence.

\item [\cs{showthe}] 
      Display the result of prefixing a token with \cs{the}.

\item [\cs{showbox}] 
      Display the contents of a box.

\item [\cs{showlists}] 
      Display
      the contents of the partial lists
      currently built in all modes.
      This is treated on
      page~\pageref{showlists}.

\item [\cs{tracingcommands}]
      If this is~1 \TeX\ displays primitive commands executed; 
      if this is 2~or more the outcome of conditionals is also recorded.

\item [\cs{tracingmacros}]
      If this is~1, \TeX\ shows expansion of macros
      that are performed and the actual values of the arguments;
      if this is 2~or more \gr{token parameter}s such as
      \cs{output} and \cs{everypar} are also traced.

\item [\cs{tracingoutput}] 
      If this is positive, the log file shows a dump of boxes 
      that are shipped to the \n{dvi} file.

\item [\cs{showboxdepth}]  
      The number of levels of box dump that are shown when 
      boxes are displayed.

\item [\cs{showboxbreadth}] 
      Number of successive elements on each level that are shown when 
      boxes are displayed.

\item [\cs{tracingonline}]
      If this parameter is positive, \TeX\ will write trace      
      information to the terminal in addition to the log file.

\item [\cs{tracingparagraphs}] 
      If this parameter is positive, \TeX\ generates      
      a trace of the line breaking algorithm.

\item [\cs{tracingpages}]
      If this parameter is positive, \TeX\ generates      
      a trace of the page breaking algorithm.

\item [\cs{tracinglostchars}]
      If this parameter is positive, \TeX\ gives      
      diagnostic messages whenever a character is accessed that      
      is not present in a font.
      Plain default:~1.

\item [\cs{tracingrestores}]
      If this parameter is positive, \TeX\ will report      
      all values that are restored when a group ends.

    \item [\cs{tracingstats}] 
      If this parameter is~1, \TeX\ reports at the      
      end of the job the usage of various internal arrays;
      if it is~2, the memory demands are given whenever
      a page is shipped out.

    \item [\csterm{tracingassigns}] \eTeX
      If this parameter is positive, all assignments are traced.

    \item [\csterm{tracinggroups}] \eTeX
      If this parameter is positive, group openings and closings are
      traced, including a report of the type of group.

    \item [\csterm{tracingifs}] \eTeX
      If this parameter is positive, conditionals are traced, with
      starting line and nesting level.
      
    \item [\csterm{tracingscantokens}] \eTeX
      This traces invocations of \cs{scantokens}.

    \item [\csterm{showifs}] \eTeX
      When issued, this command displays all active conditionals.

\end{inventory}

\section{Meaning and content: \protect\cs{show}, \protect\cs{showthe}, \protect\cs{meaning}}

The meaning of control sequences, and the contents of those
that represent internal quantities, can be obtained
by the primitive commands \cs{show}, \cs{showthe},
and~\cs{meaning}.

The control sequences \cs{show} and \cs{meaning} are similar:
\alt
the former will give
\csterm{show}%
\csterm{meaning}%
output to the log file and the terminal, whereas the latter
will produce the same tokens, but they are placed in \TeX's
input stream.

The meaning of a primitive command of \TeX\ is that command itself:
\begin{verbatim}
\show\baselineskip
\end{verbatim}
gives
\begin{verbatim}
\baselineskip=\baselineskip
\end{verbatim}
The meaning of a defined  quantity is its definition:
\begin{verbatim}
\show\pageno
\end{verbatim}
gives
\begin{verbatim}
\pageno=\count0
\end{verbatim}
The meaning of a macro is its parameter text and replacement text:
\begin{verbatim}
\def\foo#1?#2\par{\set{#1!}\set{#2?}}
\show\foo
\end{verbatim}
gives
\begin{verbatim}
\foo=macro:
#1?#2\par ->\set {#1!}\set {#2?}
\end{verbatim}
For macros without parameters the part before the arrow
(the parameter text) is empty.

The \csterm{showthe} command will display on the log file and terminal 
the tokens that \cs{the} produces. 
After \cs{show}, \cs{showthe}, \cs{showbox}, and \cs{showlists}
\TeX\ asks the user for input; this can be prevented
by specifying \cs{scrollmode}.
Characters generated
by \cs{meaning} and \cs{the} have category~12\index{category!12},
except for spaces
(see page~\pageref{cat12});
the value of \cs{escapechar} is used when control sequences
are represented.

%\point Show boxes: \cs{showbox}, \cs{tracingoutput}
\section{Show boxes: \protect\cs{showbox}, \protect\cs{tracingoutput}}

If \cs{tracingoutput} is positive the log file will
\csterm{tracingoutput}%
\csterm{showbox}%
receive a dumped representation of all boxes that are
written to the \n{dvi} file with \cs{shipout}.
The same representation is used
by the command \cs{showbox}\gr{8-bit number}.

In the first case \TeX\ will report `Completed box being shipped out';
in the second case it will enter \cs{errorstopmode}, and
tell the user `OK. (see the transcript file)'.
If \cs{tracingonline} is positive, the box is also displayed
on the terminal; if \cs{scrollmode} has been specified,
\TeX\ does not stop for input.

The upper bound on the
number of nested boxes that is dumped is \cs{showboxdepth};
\csterm{showboxdepth}%
\csterm{showboxbreadth}%
each time a level is visited at most \cs{showboxbreadth}
items are shown, the remainder of the list is summarized
with~\n{etc.}
For each box its height, depth, and width
are indicated in that order, and for characters it is
stated from what font they were taken. 

\begin{example} After
\begin{verbatim}
\font\tenroman=cmr10 \tenroman
\setbox0=\hbox{g}
\showbox0
\end{verbatim} 
the log file will show
\begin{verbatim}
\hbox(4.30554+1.94444)x5.00002
.\tenroman g
\end{verbatim}
indicating that the box was \n{4.30554pt} high,
\n{1.94444pt} deep, and \n{5.00002pt} wide, and that it contained
a character `g' from the font \cs{tenroman}. 
Note that the fifth decimal of all sizes may be rounded
because \TeX\ works with multiples of $2^{-16}$\n{pt}.
\message{ifmath: scriptfont fam0 fill!}
\end{example}

The next example has nested boxes, 
\begin{verbatim}
\vbox{\hbox{g}\hbox{o}}
\end{verbatim}
and it contains \cs{baselineskip} glue between the boxes.
After a \cs{showbox} command the log file output is:
\begin{verbatim}
\vbox(16.30554+0.0)x5.00002
.\hbox(4.30554+1.94444)x5.00002
..\tenroman g
.\glue(\baselineskip) 5.75002
.\hbox(4.30554+0.0)x5.00002
..\tenroman o
\end{verbatim}
Each time a new level is entered an extra dot is added to
the front of the line. Note that \TeX\ tells explicitly
that the glue is \cs{baselineskip} glue;
it inserts names like this for all automatically inserted glue.
The value of
the baselineskip glue here is such that the baselines of
the boxes are at 12 point distance.

Now let us look at explicit (user) glue. \TeX\ indicates the ratio
by which it is stretched or shrunk. 

\begin{example}
\begin{verbatim}
\hbox to 20pt {\kern10pt \hskip0pt plus 5pt}
\end{verbatim}
gives (indicating that the available stretch has been 
multiplied by~\n{2.0}):
\begin{verbatim}
\hbox(0.0+0.0)x20.0, glue set 2.0
.\kern 10.0
.\glue 0.0 plus 5.0
\end{verbatim}
and
\begin{verbatim}
\hbox to 0pt {\kern10pt \hskip0pt minus 20pt}
\end{verbatim}
gives (the shrink has been multiplied by~\n{0.5})
\begin{verbatim}
\hbox(0.0+0.0)x0.0, glue set - 0.5
.\kern 10.0
.\glue 0.0 minus 20.0
\end{verbatim}
respectively.
\end{example}

This is an example with infinitely stretchable or shrinkable
glue:
\begin{verbatim}
\hbox(4.00000+0.14000)x15.0, glue set 9.00000fil
\end{verbatim}
This means that the horizontal box contained \n{fil} glue, and
it was set such that its resulting width was \n{9pt}.

Underfull boxes are dumped like all other boxes, but
the usual `\n{Underfull hbox detected at line...}'
is given. Overfull horizontal boxes contain a vertical rule
of width \cs{overfullrule}:
\begin{verbatim}
\hbox to 5pt {\kern10pt}
\end{verbatim}
gives
\begin{verbatim}
\hbox(0.0+0.0)x5.0
.\kern 10.0
.\rule(*+*)x5.0
\end{verbatim}


Box leaders are not dumped completely:
\begin{verbatim}
.\leaders 40.0
..\hbox(4.77313+0.14581)x15.0, glue set 9.76852fil
...\tenrm a
...\glue 0.0 plus 1.0fil
\end{verbatim}
is the dump for
\begin{verbatim}
\leaders\hbox to 15pt{\tenrm a\hfil}\hskip 40pt
\end{verbatim}
Preceding or trailing glue around the leader
boxes is also not indicated.

%\point Global statistics
\section{Global statistics}

The parameter \csterm{tracingstats} can be used to force \TeX\
to report at the end of the job the global use of resources.
Some production versions of \TeX\ may not have this option.

As an example, here are the statistics for this book:
\begin{verbatim}
Here is how much of TeX's memory you used:
\end{verbatim}
String memory (bounded by `pool size'):
\begin{verbatim}
 877 strings out of 4649
 9928 string characters out of 61781
\end{verbatim}
Main memory, control sequences, font memory:
\begin{verbatim}
 53071 words of memory out of 262141
 2528 multiletter control sequences out of 9500
 20137 words of font info for 70 fonts,
             out of 72000 for 255
\end{verbatim}
Hyphenation:
\begin{verbatim}
 14 hyphenation exceptions out of 607
\end{verbatim}
Stacks: input, nest, parameter, buffer, and save stack respectively,
\begin{verbatim}
 17i,6n,19p,245b,422s stack positions out of 
 300i,40n,60p,3000b,4000s
\end{verbatim}

\section{Execution tracing}

If your \TeX\ run gives strange results and you want to inspect \TeX's
workings in detail, you can use:
\begin{itemize}
\item \csterm{tracingcommands} shows in detail what executable commands
  are executed; set this to~1 to see primitives that are executed, and
  2~to see if conditionals are taking the true or false branch.
\item \csterm{tracingmacros} shows you the expansion of macros if it is
  at least~1; with 2~the expansion of token parameters is also traced.
\item \csterm{tracingpages} lets you see the page breaking algorithm in action.
\item a positive value of \csterm{tracingrestores} will display values
  that are restored after a group.
\item \csterm{tracinglostchars} will cause a trace messages if you're
  trying to access a position in a font that is not filled.
\item Most of the time these traces go only to the log file, since
  they are rather voluminous. Set \csterm{tracingonline} to a positive
  value to get them also to the terminal.
\end{itemize}

\index{tracing|)}
\endofchapter
\chapter{Errors, Catastrophes,  and Help}\label{error}

When \TeX\ is running, various errors can occur.
This chapter treats how errors in the input are displayed,
and what sort of overflow of internal data structures
of \TeX\ can occur.

\label{cschap:errorcontextlines}\label{cschap:errmessage}\label{cschap:errhelp}
\begin{inventory}
\item [\cs{errorcontextlines}] 
      (\TeX3 only)
      Number of additional context lines shown in error messages.

\item [\cs{errmessage}] 
      Report an error, giving the parameter of this command as message.

\item [\cs{errhelp}] 
      Tokens that will be displayed if the user 
      asks further help after an \cs{errmessage}.

\end{inventory}

%\point Error messages
\section{Error messages}

When \TeX\ is running in \cs{errorstopmode} (which it usually is;
see Chapter~\ref{run} for the other running modes),
errors occurring are reported on the user terminal, and \TeX\
asks the user for further instructions.
Errors can occur either because of some internal condition
of \TeX, or because a macro has issued an \csterm{errmessage}
command.

If an error occurs \TeX\ shows the input 
line 
on which the error occurred. If the offending command was
not on that line but, for instance, in a macro that was
called \ldash possibly indirectly \rdash  from that line,
the line of that command is also shown.
If the offending command was indirectly called,
an additional \csterm{errorcontextlines} number of lines
is shown with the preceding macro calls.

A~value of \cs{errorcontextlines}${}=0$ causes \n{...}
to be printed as the sole indication that there is a context.
Negative values inhibit even this.

For each macro in the sequence that leads to the offending
command,
\TeX\ attempts to display some
preceding and some following tokens.
First one line is displayed ending with
the \ldash indirectly \rdash  offending command; then, one line lower
some following tokens are given. 

\begin{example}
\begin{verbatim}
This paragraph ends \vship1cm with a skip.
\end{verbatim}
gives
\begin{verbatim}
! Undefined control sequence.
l.1 This paragraph ends \vship
                              1cm with a skip.
\end{verbatim}
\end{example}

If \TeX\ is not running in some non-stop mode\label{interaction},
the user is given the chance to do some
\indexterm{error patching}, or to
ask for further information. In general the following
options are available:
\begin{description}\item [\gr{return}]
\TeX\ will continue processing. If the error was something
innocent that \TeX\ could either ignore or patch itself,
this is the easy way out.
\item [\n h]
Give further details about the error.
If the error was caused by an \cs{err\-message} command,
the \csterm{errhelp} tokens will be displayed here.
\item [\n i]
Insert. The user can insert some material. For example,
if a control sequence is misspelled, the correct command can
sometimes be inserted, as
\begin{verbatim}
i\vskip
\end{verbatim}
for the above
example. Also, this is an opportunity for inserting
\cs{show} commands to inspect \TeX's internal state.
However, if \TeX\ is in the middle of
scanning something complicated,
such commands will not be executed, or will even
add to the confusion.
\item [\n s]
    (\cs{scrollmode})
Scroll further errors, but display the messages. 
\TeX\ will patch any further errors.
This is a handy option, for instance if the error occurs
in an alignment, because the number of subsequent errors tends
to be rather large.
\item [\n r]
    (\cs{nonstopmode})
Run without stopping. \TeX\ will never stop for user interaction.
\item [\n q]
    (\cs{batchmode})
Quiet running. \TeX\ will never stop for user interaction,
and does not give any more terminal output.
\item [\n x]
Exit. Abort this run of \TeX.
\item [\n e]
Edit. This option is not available on all \TeX\ system.
If it is, the run of \TeX\ is aborted, and an editor is
started, opening with the input file, maybe even 
on the offending line.
\end{description}


%\point Overflow errors
\section{Overflow errors}
\index{overflow errors|(}

Harsh reality imposes some restrictions on how elaborate
\TeX's workings can get. Some restrictions are imposed by
compile-time constants, and are therefore fairly loose, but
some depend strongly on the actual computer implementation.

Here follows the list of all categories of overflow that
prompt \TeX\ to report `Capacity exceeded'.
Most bounds involved are (determined by) compile-time
constants; their values given here in parentheses are those
used in the source listing of \TeX\ in~\cite{Knuth:TeXbook}.
Actual values may differ, and probably will. Remember
that \TeX\ was developed in the good old days when even
big computers were fairly small.

%\spoint Buffer size {\rm(500)}
\subsection{Buffer size (500)}

Current lines of all files that are open are kept in
\TeX's input buffer, as are control sequence names
that are being built with \verb-\csname...\endcsname-.

%\spoint Exception dictionary {\rm(307)}
\subsection{Exception dictionary (307)}

The maximum number of hyphenation exceptions specified
by \cs{hyphenation} must be a prime number.
Two arrays with this many halfwords are allocated.

Changing this  number makes formats incompatible;
that is, \TeX\ can only use a format that was made by
an \IniTeX\ with the same value for this constant.

%\spoint Font memory (20$\,$000)
\subsection{Font memory (20,000)}

Information about fonts is stored in an array of
memory words. This is easily overflowed by preloading too
many fonts in \IniTeX.

%\spoint Grouping levels
\subsection{Grouping levels}

The number of open groups  should be recordable 
in a quarter word. There is no compile-time constant corresponding
to this.

%\spoint Hash size {\rm(2100)}
\subsection{Hash size (2100)}

Maximum number of control sequences. It is suggested that
this number should not exceed 10\% of the main memory size.
The values in \TeX\ and \IniTeX\ should agree; also the
\n{hash\_prime} values should agree.

This value is rather low; for macro packages that are more
elaborate than plain \TeX\ a value of about 3000 is more
realistic.

%\spoint Number of strings {\rm(3000)}
\subsection{Number of strings (3000)}

The maximum number of strings must be recordable in a half word.

%\spoint Input stack size {\rm(200)}
\subsection{Input stack size (200)}

For each input source an item is allocated on the input stack.
Typical input sources are input files (but their simultaneous
number is more limited; see below), and token lists
such as token variables, macro replacement texts, and
alignment templates. A~macro with `runaway recursion'
(for example, \verb>\def\mac{{\mac}}>)
will overflow this stack.

\TeX\ performs some optimization here: before the last call 
in a token list all token lists ending with this call are
cleared. This process is 
similar to `resolving tail recursion' (see Chapter~\ref{macro}).

%\spoint Main memory size (30$\,$000)
\subsection{Main memory size (30,000)}

Almost all `dynamic' objects of \TeX, such as macro definition
texts and all material on the current page, 
are stored in the main memory array. 
Formats may already take $20\,000$ words of
main memory for macro definitions, and complicated pages containing
for instance the \LaTeX\ picture environment may easily
overflow this array. 

\TeX's main memory is divided in words, and a half word
is supposed to be able to address the whole of the memory.
Thus on current 32-bit computers the most common choice
is to let the main memory size be at most 64K bytes.
A~half word address can then be stored in 16 bits,
half a machine word.

However, so-called `Big \TeX' implementations exist
\thecstoidxsub{TeX}{big}
that have a main memory larger than 64K words.
Most compilers will then allocate  32-bit words for
addressing this memory, even if (say) 18 bits would
suffice. Big \TeX s therefore become immediately
a lot bigger when they cross the 64K threshold.
Thus they are usually not found on microcomputers,
although virtual memory schemes for these are possible;
see for instance~\cite{Thull}.

\TeX\ can have a bigger main memory than \IniTeX;
see Chapter~\ref{TeXcomm} for further details.

%\spoint Parameter stack size {\rm(60)}
\subsection{Parameter stack size (60)}

Macro parameters may contain macro calls with
further parameters. The number of parameters that may occur
nested is bounded by the parameter stack size.

%\spoint Pattern memory {\rm(8000)}
\subsection{Pattern memory (8000)}

Hyphenation patterns are stored in a trie array.
The default size of 8000 hyphenation patterns seems sufficient
for English or Italian,  for example, but it is not for
Dutch or German.

%\spoint Pattern memory ops per language
\subsection{Pattern memory ops per language}

The number of hyphenation ops (see the literature about
hyphenation: \cite{Liang} and appendix~H of~\cite{Knuth:TeXbook})
should be recordable 
in a quarter word. There is no compile-time constant corresponding
to this. \TeX\ version~2 had the same upper bound, but gave no
error message in case of overflow. Again, for languages such
as Dutch and German this bound is too low.
There are versions of \TeX\ that have a higher bound here.

%\spoint Pool size (32$\,$000)
\subsection{Pool size (32,000)}

Strings are error messages and control sequence names.
They are stored using one byte per character.
\TeX\ has initially about $23\,000$ characters worth of
strings.

The pool will overflow if a user defines a large number of
control sequences on top of a substantial macro package.
However, even if the user does not define any new commands
\mdqon
overflow may occur: cross"-referencing schemes also
\mdqoff
work by defining control sequences. For large documents
a pool size of $40\,000$ or $60\,000$ is probably sufficient.

%\spoint Save size {\rm(600)}
\subsection{Save size {\rm(600)}}

Quantities that are assigned to inside a group must be
restored after the end of that group.
The save stack is where the values to be restored are kept;
the size of the
save stack limits the number of values that can be restored.

Alternating global and local assignments to a value
will lead to `save stack build-up': for each local
assignment following a global assignment the
previous value of the variable is saved. Thus an
alternation of such assignments will lead to
an unnecessary proliferation of items on the save stack.

%\spoint Semantic nest size {\rm(40)}
\subsection{Semantic nest size {\rm(40)}}

Each time \TeX\ switches to a mode nested inside another
mode (for instance when processing an \verb-\hbox- inside
a \verb-\vbox-) the current state is pushed on the
semantic nest stack. The semantic nest size is the maximum
number of levels that can be pushed.

%\spoint Text input levels {\rm(6)}
\subsection{Text input levels {\rm(6)}}

The number of nested \verb-\input- files 
has to be very limited,
as the current lines are all kept in the input buffer.

\index{overflow errors|)}

\endofchapter
\chapter{The Grammar of \TeX}\label{gramm}

Many  chapters in this book contain pieces of the
grammar that defines the formal syntax of \TeX.
In this chapter the structure of the rewriting rules of the
grammar is explained, and some key notions are presented.

In \TeXbook\ a grammar appears in Chapters~24--27.
An even more rigorous grammar of \TeX\ can be found in~\cite{Appelt}.
The grammar presented in this book is virtually identical 
to that of \TeXbook.

%\point Notations
\section{Notations}

Basic to the grammar are 
\begin{description}\item [grammatical terms]
These are enclosed in angle brackets:
\begin{disp}\gr{term}\end{disp}
\item [control sequences]
These are given in typewriter type with a backslash for
the escape character:
\begin{disp}\cs{command}\end{disp}
\end{description}
Lastly there are 
\begin{description}\item [keywords]
Also given in typewriter type
\begin{disp}\n{keyword}\end{disp}
This is a limited collection of words that have a special
meaning for \TeX\ in certain contexts; see below.\end{description}

The three elements of the grammar are used in syntax rules:
\begin{disp}\gr{snark} $\longrightarrow$ \n{boojum} $|$ \gr{empty}
\end{disp}
This rule says that the grammatical entity \gr{snark}
is either the keyword \n{boojum}, or the grammatical
entity \gr{empty}.

There are two other notational conventions.
The first is that the double quote
is used to indicate hexadecimal (base~16) notation.
For instance \verb>"ab56> stands for $10\times16^3+11\times16^2
+5\times16^1+6\times16^0$. The second convention
is that subscripts are used to denote category codes.
Thus \n{a}$_{12}$ denotes an `a' of category~12.

%\point[keywords] Keywords
\section{Keywords}
\label{keywords}
\index{keywords|(}

A keyword is sequence of characters (or character tokens)
of any category code but~13\index{category!13} (active).
Unlike the situation in control sequences, \TeX\ does not 
distinguish between lowercase and uppercase characters
in keywords. Uppercase characters in keywords are converted to
lowercase by adding 32 to them; the \cs{lccode} and \cs{uccode}
are not used here. Furthermore, any keyword can be preceded by
optional spaces.

Thus both \n{true cm} and \n{truecm} are legal.
By far the strangest example, however, is provided
by the grammar rule
\begin{disp}\gr{fil unit} $\longrightarrow$ \n{fil} $|$ \gr{fil unit}\n l
\end{disp} which implies that \hbox{\n{fil L l}} is also
a legal \gr{fil dimen}. Strange errors can ensue from this;
see page~\pageref{fil:l:l} for an example.

Here is the full list of all keywords: \n{at}, \n{bp},
\n{by}, \n{cc}, \n{cm}, \n{dd}, \n{depth}, \n{em}, \n{ex},
\n{fil}, \n{height}, \n{in}, \n l, \n{minus}, \n{mm}, \n{mu},
\n{pc}, \n{plus}, \n{pt}, \n{scaled}, \n{sp}, \n{spread},
\n{to}, \n{true}, \n{width}.

\index{keywords|)}

\section{Specific grammatical terms}

Some grammatical terms appear in a lot of rules.
One such term is \gr{optional spaces}.
The term \indextermbus{optional}{space} is probably clear enough,
but here is the formal definition:
\begin{disp}\gr{optional spaces} $\longrightarrow$
     \gr{empty} $|$ \gr{space token}\gr{optional spaces}
     \end{disp}
which amounts to saying that \gr{optional spaces}
is zero or more space tokens.

Other terms may not be so immediately obvious.
Below are some of them.

%\spoint \gr{equals}
\subsection{\gr{equals}}

In assignments the equals sign is optional; therefore there
is a term
\begin{disp}\gr{equals} $\longrightarrow$ \gr{optional spaces}
     $|$ \gr{optional spaces}$=_{12}$\end{disp}
in \TeX's grammar.
%% 
\begin{comment}
%% One assignment exists where the equals sign cannot
%% be left out:
%%
\begin{verbatim}
%% \let\spacetoken= %assign a space
\end{verbatim}
%% Here the space would have been skipped in \TeX's input processor
%% if the equals sign had been left out.
\end{comment}

%\spoint \gr{filler}, \gr{general text}
\subsection{\gr{filler}, \gr{general text}}

More obscure than the \gr{optional spaces} is the combination
of spaces and \cs{relax} tokens that is allowed
in some places, for instance
\begin{verbatim}
\setbox0= \relax\box1
\end{verbatim}
The quantity involved is 
\begin{disp}\gr{filler} $\longrightarrow$ \gr{optional spaces}
     $|$ \gr{filler}\cs{relax}\gr{optional spaces}\end{disp}
One important occurrence of \gr{filler} is in
\begin{disp}\gr{general text} $\longrightarrow$
     \gr{filler}\lb\gr{balanced text}\gr{right brace}
     \end{disp}
A \gr{general text} follows such control sequences as
\cs{message}, \cs{uppercase}, or \cs{mark}. The braces around
the \gr{balanced text} are explained in the next point.

%\spoint \lb\rb\ and \gr{left brace}\gr{right brace}
\subsection{\lb\rb\ and \gr{left brace}\gr{right brace}}

The \TeX\ grammar uses a perhaps somewhat unfortunate
convention for braces. First of all 
\begin{disp}\lb\ and \rb\end{disp}
stand for braces that are either explicit open/close group
characters, or control sequences defined by \cs{let},
such as
\begin{verbatim}
\let\bgroup={ \let\egroup=}
\end{verbatim}
The grammatical terms 
\begin{disp}\gr{left brace} and \gr{right brace}
\end{disp} stand for explicit open/close group characters,
that is, characters of categories 1 and~2 respectively.

Various combinations of these two kinds of braces exist.
Braces around boxes can be implicit:
\begin{disp}\cs{hbox}\gr{box specification}\lb
     \gr{horizontal mode material}\rb\end{disp}
Around a macro definition there must be explicit braces:
\begin{disp}\gr{definition text} $\longrightarrow$
     \gr{parameter text}\gr{left brace}\gr{balanced text}\gr{right brace}
     \end{disp}
Finally, the \gr{general text} that was mentioned above
has to be explicitly closed, but it can be implicitly opened:
\begin{disp}\gr{general text} $\longrightarrow$
     \gr{filler}\lb\gr{balanced text}\gr{right brace}
     \end{disp}
The closing brace of a \gr{general text} has to be explicit,
since a general text is a token list, which may
contain \cs{egroup} tokens.
\TeX\ performs expansion to find the opening 
brace of a \gr{general text}.

%\spoint \gr{math field}
\subsection{\gr{math field}}

In math mode various operations such as subscripting
or applying \cs{underline} take an argument that
is a \gr{math field}: either a single symbol, or
a group. Here is the exact definition.
\begin{disp}\gr{math field} $\longrightarrow$
    \gr{math symbol} $|$ \gr{filler}\lb\gr{math mode material}\rb\nl
 \gr{math symbol}  $\longrightarrow$ \gr{character} $|$
    \gr{math character}
\end{disp}
See page~\pageref{character} for \gr{character},
\alt
and page~\pageref{math:character} for \gr{math character}.

%\point[2vs3] Differences between \TeX\ versions 2 and 3
\section{Differences between \TeX\ versions 2 and 3}
\label{2vs3}

In 1989 Knuth released \TeX\ version~3.0, which is 
\thecstoidxsub{TeX}{version 2}
\thecstoidxsub{TeX}{version 3}
the first real change in \TeX\ since version~2.0,
which was released in~1986 (version~0 of \TeX\ was
released in 1982; see~\cite{Knuth:TeXerrors} for more about
the history of \TeX).
All intermediate versions were merely bug fixes.

The main difference between versions 2~and~3 lies
in the fact that 8-bit input has become possible.
Associated with this, various quantities that
used to be 127 or~128 have been raised to 255
or~256 respectively. Here is a short list.
The full description is in~\cite{K:TeX23}.

\begin{inventory}\message{Remove other TeX3 refernces!}
\item All `codes' (\cs{catcode}, \cs{sfcode}, et cetera;
    see page~\pageref{codename})
    now apply to 256 character codes instead of~128.
\item A character with code \cs{endlinechar}
    is appended to the line unless this parameter is negative
or more than~255 (this was~127) (see page~\pageref{append:elc}).
\item No escape character is output by \cs{write} and
    other commands if \cs{escapechar} is negative or more than~255
(this was~127) (see page~\pageref{use:escape}).
\item The \verb>^^> replacement mechanism has been extended
    (see page~\pageref{hathat}).
\item Parameters \cs{language}, \cs{inputlineno},
    \cs{errorcontextlines}, \cs{lefthyphenmin}, \cs{righthyphenmin},
\cs{badness}, \cs{holdinginserts}, \cs{emergencystretch},
and commands \cs{noboundary}, \cs{setlanguage}
have been added.
\item The value of \cs{outputpenalty} is no longer zero
    if the page break was not at a penalty item;
    instead it is~$10\,000$ (see page~\pageref{break:penalty}).
\end{inventory}

The plain format has also been updated, mostly
with default settings for parameters such as
\cs{lefthyphenmin}, but also a few macros have been added.

\endofchapter
%%%% end of input file [syntax]

\chapter{Glossary of \TeX\ Primitives}
\begin{raggedright}
\input glossary
\end{raggedright}
\endofchapter


\chapter{Tables}
%\stepcounter{chapter}\addcontentsline{toc}{section}{Tables}\pagestyle{plain}\clearpage
\input tables
\endofchapter

\chapter{Index}
%\addcontentsline{toc}{section}{Index}
\printindex
%\input \jobname.ind
%\endofchapter % this winds up on a page of its own.

\addcontentsline{toc}{section}{Bibliography}
\mdqon
\bibliography{tex}
\bibliographystyle{plain}
\mdqoff
\endofchapter

%%%% end of input file [tables]

\chapter*{Change log}

\section*{Version 1.5}

Slight clarifications.

\section*{Version 1.4}

Reinstated a couple of figures (baseline distance,
plus one one paragraph shape that didn't make it
into the original book.)

\section*{Version 1.3}

Finally managed to reinstate the tables chapter.

Starting to add more concepts to the index.

\section*{Version 1.2}

Added chapter references to glossary.

Fixed a bunch of typographic accidents.

\section*{Version 1.1}

Small remark about \cs{afterassignment} after macro definitions.

Trouble with indexing macros fixed, I hope.

Separate letter and a4 versions.

Better intro for the chapter \ref{space} on spacing.

\end{document}
