\section{Line breaking: \cs{tracingparagraphs}}

If \cs{tracingparagraphs} is positive, \TeX's line breaking
\cstoidx tracingparagraphs\par
algorithm will generate trace output. However, on some \TeX\
implementations this trace mode may have been disabled to get a 
faster running system.

Consider an example paragraph of \TeX:

\begin{verbatim}
\hsize=3in \parindent=0cm \frenchspacing
\pretolerance=500
This is a sample paragraph to show the trace output that
\TeX's line breaking algorithm produces. Some \TeX\ systems
cannot generate this trace, as the relevant piece of code
has been commented out for speed optimisation. 
With ever faster computers this won't be necessary any more.
\end{verbatim}

\TeX\ first attempts to break the paragraph without
hyphenation, and it will accept solutions where each
line has a badness less than \cs{pretolerance}.
%
\begin{verbatim}
@firstpass
\end{verbatim} Report that the first pass has started;
%
\begin{verbatim}
[]\tenrm This is a sample paragraph to show the trace 
\end{verbatim}
Apparently this is the only way to fill the first line;
%
\begin{verbatim}
@ via @@0 b=263 p=0 d=84529
\end{verbatim} and doing so
had badness~263, a zero penalty, and a resulting 84529
demerit points.
%
\begin{verbatim}
@@1: line 1.0 t=84529 -> @@0
\end{verbatim} Conclusion:
breakpoint~1 (\verb>@@1>) occurs on line~1, and it makes the
line `very loose' (indicated by the~\n{.0}), 
and the total demerits are
84529 if the previous breakpoint was number~0.

The first pass is now aborted.
%
\begin{verbatim}
@secondpass
[]\tenrm This is a sam-ple para-graph to show the trace out-
@\discretionary via @@0 b=2 p=50 d=2644
@@1: line 1.2- t=2644 -> @@0
\end{verbatim}
With a very small badness of~2, but with 50 penalty points
for breaking at a hyphen, this line is `decent' 
(indicated by the~\n{.2}), and the total of demerit points
is~2644.

The second line is also straighforward:
%
\begin{verbatim}
put that T[]X's line break-ing al-go-rithm pro-duces. 
@ via @@1 b=0 p=0 d=100
@@2: line 2.2 t=2744 -> @@1
\end{verbatim}
The demerits now derive solely from the \cs{linepenalty},
which is~10. Similarly the third line:
%
\begin{verbatim}
Some T[]X sys-tems can-not gen-er-ate this trace, as 
@ via @@2 b=1 p=0 d=121
@@3: line 3.2 t=2865 -> @@2
\end{verbatim}

For the fourth line two possibilities exist:
it can be set `loose' with 9409 demerit points
%
\begin{verbatim}
the rel-e-vant piece of code has been com-mented 
@ via @@3 b=87 p=0 d=9409
@@4: line 4.1 t=12274 -> @@3
\end{verbatim}
or, fitting in an extra word, it can be set `tight' with
2601 demerit points
%
\begin{verbatim}
out 
@ via @@3 b=41 p=0 d=2601
@@5: line 4.3 t=5466 -> @@3
\end{verbatim}

Line 5 can be set in three ways:
coming from breakpoint~4 it can be broken as
%
\begin{verbatim}
for speed op-ti-mi-sa-tion. With ever faster com-
@\discretionary via @@4 b=0 p=50 d=2600
@@6: line 5.2- t=14874 -> @@4
\end{verbatim}
and coming from breakpoint~5 there are two ways:
%
\begin{verbatim}
put-
@\discretionary via @@5 b=2 p=50 d=2644
@@7: line 5.2- t=8110 -> @@5
\end{verbatim}
and
\begin{verbatim}
ers 
@ via @@5 b=84 p=0 d=8836
@@8: line 5.3 t=14302 -> @@5
\end{verbatim}
Of the three, the last possibility is the only one that
does not involve hyphenating line~5.

As line 6 is the last line of the paragraph, coming from
breakpoints 6 or~7 gives an extra 5000 demerit points
from the \cs{finalhyphendemerits}.
%
\begin{verbatim}
this won't be nec-es-sary any more. 
@\par via @@6 b=0 p=-10000 d=5100
@\par via @@7 b=0 p=-10000 d=5100
@\par via @@8 b=0 p=-10000 d=100
@@9: line 6.2- t=13210 -> @@7
\end{verbatim}
However, coming from breakpoint 7 still gives the least
demerits.
